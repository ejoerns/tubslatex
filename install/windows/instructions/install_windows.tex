\documentclass[12pt]{scrartcl}

\usepackage[utf8]{inputenc}
\usepackage[LY1]{fontenc}
\usepackage{nexus}
\usepackage[a4paper]{tubstypearea}
\usepackage{tubslayout}
\usepackage{scrpage2}
\usepackage{enumerate}
\usepackage{listings}
\lstset{basicstyle=\ttfamily}

\cfoot{}
\ohead{\thepage}
\pagestyle{scrheadings}

\title{tubslatex -- Manuelle Installation}
\subtitle{Anleitung für MiKTeX 2.9}
\author{Enrico Jörns}


\begin{document}


\maketitle

Die folgende Installationsbeschreibung ist für MiKTeX in der Version 2.9
geschrieben. Für die Unterstützung von Vorgänger- oder Folgeversionen gibt
es keine Garantie.

Folgende Vorgehensweise wird zur manuellen Installation von tubslatex unter
MiKTeX empfohlen. Es wird vom Vorhandensein der benötigten Dateistruktur in
einer entsprechenden zip-Datei ausgegangen.

\begin{enumerate}
  \item {\bfseries Dateien kopieren}
  
    Alle zu installierende Dateien sollten entweder in ein bestehendes lokales
    texmf-Verzeichnis kopiert werden oder in ein neu angelegtes Verzeichnis
    (z.\,B. \lstinline{C:\tubslatex}).
    
    In dieses sind die Ordner \lstinline{tex}, \lstinline{doc} und 
    \lstinline{fonts} zu kopieren.
    
    \begin{enumerate}[a)]
      \item {\bfseries }
        
        Für den Fall, dass ein neues Verzeichnis angelegt wurde,
        muss dies MiKTeX noch bekannt gemacht werden.
        Dazu sind die MiKTeX-Einstellungen (Start$\to$Programme$\to$MiKTeX 2.8 
        $\to$Maintenance$\to$Settings) aufzurufen.
        Im Reiter \glqq Roots\grqq\ kann der neue Pfad hinzugefügt werden.
    \end{enumerate}
  \item {\bfseries Datenbank aktualisieren}
    
    Anschließend ist es noch ratsam den Button \glqq Refresh FNDB\grqq\
    zu drücken, um die Dateidatenbank zu aktualisieren. Alternativ kann auch 
    der Konsolenbefehl \lstinline{initexmf -u} verwendet werden.
    
  \item {\bfseries Font-Installation}
    Um die Schrift Nexus benutzen zu können, sind zunächst wie oben beschrieben
    die benötigten Dateien zu kopieren.
    
    \begin{enumerate}

      \item {\bfseries map-Dateien bekannt machen}

      Mit dem Konsolen-Befehl \lstinline{initexmf --edit-config-file updmap}
      wird ein Editor geöffnet in den folgender Text einzutragen ist:

      \begin{lstlisting}
Map NexusProSans.map
Map NexusProSerif.map
      \end{lstlisting}

      \item {\bfseries Font maps updaten}

        Danach ist ein Update der Font-Datenbank erforderlich. Dies geschieht
        mittels\\ \lstinline{initexmf --admin --mkmaps}. Die Konsole muss dazu
        mit Administratorrechten gestartet werden.
        
        \paragraph{Hinweis:} Es ist auch möglich, die Datenbank im 
          Nicht-Administrator-Modus zu erneu\-ern. Dies ist jedoch nur bei einer
          Einzelbenutzer-Installation von MiKTeX sinnvoll. Wird
          \lstinline{initexmf} ohne die Option \lstinline{--admin} aufgerufen,
          so wird eine lokale map-Liste angelegt, auf die MiKTeX anschließen für
          den aktuellen Benutzer ausschließlich zugreift.
          Eine benutzerweite Installation von Fonts ist dann noch 
          möglich, jedoch muss jedes mal manuell \lstinline{initexmf --mkmaps}
          aufgerufen werden.
%           Welche map-Liste aktuell verwendet wird, lässt sich mit dem Befehl
    \end{enumerate}
    
    Danach sollte die Installation abgeschlossen sein und alle Pakete verwendbar 
    sein.
\end{enumerate}


\end{document}
