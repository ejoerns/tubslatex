%% ToDo-List:
%% - add description for option colorblocks
%% Changelog:
%% V 0.2.1:
%%  - 
%%  - 
%% V 0.2.2:
%%  - Footline fuer mehrzeiligen Autor angepasst
%%  - pdftitle benutzt short title, wenn verfuegbar
%% V 0.2.4:
%%  - author-Fontsize auf \footnotesize korrigiert
%%  - Titelleite neu programmiert (etwas größer, Alignment-Probleme behoben)
%%  - Benutze tuGrey60, falls Option colorhead gesetzt
%%  - Abstand zwischen section und subection im toc um 0.5mm reduziert
%%  - Option tinytocinheader funktioniert (wieder)
%% V 0.2.X:
%%  - Encodingprobleme bei eulervm behoben.
%%  - Hinweis zu [plain] für Titelseite, chead gefixt, Bäker als Autor genannt
\documentclass[cmyk,a4paper,11pt,colorscheme=green,TUBStitlepage=picture]{tubsreprt}

\usepackage[utf8x]{inputenc}
\usepackage[LY1]{fontenc}
\usepackage{ngerman}
\usepackage{listings}
\lstset{basicstyle=\ttfamily}
\usepackage{xcolor}
\usepackage[colorlinks=true]{hyperref}
\usepackage{tikz} % TODO: should be RequirePackage
\usepackage{booktabs}

\chead{}  % quickfix

\input{beamerug-macros}


\author{Enrico Jörns, Martin Bäker}
\title{LaTeX-Präsentationen im Corporate Design}
% \institute{TU-Braunschweig}
\subject{Anleitung und Dokumentation}
% \institute{Institut für Lorem Ipsum}
% \address{123 Fakestreet \\ 1337 Notown}

\parindent0mm
\parskip\medskipamount

\begin{document}

\maketitle

\begin{abstract}
Es ist freilich nicht jedem Menschen zuzumuten, seine Präsentationen in
Powerpoint oder ähnlichen Programmen erstellen zu müssen.
Aus diesem Grund wurde viel Zeit und Mühe investiert, das Corporate-Design der
TU-Braunschweig in ein Vorlage für das \LaTeX-Beamer-Paket zu gießen.\medskip

Dabei wurde unter anderem Wert darauf gelegt, so wenig wie möglich an dem
Standardverhalten und Kommandoumfang des Beamer-Paketes zu ändern und darüber
hinaus so viel Flexibilität wie möglich und sinnvoll zu behalten.\medskip

Das Beamer-Theme stammt maßgeblich von Martin Bäker. Ebenfalls daran beteiligt
waren Mr X, Mrs Y und Enrico Jörns.%TODO...
\bigskip

Wir wünschen viel Erfolg und Freude bei der Arbeit mit der Vorlage.
\bigskip

{\hfill Braunschweig, \today}
\vfill
\footnotesize{Titelbild (Audimax TU Braunschweig):\\
\url{http://commons.wikimedia.org/wiki/File:Braunschweig_TU_Audimax.jpg}\\
lizenziert unter CC BY-SA 3.0 (\url{http://creativecommons.org/licenses/by-sa/3.0/deed.de})\\
von Igge (\url{http://de.wikipedia.org/wiki/Benutzer:-Igge-})}
\end{abstract}

\clearpage
\tableofcontents

\chapter{Installation}

\section{Miktex}

Unter Version 2.9 sollte es keine Probleme bei der Verwendung der Vorlage geben.
Alle benötigten Pakete werden automatisch nachgeladen.

\section{Texlive}

\subsection{}

\subsection{Fonts}

\paragraph{Installation von Arial unter Linux}

Erscheint die Warnung, dass das Paket \lstinline{uarial} nicht gefunden wurde,
so ist Arial noch zu installieren.

Dies lässt sich unter Linux mit Hilfe des Kommandos
\lstinline{getnonfreefonts-sys} durchführen.

\begin{lstlisting}
sudo getnonfreefonts-sys arial-urw
\end{lstlisting}

\begin{lstlisting}
getnonfreefonts arial-urw
\end{lstlisting}

Bezug von CTAN: \url{http://www.ctan.org/tex-archive/fonts/urw/arial/}

\input{content} % Hauptinhalt des Dokuments

\end{document}