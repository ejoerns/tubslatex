\documentclass[a4paper,colorscheme=green,TUBStitlepage=picture]{tubsreprt}

\usepackage[utf8x]{inputenc}
\usepackage[LY1]{fontenc}
\usepackage{ngerman}
\usepackage{listings}
\lstset{basicstyle=\sffamily}
\usepackage{xcolor}
\usepackage[colorlinks=true]{hyperref}
\usepackage{tikz} % TODO: should be RequirePackage

\input{beamerug-macros}


\author{Enrico Jörns}
\title{LaTeX-Präsentationen im Corporate Design}
\institute{TU-Braunschweig}
\subject{Anleitung und Dokumentation}
% \institute{Institut für Lorem Ipsum}
% \address{123 Fakestreet \\ 1337 Notown}

% \renewcommand{\maketitle}{
% \begin{TUBStitlepage}
%   \TUBSGaussGridImageBox{6}{8}{0}{6}{./titlepicture}
%   \TUBSGaussGridTextBox{6}{4}{0}{6}{\sffamily\bfseries\Huge\TUBStitle}
%     {primary_white}{secondary_dark}{top}
%   \TUBSGaussGridTextBox{6}{4}{0}{6}{\sffamily\Huge\TUBSsubject\\ \Large
%     \TUBSdate} {primary_white}{transparent}{bottom}
%   \TUBSGaussGridTextBox{3}{4}{0}{6}{\sffamily\huge
%     \TUBSauthor} {primary_white}{secondary_medium}{center}
%   \TUBSGaussGridTextBox{0}{3}{0}{6}{Lorem ipsum dolor sit amet, consetetur
%     sadipscing elitr, sed diam nonumy eirmod tempor invidunt ut labore et
% dolore
%     magna aliquyam erat, sed diam voluptua. At vero eos et accusam et justo
% duo
%     dolores et ea rebum. Stet clita kasd gubergren, no sea takimata sanctus
% est
%     Lorem ipsum dolor sit amet. Lorem ipsum dolor sit amet, consetetur
%     sadipscing elitr, sed diam nonumy eirmod tempor invidunt ut labore et
% dolore %     magna aliquyam erat, sed diam voluptua. At vero eos et accusam et
% justo duo %     dolores et ea rebum. Stet clita kasd gubergren, no sea
% takimata % sanctus est %     Lorem ipsum dolor sit
% amet.}{secondary_dark}{secondary_light}{top}
% \end{TUBStitlepage}
% }


\begin{document}

\maketitle

\tableofcontents
\begin{abstract}
Es ist freilich nicht jedem Menschen zuzumuten, seine Präsentationen in
Powerpoint oder ähnlichen Programmen erstellen zu müssen.
Aus diesem Grund wurde viel Zeit und Mühe investiert, das Corporate-Design der
TU-Braunschweig in ein Vorlage für das \LaTeX-Beamer-Paket zu gießen.\medskip

Dabei wurde unter anderem Wert darauf gelegt so wenig wie möglich an dem
Standardverhalten und Kommandoumfang des Beamer-Paketes zu ändern und darüber
hinaus so viel Flexibilität wie möglich und sinnvol zu behalten.\medskip

Das Beamer-Theme stammt maßgeblich von Martin Bäker. Ebenfalls daran beteiligt
waren Mr X, Mrs Y und Enrico Jörns.%TODO...
\bigskip

Wir wünschen viel Erfolg und Freude bei der Arbeit mit der Vorlage.
\bigskip

{\hfill Braunschweig, \today}
\end{abstract}


\chapter{Installation}

\section{Miktex}

Unter Version 2.9 sollte es keine Probleme bei der Verwendung der Vorlage geben.
Alle benötigten Pakete werden automatisch nachgeladen.

\section{Texlive}

\subsection{Fonts}

\paragraph{Installation von Arial unter Linux}

Erscheint die Warnung, dass das Paket \lstinline{uarial} nicht gefunden wurde,
so ist Arial noch zu installieren.

Dies lässt sich unter Linux mit Hilfe des Kommandos
\lstinline{getnonfreefonts-sys} durchführen.

\begin{lstlisting}
sudo getnonfreefonts-sys arial-urw
\end{lstlisting}

Bezug von CTAN: \url{http://www.ctan.org/tex-archive/fonts/urw/arial/}

\chapter{Anwendung}

Grundsätzlich handelt es sich bei der Vorlage lediglich um ein Theme für
das beamer-Paket. An der Funktionalität von beamer wird so gut wie nichts
geändert, sodass für allgemeine Fragen zur Präsentationserstellung auf die
entsprechende Dokumentation verwiesen wird. % TODO: LINK/BIB

Beschrieben werden hier alle Besonderheiten des Corporate-Design-Themes, sowie
einige damit verbundene allgemeiner Hinweise gegeben.



\section{Titelfolie}

\subsection{Bild}



\section{Fuzeile}

In der Fußzeile steht laut Vorgabe Datum und Seitenzahl.
Ist das mal nicht gewollt, einfach die Optionen \lstinline{nopagenum} bzw.
\lstinline{nodate} angeben.

Autor und Titel kommen aus dem optionalen Argument von \lstinline{\author}
und \lstinline{\title}, sodass auch Kurztitel möglich sind.
Ändert man im Babel-Paket die Sprache, ändert sich auch automatisch
die Fußzeile (\textit{page} statt Seite, anderes Datumsformat).


\section{Skalierbarkeit}

Support von aspectratio


\section{Optionen}

% \command{\usetheme\oarg{options}\marg{TU-CD}}

\subsection{Kopf-/Fußbereich}

\begin{classoption}{nopagenum}
  Blendet die mitlaufende Seitenzählung im Fuß der Folie aus
\end{classoption}

\begin{classoption}{nodate}
  Blendet das Datum im Fuß der Folie aus.
\end{classoption}

\begin{classoption}{tocinheader}
  Erzeugt ein mitlaufendes Inhaltsverzeichnis im Folienkopf,
  das auch zur Navigation im Dokument genutz werden kann
  (entsprechend der Standard-Beamer-Klassen).
  
  Angezeigt werden standardmäßig sections und subsections.
  Sollen nur sections angezeigt werden, bitte die Option
  \lstinline{nosubsectionsinheader} verwenden.
  
\end{classoption}

\begin{classoption}{tinytocinheader}
  Wie \lstinline{tocinheader}, jedoch mit kleinerer Schrift.

  Kann eingesetzt werden für den Fall, dass das mit \lstinline{tocinheader}
  erzeugte Inhaltsverzeichnis zu breit wirkt oder die Foliendimension
  überragt.
\end{classoption}

\begin{classoption}{nosubsectionsinheader}
  Deaktiviert die Anzeige von subsections im Header.
  
  Nur wirksam, wenn Option \lstinline{tocinheader} oder
  \lstinline{tinytocinheader} verwendet wird!
\end{classoption}

\begin{classoption}{nologoinfoot}
  Deaktiviert die Anzeige des Logos im Fußbereich.
\end{classoption}

\subsection{Farben}

\begin{classoption}{rgbprint}
  Ausgabe in RGB-Druckfraben
\end{classoption}

\begin{classoption}{cmyk}
  Ausgabe in CMYK-Farben.
\end{classoption}


\begin{classoption}{mathserif}
  Beschreibung
\end{classoption}

\begin{classoption}{fleqn}
  Beschreibung
\end{classoption}

\begin{classoption}{minion}
  Beschreibung
\end{classoption}

\begin{classoption}{widetoc}
  Beschreibung
\end{classoption}

\begin{classoption}{narrowtoc}
  Beschreibung
\end{classoption}

\subsection{Schriftgröße}

\begin{classoption}{9pt}
  Beschreibung
\end{classoption}

\begin{classoption}{10pt}
  Beschreibung
\end{classoption}

\begin{classoption}{11pt}
  Beschreibung
\end{classoption}

\begin{classoption}{12pt}
  Beschreibung
\end{classoption}

\begin{classoption}{14pt}
  Beschreibung
\end{classoption}





\end{document}