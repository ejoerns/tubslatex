\documentclass[cmyk,a4paper,colorscheme=green,TUBStitlepage=picture]{tubsreprt}

\usepackage[utf8x]{inputenc}
\usepackage[LY1]{fontenc}
\usepackage{ngerman}
\usepackage{listings}
\lstset{basicstyle=\ttfamily}
\usepackage{xcolor}
\usepackage[colorlinks=true]{hyperref}
\usepackage{tikz} % TODO: should be RequirePackage
\usepackage{booktabs}

\input{beamerug-macros}


\author{Enrico Jörns}
\title{LaTeX-Präsentationen im Corporate Design}
\institute{TU-Braunschweig}
\subject{Anleitung und Dokumentation}
% \institute{Institut für Lorem Ipsum}
% \address{123 Fakestreet \\ 1337 Notown}

\parindent0mm
\parskip\medskipamount

\begin{document}

\maketitle

\tableofcontents
\begin{abstract}
Es ist freilich nicht jedem Menschen zuzumuten, seine Präsentationen in
Powerpoint oder ähnlichen Programmen erstellen zu müssen.
Aus diesem Grund wurde viel Zeit und Mühe investiert, das Corporate-Design der
TU-Braunschweig in ein Vorlage für das \LaTeX-Beamer-Paket zu gießen.\medskip

Dabei wurde unter anderem Wert darauf gelegt, so wenig wie möglich an dem
Standardverhalten und Kommandoumfang des Beamer-Paketes zu ändern und darüber
hinaus so viel Flexibilität wie möglich und sinnvol zu behalten.\medskip

Das Beamer-Theme stammt maßgeblich von Martin Bäker. Ebenfalls daran beteiligt
waren Mr X, Mrs Y und Enrico Jörns.%TODO...
\bigskip

Wir wünschen viel Erfolg und Freude bei der Arbeit mit der Vorlage.
\bigskip

{\hfill Braunschweig, \today}
\end{abstract}


\chapter{Installation}

\section{Miktex}

Unter Version 2.9 sollte es keine Probleme bei der Verwendung der Vorlage geben.
Alle benötigten Pakete werden automatisch nachgeladen.

\section{Texlive}

\subsection{Fonts}

\paragraph{Installation von Arial unter Linux}

Erscheint die Warnung, dass das Paket \lstinline{uarial} nicht gefunden wurde,
so ist Arial noch zu installieren.

Dies lässt sich unter Linux mit Hilfe des Kommandos
\lstinline{getnonfreefonts-sys} durchführen.

\begin{lstlisting}
sudo getnonfreefonts-sys arial-urw
\end{lstlisting}

Bezug von CTAN: \url{http://www.ctan.org/tex-archive/fonts/urw/arial/}

\chapter{Anwendung}
fre

Grundsätzlich handelt es sich bei der Vorlage lediglich um ein Theme für
das beamer-Paket. An der Funktionalität von beamer wird so gut wie nichts
geändert, sodass für allgemeine Fragen zur Präsentationserstellung mit beamer
auf die entsprechende Dokumentation verwiesen wird. % TODO: LINK/BIB

Beschrieben werden hier alle Besonderheiten des Corporate-Design-Themes, sowie
einige damit verbundene allgemeiner Hinweise gegeben.



\section{Titelfolie}

Die Titelfolie ist im Absender/Kommunikations-Bereichslayout mit
Spiegelband-Logo im Sinne der allgemeinen Gestaltungsprinzipien des
Corporate Design gehalten.

Die Kommunikationsfläche ist nach Vorlage des Gaußrasters %TODO: ref
in 3 Bereiche aufgeteilt:
Ein Bildbereich, der ein Foto oder eine Grafik als Blickfang enthalten
sollte,
darunter der Titelbereich, der Präsentationstitel, sowie alle relevanten
Informationen trägt
und zum Abschluss ein einfarbig-roter Streifen.

Zusätzlich kann in der rechten oberen Ecke des Absenderbereich ein
Instituts-Logo platziert werden.

\subsection{Befehle}

Es können die Standardbefehle zur Titelseitenerstellung, wie
\lstinline{\title},
\lstinline{\subtitle},
\lstinline{\author}
und \lstinline{\logo} verwendet werden.

Die Titelseite wird normal mit \lstinline{\titlepage} erzeugt:

\begin{lstlisting}
\begin{frame}[plain]
  \titlepage
\end{frame}
\end{lstlisting}

Für die Einbindung des Titelbildes stehen weitere Nicht-Standard-Befehle zur
Verfügung.


\subsection{Bild}

Damit das Bild auf der Titelfolie den gesamten zur Verfügung stehenden Raum
ausfüllen kann, muss es ein korrektes Seitenverhältnis haben.
In Tabelle \ref{tab:picratio} sind Bild-Seiten\-verhältnisse für verschiedene
Seitenverhältnisse der Präsentation aufgeführt.

\begin{table}[ht]
\centering
\begin{tabular}{ll}
\toprule
\bfseries Präsentation  & \bfseries  Bild  \\
\midrule
$3:4$ &  $1:3,146$ \\
$16:10$ & \\
\bottomrule
\end{tabular}
\caption{Bild-Seitenverhältnisse}
\label{tab:picratio}
\end{table}%TODO: Tabelle füllen

Werden keine korrekten Verhältnisse verwendet, wird das Bild vertikal
skaliert und zentriert im Bildbereich dargestellt.

Der Bildbereich kann theoretisch auch freigelassen werden, er ist mit
einem Gelbton aus den Sekundärfarben eingefärbt.\medskip

Die Einbindung der Bildes erfolgt mit dem Befehl \lstinline{\titlegraphic}.

Die Verwendung von \lstinline{\titlegraphicsheight} als Höhenbeschränkung für
die Grafik stellt deren automatische korrekte vertikale Skalierung sicher.

\begin{example}
\begin{lstlisting}
\titlegraphic{\includegraphics[height=\titlegraphicsheight]
    {titlepicture}}
\end{lstlisting}
\end{example}

\subsection{Logo}

Das mittels \lstinline{\logo} eingebundene Logo wird in der rechten oberen
Ecke des Absenderbereiches angezeigt und standardmäßig auch auf allen weiteren
Folien im Fußbereich, sofern nicht die Option \lstinline{nologoinfoot}
verwendet wird.

Die Verwendung von \lstinline{\logoheight} als Höhenbeschränkung für Grafiken
stellt deren automatische korrekte vertikale Skalierung sicher.
Das Seitenverhältnis der verwendeten Grafik-Datei ist dabei relativ
frei, sollte jedoch nach Möglichkeit und Gründen der Lesbarkeit mehr Breite als
Höhe haben.

\begin{example}
\begin{lstlisting}
\logo{\includegraphics[height=\logoheight]{institut.jpg}}
\end{lstlisting}
\end{example}


\section{Fuzeile}

In der Fußzeile steht laut Vorgabe Datum und Seitenzahl.
Ist das mal nicht gewollt, einfach die Optionen \lstinline{nopagenum} bzw.
\lstinline{nodate} angeben.

Autor und Titel kommen aus dem optionalen Argument von \lstinline{\author}
und \lstinline{\title}, sodass auch Kurztitel möglich sind.
Ändert man im Babel-Paket die Sprache, ändert sich auch automatisch
die Fußzeile (\textit{page} statt Seite, anderes Datumsformat).


\section{Skalierbarkeit}

Support von aspectratio


\section{Optionen}

Auflistung aller Theme-spezieller Optionen zur Übergabe and die
Dokumentenklasse.

\subsection{Kopf-/Fußbereich}

\begin{classoption}{nopagenum}
  Blendet die mitlaufende Seitenzählung im Fuß der Folie aus
\end{classoption}

\begin{classoption}{nodate}
  Blendet das Datum im Fuß der Folie aus.
\end{classoption}

\begin{classoption}{tocinheader}
  Erzeugt ein mitlaufendes Inhaltsverzeichnis im Folienkopf,
  das auch zur Navigation im Dokument genutz werden kann
  (entsprechend der Standard-Beamer-Klassen).
  
  Angezeigt werden standardmäßig sections und subsections.
  Sollen nur sections angezeigt werden, bitte die Option
  \lstinline{nosubsectionsinheader} verwenden.
  
\end{classoption}

\begin{classoption}{tinytocinheader}
  Wie \lstinline{tocinheader}, jedoch mit kleinerer Schrift.

  Kann eingesetzt werden für den Fall, dass das mit \lstinline{tocinheader}
  erzeugte Inhaltsverzeichnis zu breit wirkt oder die Foliendimension
  überragt.
\end{classoption}

\begin{classoption}{nosubsectionsinheader}
  Deaktiviert die Anzeige von subsections im Header.
  
  Nur wirksam, wenn Option \lstinline{tocinheader} oder
  \lstinline{tinytocinheader} verwendet wird!
\end{classoption}

\begin{classoption}{nologoinfoot}
  Deaktiviert die Anzeige des Logos im Fußbereich.
\end{classoption}

\subsection{Farben}

\begin{classoption}{rgbprint}
  Ausgabe in RGB-Druckfraben
\end{classoption}

\begin{classoption}{cmyk}
  Ausgabe in CMYK-Farben.
\end{classoption}


\begin{classoption}{mathserif}
  Beschreibung
\end{classoption}

\begin{classoption}{fleqn}
  Beschreibung
\end{classoption}

\begin{classoption}{minion}
  Beschreibung
\end{classoption}

\begin{classoption}{widetoc}
  Beschreibung
\end{classoption}

\begin{classoption}{narrowtoc}
  Beschreibung
\end{classoption}

\subsection{Schriftgröße}

\begin{classoption}{9pt}
  Beschreibung
\end{classoption}

\begin{classoption}{10pt}
  Beschreibung
\end{classoption}

\begin{classoption}{11pt}
  Beschreibung
\end{classoption}

\begin{classoption}{12pt}
  Beschreibung
\end{classoption}

\begin{classoption}{14pt}
  Beschreibung
\end{classoption}


\end{document}