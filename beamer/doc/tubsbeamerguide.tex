\documentclass{beamer}

\usepackage[utf8x]{inputenc}
\usepackage{listings}
\lstset{literate=%
{Ö}{{\"O}}1
{Ä}{{\"A}}1
{Ü}{{\"U}}1
{ß}{{\ss}}2
{ü}{{\"u}}1
{ä}{{\"a}}1
{ö}{{\"o}}1,
backgroundcolor=\color{tuLightGray},
}

\usepackage[english,ngerman]{babel}
\usetheme[tocinheader,colorblocks]{tubs}

\newcommand{\tubslatex}{tubs\LaTeX}

\begin{document}

% Titelseite
\logo{\includegraphics{dummy_institut}}
\titlegraphic{\includegraphics{defaulttitlepicture}}
\title{\raggedright%
  Präsentationen im Corporate Design\newline
  mit \tubslatex}
\subtitle{Eine kleine Einführung}
\author{Enrico Jörns}

\begin{frame}[plain]
  \titlepage
\end{frame}

\begin{frame}[fragile]{Titelseite I}
Standard-Befehle verfügbar: \lstinline{\title}, \lstinline{\subtitle}, \lstinline{\author}, \ldots

Spezielle Befehle:
\begin{description}
  \item[Logo:]
    Ein Logo kann mit dem Befehl {\color{tuRed}\lstinline!\logo{}!} gesetzt werden.
    Wird eine Grafik mit \lstinline!\includegraphics{}! eingebunden,
    wird sie automatisch korrekt skaliert.
    \begin{lstlisting}
 \logo{\includegraphics{DATEI}}
    \end{lstlisting}
  \item[Titelgrafik:]
    Eine Titelgrafik kann mit dem Befehl {\leavevmode\color{tuRed}\lstinline!\titlegraphic{}!} gesetzt werden.
    Sie wird bei Verwendung von \lstinline!\includegraphics{}! ebenfalls automatisch skaliert.
    \begin{lstlisting}
 \titlegraphic{\includegraphics{DATEI}}
    \end{lstlisting}
\end{description}

\end{frame}


\begin{frame}[fragile]{Titelseite II}
  Die obige Titelseite wurde mit folgenden Inhalten erstellt
  \small\begin{lstlisting}
 \logo{\includegraphics{dummy_institut}}
 \titlegraphic{\includegraphics{defaulttitlepicture}}
 \title{\raggedright%
     Präsentationen im Corporate Design\newline
     mit \tubslatex}
 \subtitle{Eine kleine Einführung}
 \author{Enrico Jörns}
  \end{lstlisting}
\begin{lstlisting}
 \begin{frame}[plain]
   \titlepage
 \end{frame}
\end{lstlisting}
\end{frame}


\section{sectionA}
\section{sectionB}
% \date{22. Dezember 2011}
% \logo{\fboxsep0mm\fbox{\includegraphics{iva_logo.pdf}}}
% \logo{\fboxsep0mm\fbox{\includegraphics{dummy_institut.pdf}}}
% \logo{\fboxsep0mm\fbox{\parbox{3.5\logoheight}{\vbox to\logoheight{~}}}}



\subsection{subsectionA}

\begin{frame}{Dummy}
  
\end{frame}

\subsection{subsectionB}

\begin{frame}
  Titellose Folien werden wie diese dargestellt.
\end{frame}

\begin{frame}[fragile,highlight]{Wichtige Folie}
Besonders hervorzuhebende Folien können mit der Option
{\color{tuRed}\lstinline{[highlight]}} erstellt werden.
\begin{lstlisting}
 \begin{frame}[highlight]{Wichtige Folie}
   ...
 \end{frame}
\end{lstlisting}
\end{frame}


\section{sectionC}
\begin{frame}
  \begin{columns}[onlytextwidth]
    \begin{column}{0.3\textwidth}
      \begin{block}{Block}
        Hier steht der Inhalt.
      \end{block}
    \end{column}
    \begin{column}{0.3\textwidth}
      \begin{alertblock}{Alert-Block}
        Hier steht der Inhalt.
      \end{alertblock}
    \end{column}
    \begin{column}{0.3\textwidth}
      \begin{exampleblock}{Example-Block}
        Hier steht der Inhalt.
      \end{exampleblock}     
    \end{column}
  \end{columns}
\end{frame}

\end{document}
