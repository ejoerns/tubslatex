% Hinweis: 
% Alle Einstellungen, die dem Briefschreiber und seinem Institut zugeordnet sind,
% stehen jeweils in einer eigenen .lco-Datei.
% Der Brief lädt dann die Autorendatei, welche wiederum die Institutsdatei lädt
\documentclass[%
  mustermann,%  Name der zu ladenden .lco-Datei des Autoren
  nexus,%       Schriftart Arial wählen
% mono,%        Darstellung in schwarz-weiß
  10pt%         gewählte Schriftgröße (standard)
]{tubslttr2}

\usepackage[ngerman]{babel}
\usepackage[utf8]{inputenc}
\usepackage{lipsum} % Paket für Blindtextgenerierung

% Brief-Variablen setzen (optional)
\setkomavar{specialmail}{[Postvermerk]}
\setkomavar{yourref}{[xxx]}
\setkomavar{yourmail}{[Datum]}
\setkomavar{myref}{[xxx]}
\setkomavar{mymail}{[Datum]}

\begin{document}
\begin{letter}{%
  %%% Anschrift %%%
  Name\\
  Straße\\
  PLZ Ort
}
\setkomavar{subject}{[Betreff]}% Betreff-Variable setzen
\opening{Sehr geehrte Damen und Herren,}

  %%% Textteil %%%
  \lipsum[1-2]% Blindtext Seite 1
  \clearpage% \clearpage erforderlich um zweite Seite zu beginnen!
  \lipsum[5-8]% Blindtext Seite 2
  %%% Ende Textteil%%%

\closing{Mit freundlichen Grüßen}

%%% optionaler Teil %%%
\ps
P.S.: Nicht vergessen!
\encl{Anlage\,1, Anlage\,2}
\cc{Institut\,1, Institut\,2}
\end{letter}
\end{document}
