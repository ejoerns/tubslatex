\documentclass[%
landscape,violet,
  a3paper,% <a4paper/a2paper/a1paper/a0paper>
  arial,style=scifi%
  ]{tubsposter}
\usepackage[utf8]{inputenc}

\usepackage{lipsum}
%% Änderung MB:
% \usepackage{nexusserifpro}
% \renewcommand{\rmdefault}{nxsx}
% \renewcommand{\sfdefault}{nxax}
% \usepackage{ifwlogo}
%% Ende Änderung MB:
\begin{document}


\begin{tubsposter}{3.5cm,X,X}
  %% Aktiviert die horizontale Linie am oberen Ende des Kommunikationsbereichs.
  %\showtopline

  %% Stellt das TU-Logo an der gewünschten Seite (links/rechts) dar.
  \showtubslogo[left]

  %% Stellt das Institutslogo im Absenderbereich dar.
  %\showlogo{\vspace*{-0.8cm}\ifwlogo}%}
  %% Alternative Version des Logos mit Bild statt Text.
  \showlogo{\fboxsep0mm\fbox{\includegraphics{dummy_institut.pdf}}}

  %% Beginnt eine neuen Bereich mit der angegebenen Höhe im Gaußraster.
  \begin{posterrow}[bgcolor=tuViolet100]
    \vfill


    \usekomafont{headline} \LARGE   \color{tuWhite} Wärmedämmschichten\\

    \vfill
  \end{posterrow}
  \begin{posterrow}[X,X,X]
\begin{postercol}
    \large\textbf{Lebensdauer von Schichtsystemen}

  \end{postercol}
\begin{postercol}
    \large\textbf{Wärmedämmschichten für Raketentriebwerke}
    \begin{itemize}
      \item Item 1
      \item Item 2
      \item Item 3
    \end{itemize}
  \end{postercol}
\begin{postercol}
    \large\textbf{Neue Herstellungsverfahren}

    \begin{itemize}
      \item Item 1
      \item Item 2
      \item Item 3
    \end{itemize}
  \end{postercol}
\end{posterrow}
\end{tubsposter}

\end{document}
