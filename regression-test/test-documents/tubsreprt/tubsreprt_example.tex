\documentclass[%
  a4paper,%
  11pt,% <10pt, 9pt>
  %smallchapters,
  %style=screen,
  %sender=bottom,
  blue,% <orange, green, violet>
  %rgb, <cmyk>
  %mono
  ]{tubsreprt}
\usepackage[utf8x]{inputenc}

\usepackage[ngerman]{babel}

\usepackage{lipsum} % Blindtext-Paket

% Titelseiten-Elemente
\title{Beispieltitel}
\subtitle{Untertitel}
\author{Max Mustermann}
%\logo{Institut fuer Lorem Ipsum}
\logo{\includegraphics{dummy_institut.pdf}}
\titleabstract{\lipsum[2]}
\titlepicture{infozentrum.jpg}
% Rückseiten-Elemente
\address{%
  Herr Mustermann\\
  Schlossallee 1\\
  33333 Darmstadt}
\backpageinfo{%
  \lipsum[5]
}

\begin{document}

\maketitle[image,logo=right]%[<plain/image/imagetext>,<logo=left/right>]
\makebackpage[trisec]%[<plain/info/addressinfo>]

\tableofcontents


\chapter{Ut purus elit}

\textcolor{tubsSecondary}{Dies ist ein Text in \texttt{tubsSecondary}.}
\textcolor{tubsViolet}{Dies ist ein Text in \texttt{tubsViolet}.}
\textcolor{tubsGreenDark}{Dies ist ein Text in \texttt{tubsGreenDark}.}\bigskip

\lipsum[1]

\begin{itemize}
  \item Aufzählungspunkt Eins
  \item Aufzählungspunkt Zwei
    \begin{itemize}
      \item Unter-Aufzählungspunkt Eins
      \item Unter-Aufzählungspunkt Zwei
    \end{itemize}
  \item Aufzählungspunkt Drei
\end{itemize}

\chapter{Phasellus eu tellus sit amet}

\lipsum[2-5]

\chapter{Nulla malesuada porttitor diam.}

\lipsum[1-3]

\section{Donec felis erat}

\lipsum[4-7]

\end{document}
