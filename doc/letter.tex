\chapter{Briefe}

Briefe können mit der Dokumentenklasse \newdocumentclass{tubslttr2} erstellt
werden.
Der Aufbau eines Briefes unterscheidet sich zum Teil grundlegend von dem eines
normalen Dokumentes, weswegen hierfür eine spezielle Klasse vorliegt%
\footnote{Das etwas spezielle Interface liegt in der zugrunde liegenden 
KOMA-Skript-Klasse scrlttr2 begründet.}, deren Benutzung sich ein wenig von
der der anderen unterscheidet.

Die Absender-Informationen eines Briefes können mit Hilfe einer Reihe von
Variablen festgelegt werden. Näher geht darauf das Kapitel \ref{sec:lettervars}
ein.

Absender-Informationen variieren naturgemäß von Institut zu Institut
und vor allem von Person zu Person. Innerhalb eines Instituts und insbesondere
bei gleichem individuellen Absender sind diese Informationen jedoch sehr
konstant. Daher gibt es die Möglichkeit, die Absender-Informationen in
separaten Dateien als Vorlage vorzudefinieren. Mehr Informationen dazu gibt
das Kapitel~\ref{sec:lco}

\section{Paketoptionen}

Generell können die meisten Optionen, die auch die zugrunde liegenden Klasse
\texttt{scrlttr2} akzeptiert, übergeben werden. Es wird jedoch keine korrekte
Funktionalität gewährleistet und darauf hingewiesen, dass diese Optionen
ggf. nicht \acs{CD}-Konform sind.

Als Schriftgröße ist standardmäßig \PValue{10pt} eingestellt.

\begin{Declaration}
  \Option{arial}\\
  \Option{nexus}
\end{Declaration}

Standardmäßig wird ein Brief in der Schriftart \emph{Nexus} gesetzt.
Ist stattdessen eine Darstellung in der Schriftart \emph{Arial} erwünscht,
so kann dies mittels der Option \PName{arial} festgelegt werden.
Die Option \PName{nexus} sorgt für eine Darstellung in \emph{Nexus} und
kann optional angegeben werden.

\begin{Declaration}
  \Option{mono}
\end{Declaration}

Falls ein Brief nur in schwarz-weiß erstellt werden soll, können mit Hilfe
der Option \PName{mono} alle wesentlichen Elemente der Briefvorlage monochrom
dargestellt werden. Dazu zählen unter anderem Das Siegelband-Logo und die
angrenzende Trennlinie.
Auch sämtliche Elemente, die die Farbe \texttt{tuRed} verwenden, werden
in Schwarz statt in Rot dargestellt.

\section{Briefe schreiben}

Zum Erstellen eines Briefdokumentes gibt es einige spezielle Umgebungen und
Kommandos, die im Folgenden vorgestellt werden.

\begin{Declaration}
  \XMacro{begin}\PParameter{\Environment{letter}}%
  \OParameter{Optionen}\Parameter{Empfänger}\\
  \quad\dots\\
  \XMacro{end}\PParameter{letter}
\end{Declaration}

Innerhalb von \lstinline!\begin{document}! und \lstinline!\end{document}!
können mit der Umgebung \Environment{letter} beliebig viele einzelne Briefe
erstellt werden. Der grundlegende Aufbau eines solchen Briefes ist in
Abbildung~\ref{fig:scrlttr2.letter} dargestellt. Die wichtigsten Befehle
sind im Folgenden aufgelistet.

\begin{Declaration}
  \Macro{opening}\Parameter{Anrede}\\
  \Macro{closing}\Parameter{Grußfloskel}\\
  \Macro{ps}\\
  \Macro{cc}\Parameter{Verteiler}\\
  \Macro{encl}\Parameter{Anlagen}
\end{Declaration}

Die Anweisungen \Macro{opening} und \Macro{closing} sind für das Erstellen
eines Briefes unerlässlich. Mit \Macro{opening} wird sowohl die Anrede
festgelegt (z.B. \textit{Seehr geehrte Damen und Herren}) als auch der
Bereich für den Brieftext begonnen.
Dieser Bereich wird dann wiederum mit dem Befehl \Macro{closing}, über den
man zugleich eine Grußfloskel angibt (z.B. \textit{Mit freundlichen Grüßen}),
beendet.

Der Befehl \Macro{ps} leitet einen Absatz für das Postscriptum ein.
Mit \Macro{cc} lässt sich ein \PName{Verteiler} Angeben und
aufzuführende \PName{Anlagen} werden im Makro \Macro{encl} angegeben.

Für weitere Details zum Schreiben von Briefen wird auf die Dokumentation
der zugrunde liegenden Klasse \lstinline{scrlttr2} verwiesen
\cite[S.162ff]{koma-skript}.

Ein vollständiges Beispieldokument findet sich in Anhang~\ref{appendix:letter}.

\begin{important}
Eine Besonderheit beim Erstellen des Brieftextes ergibt sich aus dem wechselnden
Seitenlayout nach der ersten Seite.
Soll mehr als eine Seite beschrieben werden, so muss nach
dem Inhalt für die erste Seite ein \lstinline{\clearpage} eingefügt werden.
Ansonsten kommt es zu fehlerhafter Darstellung der Textbreite des ersten
Absatzes auf der zweiten Seite.
\end{important}


\begin{figure}[!ht]
  \centering\small
  \fbox{\parbox{.667\linewidth}{\raggedright%
      \Macro{begin}\PParameter{letter}%
      \OParameter{Optionen}\Parameter{Empfänger}\\
      \dots\\[\dp\strutbox]
      {\centering\emph{Einstellungen für diesen Brief}\\}
      \dots\\
      \Macro{opening}\Parameter{Anrede}\\
    }}\\[1pt]
  \fbox{\parbox{.667\linewidth}{\raggedright%
      \dots\\[\dp\strutbox]
      {\centering\emph{Brief"|text}\\}
      \dots\\
    }}\\[1pt]
  \fbox{\parbox{.667\linewidth}{\raggedright%
      \Macro{closing}\Parameter{Grußformel}\\
      \Macro{ps}\\
      \dots\\[\dp\strutbox]
      {\centering\emph{Postscriptum}\\}
      \dots\\
      \Macro{encl}\Parameter{Anlagen}\\
      \Macro{cc}\Parameter{Verteiler}\\
      \Macro{end}\PParameter{letter}\\
    }}\\[\dp\strutbox]
  \caption[Genereller Aufbau eines einzelnen Briefes innerhalb eines
    Briefdokuments]{Genereller Aufbau eines einzelnen Briefes
    innerhalb eines Briefdokuments nach \cite[S.171]{koma-skript} }
  \label{fig:scrlttr2.letter}
\end{figure}


\section{Variablen}\label{sec:lettervars}

Diverse Inhalte jedes Briefes können mittels einer Reihe von Variablen
definiert werden.
Eine Übersicht über alle verwendbaren Variablen geben die
Tabellen \ref{table:lettervars} und \ref{table:lettervars2}.
Es ist dabei meist nicht nötig zu wissen,
wo die entsprechenden Werte in der Ausgabe konkret gesetzt werden,
da das Design im allgemeinen festgelegt ist.
Zur besseren Veranschaulichung kann jedoch
Abbildung~\ref{fig:lettervarpos} zu Rate gezogen werden.

\begin{Declaration}
  \Macro{setkomavar}\Parameter{Name}\Parameter{Inhalt}
\end{Declaration}

Variableninhalte werden mit dem Befehl \Macro{sektomavar} gesetzt.
\PName{Name} ist dabei der Name der zu setzenden Variable, also z.B. einer der in
Tabelle~\ref{table:lettervars} definierten Namen. \PName{Inhalt} setzt den
Inhalt der Variablen, was mit Ausnahme des Logos meist ein einfacher
Text sein sollte.

Die meisten der aufgeführten Variablen müssen nicht für jeden zu erstellenden
Brief neu gesetzt werden, sondern sollten besser in eine \texttt{.lco}-Datei,
entsprechend der Beschreibung in Kapitel~\ref{sec:lco}, ausgelagert werden.
Diese Variablen finden sich in Tabelle~\ref{table:lettervars}.

Ein paar Variablen müssen (wenn sie überhaupt benutzt werden) in jedem Brief
neu gesetzt werden. Diese sind in Tabelle~\ref{table:lettervars2} aufgezählt.

\begin{example}
\begin{lstlisting}
\setkomavar{fromname}{Max Mustermann}
\end{lstlisting}
Mit dieser Anweisung wird der Name des Absenders auf \glqq Max Mustermann\grqq\ 
gesetzt.
\end{example}


\begin{desctable}
  \caption{Von der Klasse \newdocumentclass{tubslttr2} unterstützte
    Variablen und deren Bedeutung\\
    -- Inhaltsunabhängig --}\label{table:lettervars}\\
  \Endfirsthead
  \caption[]{Von der Klasse \newdocumentclass{tubslttr2} unterstützte Variablen
    (\emph{Fortsetzung})}\\
  \Endhead%
  \standardfoot%
  \ventry{fromlogo}{Anweisung(en) zum Setzen des Institutslogos (Text oder Grafik)}%
  \ventry{frominstitute}{Name des Instituts}%
  \ventry{fromdepartment}{ggf. Abteilung oder Untereinheit (eines Instituts, etc.) des Absenders}%
  \ventry{fromstreest}{Straßennahme und Hausnummer der Absenders}%
  \ventry{fromzipcode}{Postleitzahl des Absenders}%
  \ventry{fromtown}{Ort des Absenders. Voreingestellt auf \emph{Braunschweig}}%
  %
  \ventry{fromtitle}{Titel des Absenders (Dr., Prof. Dr., \ldots)}
  \ventry{fromname}{Vollständiger Name des Absenders}
  %
  \ventry{fromphonedirect}{Telefon-Durchwahl des Absenders}
  \ventry{fromfaxdirect}{Fax-Durchwahl des Absenders}%
  \ventry{fromemail}{E-Mail-Adresse des Absenders}
  \ventry{fromurl}{Website des Absenders}%
  %
  \ventry{frombank}{Bankverbindung des Absenders}%
  \ventry{fromIBAN}{IBAN des Absender}%
  \ventry{fromBIC}{BIC-Code des Absenders}%
  \ventry{fromUStID}{Umsatzsteuer-ID des Absenders}%
  \ventry{fromSteuernummer}{Steuernummer des Absenders}%
  %
  \ventry{signature}{Signatur unter Unterschrift und Grußformel}
\end{desctable}

\begin{desctable}
  \caption{Von der Klasse \newdocumentclass{tubslttr2} unterstützte
    Variablen und deren Bedeutung\\
    -- Inhaltsabhängig --}\label{table:lettervars2}\\
  \Endfirsthead
  \ventry{subject}{Inhalt der Betreffzeile}
  \ventry{specialmail}{Spezielle Versandart, wird zwischen Rücksendeadresse
    und Empfängeradresse dargestellt.}
  \ventry{yourref}{Feld \glqq Ihr Zeichen:\grqq}
  \ventry{yourmail}{Feld \glqq Ihre Nachricht vom:\grqq}
  \ventry{myref}{Feld \glqq Unser Zeichen:\grqq}
  \ventry{mymail}{Feld \glqq Unsere Nachricht vom:\grqq}
\end{desctable}


\begin{figure}[!ht]
\centering\fboxsep0mm\fbox{%
\includegraphics[width=0.9\textwidth]{examples/example_vars.pdf}
}
\caption{Darstellung der Variablen}
\label{fig:lettervarpos}
\end{figure}\clearpage


\section{lco-Dateien verwalten}\label{sec:lco}

Eine Reihe von Einstellungen und definierten Variablen für das Erstellen eines
Briefes sind meist bei allen Briefen, die man erstellt, gleichbleibend.
Während Zieladresse, Anrede, Inhalt und Grußformel meist von Brief zu Brief
abweichen werden Elemente wie Absenderadresse, Telefon-Durchwahl oder
Bankverbindung immer dieselben bleiben.

Es ist daher sinnvoll, diese Einstellungen als Vorlage in (eine) separate Datei(en)
auszulagern, die man dann bei jedem neu zu erstellenden Brief einfach laden
kann. Bei \texttt{tubslttr2} bzw. \texttt{scrlttr2} sind dafür die sogenannten
\emph{Letter-Class-Option}-Dateien, kurz \texttt{.lco}-Dateien vorgesehen.

Im Prinzip kann eine \text{.lco}-Datei alle möglichen Arten von Anweisungen
enthalten, in Zusammenhang mit den Corporate Design-Vorlagen sollten dort jedoch
hauptsächlich die im vorigen Kapitel definierten Variablen gesetzt werden.

Aufgrund der Organisationsstruktur an der Universität gibt es meist 3 Ebenen für
die jeweils ein bestimmter Satz an Informationen immer gleichbleibend ist.

\begin{itemize}
  \item Die \emph{TU Braunschweig} nimmt dabei die oberste Ebene ein.
    Alle allgemeinen Einstellung, wie die Nennung des Namens oder die Stadt Braunschweig sind
    fast immer gleich und daher bereits standardmäßig vordefiniert.

  \item Die zweite Ebene stellen die einzelnen \emph{Institute} bzw.
    zentralen Einrichtung dar. Auf dieser Ebene sind meist Elemente wie die
    Anschrift, die Web-Adresse oder die Bankverbindung gleich.

  \item Die unterste Ebene stellt dann der individuelle \emph{Mitarbeiter} dar,
    für den Elemente wie der Name, die Durchwahl oder die Mail-Adresse konstante
    Informationen darstellen.
\end{itemize}

Aufgrund dieser Aufteilung wird empfohlen zwei Arten von \texttt{.lco}-Dateien
anzulegen. Das jeweilige Institut kann eine instituts-spezifische Datei
bereit stellen, auf die alle Mitarbeiter zugriff haben. Jeder Mitarbeiter kann
sich dann eine individuelle \texttt{.lco}-Datei erstellen, die seine spezifischen
Informationen enthalten. Beides sind einmalige Vorgänge und erleichtern die
weitere Arbeit deutlich.

Im Anhang \ref{sec:examples:letter} finden sich jeweils ein Beispiel für eine
Instituts-Datei (\lstlistingname~\ref{lst:examples:letter:musterinstitut}) und eine Mitarbeiter-Datei
(\lstlistingname~\ref{lst:examples:letter:mustermann}). Da die Aufteilung der Informationen nicht immer 
gleich und eindeutig ist, können individuelle Änderungen des jeweils definierten
Variablen-Satzes nötig sein.

\paragraph{lco-Dateien laden}

Es gibt prinzipiell zwei Möglichkeiten eine \texttt{.lco}-Datei zu laden.
Die erste ist, den Namen der Datei ohne Endung als Option an
\Macro{documentclass} zu übergeben.
Die zweite besteht in der Benutzung des Befehls \Macro{LoadLetterOption}.

\begin{Declaration}
  \Macro{LoadLetterOption}\Parameter{Name}
\end{Declaration}

Diese Anweisung nimmt ebenfalls den Namen der Datei ohne Endung als
Argument und lädt sie an der entsprechenden Stelle.
Die Anweisung kann vor allem dazu benutzt werden, um Innerhalb von
\texttt{.lco}-Dateien weitere \texttt{.lco}-Dateien zu laden.

\begin{example}
  Eine Datei \lstinline{mitarbeiter.lco} könnte mit der Zeile
\begin{lstlisting}
\LoadLetterOption{institut}
\end{lstlisting}
  die übergeordnete Instituts-Vorlage (\lstinline{institut.lco}) laden.
\end{example}

\section{Weiterführende Themen}

\subsection{Seitenzahl bei einseitigen Briefen}

Standardmäßg wird auf jeder Briefseite die aktuelle Seitenzahl
im Stil \glqq Seite X von Y\grqq\ angezeigt. Dies ist sowohl bei einseitigen
als auch bei mehrseitigen Briefen der Fall.
Diese Darstellung ist zwar meist sinvoll, in Einzelfällen kann jedoch bei
einseitigen Dokumenten eine Darstellung ohne Seitenzahl gewünscht sein.
Dies kann einfach durch das Löschen des Inhalts der Variable
\Variable{firstfoot} erreicht werden: \lstinline!\setkomavar{firstfoot}{}!.

\begin{important}
  Bei Verwendung einer alten KOMA-Skript-Version ($<$ 3.0) steht die
  Variable \Variable{firstfoot} noch nicht zur Verfügung.
  Das selbe Ergebnis kann dort durch Setzen von \lstinline!\firstfoot{}!
  erreicht werden.
\end{important}



% TODO: 
% - Beispielbrief (-> Anhang!?)
