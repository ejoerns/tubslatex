\chapter{Das tubsdoc-Paket}

Die bereitgestellten Klassen decken ein breites Spektrum an Einsatzmöglichkeiten
ab. Jedoch ist es im individuellen Fall möglich, dass man eine komplett andere
Basisklasse verwenden möchte, aber trotzdem dabei nicht auf die Grundzüge
des Corporate Design verzichten mag.

Für diese und weitere Anwendungsbereiche ist das Paket \lstinline{tubsdoc}
vorgesehen. Es kann zu beliebigen Dokumentenklassen als Zusatzpaket geladen
werden und ermöglicht damit grundlegende Unterstützung von CD-Elementen.

Dazu zählen unter anderem die Schriftart Nexus, die CD-Farbklänge und das
Gauß-Layout\-system inklusive Titelseiten-Unterstützung.

\paragraph{Optionen} Es werden fast alle Optionen unterstützt, die auch
die entsprechenden Dokumentenklassen bieten. Wichtig ist hierbei,
die gewünschten Optionen auch wirklich dem Paket und nicht (nur)
der verwendeten Dokumentenklasse zu übergeben.

\begin{lstlisting}[captionpos=b,caption={Beispiel für Verwendung von tubsdoc}]
\documentclass[a4paper,11pt]{article}
\usepackage[a4paper,fontsize=11pt,colorheadings]{tubsdoc}
\end{lstlisting}
