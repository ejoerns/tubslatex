\documentclass[%
  a4paper,%
  11pt,%
  twoside=false,%
  rgb,%
  extramargin,%
  parskip=half,%
  hyperrefdark,%
  %lnum,%
  violet,
  colorheadings=tubsOrange
]{tubsbook}
% a5paper, 9pt, no extramargin, bcor=0mm

\usepackage[utf8x]{inputenc}
\usepackage{xspace}
\usepackage{amsmath}

\usepackage{framed}

% --- Tabellen
\usepackage{tabularx}
\usepackage{booktabs}
\usepackage{longtable}

% --- Indexing
\usepackage{imakeidx}
\makeindex[name=general]
\makeindex[name=cmd]
\makeindex[name=option]
\newcommand*{\Index}[2][]{\index[general]{#2}}
\newcommand*{\OptionIndex}[2][]{\index[option]{#2}}
\newcommand*{\CommandIndex}[2][]{\index[cmd]{#2@\textbackslash#2}}
\newcommand*{\EnvironmentIndex}[2][]{\index[cmd]{#2}}
\newcommand\indexbf[1]{\textbf{\hyperpage{#1}}}% to be used as \index{bla|indexbf}

% --- Abbildungen
\addtokomafont{caption}{\sffamily}

% --- Listings
\usepackage{listings}
\lstset{%
  basicstyle=\ttfamily,
  literate={Ö}{{\"O}}1
          {Ä}{{\"A}}1
          {Ü}{{\"U}}1
          {ß}{{\ss}}1
          {ü}{{\"u}}1
          {ä}{{\"a}}1
          {ö}{{\"o}}1
}
% Stil für Kommandozeile
\lstdefinestyle{cmd}{%
  frame=single,
  backgroundcolor=\color{tuGray20}
}
% Stil für Dateiinhalte
\lstdefinestyle{file}{%
  frame=single,
%   backgroundcolor=\color{tuGray20}
}
% --- Hyperref
% \usepackage[%
%   colorlinks=true,
%   linkcolor=tuRed100,
%   citecolor=tuGreenDark]{hyperref}
\usepackage{hyperref}
\usepackage[ngerman]{babel}
\usepackage{caption, subcaption}

\usepackage{enumerate,paralist}

\usepackage{tikz}

% --- Glossar
\usepackage[ngerman]{translator}
\usepackage[
nonumberlist, %keine Seitenzahlen anzeigen
acronym,      %ein Abkürzungsverzeichnis erstellen
toc,          %Einträge im Inhaltsverzeichnis
section=chapter]      %im Inhaltsverzeichnis auf section-Ebene erscheinen
{glossaries}
\let\acs\gls
\makeglossaries

% --- Beschreibungs-Makros
% Kopiert von KOMA
\makeatletter
\providecommand\marg[1]{%
  {\ttfamily\char`\{}\meta{#1}{\ttfamily\char`\}}}
\providecommand\oarg[1]{%
  {\ttfamily[}\meta{#1}{\ttfamily]}}
\def\cmd#1{\cs{\expandafter\cmd@to@cs\string#1}}
\def\cmd@to@cs#1#2{\char\number`#2\relax}
\DeclareRobustCommand\cs[1]{\texttt{\char`\\#1}}

\newenvironment{Declaration}{%
%    \end{macrocode}
% \begin{macro}{\new@element}
%   Help macro to define new Declaration elements.
%    \begin{macrocode}
  \newcommand*{\new@element}[1]{%
    \expandafter\newcommand\expandafter*\csname X##1\endcsname{}%
    \expandafter\let\csname X##1\expandafter\endcsname
    \csname ##1\endcsname
    \expandafter\newcommand\expandafter*\csname new##1\endcsname[1]{%
%      \begingroup
%        \let\ensuremath\@firstofone
%        \let\textit\@firstofone
%        \lowercase{\def\@tempa{##1}}%
%        \pdfstringdef\@tempb{\label@base.\@tempa.####1}%
%        \xdef\@currentHref{\@tempb}%
%        \Hy@raisedlink{\hyper@anchorstart{\@currentHref}\hyper@anchorend}%
%        \label{desc:\label@base.\@tempa.####1}%
%      \endgroup
      \csname X##1\endcsname{####1}\ignorespaces
    }%
    \expandafter\let\csname ##1\expandafter\endcsname\csname new##1\endcsname
  }%
  \newcommand*{\new@xelement}[2]{%
    \expandafter\newcommand\expandafter*\csname X##1\endcsname{}%
    \expandafter\let\csname X##1\expandafter\endcsname
    \csname ##1\endcsname
    \expandafter\newcommand\expandafter*\csname new##1\endcsname[2]{%
%      \begingroup
%        \let\ensuremath\@firstofone
%        \let\textit\@firstofone
%        \lowercase{\def\@tempa{##1}}%
%        \pdfstringdef\@tempb{\label@base.\@tempa.####1.####2}%
%        \xdef\@currentHref{\@tempb}%
%        \Hy@raisedlink{\hyper@anchorstart{\@currentHref}\hyper@anchorend}%
%        \label{desc:\label@base.\@tempa.####1.####2}%
%      \endgroup
      \csname X##1\endcsname{####1}{##2{####2}}\ignorespaces
    }%
    \expandafter\let\csname ##1\expandafter\endcsname\csname new##1\endcsname
  }%
%    \end{macrocode}
%    \begin{macrocode}
  \new@element{Option}%
  \new@element{Macro}%
  \new@element{Environment}%
  \new@element{Counter}%
  \new@element{FloatStyle}%
  \new@element{PLength}%
  \new@element{Variable}%
  \new@xelement{OptionValue}{\PValue}%
%    \end{macrocode}
% \end{macro}
%    \begin{macrocode}
  \ifvmode\else\par\fi\addvspace{2\baselineskip}%
  \vspace{-\baselineskip}%
  \vspace{\z@ plus \baselineskip}%
  \noindent
  \start@Declaration
  \tabular{|l|}\hline\ignorespaces
}{%
  \\\hline\endtabular\nobreak\after@Declaration\nobreak\par\nobreak
  \vspace{1.5\baselineskip}\nobreak\vspace{-\baselineskip}\nobreak%
  \vspace{0pt minus .5\baselineskip}\nobreak%
  \aftergroup\@afterindentfalse\aftergroup\@afterheading
}
\newcommand*{\start@Declaration}{\hspace{-1em}}
\newcommand*{\after@Declaration}{}

% \begin{macro}{\textsfrm}
%   Something like |\emph| using |\textsf| and |\textrm| instead of |\textit|
%   and |\textup|.
%    \begin{macrocode}
\newcommand*{\textsfrm}[1]{%
  \begingroup
    \edef\@tempa{\sfdefault}%
    \ifx\@tempa\f@family\textrm{#1}\else\textsf{#1}\fi
  \endgroup
}
%    \end{macrocode}
% \end{macro}

% \begin{macro}{\File}
% \begin{macro}{\Class}
% \begin{macro}{\Package}
%   Some markup macros for files with special meanings.
\DeclareRobustCommand*{\File}[1]{\mbox{\texttt{#1}}}
\DeclareRobustCommand*{\Class}[1]{\mbox{\textsfrm{#1}}}
\DeclareRobustCommand*{\Package}[1]{\mbox{\textsfrm{#1}}}
% \end{macro}
% \end{macro}
% \end{macro}

% \begin{macro}{\Macro}
% \begin{macro}{\Option}
% \begin{macro}{\KOption}
% \begin{macro}{\OptionValue}
% \begin{macro}{\Environment}
% \begin{macro}{\Counter}
% \begin{macro}{\Length}
% \begin{macro}{\PLength}
% \begin{macro}{\FloatStyle}
% \begin{macro}{\Pagestyle}
% \begin{macro}{\Variable}
% \begin{macro}{\FontElement}
% \begin{macro}{\PName}
% \begin{macro}{\PValue}
% \begin{macro}{\Parameter}
% \begin{macro}{\OParameter}
% \begin{macro}{\AParameter}
% \begin{macro}{\PParameter}
% \begin{macro}{\POParameter}
%   \begin{description}
%   \item[\cs{Macro}] \LaTeX{} or \TeX{} macro
%   \item[\cs{Option}] class or package option
%   \item[\cs{KOption}] |\KOMAoptions| option
%   \item[\cs{Environment}] \LaTeX{} environment
%   \item[\cs{Counter}] \LaTeX{} counter
%   \item[\cs{Length}] \LaTeX{} length
%   \item[\cs{PLength}] \KOMAScript{} pseudo length
%   \item[\cs{Variable}] \KOMAScript{} variable
%   \item[\cs{FontElement}] \KOMAScript{} element that has its own font
%     selection
%   \item[\cs{PName}] name of a parameter of a macro or environment
%   \item[\cs{PValue}] value of a parameter of a macro or environment
%   \item[\cs{Parameter}] the mandatory parameter of a macro or environment
%   \item[\cs{OParameter}] the optional parameter of a macro or environment
%   \item[\cs{AParameter}] the alternativ parameter of a macro or environment
%   \item[\cs{PParameter}] the part-of-command parameter of a macro or
%     environment
%   \end{description}
%    \begin{macrocode}
\DeclareRobustCommand*{\Macro}[1]{\mbox{\texttt{\char`\\#1}}\index{cmd}{#1@\textbackslash#1}}
\DeclareRobustCommand*{\Option}[1]{\mbox{\texttt{#1}}\index{option}{#1}}
\DeclareRobustCommand*{\KOption}[1]{\mbox{\Option{#1}\texttt=}}
\DeclareRobustCommand*{\OptionValue}[2]{\mbox{\texttt{#1=#2}}}
\DeclareRobustCommand*{\FloatStyle}[1]{\mbox{\texttt{#1}}}
\DeclareRobustCommand*{\Pagestyle}[1]{\mbox{\texttt{#1}}}
\DeclareRobustCommand*{\Environment}[1]{\mbox{\texttt{#1}}\index{cmd}{#1}}
\DeclareRobustCommand*{\Counter}[1]{\mbox{\texttt{#1}}}
\DeclareRobustCommand*{\Length}[1]{\mbox{\texttt{\char`\\#1}}}
\DeclareRobustCommand*{\PLength}[1]{\mbox{\PValue{#1}}}
\DeclareRobustCommand*{\Variable}[1]{\mbox{\PValue{#1}}}
\DeclareRobustCommand*{\FontElement}[1]{\PValue{#1}}
\DeclareRobustCommand*{\PName}[1]{\texttt{\textit{#1}}}
\DeclareRobustCommand*{\PValue}[1]{\texttt{#1}}
\DeclareRobustCommand*{\Parameter}[1]{\texttt{\{}\PName{#1}\texttt{\}}}
\DeclareRobustCommand*{\OParameter}[1]{%
  \texttt{[%]
  }\PName{#1}\texttt{%[
    ]}}
\DeclareRobustCommand*{\AParameter}[1]{%
  \texttt{(%)
  }\PName{#1}\texttt{%(
    )}}
\DeclareRobustCommand*{\PParameter}[1]{\texttt{\{#1\}}}
\DeclareRobustCommand*{\POParameter}[1]{\texttt{[#1]}}
\DeclareRobustCommand*{\Color}[1]{\mbox{\texttt{#1}}}
%    \end{macrocode}
% \end{macro}
% \end{macro}
% \end{macro}
% \end{macro}
% \end{macro}
% \end{macro}
% \end{macro}
% \end{macro}
% \end{macro}
% \end{macro}
% \end{macro}
% \end{macro}
% \end{macro}
% \end{macro}
% \end{macro}
% \end{macro}
% \end{macro}
% \end{macro}
% \end{macro}
% NOTE: taken from scrguide.cls

% \begin{environment}{desctable}
%   This is almost the same like \texttt{desctabular} but it uses a longtable
%   to allow page breaks.
%    \begin{macrocode}
\newenvironment{desctable}[1][2em]{%
  \onelinecaptionsfalse
  \start@desctab{#1}%
  \newcommand{\Endfirsthead}{\toprule\endfirsthead}%
  \newcommand{\Endhead}{\midrule\endhead}%
  \newcommand*{\standardfoot}{%
    \addlinespace[-.5\normalbaselineskip]\midrule
    \multicolumn{2}{r@{}}{\dots}\\
    \endfoot
    \addlinespace[-.5\normalbaselineskip]\bottomrule
    \endlastfoot
  }%
  \longtable{lp{\descwidth}}%
}{%
  \endlongtable
}
%    \end{macrocode}
% \end{environment}

% \begin{length}{\descwidth}
%   I need a length of local usage. I could have used |\@tempdima| or
%   another local length from kernel. But I've decided not to try to find a
%   unused length at \texttt{tabular} environment.
%    \begin{macrocode}
\newlength{\descwidth}
%    \end{macrocode}
% \end{length}

% \begin{macro}{\start@desctab}
%   This is the \emph{worker} macro of \texttt{desctable} and
%   \texttt{desctabular}. It does the complete calculations and definition of
%   the entry (something like |\item|) commands.
%    \begin{macrocode}
\newcommand*{\start@desctab}[1]{%
  \setlength{\descwidth}{\linewidth}%
  \addtolength{\descwidth}{-4\tabcolsep}%
  \addtolength{\descwidth}{-#1}%
  \setlength{\labelwidth}{\linewidth}%
  \addtolength{\labelwidth}{-2\tabcolsep}%
  \newcommand{\nentry}[2]{%
    \multicolumn{2}{p{\labelwidth}}{\raggedright##1}\\*%
    \hspace*{#1} & ##2\tabularnewline%
  }%
  \newcommand{\entry}[2]{\nentry{##1}{##2}[.5\baselineskip]}%
  \newcommand*{\pentry}[1]{%
    \entry{\PLength{##1}\IndexPLength[indexmain]{##1}}}%
  \newcommand*{\pventry}[1]{\entry{\PValue{##1}}}%
  \newcommand*{\mentry}[1]{\entry{\Macro{##1}}}%
  \newcommand*{\ventry}[1]{%
    \entry{\Variable{##1}%\IndexVariable[indexmain]{##1}%
    }%
  }%
  \newcommand*{\feentry}[1]{%
    \entry{\FontElement{##1}\IndexFontElement[indexmain]{##1}}%
  }%
  \newcommand*{\oentry}[1]{%
    \entry{\Option{##1}\IndexOption[indexmain]{##1}}%
  }%
}
% \end{macro}

\makeatother

% xspace für tubslatex-Logo
\makeatletter
\g@addto@macro{\tubslatex}{\xspace}
\makeatother

% \def\example{\par\smallskip\noindent\textit{Beispiel: }}

% --- Umgebung: 'Beispiel'
% Schreibt 'Beispiel' vor den folgenden Inhalt und rückt alles nach dem ersten
% Absatz um 2em ein.
% \newenvironment{Example}{%
% \begingroup
% \leftskip2em
% \par\smallskip\noindent\hspace*{-2em}\textit{Beispiel: }
% }{%
% \par\endgroup
% }

\reversemarginpar\setlength{\marginparwidth}{0.5cm}% TODO: needs fix?
% --- Umgebung: 'Wichtig:'
% \newenvironment{important}{%
%   \begin{description}
%     \item[\itshape\mdseries\rmfamily Wichtig:]
%       \leavevmode\marginpar{\LARGE\bfseries\sffamily\textcolor{tuGrey}{%
%       \raisebox{-1ex}{!}}}
% }{%
%   \end{description}
% }

% --- Umgebung: 'Hinweis'
% \newenvironment{hint}{%
%   \begin{description}
%     \item[\itshape\mdseries\rmfamily Hinweis:]
% }{%
%   \end{description}
% }


% \newenvironment{newimportant}[1][\hsize]
% {%
%     \def\FrameCommand
%     {%
%         {\color{red}\vrule width 3pt~~}%
%         \hspace{0pt}%must no space.
%         \fboxsep=0mm
% %         \colorbox{yellow}%
%     }%
%     \MakeFramed{\hsize#1\advance\hsize-\width\FrameRestore}%
%     \begin{description}
%       \item[Beispiel:]
% }
% {
%     \end{description}
%     \endMakeFramed}

\usepackage{lipsum} 

%\marg{Farbe}\marg{Label}
\newenvironment{emphlist}[2]{%
  \def\FrameCommand{{\color{#1}\vrule width 3pt \hspace{8pt}}}%
  \MakeFramed {\advance\hsize-\width \FrameRestore}%
   \vspace{-\parskip}%
   \begingroup
  \list{\sffamily\itshape #2:}{}%
  \item
  }{%
  \endlist
  \endgroup
  \endMakeFramed
   \vspace{-\parskip}%
  }
  
\newenvironment{important}{%
  \emphlist{tuRed}{Wichtig}
}{%
  \endemphlist
}


\newenvironment{example}{%
  \emphlist{tuGreen}{Beispiel}
}{%
  \endemphlist
}

\newenvironment{hint}{%
  \emphlist{tuBlue}{Hinweis}
}{%
  \endemphlist
}

\newcommand{\zB}{\mbox{z.\,B.}\xspace}


% Bindet Grafik mit Rahmen ein
\newcommand{\examplegraphic}[2][]{%
  \fboxsep0mm\fbox{\includegraphics[#1]{#2}}%
}

% URLs
\newcommand\urltoolbox{\url{https://www.tu-braunschweig.de/presse/cd/toolbox}}
\newcommand\urltubslatex{\url{http://tubslatex.ejoerns.de}}

