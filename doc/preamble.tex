\documentclass[%
  a4paper,%
  11pt,%
  twoside=false,%
  rgb,%
  extramargin,%
  parskip=half,%
]{tubsbook}

\usepackage[utf8x]{inputenc}
\usepackage{xspace}
\usepackage{amsmath}
% --- Tabellen
\usepackage{tabularx}
\usepackage{booktabs}
\usepackage{longtable}
% --- Indexing
\usepackage{multind}
\makeindex{general}
\makeindex{cmd}
\makeindex{option}
\newcommand*{\Index}[2][]{\index{general}{#2}}
\newcommand*{\OptionIndex}[2][]{\index{option}{#2}}
\newcommand*{\CommandIndex}[2][]{\index{cmd}{#2@\textbackslash#2}}
\newcommand*{\EnvironmentIndex}[2][]{\index{cmd}{#2}}
% --- Listings
\usepackage{listings}
\lstset{basicstyle=\ttfamily}
\lstdefinestyle{cmd}{%
  frame=single,
  backgroundcolor=\color{tuGray20}
}
\lstdefinestyle{file}{%
  frame=single,
%   backgroundcolor=\color{tuGray20}
}
% --- Hyperref
\usepackage[%
  colorlinks=true,
  linkcolor=tuRed100,
  citecolor=tuGreenDark]{hyperref}
\usepackage[ngerman]{babel}
\usepackage{caption, subcaption}

\usepackage{enumerate,paralist}

\usepackage{tikz}

% --- Glossar
\usepackage[ngerman]{translator}
\usepackage[
nonumberlist, %keine Seitenzahlen anzeigen
acronym,      %ein Abkürzungsverzeichnis erstellen
toc,          %Einträge im Inhaltsverzeichnis
section]      %im Inhaltsverzeichnis auf section-Ebene erscheinen
{glossaries}
\let\acs\gls
\makeglossaries

% --- Beschreibungs-Makros
% Kopiert von KOMA
\makeatletter
\providecommand\marg[1]{%
  {\ttfamily\char`\{}\meta{#1}{\ttfamily\char`\}}}
\providecommand\oarg[1]{%
  {\ttfamily[}\meta{#1}{\ttfamily]}}
\def\cmd#1{\cs{\expandafter\cmd@to@cs\string#1}}
\def\cmd@to@cs#1#2{\char\number`#2\relax}
\DeclareRobustCommand\cs[1]{\texttt{\char`\\#1}}

\newenvironment{Declaration}{%
%    \end{macrocode}
% \begin{macro}{\new@element}
%   Help macro to define new Declaration elements.
%    \begin{macrocode}
  \newcommand*{\new@element}[1]{%
    \expandafter\newcommand\expandafter*\csname X##1\endcsname{}%
    \expandafter\let\csname X##1\expandafter\endcsname
    \csname ##1\endcsname
    \expandafter\newcommand\expandafter*\csname new##1\endcsname[1]{%
%      \begingroup
%        \let\ensuremath\@firstofone
%        \let\textit\@firstofone
%        \lowercase{\def\@tempa{##1}}%
%        \pdfstringdef\@tempb{\label@base.\@tempa.####1}%
%        \xdef\@currentHref{\@tempb}%
%        \Hy@raisedlink{\hyper@anchorstart{\@currentHref}\hyper@anchorend}%
%        \label{desc:\label@base.\@tempa.####1}%
%      \endgroup
      \csname X##1\endcsname{####1}\ignorespaces
    }%
    \expandafter\let\csname ##1\expandafter\endcsname\csname new##1\endcsname
  }%
  \newcommand*{\new@xelement}[2]{%
    \expandafter\newcommand\expandafter*\csname X##1\endcsname{}%
    \expandafter\let\csname X##1\expandafter\endcsname
    \csname ##1\endcsname
    \expandafter\newcommand\expandafter*\csname new##1\endcsname[2]{%
%      \begingroup
%        \let\ensuremath\@firstofone
%        \let\textit\@firstofone
%        \lowercase{\def\@tempa{##1}}%
%        \pdfstringdef\@tempb{\label@base.\@tempa.####1.####2}%
%        \xdef\@currentHref{\@tempb}%
%        \Hy@raisedlink{\hyper@anchorstart{\@currentHref}\hyper@anchorend}%
%        \label{desc:\label@base.\@tempa.####1.####2}%
%      \endgroup
      \csname X##1\endcsname{####1}{##2{####2}}\ignorespaces
    }%
    \expandafter\let\csname ##1\expandafter\endcsname\csname new##1\endcsname
  }%
%    \end{macrocode}
%    \begin{macrocode}
  \new@element{Option}%
  \new@element{Macro}%
  \new@element{Environment}%
  \new@element{Counter}%
  \new@element{FloatStyle}%
  \new@element{PLength}%
  \new@element{Variable}%
  \new@xelement{OptionValue}{\PValue}%
%    \end{macrocode}
% \end{macro}
%    \begin{macrocode}
  \ifvmode\else\par\fi\addvspace{2\baselineskip}%
  \vspace{-\baselineskip}%
  \vspace{\z@ plus \baselineskip}%
  \noindent
  \start@Declaration
  \tabular{|l|}\hline\ignorespaces
}{%
  \\\hline\endtabular\nobreak\after@Declaration\nobreak\par\nobreak
  \vspace{1.5\baselineskip}\nobreak\vspace{-\baselineskip}\nobreak%
  \vspace{0pt minus .5\baselineskip}\nobreak%
  \aftergroup\@afterindentfalse\aftergroup\@afterheading
}
\newcommand*{\start@Declaration}{\hspace{-1em}}
\newcommand*{\after@Declaration}{}
% \begin{macro}{\Macro}
% \begin{macro}{\Option}
% \begin{macro}{\KOption}
% \begin{macro}{\OptionValue}
% \begin{macro}{\Environment}
% \begin{macro}{\Counter}
% \begin{macro}{\Length}
% \begin{macro}{\PLength}
% \begin{macro}{\FloatStyle}
% \begin{macro}{\Pagestyle}
% \begin{macro}{\Variable}
% \begin{macro}{\FontElement}
% \begin{macro}{\PName}
% \begin{macro}{\PValue}
% \begin{macro}{\Parameter}
% \begin{macro}{\OParameter}
% \begin{macro}{\AParameter}
% \begin{macro}{\PParameter}
% \begin{macro}{\POParameter}
%   \begin{description}
%   \item[\cs{Macro}] \LaTeX{} or \TeX{} macro
%   \item[\cs{Option}] class or package option
%   \item[\cs{KOption}] |\KOMAoptions| option
%   \item[\cs{Environment}] \LaTeX{} environment
%   \item[\cs{Counter}] \LaTeX{} counter
%   \item[\cs{Length}] \LaTeX{} length
%   \item[\cs{PLength}] \KOMAScript{} pseudo length
%   \item[\cs{Variable}] \KOMAScript{} variable
%   \item[\cs{FontElement}] \KOMAScript{} element that has its own font
%     selection
%   \item[\cs{PName}] name of a parameter of a macro or environment
%   \item[\cs{PValue}] value of a parameter of a macro or environment
%   \item[\cs{Parameter}] the mandatory parameter of a macro or environment
%   \item[\cs{OParameter}] the optional parameter of a macro or environment
%   \item[\cs{AParameter}] the alternativ parameter of a macro or environment
%   \item[\cs{PParameter}] the part-of-command parameter of a macro or
%     environment
%   \end{description}
%    \begin{macrocode}
\DeclareRobustCommand*{\Macro}[1]{\mbox{\texttt{\char`\\#1}}}
\DeclareRobustCommand*{\Option}[1]{\mbox{\texttt{#1}}}
\DeclareRobustCommand*{\KOption}[1]{\mbox{\Option{#1}\texttt=}}
\DeclareRobustCommand*{\OptionValue}[2]{\mbox{\texttt{#1=#2}}}
\DeclareRobustCommand*{\FloatStyle}[1]{\mbox{\texttt{#1}}}
\DeclareRobustCommand*{\Pagestyle}[1]{\mbox{\texttt{#1}}}
\DeclareRobustCommand*{\Environment}[1]{\mbox{\texttt{#1}}}
\DeclareRobustCommand*{\Counter}[1]{\mbox{\texttt{#1}}}
\DeclareRobustCommand*{\Length}[1]{\mbox{\texttt{\char`\\#1}}}
\DeclareRobustCommand*{\PLength}[1]{\mbox{\PValue{#1}}}
\DeclareRobustCommand*{\Variable}[1]{\mbox{\PValue{#1}}}
\DeclareRobustCommand*{\FontElement}[1]{\PValue{#1}}
\DeclareRobustCommand*{\PName}[1]{\texttt{\textit{#1}}}
\DeclareRobustCommand*{\PValue}[1]{\texttt{#1}}
\DeclareRobustCommand*{\Parameter}[1]{\texttt{\{}\PName{#1}\texttt{\}}}
\DeclareRobustCommand*{\OParameter}[1]{%
  \texttt{[%]
  }\PName{#1}\texttt{%[
    ]}}
\DeclareRobustCommand*{\AParameter}[1]{%
  \texttt{(%)
  }\PName{#1}\texttt{%(
    )}}
\DeclareRobustCommand*{\PParameter}[1]{\texttt{\{#1\}}}
\DeclareRobustCommand*{\POParameter}[1]{\texttt{[#1]}}
%    \end{macrocode}
% \end{macro}
% \end{macro}
% \end{macro}
% \end{macro}
% \end{macro}
% \end{macro}
% \end{macro}
% \end{macro}
% \end{macro}
% \end{macro}
% \end{macro}
% \end{macro}
% \end{macro}
% \end{macro}
% \end{macro}
% \end{macro}
% \end{macro}
% \end{macro}
% \end{macro}
% NOTE: taken from scrguide.cls

% \begin{environment}{desctable}
%   This is almost the same like \texttt{desctabular} but it uses a longtable
%   to allow page breaks.
%    \begin{macrocode}
\newenvironment{desctable}[1][2em]{%
  \onelinecaptionsfalse
  \start@desctab{#1}%
  \newcommand{\Endfirsthead}{\toprule\endfirsthead}%
  \newcommand{\Endhead}{\midrule\endhead}%
  \newcommand*{\standardfoot}{%
    \addlinespace[-.5\normalbaselineskip]\midrule
    \multicolumn{2}{r@{}}{\dots}\\
    \endfoot
    \addlinespace[-.5\normalbaselineskip]\bottomrule
    \endlastfoot
  }%
  \longtable{lp{\descwidth}}%
}{%
  \endlongtable
}
%    \end{macrocode}
% \end{environment}

% \begin{length}{\descwidth}
%   I need a length of local usage. I could have used |\@tempdima| or
%   another local length from kernel. But I've decided not to try to find a
%   unused length at \texttt{tabular} environment.
%    \begin{macrocode}
\newlength{\descwidth}
%    \end{macrocode}
% \end{length}

% \begin{macro}{\start@desctab}
%   This is the \emph{worker} macro of \texttt{desctable} and
%   \texttt{desctabular}. It does the complete calculations and definition of
%   the entry (something like |\item|) commands.
%    \begin{macrocode}
\newcommand*{\start@desctab}[1]{%
  \setlength{\descwidth}{\linewidth}%
  \addtolength{\descwidth}{-4\tabcolsep}%
  \addtolength{\descwidth}{-#1}%
  \setlength{\labelwidth}{\linewidth}%
  \addtolength{\labelwidth}{-2\tabcolsep}%
  \newcommand{\nentry}[2]{%
    \multicolumn{2}{p{\labelwidth}}{\raggedright##1}\\*%
    \hspace*{#1} & ##2\tabularnewline%
  }%
  \newcommand{\entry}[2]{\nentry{##1}{##2}[.5\baselineskip]}%
  \newcommand*{\pentry}[1]{%
    \entry{\PLength{##1}\IndexPLength[indexmain]{##1}}}%
  \newcommand*{\pventry}[1]{\entry{\PValue{##1}}}%
  \newcommand*{\mentry}[1]{\entry{\Macro{##1}}}%
  \newcommand*{\ventry}[1]{%
    \entry{\Variable{##1}%\IndexVariable[indexmain]{##1}%
    }%
  }%
  \newcommand*{\feentry}[1]{%
    \entry{\FontElement{##1}\IndexFontElement[indexmain]{##1}}%
  }%
  \newcommand*{\oentry}[1]{%
    \entry{\Option{##1}\IndexOption[indexmain]{##1}}%
  }%
}
% \end{macro}

\makeatother

% xspace für tubslatex-Logo
\makeatletter
\g@addto@macro{\tubslatex}{\xspace}
\makeatother

\def\example{\par\smallskip\noindent\textit{Beispiel: }}

% Schreibt 'Beispiel vor den folgenden Inhalt und rückt alles nach dem ersten
% Absatz um 2em ein.
\newenvironment{Example}{%
\begingroup
\leftskip2em
\par\smallskip\noindent\hspace*{-2em}\textit{Beispiel: }
}{%
\par\endgroup
}

% Umgebung: 'Wichtig:' ...
\newenvironment{important}{%
  \begin{description}
    \item[\itshape\mdseries\rmfamily Wichtig:]
}{%
  \end{description}
}

% Umgebung: 'Hinweis'
\newenvironment{hint}{%
  \begin{description}
    \item[\itshape\mdseries\rmfamily Hinweis:]
}{%
  \end{description}
}


\newcommand{\zB}{\mbox{z.\,B.}\xspace}

% Beamer-Macros
% Copyright 2003--2007 by Till Tantau
% Copyright 2010 by Vedran Mileti\'c
%
% This file may be distributed and/or modified
%
% 1. under the LaTeX Project Public License and/or
% 2. under the GNU Free Documentation License.
%
% See the file doc/licenses/LICENSE for more details.

% $Header: /home/vedranm/bitbucket/beamer/doc/beamerug-macros.tex,v 14743c450e2c 2010/06/17 09:25:56 rivanvx $

\def\beamer{\textsc{beamer}}
\def\pdf{\textsc{pdf}}
\def\pgfname{\textsc{pgf}}
\def\translatorname{\textsc{translator}}
\def\pstricks{\textsc{pstricks}}
\def\prosper{\textsc{prosper}}
\def\seminar{\textsc{seminar}}
\def\texpower{\textsc{texpower}}
\def\foils{\textsc{foils}}

{
  \makeatletter
  \global\let\myempty=\@empty
  \global\let\mygobble=\@gobble
  \catcode`\@=12
  \gdef\getridofats#1@#2\relax{%
    \def\getridtest{#2}%
    \ifx\getridtest\myempty%
      \expandafter\def\expandafter\strippedat\expandafter{\strippedat#1}
    \else%
      \expandafter\def\expandafter\strippedat\expandafter{\strippedat#1\protect\printanat}
      \getridofats#2\relax%
    \fi%
  }

  \gdef\removeats#1{%
    \let\strippedat\myempty%
    \edef\strippedtext{\stripcommand#1}%
    \expandafter\getridofats\strippedtext @\relax%
  }

  \gdef\stripcommand#1{\expandafter\@gobble\string#1}
}

\providecommand\href[2]{\texttt{#1}}

\def\printanat{\char`\@}

\def\declare#1{{\color{red!75!black}#1}}
%\def\declare{\afterassignment\translatormanualdeclare\let\next=}
%\def\translatormanualdeclare{\ifx\next\bgroup\bgroup\color{red!75!black}\else{\color{red!75!black}\next}\fi}

\def\command#1{\list{}{\leftmargin=2em\itemindent-\leftmargin\def\makelabel##1{\hss##1}}%
\item\extractcommand#1@\par\topsep=0pt}
\def\endcommand{\endlist}
\def\extractcommand#1#2@{\strut\declare{\texttt{\string#1}}#2%
  \index{\stripcommand#1@\protect\myprintocmmand{\stripcommand#1}}}

%\let\textoken=\command
%\let\endtextoken=\endcommand

\def\myprintocmmand#1{\texttt{\char`\\#1}}

\def\example{\par\smallskip\noindent\textit{Beispiel: }}
\def\themeauthor{\par\smallskip\noindent\textit{Theme author: }}

\def\environment#1{\list{}{\leftmargin=2em\itemindent-\leftmargin\def\makelabel##1{\hss##1}}%
\extractenvironement#1@\par\topsep=0pt}
\def\endenvironment{\endlist}
\def\extractenvironement#1#2@{%
\item{{\ttfamily\char`\\begin\char`\{\declare{#1}\char`\}}#2}%
  {\itemsep=0pt\parskip=0pt\item{\meta{environment contents}}%
  \item{\ttfamily\char`\\end\char`\{\declare{#1}\char`\}}}%
  \index{#1@\protect\texttt{#1} environment}%
  \index{Environments!#1@\protect\texttt{#1}}}

\def\classoption#1{\list{}{\leftmargin=2em\itemindent-\leftmargin\def\makelabel##1{\hss##1}}%
\item{{\ttfamily\char`\\documentclass[\declare{#1}]\char`\{beamer\char`\}}}
  \index{#1@\protect\texttt{#1} class option}%
  \index{Class options for \textsc{beamer}!#1@\protect\texttt{#1}}%
  \par\topsep=0pt}
\def\endclassoption{\endlist}


\newcommand\beameroption[2]{\list{}{\leftmargin=2em\itemindent-\leftmargin\def\makelabel##1{\hss##1}}%
\item{{\ttfamily\char`\\setbeameroption\char`\{\declare{#1}{\normalfont\opt{#2}}\char`\}}}
  \index{#1@\protect\texttt{#1} beamer option}%
  \index{Beamer options!#1@\protect\texttt{#1}}%
  \par\topsep=0pt}
\def\endbeameroption{\endlist}


\def\smallpackage{\vbox\bgroup\package}
\def\endsmallpackage{\egroup\endpackage}

\def\package#1{\list{}{\leftmargin=2em\itemindent-\leftmargin\def\makelabel##1{\hss##1}}%
\extracttheme#1@usepackage@package@Packages@\par\topsep=0pt}
\def\endpackage{\endlist}
%\def\extracttheme#1#2@{%
%\item{{{\ttfamily\char`\\usepackage}#2{\ttfamily\char`\{\declare{#1}\char`\}}}}}

\def\theme#1#2#3#4{\list{}{\leftmargin=2em\itemindent-\leftmargin\def\makelabel##1{\hss##1}}%
\extracttheme#2@#1@#3@#4@\par\topsep=0pt}
\def\endtheme{\endlist}
\def\extracttheme#1#2@#3@#4@#5@{%
\item{{{\ttfamily\char`\\#3}#2{\ttfamily\char`\{\declare{#1}\char`\}}}}%
  \index{#1@\protect\texttt{#1} #4}%
  \index{#5!#1@\protect\texttt{#1}}
}

\def\class#1{\list{}{\leftmargin=2em\itemindent-\leftmargin\def\makelabel##1{\hss##1}}%
\extractclass#1@\par\topsep=0pt}
\def\endclass{\endlist}
\def\extractclass#1#2@{%
\item{{{\ttfamily\char`\\documentclass}#2{\ttfamily\char`\{\declare{#1}\char`\}}}}%
  \index{#1@\protect\texttt{#1} class}%
  \index{Classes!#1@\protect\texttt{#1}}}

\def\typesetsol#1{\texttt{\def\_{\char`\_}#1}}

\def\solution#1{\list{}{\leftmargin=2em\itemindent-\leftmargin\def\makelabel##1{\hss##1}}%
\item \textbf{Solution Template }\declare{\typesetsol{#1}}\par\topsep=0pt%
  \index{#1@\protect\typesetsol{#1} solution}%
  \index{Solutions!#1@\protect\typesetsol{#1}}}
\def\endsolution{\endlist}

\def\template#1{\list{}{\leftmargin=2em\itemindent-\leftmargin\def\makelabel##1{\hss##1}}%
\item {\ttfamily\char`\\setbeamertemplate\char`\{\declare{#1}\char`\}}\oarg{options}\opt{\meta{args}}\par\topsep=0pt}
\def\endtemplate{\endlist}
\newenvironment{template*}[1]{\list{}{\leftmargin=2em\itemindent-\leftmargin\def\makelabel##1{\hss##1}}%
\item \leavevmode\llap{\color{blue}\vtop
    to0pt{\llap{\textsc{appear-\!}}\vskip-3pt\llap{\textsc{ance}}\vss}\ \ }{\ttfamily\char`\\setbeamertemplate\char`\{\declare{#1}\char`\}}\oarg{options}\opt{\meta{args}}\par\topsep=0pt}
{\endlist}

\newenvironment{element}[4]{\list{}{\leftmargin=2em\itemindent-\leftmargin\def\makelabel##1{\hss##1}}%
\item \textbf{\ifx#2\semiyes Parent Beamer-Template\else%
    Beamer\applier#2{-Template}\applier#3{\applier#2{/}-Color}\applier#4{\ifx#2\yes/\else\ifx#3\yes/\fi\fi
      -Font}\fi}
    {\ttfamily{\declare{#1}}}\par\topsep=0pt%
  \edef\parameters{%
    \ifx#2\semiyes parent template\else%
    \applier#2{template}\applier#3{\applier#2{/}color}\applier#4{\ifx#2\yes/\else\ifx#3\yes/\fi\fi font}\fi}
  \index{#1@\protect\texttt{#1} \parameters}%
  \applier#2{\index{Beamer templates!#1@\protect\texttt{#1}}}%
  \applier#3{\index{Beamer colors!#1@\protect\texttt{#1}}}%
  \applier#4{\index{Beamer fonts!#1@\protect\texttt{#1}}}%
}
{\endlist}

\def\applier#1#2{\ifx#1\yes#2\fi}

\def\templateoptions{\par
  The following template options are predefined:
  \begin{itemize}}
\def\endtemplateoptions{\end{itemize}}

\def\itemoption#1#2{\item {\texttt{[\declare{#1}]}}#2}

%\def\itemoption#1{\item \declare{\texttt{#1}}%
%  \indexoption{#1}%
%}

%\def\indexoption#1{%
%  \index{#1@\protect\texttt{#1} option}%
%  \index{Options!#1@\protect\texttt{#1}}%
%}

\def\yes{\hbox to .6cm{\ding{51}\hfil}}
\def\semiyes{\hbox to .6cm{(\ding{51})\hfil}}
\def\no{\hbox to .6cm{\ding{55}\hfil}}

\def\choosecol#1{}%\ifx#1\yes\color{green!50!black}\else\color{red!50!black}\fi}

\def\templatefontcolor#1#2#3#4{%
  \item\declare{\texttt{#1}}\hfill%
  {\choosecol#2Template #2} {\choosecol#3Color #3} {\choosecol#4Font #4}\par}

\def\fontparents#1{Font parents: \texttt{#1}\par}
\def\colorparents#1{Color parents: \texttt{#1}\par}
\def\colorfontparents#1{Color/font parents: \texttt{#1}\par}

\def\templateinserts{\begin{itemize}}
\def\endtemplateinserts{\end{itemize}}

\def\iteminsert#1{\item {\texttt{\declare{\string#1}}}%
  \index{Inserts!\stripcommand#1@\protect\myprintocmmand{\stripcommand#1}}}

\newcommand\opt[1]{{\color{black!50!green}#1}}
\newcommand\oarg[1]{\opt{{\ttfamily[}\meta{#1}{\ttfamily]}}}
\newcommand\ooarg[1]{{\ttfamily[}\meta{#1}{\ttfamily]}}
\newcommand\sarg[1]{\opt{{\ttfamily\char`\<}\meta{#1}{\ttfamily\char`\>}}}
\newcommand\ssarg[1]{{\ttfamily\char`\<}\meta{#1}{\ttfamily\char`\>}}

%\def\opt{\afterassignment\translatormanualopt\let\next=}
\def\translatormanualopt{\ifx\next\bgroup\bgroup\color{black!50!green}\else{\color{black!50!green}\next}\fi}

\providecommand{\LyX}{L\kern-.1667em\lower.25em\hbox{Y}\kern-.125emX\@}

\newcommand{\beamernote}{\par\smallskip\noindent\llap{\color{blue}\vtop to0pt{\llap{\textsc{presen-\!}}\vskip-3pt\llap{\textsc{tation}}\vss}\ \ }}
\newcommand{\articlenote}{\par\smallskip\noindent\llap{\color{blue}\textsc{article}\ \ }}
\newcommand{\lyxnote}{\par\smallskip\noindent\llap{\color{blue}\textsc{lyx}\ \ }}
\newcommand{\appearancenote}{\par\smallskip\noindent\appearancenotetext}

\def\appearancenotetext{\llap{\color{blue}\vtop
    to0pt{\llap{\textsc{appear-\!}}\vskip-3pt\llap{\textsc{ance}}\vss}\ \ }}

\newcommand{\templatenote}{\par\smallskip\noindent\llap{\color{blue}\textsc{template}\ \ }}
\newcommand{\colornote}{\par\smallskip\noindent\llap{\color{blue}\textsc{color}\ \ }}
\newcommand{\fontnote}{\par\smallskip\noindent\llap{\color{blue}\textsc{font}\ \ }}

\newcommand{\genericthemeexample}[2][]{%
  \smallskip\par\noindent
  \pgfimage[width=.45\textwidth,page=1]{beamerug#2}\qquad\pgfimage[width=.45\textwidth,page=2]{beamerug#2}
  \smallskip\par}
\newenvironment{themeexample}[2][]
{\begin{theme}{usetheme}{{#2}#1}{presentation theme}{Presentation themes}
    \example\genericthemeexample{theme#2}
  }
{\end{theme}}
\newenvironment{innerthemeexample}[2][]
{\begin{theme}{useinnertheme}{{#2}#1}{inner theme}{Inner themes}
    \example\genericthemeexample{innertheme#2}
  }
{\end{theme}}
\newenvironment{outerthemeexample}[2][]
{\begin{theme}{useoutertheme}{{#2}#1}{outer theme}{Outer themes}
    \example\genericthemeexample{outertheme#2}
  }
{\end{theme}}
\newenvironment{colorthemeexample}[2][]
{\begin{theme}{usecolortheme}{{#2}#1}{color theme}{Color themes}
    \example\genericthemeexample{colortheme#2}
  }
{\end{theme}}
\newenvironment{fontthemeexample}[2][]
{\begin{theme}{usefonttheme}{{#2}#1}{font theme}{Font themes}
    \example\genericthemeexample{fonttheme#2}
  }
{\end{theme}}
\newenvironment{fontthemeexample*}[2][]
{\begin{theme}{usefonttheme}{{#2}#1}{font theme}{Font themes}}
{\end{theme}}

\def\partname{Part}

\colorlet{examplefill}{yellow!80!black}
\definecolor{graphicbackground}{rgb}{0.96,0.96,0.8}
\definecolor{codebackground}{rgb}{0.8,0.8,1}

\newenvironment{translatormanualentry}{\list{}{\leftmargin=2em\itemindent-\leftmargin\def\makelabel##1{\hss##1}}}{\endlist}
\newcommand\translatormanualentryheadline[1]{\itemsep=0pt\parskip=0pt\item\strut#1\par\topsep=0pt}
\newcommand\translatormanualbody{\parskip3pt}


%\newenvironment{command}[1]{
%  \begin{translatormanualentry}
%    \extractcommand#1\@@
%    \translatormanualbody
%}
%{
%  \end{translatormanualentry}
%}

%\def\extractcommand#1#2\@@{%
%  \translatormanualentryheadline{\declare{\texttt{\string#1}}#2}%
%  \removeats{#1}%
%  \index{\strippedat @\protect\myprintocmmand{\strippedat}}}


\renewenvironment{environment}[1]{
  \begin{translatormanualentry}
    \extractenvironement#1\@@
    \translatormanualbody
}
{
  \end{translatormanualentry}
}

\def\extractenvironement#1#2\@@{%
  \translatormanualentryheadline{{\ttfamily\char`\\begin\char`\{\declare{#1}\char`\}}#2}%
  \translatormanualentryheadline{{\ttfamily\ \ }\meta{environment contents}}%
  \translatormanualentryheadline{{\ttfamily\char`\\end\char`\{\declare{#1}\char`\}}}%
  \index{#1@\protect\texttt{#1} environment}%
  \index{Environments!#1@\protect\texttt{#1}}}



%\newenvironment{package}[1]{
%  \begin{translatormanualentry}
%    \translatormanualentryheadline{{\ttfamily\char`\\usepackage\opt{[\meta{options}]}\char`\{\declare{#1}\char`\}}}
%    \index{#1@\protect\texttt{#1} package}%
%    \index{Packages and files!#1@\protect\texttt{#1}}%
%    \translatormanualbody
%}
%{
%  \end{translatormanualentry}
%}



\newenvironment{filedescription}[1]{
  \begin{translatormanualentry}
    \translatormanualentryheadline{File {\ttfamily\declare{#1}}}%
    \index{#1@\protect\texttt{#1} file}%
    \index{Packages and files!#1@\protect\texttt{#1}}%
    \translatormanualbody
}
{
  \end{translatormanualentry}
}


\newenvironment{packageoption}[1]{
  \begin{translatormanualentry}
    \translatormanualentryheadline{{\ttfamily\char`\\usepackage[\declare{#1}]\char`\{translator\char`\}}}
    \index{#1@\protect\texttt{#1} package option}%
    \index{Package options for \textsc{translator}!#1@\protect\texttt{#1}}%
    \translatormanualbody
}
{
  \end{translatormanualentry}
}

\makeatletter
\def\index@prologue{\section*{Index}\addcontentsline{toc}{section}{Index}
  This index only contains automatically generated entries, sorry. A good
  index should also contain carefully selected keywords.
  \bigskip
}
% \c@IndexColumns=2
%   \def\theindex{\@restonecoltrue
%     \columnseprule \z@  \columnsep 35\p@
%     \twocolumn[\index@prologue]%
%        \parindent -30pt
%        \columnsep 15pt
%        \parskip 0pt plus 1pt
%        \leftskip 30pt
%        \rightskip 0pt plus 2cm
%        \small
%        \def\@idxitem{\par}%
%     \let\item\@idxitem \ignorespaces}
%   \def\endtheindex{\onecolumn}
% \def\noindexing{\let\index=\@gobble}

\makeatother


% Bindet Grafik mit Rahmen ein
\newcommand{\examplegraphic}[2][]{%
  \fboxsep0mm\fbox{\includegraphics[#1]{#2}}
}

