{%
\footnotesize
\fboxsep0mm
\lstset{%
  language=[LaTeX]TeX,
  captionpos=b,
  keywordstyle=\color{tubsBlueDark},
  commentstyle=\color{tubsGray},
  stringstyle=\color{tubsRed},
  frame=lefline,
  numbers=left, numberstyle={\sffamily}, numbersep=0.8em,
  numberblanklines=false
}%

\chapter{Beispiele}

Im Folgenden sind einige Beispiele zur Benutzung der verschiedenen
Klassen aufgeführt.

\section{tubsreprt}

  \lstinputlisting[
    caption={Beispiel für ein Dokument mit \Class{tubsreprt}},
    label=lst:examples:report]%
    {../examples/tubsreprt/tubsreprt_example.tex}

  \begin{figure}\centering
    \begin{minipage}{0.49\textwidth}
    \fbox{\includegraphics[width=\textwidth,page=1]%
      {../examples/tubsreprt/tubsreprt_example.pdf}}
    \end{minipage}\hfill
    \begin{minipage}{0.49\textwidth}
    \fbox{\includegraphics[width=\textwidth,page=2]%
      {../examples/tubsreprt/tubsreprt_example.pdf}}
    \end{minipage}
    \begin{minipage}{0.49\textwidth}
    \fbox{\includegraphics[width=\textwidth,page=3]%
      {../examples/tubsreprt/tubsreprt_example.pdf}}
    \end{minipage}\hfill
    \begin{minipage}{0.49\textwidth}
    \fbox{\includegraphics[width=\textwidth,page=4]%
      {../examples/tubsreprt/tubsreprt_example.pdf}}
    \end{minipage}
    \caption{Ausgabe-Beispiel zu \lstlistingname~\ref{lst:examples:report}}
    \label{fig:examples:report}
  \end{figure}

%-----------------------------------------------------------------------------

\clearpage
\section{tubslttr2}\label{sec:examples:letter}

  \lstinputlisting[%
    caption={Beispiel für eine lco-Datei auf Institutsebene},
    label=lst:examples:letter:musterinstitut]%
    {../examples/tubslttr2/musterinstitut.lco}
  
  \lstinputlisting[%
    caption={Beispiel für eine individuelle lco-Datei, die die Instituts-lco aus
            \lstlistingname~\ref{lst:examples:letter:musterinstitut} verwendet},
    label=lst:examples:letter:mustermann]%
    {../examples/tubslttr2/mustermann.lco}
  
\clearpage
  \lstinputlisting[%
    caption={Beispielbrief mit \Class{tubslttr2} unter Verwendung der
    individuellen lco-Datei aus \lstlistingname~\ref{lst:examples:letter:mustermann}},
    label=lst:examples:letter]%
    {../examples/tubslttr2/tubslttr2_example.tex}

  \begin{figure}\centering
    \begin{minipage}{0.49\textwidth}
    \fbox{\includegraphics[width=\textwidth,page=1]%
      {../examples/tubslttr2/tubslttr2_example.pdf}}
    \end{minipage}\hfill
    \begin{minipage}{0.49\textwidth}
    \fbox{\includegraphics[width=\textwidth,page=2]%
      {../examples/tubslttr2/tubslttr2_example.pdf}}
    \end{minipage}
    \caption{Ausgabe-Beispiel zu \lstlistingname~\ref{lst:examples:letter}}
    \label{fig:examples:letter:mustermann}
  \end{figure}

%-----------------------------------------------------------------------------
  
\clearpage
\section{beamer-Theme}

  \lstinputlisting[
    linerange={1-42,50-87,186-187},
    caption={Beispiel für eine Präsentation mit dem \Class{beamer}-Theme \Package{tubs}},
    label=lst:examples:beamer]%
    {../examples/beamer/beamer_example.tex}
  
    \begin{figure}
    \begin{minipage}{0.49\textwidth}\centering
    \fbox{\includegraphics[width=\textwidth,page=1]%
      {../examples/beamer/beamer_example.pdf}}
    \end{minipage}\hfill
    \begin{minipage}{0.49\textwidth}\centering
    \fbox{\includegraphics[width=\textwidth,page=3]%
      {../examples/beamer/beamer_example.pdf}}
    \end{minipage}\\[3mm]
    \begin{minipage}{0.49\textwidth}\centering
    \fbox{\includegraphics[width=\textwidth,page=4]%
      {../examples/beamer/beamer_example.pdf}}
    \end{minipage}\hfill
    \begin{minipage}{0.49\textwidth}\centering
    \fbox{\includegraphics[width=\textwidth,page=5]%
      {../examples/beamer/beamer_example.pdf}}
    \end{minipage}
    \caption{Ausgabe-Beispiel zu \lstlistingname~\ref{lst:examples:beamer}}
    \label{fig:examples:beamer}
  \end{figure}

%-----------------------------------------------------------------------------

\clearpage
\section{tubsposter}

  \lstinputlisting[
    linerange={1-42},
    caption={Beispiel für ein Veranstaltungsplakat mit \Class{tubsposter}},
    label=lst:examples:tubsposter]%
    {../examples/tubsposter/tubsposter_example.tex}


  \begin{figure}[!ht]\centering
    \fbox{\includegraphics[width=0.8\textwidth,page=1]%
      {../examples/tubsposter/tubsposter_example.pdf}}
    \caption{Ausgabe-Beispiel zu \lstlistingname~\ref{lst:examples:tubsposter}}
    \label{fig:examples:tubsposter}
  \end{figure}
  
\clearpage
  \lstinputlisting[
    linerange={1-6,59-100},
    caption={Beispiel für ein wissenschaftliches Plakat mit \Class{tubsposter}},
    label=lst:examples:scifiposter]%
    {../examples/tubsposter/scifiposter_example.tex}

  \begin{figure}[!ht]\centering
    \fbox{\includegraphics[width=0.8\textwidth,page=2]%
      {../examples/tubsposter/scifiposter_example.pdf}}
    \caption{Ausgabe-Beispiel zu \lstlistingname~\ref{lst:examples:scifiposter}}
    \label{fig:examples:scifiposter}
  \end{figure}

%-----------------------------------------------------------------------------

\clearpage
\section{tubsleaflet}

  \lstinputlisting[
    caption={Beispiel für einen Flyer mit \Class{tubsleaflet}},
    label=lst:examples:leaflet]%
    {../examples/tubsleaflet/tubsleaflet-example.tex}
  
    \begin{figure}
    \begin{minipage}{\textwidth}\centering
    \fbox{\includegraphics[width=0.8\textwidth,page=1]%
      {../examples/tubsleaflet/tubsleaflet-example.pdf}}
    \end{minipage}\\[1ex]
    \begin{minipage}{\textwidth}\centering
    \fbox{\includegraphics[width=0.8\textwidth,page=2]%
      {../examples/tubsleaflet/tubsleaflet-example.pdf}}
    \end{minipage}
    \caption{Ausgabe-Beispiel zu \lstlistingname~\ref{lst:examples:leaflet}}
    \label{fig:examples:leaflet}
  \end{figure}

}
