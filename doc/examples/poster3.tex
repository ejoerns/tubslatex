\documentclass[a3paper,13pt]{tubsposter}
\usepackage[utf8]{inputenc}

\usepackage{lipsum}

\pagestyle{empty}% TODO: add to class!!!

\begin{document}
\begin{tubsposter}[sender=bottom]
  %% aktiviert die horizontale Linie am oberen Ende des Kommunikationsbereichs.
  %   \showtopline
  %% Stellt das TU-Logo an der gewünschten Seite (links/rechts) dar.
  \showtubslogo[left]
  %% Stellt das Institutslogo im Absenderbereich dar.
  \showlogo{Institutslogo einfügen oder Institutsname/\\
    zentrale Einrichtung als Text eingeben}
  %% Alternative Version des Logos mit Bild statt Text.
  %   \showlogo{\includegraphics{dummy_institut.jpg}}
  %% Beginnt eine neuen Bereich mit der angegebenen Höhe im Gaußraster.
  \begin{posterrow}[bgimage=infozentrum.jpg]{6}
  \end{posterrow}
  \begin{posterrow}[bgcolor=tuGreenDark80]{2}
    \color{tuWhite}
    {\usekomafont{headline} Headline xxpt\bigskip}
      
    {\usekomafont{subheadline} Subheadline\\
    Mitant dur Wolche to illemit drusi puzen, um brackl jaun utten}
    \vfill
  \end{posterrow}
\end{tubsposter}
\end{document}
