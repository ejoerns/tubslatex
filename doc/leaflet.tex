\chapter{Broschüren}

Mit der Klasse \newdocumentclass{tubsleaflet} können dreigeteilte Broschüren
im Corporate Design erstellt werden.
Die Vorlage basiert auf der Klasse \texttt{leaflet}.
Alle allgemeinen Informationen und Möglichkeiten sind daher in der
entsprechenden Dokumentation\cite{cls:leaflet} nachzuschlagen.

Die Ausgabe erfolgt zweiseitig auf DINA4-Größe.
Andere Formate können gewählt werden, sind aber nicht voll unterstützt.

\begin{important}
  Als Papierformat sollte nicht \Option{a4paper} gewählt werden,
  da so falsche Größen geladen werden. Die korrekte Option lautet
  \Option{langpaper} (\gls{glos:dinlang}),
  ist aber optional, da sie automatisch vorgeladen wird.
\end{important}


\begin{figure}[!ht]
  \examplegraphic[width=\textwidth]{examples/leaflet.pdf}
  \caption{Einfacher Flyer im CD (erste Seite)}
  \label{fig:leaflet:example}
\end{figure}

% TODO: wird aktuell nicht richtig unterstützt
% \paragraph{Darstellungsstil}\hfill
% 
% \begin{Declaration}
%   \KOption{style}\PName{<simple/none>}
% \end{Declaration}
% 
% Der Darstellungsstil \PName{simple} setzt standardmäßig das Frontseiten- und das
% Rückseiten-Logo.
% Im Darstellungsstil \PName{none} werden standardmäßig keine Elemente gesetzt.
% 
% Hintergrundelemente können manuell mittels \Macro{AddToBackground} gesetzt
% werden.

\section{Seitenerstellung}

Die einzelnen Seiten können wie bei den anderen Dokumentenklassen
mit der Umgebung \Environment{gausspage} erstellt werden.
Diese erlaubt das Erstellen von Seiten im Gaußraster.
Da sie ein universelles Element von \tubslatex ist, kann hier auf die
allgemeinene Beschreiung in \chaptername~\ref{sec:gausslayout:gausspage}
verwiesen werden.

\paragraph{Reihenfolge}\hfill

Die verschiedenen Seiten des Flyers müssen in der Reihenfolge
geschrieben werden in der sie später gelesen werden sollen.
Die korrekte Darstellungsreihenfolge der Seiten für den Druck
wird automatisch gesetzt.
Die Reihenfolge geht dabei von der Deckseite des Flyers
über die Innenseite des aufgeklappten Flyers bis hin zur Rückseite
des aufgeklappten Flyers.
Die Seite, die im zugefalteten Zustand der Titelseite gegenüberliegt,
ist somit auch die zuletzt einzugebende Seite. Anschaulich wird dies
in \figurename~\ref{fig:leaflet:pageorder} dargestellt.

Zu beachten ist auch, dass in \Class{leaflet} die Ausgabe der zweiten 
Seite standardmäßig überkopf erfolgt.
Ist dies nicht gewünscht, kann dies mit der Klassenoption
\Option{notumble} deaktiviert werden.

\begin{figure}[!ht]
\begin{minipage}{0.475\textwidth}
\fboxsep0mm%
\fbox{\includegraphics[width=\textwidth,page=1]{examples/tubsleaflet_order}}
\end{minipage}%
\hfill
\begin{minipage}{0.475\textwidth}
\fboxsep0mm%
\fbox{\includegraphics[width=\textwidth,page=2]{examples/tubsleaflet_order}}
\end{minipage}
\caption{Ausgabereihenfolge und -orientierung der einzelnen Seiten, wobei 
die Zahl der Seitenposition im tex-Dokument entspricht.}
\label{fig:leaflet:pageorder}
\end{figure}

\section{leaflet-Klassenoptionen}

Zur einfacheren Orientierung werden hier einmal kurz die wichtigsten Optionen,
die die Klasse \Class{leaflet}, welche die Basisklasse von \Class{tubsleaflet}
ist, zur Verfügung stellt.

\begin{Declaration}
  \Option{notumble}
\end{Declaration}
Normalerweise wird die zweite zu druckende A4-Seite auf dem Kopf stehend erzeugt,
um in einigen Fällen den Druck zu erleichtern.
Sollte dies Verhalten nicht gewünscht sein, kann es mit der Option
\Option{notumble} deaktiviert werden, sodass beide Seiten normal (lesbar)
ausgegeben werden.

\begin{Declaration}
  \Option{nofoldmark}
\end{Declaration}
Um das Falten zu erleichtern werden an den Faltpositionen Falzmarken
eingezeichnet. Sollen diese nicht dargestellt werden, kann ihre
Darstellung mit der Option \Option{nofoldmark} deaktiviert werden.

\begin{Declaration}
  \Option{portrait}
\end{Declaration}

Standardmäßig werden die Seiten im PDF im Querformat ausgegeben.
Mit der Option \Option{portrait} werden die PDF-Seiten im Hochformat dargestellt.
Das Format des Inhalts verändert sich dagegen nicht.

\begin{Declaration}
  \Option{nocombine}
\end{Declaration}
Normalerweise werden die einzelnen Flyer-Seiten kombiniert auf 2 A4-Seiten
ausgegeben, um sie so einfach ausdrucken und falten zu können.
Mit der Option \Option{nocombine} wird dagegen für jede Flyer-Seite eine
einzelne pdf-Seite erzeugt.

\begin{Declaration}
  \Option{frontside}\\
  \Option{backside}\\
  \Option{bothsides}
\end{Declaration}
Diese Optionen steuern, welche Seiten im PDF ausgegeben werden sollen.
Normal werden Vorder- und Rückseite ausgegeben, was der Optionen
\Option{bothsides} entspricht.
Soll nur die Vorderseite erzegut werden, kann dies mit der Option
\Option{frontside} bewirkt werden.
Soll nur die Rückseite erzeugt werden, kann dies mit der Option
\Option{backside} bewirkt werden.
