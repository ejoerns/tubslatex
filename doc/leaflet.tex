\chapter{Broschüren}

Mit der Klasse \newdocumentclass{tubsleaflet} können dreigeteilte Broschüren
im Corporate Design erstellt werden.
Die Vorlage basiert auf der Klasse \texttt{leaflet}.
Alle allgemeinen Informationen und Möglichkeiten sind daher in der
entsprechenden Dokumentation\cite{cls:leaflet} nachzuschlagen.

Die Ausgabe erfolgt zweiseitig auf DINA4-Größe.
Andere Formate können gewählt werden, sind aber nicht voll unterstützt.

\begin{important}
  Als Papierformat sollte nicht \Option{a4paper} gewählt werden,
  da so falsche Größen geladen werden. Die korrekte Option lautet
  \Option{langpaper}, ist aber optional, da sie automatisch vorgeladen wird.
\end{important}


\begin{figure}[!ht]
  \examplegraphic[width=\textwidth]{examples/leaflet.pdf}
  \caption{Einfacher Flyer im CD (erste Seite)}
\end{figure}

\paragraph{Darstellungsstil}\hfill

\begin{Declaration}
  \KOption{style}\PName{<simple/none>}
\end{Declaration}

Der Darstellungsstil \PName{simple} setzt standardmäßig das Frontseiten- und das
Rückseiten-Logo.
Im Darstellungsstil \PName{none} werden standardmäßig keine Elemente gesetzt.

Hintergrundelemente können manuell mittels \Macro{AddToBackground} gesetzt
werden.

\paragraph{Seiten}\hfill

Die einzelnen Seiten können mit der Umgebung \Environment{gausspage} erstellt
werden. Diese erlaubt das Erstellen von Seiten im Gaußraster.
Da sie ein universelles Element von \tubslatex ist, ist ihr ein spezielles
\href{chap:gausspage}{Kapitel} gewidmet.