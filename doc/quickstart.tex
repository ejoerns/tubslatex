\chapter{Schnellstart}\label{chap:rapid}

Dieses Kapitel soll einen kurzen Schnelleinstieg in die Benutzung von
\tubslatex bieten. Eine ausführliche Beschreibung der erwähnten
Klassen und Funktionalitäten, sowie weiterer Möglichkeiten der Vorlagen
bieten die entsprechenden Einzel-Kapitel.

Für eine schnelle Einarbeitung ist es in jedem Fall empfehlenswert, die
zur Verfügung stehenden Beispieldateien zu studieren, die aus der
CD-Toolbox oder der Projektwebsite bezogen werden können.

\section{Dokumente}
\tubslatex bietet für die Erstellung von Textdokumenten drei auf den
Standardklassen aufbauende Klassen. Dies sind \newdocumentclass{tubsartcl},
\newdocumentclass{tubsreprt} und \newdocumentclass{tubsbook}.

\textbf{Titelseiten} können wie in den Standardklassen mit \lstinline{\maketitle}
erzeugt werden, ein optionales Argument bietet jedoch noch die Möglichkeit
vordefinierte Styles zu laden.
Mit der Umgebung \lstinline{titlepage} können individuelle Titelseiten
im Gaußraster erstellt werden.

\textbf{Rückseiten} können ähnlich den Titelseiten mit dem Befehl
\lstinline{\makebackpage} bzw. der Umgebung \lstinline{backpage} erstellt werden.



\paragraph{Poster}
Die Klasse \newdocumentclass{tubsposter} kann zum Erstellen von Postern verwendet werden.
% tubsflyer?

\paragraph{Papierformat}
Das zu verwendende Papierformat sollte dabei als optionales Argument mit
übergeben werden. Für ein Dokument in DIN A4 ist dies \texttt{a4paper}.
Zur Verfügung stehen alle Papierformate von A0 bis A6.

Standardmäßig wird die Titelseite in Dokumenten einfach mit dem Logo und einer
roten Trennlinie zwischen Kommunikations- und Absenderbereich versehen.

\paragraph{Briefe}

Die Klasse \newdocumentclass{tubslttr2} ist für die Erstellung von Briefen
vorgesehen.

\paragraph{Präsentationen}
Um Präsentation zu erstellen existiert ein Style für \LaTeX-Beamer.
Dieser wird einfach mit \lstinline!\usetheme{tubs}! geladen.
