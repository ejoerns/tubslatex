\chapter{Schnellstart}\label{chap:rapid}

% TODO: Schnelleinführung Corporate Desing (Gaußraster, etc.)

\paragraph{Dokumente}
Die schnellste Methode ein Dokument im Corporate Design zu erstellen ist
das Laden einer der zur Verfügung stehenden Dokumentenklassen.
Für die Erstellung von Textdokumenten sind dies \newdocumentclass{tubsartcl},
\newdocumentclass{tubsreprt} und \newdocumentclass{tubsbook}.

\paragraph{Poster}
Die Klasse \newdocumentclass{tubsposter} kann zum Erstellen von Postern verwendet werden.
% tubsflyer?

\paragraph{Papierformat}
Das zu verwendende Papierformat sollte dabei als optionales Argument mit
übergeben werden. Für ein Dokument in DIN A4 ist dies \texttt{a4paper}.
Zur Verfügung stehen alle Papierformate von A0 bis A6.

Standardmäßig wird die Titelseite in Dokumenten einfach mit dem Logo und einer
roten Trennlinie zwischen Kommunikations- und Absenderbereich versehen.

\paragraph{Briefe}

Die Klasse \newdocumentclass{tubslttr2} ist für die Erstellung von Briefen
vorgesehen.

\paragraph{Präsentationen}
Um Präsentation zu erstellen existiert ein Style für \LaTeX-Beamer.
Dieser wird einfach mit \lstinline!\usetheme{tubs}! geladen.
