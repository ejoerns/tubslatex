\chapter{Einleitung}

Das vorliegende Dokument beschreibt die Möglichkeiten von \tubslatex.
Mit \tubslatex\ ist im Folgenden immer die Sammlung aller Vorlagen in Form
von \LaTeX-Paketen und -Klassen bezeichnet, welche für die Dartellung von
Dokumenten im Coporate Design der TU Braunschweig vorgesehen sind.

Die Vorlagen bauen allesamt auf Standard-\LaTeX-Klassen und Paketen auf.
Insbesondere sind alle Dokumentenklassen von Klassen des KOMA-Scripts
abgeleitet, was sich auch in ihrem Benennungsschema wiederspiegelt.
Konkret bauen die Vorlagen für Poster (\texttt{tubsposter}) und Dokumente
(\texttt{tubsartcl}, \texttt{tubsreprt}, \texttt{tubsboox}) auf
KOMA-Skript-Klassen auf. Die Vorlage für Broschüren (\texttt{tubsleaflet})
ist von der Klasse \texttt{leaflet} abgeleitet. Für Präsentationen wird
die Klasse \texttt{beamer} verwendet, hier gibt es keine spezielle Klasse,
die Vorlagen werden als \texttt{beamer-theme} geladen. Die Briefklasse
\texttt{tubslttr2} ist dafür wieder direkt von der KOMA-Briefklasse
\texttt{scrlttr2} abgeleitet.

Einige Grundelemente sind in Paketen untergebracht, sodass sie bei Bedarf auch
von anderen \LaTeX-Klassen benutzt werden könnten. Dazu gehören unter anderem
die Schriftart \emph{Nexus}, das Siegelbandlogo und die Farbdefinitionen.
Auf diesen Aspekt geht der zweite Teil des Dokumentes näher ein.

Der erste Teil des Dokumentes beschäftigt sich mit der allgemeinen Verwendung
der vorgestellten Klassen. Für besonders Eilige empfiehlt sich das
Kapitel~\ref{chap:rapid}, welches in aller Kürze die jeweils wichtigsten
Befehle und Optionen zusammenfasst.
