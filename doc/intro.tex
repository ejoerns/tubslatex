\chapter{Einleitung}

Das vorliegende Dokument beschreibt die Verwendung und die Möglichkeiten von \tubslatex.
Mit \tubslatex wird im Folgenden immer die Sammlung aller \LaTeX-Vorlagen
(in Form von Paketen und Klassen) zur Erstellung von Dokumenten im
Corporate Design der TU Braunschweig bezeichnet.

Dieses einführende Kapitel stellt zuerst die wesentlichen Elemente und Richtlinien
des Corporate Designs vor und vermittelt so ein erstes Verständnis für
die bereitgestellten Gestaltungsmöglichkeiten.

Im Kapitel~\ref{chap:install} wird dann die automatische und manuelle
Installation von \tubslatex auf verschiedenen Systemen erläutert.

Der \hyperlink{part:user}{erste} Hauptteil des Dokumentes beschäftigt sich mit
der allgemeinen Verwendung der bereitgestellten Dokumentenklassen.
Dabei wurde versucht, für jede Klasse eine möglichst vollständige Beschreibung
zu liefern. Da sich allerdings viele Funktionen gleichen
(da sie auf die selben Grundfunktionen zurück greifen)
sind manche Funktionen an einer Stelle nur kurz erwähnt mit einem Verweis
auf die Stelle mit ausführlicherer Beschreibung.

Einige Grundelemente von \tubslatex sind in eigenständigen Paketen untergebracht,
sodass sie bei Bedarf auch von anderen \LaTeX-Klassen benutzt werden könnten.
Dazu gehören unter anderem die Schriftart \emph{Nexus}, das Siegelband-Logo
und die Farbdefinitionen.
Auf diesen Aspekt geht der \hyperlink{part:packages}{zweite}
Teil des Dokumentes näher ein.

Für besonders Eilige empfiehlt sich das Kapitel~\ref{chap:rapid},
welches in aller Kürze die jeweils wichtigsten Befehle und Optionen einer Klasse
zusammenfasst und so einen raschen Einstieg ermöglicht.

\clearpage
\section{Das Corporate Design der TU Braunschweig}

\paragraph{Seitenlayout}\hfill\\
\begin{minipage}[t]{0.45\textwidth}
\vspace*{0pt}
\centering\fboxsep0mm%
\fbox{\includegraphics[width=0.85\textwidth,page=1]{examples/thecd.pdf}}
\end{minipage}
\begin{minipage}[t]{0.55\textwidth}
\vspace*{0pt}
Das Corporate Design setzt sich aus ein paar wesentlichen Elementen und
Gestaltungsmerkmalen zusammen.
Eine Seite teilt sich grundsätzlich in einen \emph{Absenderbereich} (1) und einen
\emph{Kommunikationsbereich} (2).
Das markante rote \emph{Siegelband-Logo} dient dabei als Bindeglied zwischen
diesen beiden Bereichen. 
Der Absenderbereich kann sowohl am oberen als auch am unteren Blattende
platziert werden.
Das Siegelband-Logo passt sich in seiner vertikalen Position an und kann darüber hinaus
jeweils rechts- oder linksseitig platziert werden.
Daraus ergeben sich insgesamt 4 Darstellungsvarianten.

Der Kommunikationsbereich wird von einem Rand fester Breite eingerahmt,
der der Gestaltung einen edlen Wiedererkennungswert verleiht und gleichzeitig
den Inhalt absetzt, um Druck- und Kopierprobleme zu vermeiden.
\end{minipage}\bigskip
% TODO: note landscape?

\paragraph{Absenderfeld und Gaußraster}\hfill\\
\begin{minipage}[t]{0.45\textwidth}
\vspace*{0pt}
\centering\fboxsep0mm%
\fbox{\includegraphics[width=0.85\textwidth,page=2]{examples/thecd.pdf}}
\end{minipage}
\begin{minipage}[t]{0.55\textwidth}
\vspace*{0pt}
Das im Absenderbereich befindliche \emph{Absenderfeld} (1) bietet Platz für
einen Instituts-/Abteilungsnamen bzw. ein entsprechendes Logo.

Der Kommunikationsbereich kann in einem an die gaußsche Summenformel
angelehnten Raster (\emph{Gaußraster}) unterteilt werden.
Dabei ergibt sich die Höhe eines Grundsegmentes ((2), (3), \ldots)
immer aus der Summer der Höhe der beiden vorhergehenden Grundsegmente.
Zur Darstellung können beliebig viele Grundsegmente zusammengefasst werden,
womit sich ein zugleich flexibles aber trotzdem charakteristisches
Gesamtbild ergibt.

Das Gaußraster ist immer so ausgerichtet, dass sich das breiteste Grundsegment
an der Grenze zum Absenderbereich befindet.
\end{minipage}

\paragraph{Schrift und Farbwelt}\hfill\\
\begin{minipage}[t]{0.45\textwidth}
\vspace*{0pt}
\centering\fboxsep0mm%
\fbox{\includegraphics[width=0.85\textwidth,page=3]{examples/thecd.pdf}}
\end{minipage}
\begin{minipage}[t]{0.55\textwidth}
\vspace*{0pt}
Zum Umfang des Corporate Design gehört neben den verschiedenen
Gestaltungsmerkmalen auch eine unverkennbare Schrift.
Dieses ist mit der modernen Hausschrift \emph{Nexus} gegeben, die sowohl mit als auch
ohne Serifen in jeweils drei Schriftschnitten zur Verfügung steht.

Ein weiteres wichtiges Merkmal von Dokumenten im Corporate Design sind
die verwendeten Farben, die sich aus einer Reihe fest definierter Paletten
zusammen setzen.
Als wichtige Strukturfarben sind das TU-Rot, sowie Schwarz und Weiß vorgesehen.
Zur individuellen Gestaltung stehen 4 Farbklänge (Gelb/Orange, Grün, Blau, Violett)
mit jeweils 3 Farben und 12 Abstufungen zur Verfügung.
\end{minipage}

\clearpage
\section{Übersicht über die Vorlagen}

Die Vorlagen bauen allesamt auf Standard-\LaTeX-Klassen und Paketen auf.
Insbesondere sind alle Dokumentenklassen von Klassen des KOMA-Skripts
abgeleitet, was sich auch in ihrem Benennungsschema widerspiegelt.
Konkret bauen die Vorlagen für Poster (\texttt{tubsposter}) und Dokumente
(\texttt{tubsartcl}, \texttt{tubsreprt}, \texttt{tubsboox}) auf
den entsprechenden KOMA-Skript-Klassen auf,%
\footnote{\texttt{tubsposter} verwendet \texttt{scrartcl}}.
Dabei wurden einige Optionen und Befehle, die die Basis-Klassen bieten, erhalten
und können in der ausführlichen Dokumentation zum KOMA-Skript\cite{koma-skript}
nachgeschlagen werden.1
Einige Funktionalitäten wie etwa die Satzspiegelberechnung sind dagegen
komplett ersetzt wurden.
Die Dokumentation geht an einigen Stellen noch expliziter auf Unterschiede ein.

Ebenfalls von einer KOMA-Klasse abgeleitet ist die Briefklasse
\texttt{tubslttr2}.
Diese nutzt (wie der Name vermuten lässt) die noch verhältnismäßig junge
Klasse \texttt{scrlttr2}. Dies spiegelt sich vor allem darin wieder, dass sich
das verwendete Interface in Teilen stark von den anderen Klassen unterscheidet.
Die Dokumentation geht dabei hauptsächlich auf die Besonderheiten in
\texttt{tubslatex} ein.
Für detailliertere Informationen kann hier auch wieder die Dokumentation von
KOMA\cite{koma-skript} zu Rate gezogen werden.

Für Präsentationen wird die relativ bekannte \texttt{beamer}-Klasse verwendet.
Hier stellt \tubslatex keine eigene Klasse zur Verfügung,
sondern bietet entsprechend der Beamer-Philosophie ein \emph{Beamer-Theme} an,
das im Dokument geladen werden kann und so viel Flexibilität erlaubt.
Bei der Verwendung ergeben sich daher lediglich eine minimale Anzahl an
Veränderungen und Erweiterungen des Funktionsumfangs von beamer.

Die Vorlage für Broschüren (\texttt{tubsleaflet}) ist
von der Klasse \texttt{leaflet} abgeleitet.


