\chapter{Einleitung}

Das vorliegende Dokument beschreibt die Möglichkeiten von \tubslatex.
Mit \tubslatex\ ist im Folgenden immer die Sammlung aller Vorlagen in Form
von \LaTeX-Paketen und -Klassen bezeichnet, welche für die Dartellung von
Dokumenten im Coporate Design der TU Braunschweig vorgesehen sind.

Das Kapitel~\ref{chap:install} bietet einige kleine Hinweise, die bei
der Installation von \tubslatex nützlich sein könnten.
Es beschreibt außerdem Möglichkeiten zur manuellen Installation, falls
kein Installer für das gewünschte Betriebssystem installiert oder eine
individuelle Installation durchgeführt werden soll.

Der erste Teil des Dokumentes beschäftigt sich mit der allgemeinen Verwendung
der vorgestellten Klassen.
Dabei wurde versucht, für jede Klasse eine möglichst vollständige Beschreibung
zu liefern. Da sich allerdings viele Funktionen gleichen, das sie auf die selben
Grundfunktionen zurück greifen sind manche Funktionen an einer Stelle nur kurz erwähnt mit einem Verweis auf die ausführlichere Beschreibung.

Einige Grundelemente von \tubslatex sind in eigenständigen Paketen untergebracht, sodass sie bei Bedarf auch von anderen \LaTeX-Klassen benutzt werden könnten. Dazu gehören unter anderem die Schriftart \emph{Nexus}, das Siegelbandlogo und die Farbdefinitionen.
Auf diesen Aspekt geht der zweite Teil des Dokumentes näher ein.% TODO: Link

Für besonders Eilige empfiehlt sich das Kapitel~\ref{chap:rapid}, welches in aller Kürze die jeweils wichtigsten Befehle und Optionen einer Klasse zusammenfasst und so einen raschen Einstieg ermöglicht.

\section{Übersicht über die Vorlagen}

Die Vorlagen bauen allesamt auf Standard-\LaTeX-Klassen und Paketen auf.
Insbesondere sind alle Dokumentenklassen von Klassen des KOMA-Scripts
abgeleitet, was sich auch in ihrem Benennungsschema wiederspiegelt.
Konkret bauen die Vorlagen für Poster (\texttt{tubsposter}) und Dokumente
(\texttt{tubsartcl}, \texttt{tubsreprt}, \texttt{tubsboox}) auf
den entsprechenden KOMA-Skript-Klassen auf%
\footnote{\texttt{tubsposter} verwendet \texttt{scrartcl}}.
Dabei wurden einige Optionen und Befehle, die die Basis-Klassen bieten erhalten
und können in der ausführlichen Dokumentation zum KOMA-Skript\cite{koma-skript}
nachgeschlagen werden.
Einige Funktionalitäten wie etwa die Satzspiegelberechnung sind dagegen
komplett ersetzt wurden.
Die Dokumentation geht an einigen Stellen noch expliziter auf Unterschiede ein.

Ebenfalls von einer KOMA-Klasse abgeleitet ist die Briefklasse
\texttt{tubslttr2}.
Diese nutzt (wie der Name vermuten lässt) die noch verhältnismäßig junge
Klasse \texttt{scrlttr2}. Dies spiegelt sich vor allem darin wieder, dass
das verwendete Interface sich in Teilen stark von den anderen Klassen unterscheidet. Die Dokumentation geht dabei hauptsächlich auf die Besonderheiten in \texttt{tubslatex} ein. Für detailliertere Informationen empiehlt sich auch hier wieder die Dokumentation von KOMA\cite{koma-skript}.

Für Präsentationen wird die relativ bekannt \texttt{beamer}-Klasse verwendet. Hier stellt \tubslatex keine eigene Klasse zur Verfügung, sondernd bietet entsprechend der Beamer-Philosophie ein \emph{Beamer-Theme} an, dass
im Dokument geladen werden kann und so viel Flexibilität erlaubt.
Bei der Verwendung ergeben sich daher lediglich eine minimale Anzahl an
Veränderungen und Erweiterungen des Funktionsumfangs von beamer.

Die Vorlage für Broschüren (\texttt{tubsleaflet}) ist
von der Klasse \texttt{leaflet} abgeleitet.\par


