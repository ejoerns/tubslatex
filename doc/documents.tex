\chapter{Dokumente}

Einfache Textdokumente können mit den Klassen \newdocumentclass{tubsartcl},
\newdocumentclass{tubsreprt} und \newdocumentclass{tubsbook} erstellt werden.

% \section{Bindekorrektur und Marginalen}

Da Dokumente normalerweise gedruckt werden und ggf. auch gebunden, werden
sie standardmäßig mit einer kleinen Bindekorrektur gesetzt, sodass das
Logo beim Drucken oder Abheften nicht abgeschnitten wird, außerdem wird
es als zweiseitiges Dokument in CMYK-Farben ausgegeben.
Für eine Optimierung zur Ausgabe auf einem Bildschirm steht die Option
\OptionValue{style}{screen} zu Verfügung. Weitere Paket-Optionen dienen der
genauen individuellen Anpassung der Darstellung. Eine genauere Darstellung
liefert \mbox{u.\,a.}\xspace das Kapitel~\ref{sec:pagelayout}.

Für die Darstellung von Titelseiten stellt \tubslatex verschiedene Möglichkeiten
zur Verfügung, die an die konventionelle Erstellung von Titelseiten angelehnt
sind. So sowohl sowohl der Befehl \Macro{maketitle} als auch die Umgebung
\Environment{titlepage} so umdefiniert, dass man mit ihnen einfach CD-konforme
Titelseiten aus Vorlagen auswählen bzw. individuell erstellen kann.
Darüber hinaus steht auch die Möglichkeit zur Verfügung mittels
\Macro{makebackpage} oder der Umgebung \Environment{backpage}
Rückseiten zu erstellen. Kapitel~\ref{sec:titlepage} geht darauf näher ein.

Darüber hinaus ist die Schrift Nexus, sowie alle Farben des Corporate Designs
in allen Dokumenten vordefiniert.
Befehle für die Erstellung von Kopf- bzw. Fußzeilen sind CD-konform angepasst und
werden im Kapitel~\ref{sec:headline} beschrieben.

Die Darstellung von Seiten basiert im CD-Layout im Prinzip immer auf den
sogenannten Gaußraster. Dies unterteilt die Seite in verschieden große
Segmente, deren Höhe durch die gaußsche Summenformel berechnet werden kann.
Außerdem könne Seiten ein Reihe von Standardelementen wie das Sigellogo oder
ein zusätzliches individuelles Logo aufweisen.
Alle Seiten, die im Stil des \gls{glos:gaussraster} erzeugt werden basieren letzendlich
auf dem Selben backend. Dies wird in Kapitel~\ref{sec:gausspage} ausführlicher
beschrieben.


\section{Titelseite}\label{sec:titlepage}

Titelseiten können bei \LaTeX\ generell auf zwei verschiedene Arten erstellt
werden; entweder mit Hilfe des Befehls \lstinline{\maketitle} oder mit
der Umgebung \lstinline{titlepage}. Beide Varianten werden von \tubslatex
in modifizierter Weise unterstützt.

\begin{Declaration}
  \Macro{maketitle}\OParameter{style}
\end{Declaration}

Die einfache Verwendung von \lstinline{\maketitle} erzeugt eine Titelseite
mit dem TU-\gls{glos:siegelbandlogo} und einer roten Trennlinie zwischen Absender- und
Kommunikationsbereich.

Mit Hilfe des optionalen Arguments \PName{style} kann die Darstellung
der Titelseite geändert werden, indem aus einer Reihe vordefinierter Styles
ausgewählt wird. Eine weiterführende Erklärung findet
sich dazu in Kapitel~\ref{sec:titlestyles}.\bigskip

Neben den standardmäßig definierten Elementen für Titelseiten wie \Macro{author}
oder \Macro{title} werden in \tubslatex noch ein paar Zusätzliche definiert.

\begin{Declaration}
  \Macro{logo}\Parameter{logo}\\
  \Macro{titlepicture}\Parameter{file}\\
  \Macro{titleabstract}\Parameter{text}
\end{Declaration}

\Macro{logo} dient zur Darstellung eines zusätzlichen Absenders als Schrift
oder Bild. Es wird in allen Stilen im Absenderbereich auf der dem
TU \gls{glos:siegelbandlogo} gegenüberliegenden Seite dargestellt. Eingefügte Grafiken
werden automatisch auf die Höhe des Kopfbereichs skaliert.

Mit \Macro{titlepicture} kann eine Bilddatei angegeben werden, die bei
Verwendung eines entsprechenden Styles auf der Titelseite dargestellt wird.

Der Befehl \Macro{titeabstract} erlaubt die Darstellung eines kurzen zusammenfassenden Textes auf der Titelseite, sofern der gewählte Stil dies unterstützt.
Weitere Erläuterungen finden sich jeweils auch im Kapitel~\ref{sec:titlestyles}.


\begin{Declaration}
  \XMacro{begin}\PParameter{\Environment{titlepage}}\\
  \quad\dots\\
  \XMacro{end}\PParameter{titlepage}
\end{Declaration}

Mit der \Environment{titlepage}-Umgebung können, wie von den Standardklassen
gewohnt, Titelseiten definiert werden. Sie bietet aber in \tubslatex noch einige
zusätzliche Darstellungsmöglichkeiten. \Environment{titlepage} entspricht
im Prinzip der Umgebung \Environment{gausspage}, die allgemein zum Erstellen
von Seiten im \gls{glos:gaussraster} konzipiert ist. Daher finden sich
weiterführende Information zu Definitionsmölighckeiten auch in
Abschnitt~\ref{sec:gausspage}.

\begin{Declaration}
  \Macro{showtubslogo}\OParameter{Seite}\\
  \Macro{showlogo}\Parameter{Inhalt}\\
  \Macro{showtopline}
\end{Declaration}

Die Verwendung einer dieser Befehle innerhalb einer Titlepage-Umgebung sorgt
dafür, dass das entsprechende Element auf der Seite dargestellt wird.
\Macro{showtubslogo} stellt das Siegelbandlogo dar. Der optionale Parameter
\PName{Seite} akzeptiert legt dabei fest auf welcher Seite des Blattes
das Logo gezeigt werden soll. Der Wert \PValue{left} entspricht der Standardeinstellung und stellt das Logo auf der linken Seite,
% bzw. bei zweiseitigem Layout auf der Innenseite % NOTE: für Titelseite irrelevant!?
dar.
Mit dem Wert \PValue{right} wird dagegen eine Darstellung auf der rechten
% bzw. äußeren
Seite des Blattes erwirkt. Gleichzeitig hat dies Auswirkungen
auf die Positionierung des Zweitlogos, das automatisch immer auf der 
dem Siegelbandlogo gegenüberliegenden Seite platziert wird.
% TODO: inner, outer?

\begin{Declaration}
  \Macro{showdesignhelper}
\end{Declaration}

% TODO...

\begin{Declaration}
  \XMacro{begin}\PParameter{\Environment{titlerow}}%
    \OParameter{options}%
    \Parameter{gaussheight}\\
  \quad\dots\\
  \XMacro{end}\PParameter{titlerow}
\end{Declaration}

Die Umgebung \Environment{titlerow} erlaubt es dabei, die Titelelemente im 
Gaußraster anzulegen. Der Parameter \PName{gaussheight} gibt dabei
die Höhe des jeweiligen Elements in Segmenten an. Die Position der Elemente
ergibt sich aus der Reihenfolge der Definition.
Mit dem optionalen Parameter \PName{options} können Einstellung wie die Hintergrundfarbe oder ein Hintergrundbild übergeben werden.

\subsection{Rückseiten}

Neben der Möglichkeit Titelseiten zu definieren, gibt es in \tubslatex auch
einen ähnlichen Mechanismus zur Erstellung von Rückseiten derselbigen, sodass
ohne Mühe ein komplettes Titelblatt, dass auf die Möglichkeiten des
Corporate Design abgestimmt ist, erstellt werden kann.

\begin{Declaration}
  \Macro{makebackpage}\OParameter{Style}
\end{Declaration}

Dieser Befehl stellst das Gegenstück zu \Macro{maketitle} auf und erstellt eine
Rückseite. Genau wie bei \Macro{maketitle} kann über den Optionalen Parameter
\PName{style} wieder ein vordefinierter Stil geladen werden.

Außerdem stehen ein paare Befehle zur Verfügung, mit denen sich der Inhalt
der Rückseite festlegen lässt. Die konkrete Darstellung ist dabei wieder
von dem verwendeten Style abhängig. Weiteres findet sich in 
Kapitel~\ref{sec:backstyles}.


\begin{Declaration}
  \Macro{address}\Parameter{Adressdaten}\\
  \Macro{backpageinfo}\Parameter{Inhalt}
\end{Declaration}

Mit \Macro{address} kann eine Herausgeber-Adresse für das Dokument
dargestellt werden.
mit \Macro{backpageinfo} Lässt sich ganz allgemein der verbleibende
Inhalt der Rückseite nach Belieben füllen.
Siehe hierzu jeweils auch Kapitel~\ref{sec:backstyles}.

\subsection{Vordefinierte Titel-Styles}\label{sec:titlestyles}

%TODO: Hinweis Schriftgrößen, KOMA-Fonts

Es sind 3 einfache Styles definiert.% vordefiniert
% TODO: styles selbst definieren.
% Hinweis zu Farbdarstellung!
% TODO: Default-Style in Config-Datei speichern

\begin{center}
  \fboxsep0mm
  \begin{minipage}[t]{0.33\textwidth}
    \centering\sffamily
    \fbox{%
      \includegraphics[width=0.95\textwidth,page=1]{examples/titlestyles.pdf}}
    [default]
  \end{minipage}%
  \begin{minipage}[t]{0.33\textwidth}
    \centering\sffamily
    \fbox{%
      \includegraphics[width=0.95\textwidth,page=2]{examples/titlestyles.pdf}}
    [image]
  \end{minipage}%
  \begin{minipage}[t]{0.33\textwidth}
    \centering\sffamily
    \fbox{%
      \includegraphics[width=0.95\textwidth,page=3]{examples/titlestyles.pdf}}
    [imagetext]
  \end{minipage}
\end{center}

\subsection{Vordefinierte Rückseiten-Styles}\label{sec:backstyles}

Es sind 3 einfache Styles definiert.% vordefiniert

\begin{center}
  \fboxsep0mm
  \begin{minipage}[t]{0.33\textwidth}
    \centering\sffamily
    \fbox{%
      \includegraphics[width=0.95\textwidth,page=1]{examples/backstyles.pdf}}
    [plain] (default)
  \end{minipage}%
  \begin{minipage}[t]{0.33\textwidth}
    \centering\sffamily
    \fbox{%
      \includegraphics[width=0.95\textwidth,page=2]{examples/backstyles.pdf}}
    [info]
  \end{minipage}%
  \begin{minipage}[t]{0.33\textwidth}
    \centering\sffamily
    \fbox{%
      \includegraphics[width=0.95\textwidth,page=3]{examples/backstyles.pdf}}
    [addressinfo]
  \end{minipage}
  \begin{minipage}[t]{0.33\textwidth}
    \centering\sffamily
    \fbox{%
      \includegraphics[width=0.95\textwidth,page=4]{examples/backstyles.pdf}}
    [trisec]
  \end{minipage}
\end{center}

\section{Kopf-/ Fußzeile}\label{sec:headline}
\marginpar[Neu in 0.3-alpha3]{}

Bei Dokumenten mit Absenderbereich am oberen Rand wird die Fußzeile
standardmäßig komplett leer gelassen und die Kopfzeile
wird mit Seitennummer außen und Kapitelname auf ungeraden Seiten außen gesetzt.
Zusätzlich wird noch ein schmaler Strich am oberen äußeren Ende des
Absenderbereiches gesetzt.
Bei Dokumenten mit Absenderbereich am unteren Rand wird dagegen die Kopfzeile
komplett leer gelassen und die Fußzeile entsprechend der eben aufgeführten
Vorgaben gesetzt. Der kleine Trensstrich befindet sich dabei am unteren
äußeren Rand.

Die Darstellung der Kopf- und Fußzeilen lässt sich jedoch mit einigen
speziellen Befehlen individuell anpassen.

\begin{Declaration}
  \Macro{outersender}\OParameter{links}\Parameter{recht}\\
  \Macro{innersender}\OParameter{rechts}\Parameter{links}\\
  \Macro{outerhead}\OParameter{links}\Parameter{rechts}\\
  \Macro{innerhead}\OParameter{rechts}\Parameter{links}\\
  \Macro{outerfoot}\OParameter{links}\Parameter{rechts}\\
  \Macro{innerfoot}\OParameter{rechts}\Parameter{links}
\end{Declaration}

Mit dem Befehl \Macro{outersender}\OParameter{links}\Parameter{recht}
wird der Inhalt des äußeren Absenderbereichs festgelegt.
Die Parameter \PName{rechts} stellt dabei den darzustellenden Inhalt dar.
Mit Hilfe des optionalen Parameters \PName{links} können unterschiedliche
Inhalte gesetzt werden, abhängig davong, ob (bei zweiseitigem Layout)
der äußere Rand auf der jeweiligen Seite der linke oder rechte Rand ist.
Abhängig davon, ob der Absenderbereicht am oberen oder unteren Ende des Blattes
dargestellt ist wird der Inhalt entweder in der Kopf- oder in der Fußzeile
dargestellt.

Mit den Befehlen \Macro{outerhead} und \Macro{innerhead}, die ansonsten genause
wie die beiden zuvor erläuterten Befehle funktionieren, kann man dagegen
Inhalte festlegen, die explizit nur als Kopfzeile gesetzt werden, also
im Falle eines am unteren Rand platzierten Absenderbereiches nicht dargestellt 
werden.

Analog dazu gibt es noch die zwei Befehle \Macro{outerfoot}
und \Macro{innerfoot}, die selbiges Verhalten für Fußzeilen aufweisen.
Der so festgelegte Inhalt wird dann im Falle eines am oberen Blattrand
befindlichen Absenderbereichs nicht dargestellt.

\begin{Declaration}
  \Macro{headtopline}
\end{Declaration}


Die standardmäßig gesetzte kurze Linien am oberen äußeren Ende des
Kopfbereiches kann mit diesem aus dem Koma-Skript stammenden Befehl verändert
werden. Näheres ist in der zugehörigen Dokumenation zu
entnehmen\cite{koma-skript}.


\section{Seitenlayout}\label{sec:pagelayout}

Eine Seite im Corporate Design unterteilt sich grundlegend in 2 Bereiche.
Der Absenderbereich und der Kommunikationsbereich. Der Absenderbereich wird zur
Darstellung des TU-Siegelbandlogos und des Logos eines speziellen Instituts
oder einer Abteilung verwendet. Im Textteil dient er teilweise als Bereich für
Inhalte einer Kopf-/Fußzeile.

Der Absenderbereich kann entweder am Anfang oder am Ende der Seite platziert
werden. Dies beeinflusst auch das Gaußraster, es beginnt jeweils mit dem
größten Segment am Absenderbereich.

\begin{Declaration}
  \Option{a0paper}\\
  \Option{a1paper}\\
  \Option{a2paper}\\
  \Option{a3paper}\\
  \Option{a4paper}\\
  \Option{a5paper}\\
  \Option{a6paper}
\end{Declaration}

Für die Auswahl des verwendeten Papierformats stehen alle DIN A-Größen von
0 bis 6 zur Verfügung. Standardeinstellung ist \Option{a4paper}.

\begin{Declaration}
  \Option{landscape}
\end{Declaration}

Schaltet das Dokument in Querformat-Darstellung.

\begin{minipage}{0.45\textwidth}\sffamily\centering
  {\fboxsep0mm\fbox{\includegraphics[width=\textwidth]{examples/article_landscape.pdf}}}\\
  Dokument im Querformat
\end{minipage}


\begin{important}
  Im Querformat hat das Gaußraster eine abweichende
  Segmentanzahl. Für alle gängigen Formate sind im Querformat \emph{6 Segmente}
  definiert. Daher lösen Layouts, die für hochformatige Dokumente
  geschrieben wurden, im Querformat einen Fehler aus.
\end{important}


\begin{Declaration}
  \KOption{sender}\PName{Position}
\end{Declaration}

Mit der Option \Option{sender} kann die Position des Absenderbereiches
festgelegt werden. Mit der Einstellung \OptionValue{sender}{top} wird
der Absenderbereich am oberen Seitenende dargestellt. Dies ist auch die
Standardeinstellung. Wählt man dagegen \OptionValue{sender}{bottom}, so
wird der Absenderbereich am unteren Ende der Seite platziert

\begin{important}
  Die Positionierung des Absenderbereichs beeinflusst die Orientierung des
  Gaußrasters. Das größte Segment wird dabei immer so gesetzt, dass es direkt
  an den Absenderbereich anschließt und alle folgende von absteigender Höhe sind.
\end{important}


\begin{Declaration}
  \KOption{style}\PName{<print/screen>}
\end{Declaration}

Die Option \Option{style} bietet die automatische Möglichkeit diverse Elemente
des Layouts so anzupassen, dass sie die Darstellung auf dem gewünschten
Zielmedium optimieren.

Mit \OptionValue{style}{print}, was standardmäßig voreingestellt ist,
wird die Darstellung für den Druck optimiert.

Mit \OptionValue{style}{screen} wird die Darstellung
für die Ausgabe auf Bildschirmen optimiert.

\begin{center}
\begin{tabular}{lll}
                      & \textbf{\ttfamily print} & \textbf{\ttfamily screen}  \\
  \midrule
Bindekorrektur        & 15mm  & 0mm     \\
Zweisitige Dartellung & Ja    & Nein    \\
Farbdarstellung       & CMYK  & RGB
\end{tabular}
\end{center}

Die einzelnen Einstellungen können natürlich auch einzeln angepasst oder
nachträglich geändert werden. Es entscheidet die Reihenfolge der Optionen.
Die zuletzt gesetzte Option hat dabei die höchste Priorität.

\begin{Declaration}
  \KOption{bcor}\PName{Wert}
\end{Declaration}

Die Bindekorrektur beschreibt einen zusätzlichen Abstand des eigentlichen
Darstellungsbereichs vom inneren Formatrand. Die Bindekorrektur
ist bei Textdokumenten standardmäßig auf Rahmenbreite voreingestellt.

Sinnvoll ist eine Bindekorrektur selbstverständlich zum einen für Bindungen,
wo sie der Breite des durch die Bindung verdeckten Bereiches entsprechen sollte.
Zum anderen ist sie aber auch für den Druckvorgang sinnvoll, da normale Drucker
keinen randlosen Druck ermöglichen. Die Bindekorrektur verhindert so \zB ein 'Abschneiden' des Siegellogos.
Soll das Dokument ohne Bindekorrektur dargestellt werden, so ist dies
mit \OptionValue{bcor}{0mm} möglich.

\begin{Declaration}
  \KOption{twoside}\PName{bool}
\end{Declaration}

Für das Setzen von zweiseitige Dokumente ist die Option \Option{twoside}
vorgesehen. Mit ihr wird von einseitigem Layout auf zweiseitiges Layout
umgeschaltet, was bedeutet, dass die Innenseite eines Dokuments abwechselnd
auf der linken und rechten Seite definiert ist. Die hat unter anderem Einfluss
auf definierte Ränder (Bindungskorrektur, Marginale).

Die Angabe eines Wertes ist Optional. \OptionValue{twoside}{true} entspricht
dabei der einfachen Benutzung von \Option{twoside}.
\OptionValue{twoside}{false} sorgt dagegen für die Deaktivierung der Option.

\begin{Declaration}
  \Option{marginleft}\\
  \Option{marginright}
\end{Declaration}

Das setzen einer Marginale wird durch die Befehle \PName{marginleft} und
\PName{marginright} vereinfacht. Diese setzen jeweils auf der linken (inneren)
bzw. rechten (äußeren) Seite des Dokumentes eine Marginale, deren Breite
einem Element im Spaltenraster entspricht.

\begin{Declaration}
  \Option{extramargin}
\end{Declaration}

Die Vorgaben des CD definieren einen recht breiten Textbereich, der im Prinzip
für den zweispaltigen Textsatz vorgesehen ist. Dieser ist für einspaltige
Darstellung, welche häufig Verwendung findet jedoch deutlich zu breit.
Die Option \Option{extramargin} definiert dafür einen Textbereich, der um
doppelte Rahmenstärke schmaler ist als der herkömmliche Textbereich.
\begin{important}
Dies hat keine Auswirkung auf Darstellungen, die die Umgebung
\Environment{gausspage} benutzen.
\end{important}


\section{Schrift}

%TODO: Wahl Nexus/Arial

Dokumente im Corporate Design werden allgemein in der Schrift \emph{Nexus}
gesetzt. Diese wird standardmäßig bei allen Dokumenten in der Schriftgröße
11pt geladen. Außerdem definiert sind Schriftgrößen von 10pt und 9pt.
Auf die Verwendung einer Schriftgröße außerhalb des Definitionsbereiches sollte
nach Möglichkeit verzichtet werden, da dies zu starken Abweichungen in der
Größendarstellung und somit im Gesamtlayout führt.


Für wichtige Standard-Elemente gibt es des Weiteren vordefinierte Koma-Fonts.
Diese können bei Bedarf mit dem Befehl \Macro{setkomafont}\Parameter{Parameter} 
geändert werden, wovon aber abgeraten wird, da sie entsprechend des CD korrekt
definiert sind. Teilweise sind Namen auch doppelt definiert, um sowohl
CD- als auch \LaTeX-Konventionen zu entsprechen.

Die folgende Liste bietet eine Übersicht über alle verfügbaren bzw.
definierten KOMA-Font-Elemente:

\begin{desctable}
\entry{\PValue{headline}\\
  \PValue{title}}{%
  Einfache Überschrift (groß) (Titelseite)
}
\entry{\PValue{headlinesmall}\\
  \PValue{title}}{%
  Einfache Überschrift (klein)
}
\entry{\PValue{subheadline}\\
  \PValue{subtitle}}{%
  Unterüberschrift (Titelseite)
}
\entry{\PValue{institute}}{%
  Institutsname im Logo-Bereich des Absenders.
}
\entry{\PValue{author}}{%
  Autorenname auf der Titelseite
}
\entry{\PValue{date}}{%
  Datum auf der Titelseite
}
% \entry{\PValue{infotext}}{%
% }
% \entry{\PValue{infotextitem}}{%
% }
\entry{\PValue{copytext}}{%
  Mengentext
}
\end{desctable}


\paragraph{Aufzählungszeichen}

Aufzählungszeichen für die \Environment{itemize}-Umgebung werden in \tubslatex
als kleine Kästchen dargestellt.
Diese sind für die ersten zwei Ebenen einer Aufzählung definiert, wobei
die zweite Ebene kleinere Kästchen in der Darstellung aufweißt:
\begin{center}
\fbox{\parbox{0.5\textwidth}{\begin{itemize}
  \item Item 1
  \item Item 2
  \begin{itemize}
    \item Sub-Item 1
    \item Sub-Item 2
  \end{itemize}
  \item Item 3
\end{itemize}}}
\end{center}



\section{Farben}

Diese Kapitel gibt nur einen kurzen Einblick über die wichtigsten Punkte
zum Thema Farben. Für eine detaillierte Übersicht über die verfügbaren Farben
und weitere Möglichkeiten siehe Kapitel~\ref{sec:colors}.%TODO...

Die Dokumentenklassen in \tubslatex unterstüzen im Prinzip 2 Farbmodelle:
Das für Bildschirmdarstellungen verwendeten RGB-Modell und das im
Druck gebräuchliche CMYK-Modell. Zwischen diesen beiden kann in den Optionen
der Dokumentenklasse gewählt werden. Darüber hinaus gibt es noch die Unterstützung
für die Darstellung von Schwarz-Weiß-Dokumenten. Dabei werden automatisch
alle Elemente, die sonst in der roten Primärfarbe des CD dargestellt werden,
schwarz dargestellt.

\begin{Declaration}
  \Option{cmyk}
\end{Declaration}

Diese Klassenoption erwirkt eine CMYK-Darstellung des Siegelbandlogos und der
CD-Standardfarben. Sie ist für den Druck zu bevorzugen.

\begin{important}
  Bei der Verwendung von \Option{cmyk} ist zu beachten,
  dass das die Farbe Schwarz (\texttt{tuBlack})
  aus den Farbdefinitionen der Mischung 0\%C, 0\%M, 0\%Y, 100\%K entspricht,
  was in der Bildschirmdarstellung allgemein als dunkles grau wirkt. Dies
  gilt somit auch für den Text im Dokument und ist normal.

  Sollte dagegen die Ausgabe aus dem Drucker zu hell erscheinen, so kann dies
  durch einen manuelle Anpassung der Farbe \texttt{tuBlack} korrigiert werden,
  indem zusätzlich die CMY-Farben benutzt werden. Dies kann jedoch auch zu
  schlechteren Druckergebnissen durch übermäßigen Tinteneinsatz führen!
  
  Beispiel zur Umdefinierung:
  \lstinline!\definecolor{tuBlack}{CMYK}{0.5,0.5,0.5,1.0}!
\end{important}

\begin{Declaration}
  \Option{rgb}
\end{Declaration}

Diese Klassenoption erwirkt eine RGB-Darstellung des Siegelbandlogos und der
CD-Standardfarben. Sie ist für Bildschirmdarstellungen zu bevorzugen.

\begin{Declaration}
  \Option{mono}
\end{Declaration}

Diese Klassenoption stellt sowhl das Siegelbandlogo als auch alle in der
roten Primärfarbe (\texttt{tuRed}) des CD definierten Elemente schwarz dar.
Sie entscheidet dabei \emph{nicht} über das verwendete Farbmodell (RGB/CMYK).
\begin{important}
  Bei der Verwendung von \Option{mono} in Verbindung mit \Option{cmyk}
  ist zu beachten, dass das die Farbe Schwarz (\texttt{tuBlack})
  aus den Farbdefinitionen (wie oben beschrieben) nicht dem 'wirklichen' Schwarz
  des Siegelbandlogos entspricht.
  Im Zweifelsfall ist also eine Verwendung des RGB-Farbmodells in Verbindung
  mit \Option{mono} oder eine Neudefinition von \texttt{tuBlack} vorzuziehen.
\end{important}

\subsection{Farbschema}

Das Farbschema des CD gliedert sich grob in zwei Kategorieren: Primärfarben
und Sekundärfarben. Während zu den Primärfarben nur das TU-Rot (\texttt{tuRed})
und Schwarz, Weiß, sowie weitere Grautöne gehören, bieten die Sekundärfarben
eine größere Auswahl. Sie sind nochmals in 4 verschiedene Farbklänge
(orange, grün, blau, violett) unterteilt, die jeweils 3 Farben aufeinander
abgesetimmte Farben enthalten, die wiederum noch 20-Prozent-Schritten aufgehellt
werden können.

\begin{Declaration}
  \Option{orange}\\
  \Option{blue}\\
  \Option{green}\\
  \Option{violet}
\end{Declaration}

Da Oftmals nur mit einem Hauptfarbklang gearbeitet wird, gibt es die Möglichkeit
diesen in den Klassenoptionen zu wählen. Die Farben sind dann unter den
Aliasnamen
\texttt{tuSecondary\textit{<Light//Medium/Dark>}\textit{<100//80/60/40/20}}
verwendbar.
So entspricht \texttt{tuSecondaryDark80} in Verbindung mit Option
\Option{blue} \texttt{tuBlueDark80}, in Verbindung mit Option \Option{green}
jedoch \texttt{tuGreenDark80}.

Viele der standardmäßig definierte Styles und Vorlagen nutzen ebenfalls diese
Alias-Farben, sodass deren darstellung durch die Optionen ebenfalls beeinflusst 
werden kann.

\begin{Declaration}
  \Macro{colorshow}\OParameter{Breite}\Parameter{Farbe}\Parameter{Helligkeit}
\end{Declaration}

Hilfreicher Befehl für den Design-Prozess. Er zeigt alle Abstufungen
einer \PName{Farbe} \PValue{Orange/Blue/Green/Violet} mit gewählter 
\PName{Helligkeit} \PValue{Light/Medium/Dark}.
Optional kann noch die \PName{Breite} der Darstellung gewählt werden,
ist kein Wert angegeben, so wird \Macro{textwidth} verwendet.

\begin{figure}
  tuOrange\ldots\\
  \colorshow{Orange}{Light}
  \colorshow{Orange}{Medium}
  \colorshow{Orange}{Dark}\\[-1ex]
  tuBlue\ldots\\
  \colorshow{Blue}{Light}
  \colorshow{Blue}{Medium}
  \colorshow{Blue}{Dark}\\[-1ex]
  tuGreen\ldots\\
  \colorshow{Green}{Light}
  \colorshow{Green}{Medium}
  \colorshow{Green}{Dark}\\[-1ex]
  tuViolet\ldots\\
  \colorshow{Violet}{Light}
  \colorshow{Violet}{Medium}
  \colorshow{Violet}{Dark}
  \caption{Im CD definierte Farben und deren Benennung (Auszug)}
\end{figure}


\subsection{Hinweis zur Verwendung}

Für Typografie auf farbigen Flächen gilt: Bis zu einer Abstufung von 60\% kann die Schrift in Weiß auf diesem Farbton stehen. Ab einem Tonwert von 40\% sollte die Schrift, im Sinne einer optimalen Lesbarkeit, in Schwarz dargestellt werden. 

Weitere Hinweise zur allgemeinen Verwendung sind bitte der
CD-Toolbox\cite{toolbox} zu entnehmen.


\section{Arbeiten mit \tubslatex}

\subsection{Text schreiben}



% TODO: margins, textwidth-Problem, etc.


