\chapter{Dokumente}

Für das Erstellen von Dokumenten im \acs{CD} gibt es diverse Richtlinien
und Gestaltungsmöglichkeiten, die unter Einsatz von Standard-Klassen nur sehr
mühselig einzuhalten und zu verwirklichen sind.
Daher stellt \tubslatex drei eigene Dokumentenklassen bereit, die als Ersatz für
die Basisklassen \lstinline{article}, \lstinline{report} und \lstinline{book} 
verwendet werden können. Sie heißen \newdocumentclass{tubsartcl},
\newdocumentclass{tubsreprt} und \newdocumentclass{tubsbook}.

Einige Funktionalitäten, welche diese Klassen bieten:
\begin{itemize}
  \item Anpassung der Seitengeometrie
  \item Angepasste Schriftart- und größe für 9pt, 10pt und 11pt
  \item Vorlagenoptimierte Darstellung von Kopf- bzw. Fußzeilen
  \item Vorlagen für Titelseiten im \gls{glos:gaussraster}
  \item Umgebungen zur individuellen Erstellung von (Titel-)Seiten im Gaußraster.
  \item Vordefinierte Farben aus der \acs{CD}-Farbpalette.
  \item Einfache Darstellungsstile für Bildschirmdarstellung und Druck
\end{itemize}

\begin{hint}
  Um die Definitionen von \tubslatex mit anderen Klassen nutzen zu können,
  steht das Paket \newpackage{tubsdoc} zu Verfügung.
  Dies stellt viele der hier beschriebenen Funktionen zur Verfügung, ändert
  aber das Seitenlayout nicht.% TODO: Iwo genauer beschreiben
\end{hint}


\paragraph{Seitenlayout}
\Index{Seitenlayout}

Der im \acs{CD} festgelegte \gls{glos:absenderbereich} findet sich sowohl auf
Titelseiten als auch auf Inhaltsseiten (als Kopf- bzw. Fußzeile) wieder.
Der Absenderbereich wird normalerweise am oberen Seitenende platziert,
kann mittels der Klassenoption \OptionValue{sender}{bottom} aber auch am unteren
Seitenende platziert werden.

Da Dokumente normalerweise gedruckt und ggf. auch gebunden werden,
wird standardmäßig eine kleine Bindekorrektur gesetzt, sodass das
Logo beim Drucken oder Abheften nicht abgeschnitten wird.
Außerdem wird das Dokument als zweiseitig und im \gls{glos:cmyk} gesetzt.
Für eine Optimierung zur Ausgabe auf einem Bildschirm steht die Option
\OptionValue{style}{screen} zu Verfügung. Weitere Optionen dienen der
genauen individuellen Anpassung der Darstellung. Eine genauere Darstellung
liefert \mbox{u.\,a.}\xspace das Kapitel~\ref{sec:pagelayout}.

\paragraph{Titelseiten}
\Index{Titelseiten}

Für die Darstellung von Titelseiten stellt \tubslatex verschiedene Möglichkeiten
zur Verfügung, die an die konventionelle Erstellung von Titelseiten angelehnt
sind. So werden sowohl sowohl der Befehl \Macro{maketitle} als auch die Umgebung
\Environment{titlepage} so umdefiniert, dass man mit ihnen einfach CD-konforme
Titelseiten aus Vorlagen auswählen bzw. individuell erstellen kann.
Darüber hinaus steht auch die Möglichkeit zur Verfügung mittels
\Macro{makebackpage} oder der Umgebung \Environment{backpage}
Rückseiten zu erstellen. Kapitel~\ref{sec:titlepage} geht darauf näher ein.
\bigskip

% \paragraph{Gaußraster}
% Die Darstellung von Seiten basiert im CD-Layout im Prinzip immer auf den
% sogenannten Gaußraster. Dies unterteilt die Seite in verschieden große
% Segmente, deren Höhe durch die gaußsche Summenformel berechnet werden kann.
% Außerdem könne Seiten ein Reihe von Standardelementen wie das Siegellogo oder
% ein zusätzliches individuelles Logo aufweisen.
% Alle Seiten, die im Stil des \gls{glos:gaussraster} erzeugt werden basieren letztendlich
% auf dem Selben backend. Dies wird in Kapitel~\ref{sec:gausspage} ausführlicher
% beschrieben.\bigskip

\paragraph{Schrift, Kopf-/Fußzeilen}

Darüber hinaus ist die Schrift Nexus, sowie alle Farben des Corporate Designs
in allen Dokumenten vordefiniert. Die Alternativschrift Arial kann durch
Angabe der Klassenoption \Option{arial} gehält werden.
Befehle für die Erstellung von Kopf- bzw. Fußzeilen sind CD-konform angepasst und
werden im Kapitel~\ref{sec:headline} beschrieben.


\section{Seitenlayout}\label{sec:pagelayout}
\Index{Seitenlayout}
% TODO: notieren, dass auch Standard-Koma-Optionen funktionieren
% TODO: Tabelle mit Schriftgrößen? (-> Anhang oder so)
% TODO: Allgemein Zusammenhang Schriftgrößen -> Papiergröße erwähnen?

Eine Seite im Corporate Design unterteilt sich grundlegend in 2 Bereiche.
Der Absenderbereich und der Kommunikationsbereich. Der Absenderbereich wird zur
Darstellung des TU-Siegelband-Logos und des Logos eines speziellen Instituts
oder einer Abteilung verwendet. Im Textteil dient er teilweise als Bereich für
Inhalte einer Kopf-/Fußzeile.

Der Absenderbereich kann entweder am Anfang oder am Ende der Seite platziert
werden. Dies beeinflusst auch das Gaußraster, es beginnt jeweils mit dem
größten Segment am Absenderbereich.
\bigskip

Im Folgenden werden alle \tubslatex-Klassenoptionen beschrieben,
die das Layout beeinflussen. Die Reihenfolge der Beschreibung richtet sich dabei
nach einer möglichst sinnvollen Auswahlreihenfolge bei der Gestaltung des Gesamtlayouts.

\paragraph{Papierformat}\hfill

Das Corporate Design definiert allgemein feste Zusammenhänge zwischen dem
verwendeten Papierformat und den zu verwendeten Basis-Schriftgrößen.
In \tubslatex sind diese Werte größtenteils $1:1$ übernommen, jedoch
werden für die gängigsten Dokumentenformate hier noch weitere Abstufungen 
bereit gestellt. Details zu Schriftgrößen gibt das Kapitel~\ref{subsec:documents:fonts}

\begin{Declaration}
  \Option{a0paper}\\
  \Option{a1paper}\\
  \Option{a2paper}\\
  \Option{a3paper}\\
  \Option{a4paper}\\
  \Option{a5paper}\\
  \Option{a6paper}
\end{Declaration}
\OptionIndex{a0paper}
\OptionIndex{a1paper}
\OptionIndex{a2paper}
\OptionIndex{a3paper}
\OptionIndex{a4paper}
\OptionIndex{a5paper}
\OptionIndex{a6paper}
\Index{Papierformat!Option}

Für die Auswahl des verwendeten Papierformats stehen alle DIN\,A-Größen von
0 bis 6 zur Verfügung. Standardeinstellung ist \Option{a4paper}.
Die Standard-Schriftgröße wird dabei jeweils automatisch mit umgestellt.
Die verfügbaren \acs{CD}-konformen Schriftgrößen und die jeweils eingestellten
Standard-Schriftgrößen können der Tabelle~\ref{table:fontsizes}
entnommen werden.

\begin{Declaration}
  \Option{landscape}
\end{Declaration}
\OptionIndex{landscape}
\Index{Querformat!Option}

Die Option \Option{landscape} schaltet das Dokument in Querformat-Darstellung.

\begin{center}
  \fboxsep0mm\fbox{\includegraphics[width=0.45\textwidth]{examples/article_landscape.pdf}}\\
  Dokument im Querformat
\end{center}

\begin{important}
  Im Querformat hat das Gaußraster eine abweichende
  Segmentanzahl. Für alle gängigen Formate sind im Querformat \emph{6 Segmente}
  definiert. Daher lösen Layouts, die für hochformatige Dokumente
  geschrieben wurden, im Querformat einen Fehler aus. %TODO: Fehlermeldung benennen bzw. Referenz
\end{important}

\paragraph{Inhaltsdarstellung}\hfill

\begin{Declaration}
  \KOption{sender}\PName{Position}
\end{Declaration}
\OptionIndex{sender}
\Index{Absenderbereich!Option}

Mit der Option \Option{sender} kann die Position des Absenderbereichs
festgelegt werden. Mit der Einstellung \OptionValue{sender}{top} wird
der Absenderbereich am oberen Seitenende dargestellt. Dies ist auch die
Standardeinstellung. Wählt man dagegen \OptionValue{sender}{bottom}, so
wird der Absenderbereich am unteren Ende der Seite platziert

\begin{important}
  Die Positionierung des Absenderbereichs beeinflusst die Orientierung des
  Gaußrasters. Das größte Segment wird dabei immer so gesetzt, dass es direkt
  an den Absenderbereich anschließt und alle Folgenden von absteigender Höhe sind.
\end{important}


\begin{Declaration}
  \KOption{style}\PName{<print/screen/printdev>}
\end{Declaration}
\OptionIndex{style}

Die Option \Option{style} bietet die Möglichkeit, diverse Elemente
des Layouts so anzupassen, dass sie für die Darstellung auf dem gewünschten
Zielmedium optimiert sind.

Mit \OptionValue{style}{print}, was standardmäßig voreingestellt ist,
wird die Darstellung für den Druck optimiert.
Mit \OptionValue{style}{screen} wird die Darstellung
für die Ausgabe auf Bildschirmen optimiert.

Die Option \OptionValue{style}{printdev} entspricht in der Darstellung
der Option \PValue{print}, setzt das Dokument aber nur einseitig.
Damit lassen sich zu druckende Dokumente am Bildschirm leichter entwickeln.
Bei Wechsel zu \OptionValue{style}{print} ändern sich dann keine Umbrüche etc.
mehr, da der Darstellungsbereich gleich groß bleibt.

Eine Übersicht über die Einstellungen liefert folgende Tabelle:

\begin{center}
\begin{tabular}{llll}
  & \textbf{\ttfamily print} & \textbf{\ttfamily screen} & \textbf{\ttfamily printdev} \\
  \midrule
Bindekorrektur          & 15mm  & 0mm   & 15mm\\
Zweiseitige Darstellung & Ja    & Nein  & Nein\\
Farbmodell              & CMYK  & RGB   & CMYK
\end{tabular}
\end{center}

Die einzelnen Einstellungen können natürlich auch einzeln angepasst oder
nachträglich geändert werden. Es entscheidet die Reihenfolge der Optionen.
Die zuletzt gesetzte Option hat dabei die höchste Priorität.

\begin{Declaration}
  \KOption{twoside}\PName{bool}\\
  \Option{oneside}
\end{Declaration}
\OptionIndex{twoside}
\OptionIndex{oneside}
\Index{zweiseitig}
\Index{einseitig}

\Index{Dokumente!einseitig}
Für das Setzen von zweiseitigen Dokumente ist die Option \Option{twoside}
vorgesehen. Mit ihr wird von einseitigem Layout auf zweiseitiges Layout
umgeschaltet, was bedeutet, dass die Innenseite eines Dokuments abwechselnd
auf der linken und rechten Seite definiert ist. Dies hat unter anderem Einfluss
auf definierte Ränder (Bindekorrektur, Marginale) und den Inhalt der
Kopf-/Fußzeilen.
Die Angabe eines Wertes ist Optional. \OptionValue{twoside}{true} entspricht
dabei der einfachen Benutzung von \Option{twoside}.

\Index{Dokumente!zweiseitig}
Durch Verwendung von \OptionValue{twoside}{false} oder alternativ
von \Option{oneside} werden Dokumente einseitig gesetzt.

\begin{Declaration}
  \KOption{bcor}\PName{Wert}
\end{Declaration}
\OptionIndex{bcor}
\Index{Bindekorrektur}

Die Bindekorrektur beschreibt einen zusätzlichen Abstand des eigentlichen
Darstellungsbereichs vom inneren Formatrand.
Sie ist bei Textdokumenten standardmäßig auf Rahmenbreite voreingestellt.

Sinnvoll ist eine Bindekorrektur selbstverständlich zum einen für Bindungen,
wo sie der Breite des durch die Bindung verdeckten Bereiches entsprechen sollte.
Zum anderen ist sie aber auch für den Druckvorgang sinnvoll, da normale Drucker
keinen randlosen Druck ermöglichen.
Die Bindekorrektur verhindert so \zB ein 'Abschneiden' des Siegellogos.
Soll das Dokument ohne Bindekorrektur dargestellt werden, so ist dies
mit \OptionValue{bcor}{0mm} möglich.


\begin{Declaration}
  \Option{marginleft}\\
  \Option{marginright}
\end{Declaration}
\OptionIndex{marginleft}
\OptionIndex{marginright}
\Index{Marginale}

Das setzen einer Marginale wird durch die Befehle \PName{marginleft} und
\PName{marginright} vereinfacht. Diese setzen jeweils auf der linken (inneren)
bzw. rechten (äußeren) Seite des Dokumentes eine Marginale, deren Breite
einem Element im sechsgeteilten \gls{glos:spaltenraster} entspricht.

\begin{Declaration}
  \Option{extramargin}
\end{Declaration}% TODO: Namenskonflikt mit marginleft, ..?
\OptionIndex{extramargin}

% TODO: noch woanders Präferenz für zweispaltiges Layout erwähnen?
Die Vorgaben des CD definieren einen recht breiten Textbereich, der im Prinzip
für den zweispaltigen Textsatz vorgesehen ist. Jedoch ist dieser für die häufig
verwendete einspaltige darstellung deutlich zu breit.
Die Option \Option{extramargin} definiert dafür einen Textbereich, der um
doppelte Rahmenstärke schmaler und um einfache Rahmenstärke flacher
ist als der herkömmliche Textbereich.
\begin{important}
Dies hat keine Auswirkung auf Darstellungen, die die Umgebung
\Environment{gausspage} benutzen.
\end{important}
% TODO: Notiz, Spaltenraster, \twocolumn, etc.
% TODO: Rahmenstärke?

\section{Titelseite}\label{sec:titlepage}
\Index{Titelseite}

Titelseiten können bei \LaTeX\ generell auf zwei verschiedene Arten erstellt
werden; entweder mit Hilfe des Befehls \Macro{maketitle} oder mit
der Umgebung \Environment{titlepage}. Beide Varianten werden von \tubslatex
in modifizierter Weise unterstützt.

Darüber hinaus können auch Rückseiten automatisch bzw. manuell eingefügt
werden. Das Kapitel~\ref{subsec:backpages} geht hierauf näher ein.

\begin{Declaration}
  \Macro{maketitle}\OParameter{style}
\end{Declaration}
\CommandIndex{maketitle}

Die einfache Verwendung von \Macro{maketitle} erzeugt eine Titelseite
mit dem TU-\gls{glos:siegelbandlogo} und einer roten Trennlinie zwischen Absender- und
Kommunikationsbereich.

Mit Hilfe des optionalen Arguments \PName{style} kann die Darstellung
der Titelseite geändert werden, indem aus einer Reihe vordefinierter Styles
ausgewählt wird. Eine weiterführende Erklärung zu verfügbaren Style samt
Beispielen findet sich in Kapitel~\ref{sec:titlestyles}.\bigskip

\begin{hint}
  Aktuell sind die vordefinierten Styles nur korrekt nutzbar, wenn sich der
  Absenderbereich am oberen Blattrand befindet, was jedoch für fast alle
  Anwendungsfälle angemessen sein sollte.
\end{hint}


Neben den standardmäßig definierten Elementen für Titelseiten wie \Macro{author}
oder \Macro{title} werden in \tubslatex noch ein paar Zusätzliche definiert.

\begin{Declaration}
  \Macro{logo}\Parameter{logo}\\
  \Macro{titlepicture}\OParameter{fitting}\Parameter{file}\\
  \Macro{titleabstract}\Parameter{text}
\end{Declaration}
\CommandIndex{logo}
\CommandIndex{titlepicture}
\CommandIndex{titleabstract}

\begin{minipage}[t]{0.6\textwidth}
  \Macro{logo} dient zur Darstellung eines zusätzlichen Absender-Logos als Schrift
  oder Bild. Es wird in allen Stilen im Absenderbereich auf der dem
  TU-\gls{glos:siegelbandlogo} gegenüberliegenden Seite dargestellt.
  
  Mittels \Macro{includegraphics} eingefügte Grafiken werden automatisch
  passend auf die Höhe des Kopfbereichs (\Macro{headheight}) und
  halbe Textbereichsbreite (0.5\Macro{textwidth}) skaliert
  (Seitenverhältnis wird beibehalten).
\end{minipage}
\hfill
\begin{minipage}[t]{0.3\textwidth}
  \vspace*{-1ex}%
  \fboxsep0mm\fbox{\includegraphics[width=\textwidth,page=1]{examples/doc_layout.pdf}}
\end{minipage}

Mit \Macro{titlepicture} kann eine Bilddatei angegeben werden, die bei
Verwendung eines entsprechenden Styles auf der Titelseite dargestellt wird.
Diese wird standardmäßig so skaliert und zurechtgeschnitten, dass sie den ihr
zur Verfügung stehenden Bereich optimal ausfüllt.
Mit der Option \Option{fitting} kann die Einpassung beeinflusst werden.
Hier können die in Kapitel~\ref{subsec:gausspage:bgelement} beschriebenen Werte
verwendet werden.

Der Befehl \Macro{titleabstract} erlaubt die Darstellung eines kurzen
zusammenfassenden Textes auf der Titelseite, sofern der gewählte Style
dies unterstützt.

Weitere Erläuterungen und eine Übersicht über die vordefinierten Styles
finden sich in Kapitel~\ref{sec:titlestyles}.


\begin{Declaration}
  \XMacro{begin}\PParameter{\Environment{titlepage}}\\
  \quad\dots\\
  \XMacro{end}\PParameter{titlepage}
\end{Declaration}
\EnvironmentIndex{titlepage}

Mit der \Environment{titlepage}-Umgebung können, wie von den Standardklassen
gewohnt, Titelseiten definiert werden. Sie bietet aber in \tubslatex noch einige
zusätzliche Darstellungsmöglichkeiten. \Environment{titlepage} entspricht
im Prinzip der Umgebung \Environment{gausspage}, die allgemein zum Erstellen
von Seiten im \gls{glos:gaussraster} konzipiert ist. Daher finden sich
weiterführende Information zu Definitionsmöglichkeiten auch in
Abschnitt~\ref{chap:gausspage}.

\begin{Declaration}
  \Macro{showtubslogo}\OParameter{Seite}\\
  \Macro{showlogo}\Parameter{Inhalt}\\
  \Macro{showtopline}
\end{Declaration}
\CommandIndex{showtubslogo}
\CommandIndex{showlogo}
\CommandIndex{showtopline}

Die Verwendung einer dieser Befehle innerhalb einer \Environment{titlepage}-Umgebung
sorgt dafür, dass das entsprechende Element auf der Seite dargestellt wird.
\Macro{showtubslogo} stellt das Siegelbandlogo dar. Der optionale Parameter
\PName{Seite} legt dabei fest auf welcher Seite des Blattes
das Logo gezeigt werden soll.
Der Wert \PValue{left} entspricht der Standardeinstellung und platziert das Logo auf der linken Seite.
Mit dem Wert \PValue{right} wird dagegen eine Darstellung auf der rechten
Seite des Blattes erwirkt. Gleichzeitig hat dies Auswirkungen
auf die Positionierung des Zweitlogos, das automatisch immer auf der 
dem Siegelbandlogo gegenüberliegenden Seite platziert wird.

\begin{Declaration}
  \Macro{showdesignhelper}
\end{Declaration}
\CommandIndex{showdesignhelper}

Dieser Befehl eignet sich für die Gestaltungsphase der Titelseite.
Er stellt alle verfügbaren Gauß-Segmente, sowie die möglichen Positionen
für Siegelbandlogo und Zweitlogo dar.

\begin{Declaration}
  \XMacro{begin}\PParameter{\Environment{titlerow}}%
    \OParameter{options}%
    \Parameter{gaussheight}\\
  \quad\dots\\
  \XMacro{end}\PParameter{titlerow}
\end{Declaration}
\EnvironmentIndex{titlerow}

Die Umgebung \Environment{titlerow} erlaubt es, Titelelemente im 
Gaußraster anzulegen. Der Parameter \PName{gaussheight} gibt dabei
die Höhe des jeweiligen Elements in Segmenten an. Die Position der Elemente
ergibt sich aus der Reihenfolge der Definition.
Mit dem optionalen Parameter \PName{options} können Einstellung wie
die Hintergrundfarbe oder ein Hintergrundbild übergeben werden.
Die Umgebung basiert auf der \Environment{segment}-Umgebung.
Detailliertere Informationen sind daher deren Beschreibung
in Kapitel~\ref{subsec:gausspage:bgelement} zu entnehmen.

\subsection{Rückseiten}\label{subsec:backpages}

Neben der Möglichkeit Titelseiten zu definieren, gibt es in \tubslatex auch
einen ähnlichen Mechanismus zur Erstellung von Rückseiten.
Sodass ohne Mühe ein komplettes Titelblatt, dass auf die Möglichkeiten des
Corporate Design abgestimmt ist, erstellt werden kann.

\begin{Declaration}
  \Macro{makebackpage}\OParameter{Style}
\end{Declaration}

Dieser Befehl stellst das Gegenstück zu \Macro{maketitle} dar und erstellt eine
Rückseite. Genau wie bei \Macro{maketitle} kann über den Optionalen Parameter
\PName{style} wieder ein vordefinierter Style geladen werden.

Außerdem stehen ein paar Befehle zur Verfügung, mit denen sich der Inhalt
der Rückseite festlegen lässt. Die konkrete Darstellung ist dabei wieder
von dem verwendeten Style abhängig.
Beschreibungen und Beispiele zu den verfügbaren Styles finden sich in
Kapitel~\ref{sec:backstyles}.


\begin{Declaration}
  \Macro{address}\Parameter{Adressdaten}\\
  \Macro{backpageinfo}\Parameter{Inhalt}
\end{Declaration}
\CommandIndex{address}
\CommandIndex{backpageinfo}

Mit \Macro{address} kann eine Herausgeber-Adresse für das Dokument
dargestellt werden.
mit \Macro{backpageinfo} Lässt sich ganz allgemein der verbleibende
Inhalt der Rückseite nach Belieben füllen.
Siehe hierzu jeweils auch Kapitel~\ref{sec:backstyles}.

\begin{Declaration}
  \XMacro{begin}\PParameter{\Environment{backpage}}\\
  \quad\dots\\
  \XMacro{end}\PParameter{backpage}
\end{Declaration}
\EnvironmentIndex{backpage}

Die Umgebung \Environment{backpage} erlaubt alternativ zu
\Macro{makebackpage} das Erstellen komplett individueller
Rückseiten. Die Verwendung entspricht der \Environment{titlepage}-Umgebung
(siehe vorheriges Kapitel),
jedoch wird das Siegelband-Logo bei Verwendung standardmäßig im
\Option{plain}-Stil dargestellt, also schriftlos.

Die Einzelnen Segmente können mit der Umgebung \Environment{segment}
erstellt werden, welche in Kapitel~\ref{sec:gausspage:kombi} näher beschrieben ist.

\subsection{Vordefinierte Titel-Styles}\label{sec:titlestyles}

%TODO: Hinweis Schriftgrößen, KOMA-Fonts

Für Titelseiten gibt es 3 vordefinierte Styles, die durch Angabe des
Style-Namens als Option von \Macro{maketitle} verwendet werden können

\begin{Declaration}
  \KOption{logo}\PName{Position}
\end{Declaration}

Bei allen Styles kann die Darstellungsseite des Siegelbandlogos
(und somit auch des Zweitlogos) durch Angabe der Option
\OptionValue{logo}{Position} angepasst werden.
Mögliche Werte sind \PValue{left} bzw. \PValue{inside} für die Darstellung
auf der linken bzw. inneren Blattseite und \PValue{right} bzw. \PValue{outside}
für die Darstellung auf der rechten bzw. äußeren Blattseite.

\paragraph{Farbe}
Da als Darstellungsfarbe der \gls{glos:sekundaerfarbklang} verwendet wird,
kann die Farbgebung der Titelseite z.B. durch eine entsprechende Klassenoption
angepasst werden.
% TODO: styles selbst definieren.
% Hinweis zu Farbdarstellung!
% TODO: Default-Style in Config-Datei speichern

Die folgenden Styles sind jeweils im blauen Standardfarbklang dargestellt.

% \begin{center}

  \begin{minipage}[t]{0.33\textwidth}
    \null\centering\sffamily
    \fboxsep0mm\fbox{%
      \includegraphics[width=0.95\textwidth,page=1]{examples/titlestyle_plain.pdf}}
%     [default]
  \end{minipage}%
  \hfill
  \begin{minipage}[t]{0.6\textwidth}
    \paragraph{default}
    Einfacher Stil, der Darstellung der Standardklassen imitiert und Logos,
    sowie eine Trennline zwischen Absender- und Kommunikationsbereich hinzufügt.
    \\
    Er wird benutzt, wenn kein Style explizit gewählt wurde.
    \par\bigskip
    \Macro{maketitle}\OParameter{default}
    \par\bigskip
    Dargestellte Zusatz-Elemente:
    \begin{compactitem}\ttfamily
      \item logo
    \end{compactitem}
  \end{minipage}
  
  \begin{minipage}[t]{0.33\textwidth}
    \null\centering\sffamily
    \fboxsep0mm\fbox{%
      \includegraphics[width=0.95\textwidth,page=1]{examples/titlestyles.pdf}}
  \end{minipage}%
  \hfill
  \begin{minipage}[t]{0.6\textwidth}
    \paragraph{image}
    
    Einfacher dreigeteilter Stil im Gaußraster mit großem Titelbild.
    \par\bigskip
    \Macro{maketitle}\OParameter{image}
    \par\bigskip
    Dargestellte Zusatz-Elemente:
    \begin{compactitem}\ttfamily
      \item logo
      \item titlepicture
    \end{compactitem}
  \end{minipage}
  
  \begin{minipage}[t]{0.33\textwidth}
    \null\centering\sffamily
    \fboxsep0mm\fbox{%
      \includegraphics[width=0.95\textwidth,page=2]{examples/titlestyles.pdf}}
  \end{minipage}
  \hfill
  \begin{minipage}[t]{0.6\textwidth}
    \paragraph{imagetext}
    
    Viergeteilter Stil im Gaußraster mit Titelbild und schmalem Abstract-Bereich.
    \par\bigskip
    \Macro{maketitle}\OParameter{imagetext}
    \par\bigskip
    Dargestellte Zusatz-Elemente:
    \begin{compactitem}\ttfamily
      \item logo
      \item titlepicture
      \item titleabstract
    \end{compactitem}
  \end{minipage}

\paragraph{Gestaltungsbeispiele:}\hfill

\begin{minipage}{0.33\textwidth}\centering
  \fboxsep0mm\fbox{\includegraphics[width=0.95\textwidth,page=3]{examples/titlestyles.pdf}}
  Farbklang: \texttt{orange}\\
  \OParameter{imagetext,logo=right}
\end{minipage}
\begin{minipage}{0.33\textwidth}\centering
  \fboxsep0mm\fbox{\includegraphics[width=0.95\textwidth,page=4]{examples/titlestyles.pdf}}
  Farbklang: \texttt{violet}\\
  \OParameter{imagetext}
\end{minipage}
\begin{minipage}{0.33\textwidth}\centering
  \fboxsep0mm\fbox{\includegraphics[width=0.95\textwidth,page=5]{examples/titlestyles.pdf}}
  Farbklang: \texttt{green}\\
  \OParameter{image,logo=right}
\end{minipage}

\subsection{Vordefinierte Rückseiten-Styles}\label{sec:backstyles}

Für Rückseiten gibt es 4 vordefinierte Styles, die durch Angabe des
Style-Namens als Option von \Macro{makebackpage} verwendet werden können.
Auf Rückseiten wird das Siegelbandlogo immer als einfarbige Fläche fortgesetzt.

Die Positionierung der Logos ist abhängig von der Wahl in \Macro{maketitle}.
Die Farbgebung ergibt sich wie bei Titelseiten aus dem aktiven Sekundärfarbklang.
% TODO: wie wird Adresse dargestellt

  \begin{minipage}[t]{0.33\textwidth}
    \fboxsep0mm
    \null\centering\sffamily
    \fbox{%
      \includegraphics[width=0.95\textwidth,page=1]{examples/backstyles.pdf}}
  \end{minipage}%
  \hfill
  \begin{minipage}[t]{0.6\textwidth}
    \paragraph{plain}
    \par
    Einfache Darstellung ohne Hintergrundfarbe aber mit
    roter Trennlinie zwischen Absender- und Kommunikationsbereich.
    Lediglich Absenderinformationen werden (wenn vorhanden) dargestellt.
    \par\bigskip
    \Macro{makebackpage}\OParameter{plain}
    \par\bigskip
    Dargestellte Zusatz-Elemente:
    \begin{compactitem}\ttfamily
      \item address
    \end{compactitem}
  \end{minipage}

  \begin{minipage}[t]{0.33\textwidth}
    \fboxsep0mm
    \null\centering\sffamily
    \fbox{%
      \includegraphics[width=0.95\textwidth,page=2]{examples/backstyles.pdf}}
  \end{minipage}%
  \hfill
  \begin{minipage}[t]{0.6\textwidth}
    \paragraph{info}
    \par
    Einfarbiger Hintergrund mit Darstellung der backpageinfo.
    \par\bigskip
    \Macro{makebackpage}\OParameter{info}
    \par\bigskip
    Dargestellte Zusatz-Elemente:
    \begin{compactitem}\ttfamily
      \item backpageinfo
    \end{compactitem}
  \end{minipage}

  \begin{minipage}[t]{0.33\textwidth}
    \fboxsep0mm
    \null\centering\sffamily
    \fbox{%
      \includegraphics[width=0.95\textwidth,page=3]{examples/backstyles.pdf}}
  \end{minipage}
  \hfill
  \begin{minipage}[t]{0.6\textwidth}
    \paragraph{addressinfo}
    \par
    Kontrastreiche zweigeteilte Darstellung von Adresse und backpageinfo.
    \par\bigskip
    \Macro{makebackpage}\OParameter{addressinfo}
    \par\bigskip
    Dargestellte Zusatz-Elemente:
    \begin{compactitem}\ttfamily
      \item address
      \item backpageinfo
    \end{compactitem}
  \end{minipage}

  \begin{minipage}[t]{0.33\textwidth}
    \fboxsep0mm
    \null\centering\sffamily
    \fbox{%
      \includegraphics[width=0.95\textwidth,page=4]{examples/backstyles.pdf}}
  \end{minipage}
  \hfill
  \begin{minipage}[t]{0.6\textwidth}
    \paragraph{trisec}
    \par
    Kontrastreiche dreigeteilte Darstellung, die aber nur Adresse anzeigt.
    \par\bigskip
    \Macro{makebackpage}\OParameter{trisec}
    \par\bigskip
    Dargestellte Zusatz-Elemente:
    \begin{compactitem}\ttfamily
      \item address
    \end{compactitem}
  \end{minipage}


\section{Kopf-/Fußzeile}\label{sec:headline}
\Index{Kopfzeile}
\Index{Fußzeile}
\Index{Absenderbereich}
% \marginpar[Neu in 0.3-alpha3]{}

Durch das im \acs{CD} festgeschriebene Layout ist eine klassische Behandlung von
Kopf- und Fußzeile nicht möglich.
Stattdessen wird jeweils der Absenderbereich zur Darstellung entsprechender
Inhalte benutzt. Dies entspricht je nach dessen Platzierung
(Seitenanfang/Seitenende) einer Kopf- oder Fußzeile.

Bei Dokumenten mit Absenderbereich am oberen Seitenrand wird daher die Fußzeile
komplett leer gelassen und die Kopfzeile wird auf ungeraden Seiten
(bzw. bei einseitigem Layout auf allen Seiten)
mit Seitennummer und (Unter-)Kapitelname rechts bzw. außen gesetzt.
Bei zweiseitiger Darstellung wir auf geraden Seiten die Seitennummer außen und
der Kapitelname innen gesetzt.
Zusätzlich wird noch ein schmaler Strich am oberen äußeren Ende des
Absenderbereichs gesetzt.
Bei Dokumenten mit Absenderbereich am unteren Rand wird dagegen die Kopfzeile
komplett leer gelassen und die Fußzeile entsprechend der eben aufgeführten
Vorgaben gesetzt. Der kleine Trennstrich befindet sich dabei am unteren
äußeren Rand.

Die Darstellung der Kopf- und Fußzeilen lässt sich mit einigen
speziellen Befehlen individuell anpassen.

\begin{Declaration}
  \Macro{outersender}\OParameter{links}\Parameter{rechts}\\
  \Macro{innersender}\OParameter{rechts}\Parameter{links}\\
  \Macro{outerhead}\OParameter{links}\Parameter{rechts}\\
  \Macro{innerhead}\OParameter{rechts}\Parameter{links}\\
  \Macro{outerfoot}\OParameter{links}\Parameter{rechts}\\
  \Macro{innerfoot}\OParameter{rechts}\Parameter{links}
\end{Declaration}
\CommandIndex{outersender}
\CommandIndex{innersender}
\CommandIndex{outerhead}
\CommandIndex{innerhead}
\CommandIndex{outerfoot}
\CommandIndex{innerfoot}

Mit dem Befehl \Macro{outersender}\OParameter{links}\Parameter{rechts}
wird der Inhalt des äußeren Absenderbereichs festgelegt.
Die Parameter \PName{rechts} stellt dabei den darzustellenden Inhalt dar.
Mit Hilfe des optionalen Parameters \PName{links} können unterschiedliche
Inhalte gesetzt werden, abhängig davon, ob (bei zweiseitigem Layout)
der äußere Rand auf der jeweiligen Seite der linke oder rechte Rand ist.
Abhängig davon, ob der Absenderbereich am oberen oder unteren Ende des Blattes
dargestellt ist wird der Inhalt entweder in der Kopf- oder in der Fußzeile
dargestellt.

Mit den Befehlen \Macro{outerhead} und \Macro{innerhead}, die ansonsten genauso
wie die beiden zuvor erläuterten Befehle funktionieren, kann man dagegen
Inhalte festlegen, die explizit nur als Kopfzeile gesetzt werden, also
im Falle eines am unteren Rand platzierten Absenderbereichs nicht dargestellt 
werden.

Analog dazu gibt es noch die zwei Befehle \Macro{outerfoot}
und \Macro{innerfoot}, die selbiges Verhalten für Fußzeilen aufweisen.
Der so festgelegte Inhalt wird dann im Falle eines am oberen Blattrand
befindlichen Absenderbereichs nicht dargestellt.

\begin{Declaration}
  \Macro{headtopline}
\end{Declaration}
\CommandIndex{headtopline}

Die standardmäßig gesetzte kurze Linie am oberen äußeren Ende des
Kopfbereichs kann mit diesem aus dem Koma-Skript stammenden Befehl verändert
werden. Näheres ist in der zugehörigen Dokumentation zu
entnehmen\cite{koma-skript}.


\section{Schrift}\label{subsec:documents:fonts}
\Index{Schrift}

Dokumente im \acs{CD} werden allgemein in der Schriftart \emph{Nexus}
gesetzt.

\begin{Declaration}
  \Option{nexus}\\
  \Option{arial}
\end{Declaration}
\Index{Schrift!Nexus}
\Index{Schrift!Arial}

Als Alternativschrift ist im \acs{CD} Arial festgelegt. Durch die Klassenoption
\Option{arial} kann die Dokumentenschriftart auf die serifenlose Schrift
Arial umgestellt werden. Dies ist bei textlastigen Darstellungen aus
typographischer Sicht jedoch nicht zu empfehlen.
Die vollständigkeitshalber definierte Option \Option{nexus} wählt Nexus als 
Dokumentenschriftart, welche jedoch standardmäßig bereits voreingestellt ist.


\paragraph{Schriftgröße}
\Index{Schrift!Größe}
% TODO: List Format <-> Default font size !?

Die Standardschriftgröße bei allen DINA4-Dokumenten ist 11pt.
Für diese sind die verschiedenen Größenabstufungen
(\Macro{small}, \Macro{large}, \ldots) \acs{CD}-konform definiert.
Dies gilt ebenfalls für die Basis-Schriftgrößen von 10pt und 9pt.
Für andere Formate ist jeweils ebenfalls eine Basis-Schriftgröße definiert.
Eine Übersicht darüber bietet Tabelle~\ref{table:fontsizes}
Auf die Verwendung einer Schriftgröße außerhalb des Definitionsbereichs sollte
nach Möglichkeit verzichtet werden, da dies zu starken Abweichungen in der
Größendarstellung und somit im Gesamtlayout führt.
% \begin{important}
%   Bei Abweichung von diesen vordefinierten Basis-Schriftgrößen kann es zu deutlichen
%   Abweichungen bei größeren Schriftelementen (\Macro{large}, \Macro{huge}, \ldots)
%   kommen, da die CD-Definitionen teilweise stark von den Standard-\LaTeX-Definitionen
%   abweichen.
% \end{important}

% TODO: Hier weiter machen!
\begin{table}\centering
\begin{tabular}{ll}
  Format  & Basis-Schriftgröße  \\
  \midrule
  DIN A0  & 40pt \\
  DIN A1  & 25pt \\
  DIN A2  & 18pt \\
  DIN A3  & 13pt \\
  DIN A4  & 9pt,10pt,11pt \\
  DIN A5  & 9pt \\
  DIN lang& 9pt \\
\end{tabular}
\caption{Definierte Basis-Schriftgrößen}\label{table:fontsizes}
\end{table}


Für wichtige Standard-Elemente gibt es des Weiteren vordefinierte Koma-Fonts.
Diese können bei Bedarf mit dem Befehl \Macro{setkomafont}\Parameter{Parameter} 
geändert werden, wovon aber abgeraten wird, da sie entsprechend des CD korrekt
definiert sind. Teilweise sind Namen auch doppelt definiert, um sowohl
CD- als auch \LaTeX-Konventionen zu entsprechen.

\paragraph{Überschriften}
Alle Überschriftsebenen werden in einer im CD definierten
Schriftgröße dargestellt.
Die Darstellung erfolgt dabei in der Farbe \lstinline{tubsheadings},
welche normalerweise auf \lstinline{tuBlack} voreingestellt ist.
Sie kann mittels
\Macro{colorlet}\Parameter{tubsheadings}\Parameter{{\textit Farbe}} individuell
angepasst werden.

\begin{Declaration}
  \Option{colorheadings}
\end{Declaration}
\OptionIndex{colorheadings}

Diese Klassenoption stellt alle Überschriftsebenen
(\Macro{chapter}, \ldots, \Macro{subsubsection})
in Sekundärfarbe dar, indem die Farbe \lstinline{tubsheadings} zu
\lstinline{tuSecondary} gesetzt wird (siehe vorheriger Abschnitt).

\paragraph{Schriftauszeichnung}
\begin{sloppypar}
Die Schriften für die verschiedenen Elemente sind als benannte Schrift-Elemente
definiert und können so einfach und universell verwendet werden.
Dabei wird auf das aus dem KOMA-Skript\cite{koma-skript} bekannte Schrift-Interface
aufgebaut. Die verschiedenen Befehle können daher größtenteils wie ihre KOMA-Pendants
benutzt werden. Eine kurze Benutzungsbeschreibung ist trotzdem sinnvoll.
\end{sloppypar}

\begin{Declaration}
  \Macro{usetubsfont}\Parameter{Element}\\
  \Macro{settubsfont}\Parameter{Element}\Parameter{Befehle}\\
  \Macro{addtotubsfont}\Parameter{Element}\Parameter{Befehle}
\end{Declaration}
\CommandIndex{usetubsfont}
\CommandIndex{settubsfont}
\CommandIndex{addtotubsfont}

Mit \Macro{usetubsfont} können die einzelnen Schrift-\PName{Element}e benutzt
werden, um die aktuelle Schriftart zu ändern.
Die Schriftart für Introtexte wird beispielsweise mit\\
\lstinline!\usetubsfont{introtext}! ausgewählt.
Auch wenn es allgemein nicht zu empfehlen ist, können die einzelnen
\PName{Element}e auch verändert oder individuell neu definiert werden.
Dies geschieht mit den Befehlen \Macro{addtotubsfont} bzw.
\Macro{settubsfont}.



Die folgende Liste bietet eine Übersicht über alle verfügbaren bzw.
vordefinierten Schriftart-Elemente. Identisch definierte Elemente sind
dabei zusammengefasst dargestellt:

\begin{desctable}
\toprule
\entry{%
  \PValue{headline}, \PValue{title}}{%
  Einfache Überschrift (groß) wie sie hauptsächlich auf Titelseiten verwendet wird.
  Die Schriftfarbe ist abhängig von der Klassenoption \Option{colorheadings} bzw.
  der Farbe \lstinline{tubsheadings}.
}
\entry{%
  \PValue{headlinesmall}, \PValue{chapter}}{%
  Einfache Überschrift (klein).
  Sie wird auch als Schrift der Gliederungsebene \Macro{chapter}
  verwendet.
  Die Schriftfarbe ist abhängig von der Klassenoption \Option{colorheadings} bzw.
  der Farbe \lstinline{tubsheadings}.
}
\entry{%
  \PValue{subheadline}, \PValue{subtitle}, \PValue{section}}{%
  Unterüberschrift wie sie unter anderem auf Titelseiten verwendet wird.
  Sie wird auch als Schrift der Gliederungsebene \Macro{section} verwendet.
  Die Schriftfarbe ist abhängig von der Klassenoption \Option{colorheadings} bzw.
  der Farbe \lstinline{tubsheadings}.
}
\entry{%
  \PValue{subheadlinesmall}, \PValue{subsection}}{%
  Unterüberschrift (klein).
  Sie wird auch als Schrift der Gliederungsebene \Macro{subsection} verwendet.
  Die Schriftfarbe ist abhängig von der Klassenoption \Option{colorheadings} bzw.
  der Farbe \lstinline{tubsheadings}.
}
\entry{%
  \PValue{intro}, \PValue{subsubsection}}{%
  Schrift für Introtext.
  Sie wird auch als Schrift der Gliederungsebene \Macro{subsubsection}
  verwendet.
  Die Schriftfarbe ist abhängig von der Klassenoption \Option{colorheadings} bzw.
  der Farbe \lstinline{tubsheadings}.
}
\entry{%
  \PValue{institute}}{%
  Institutsname im Logo-Bereich des Absenders.
}
\entry{%
  \PValue{author}}{%
  Autorenname auf der Titelseite
}
\entry{%
  \PValue{date}}{%
  Datum auf der Titelseite
}
% \entry{\PValue{infotext}}{%
% }
% \entry{\PValue{infotextitem}}{%
% }
\entry{\PValue{copytext}}{%
  Mengentext in Basis-Schriftgröße
}
\bottomrule
\caption{In \tubslatex-Dokumentenklassen definierte Schriften (KOMA-Fonts)}
\end{desctable}


\paragraph{Aufzählungszeichen}

Aufzählungszeichen für die \Environment{itemize}-Umgebung werden in \tubslatex
als kleine Kästchen dargestellt.
Diese sind für die ersten zwei Ebenen einer Aufzählung definiert, wobei
die zweite Ebene kleinere Kästchen in der Darstellung aufweist:
\begin{center}
\fbox{\parbox{0.5\textwidth}{\begin{itemize}
  \item Item 1
  \item Item 2
  \begin{itemize}
    \item Sub-Item 1
    \item Sub-Item 2
  \end{itemize}
  \item Item 3
\end{itemize}}}
\end{center}



\section{Farben}

Diese Kapitel gibt nur einen kurzen Einblick über die wichtigsten Punkte
zum Thema Farben. Für eine detaillierte Übersicht über die verfügbaren Farben
und weitere Möglichkeiten siehe Kapitel~\ref{chap:tubscolors}.%TODO...

Die Dokumentenklassen in \tubslatex unterstützen im Prinzip 2 Farbmodelle:
Das für Bildschirmdarstellungen verwendeten RGB-Modell und das im
Druck gebräuchliche CMYK-Modell. Zwischen diesen beiden kann in den Optionen
der Dokumentenklasse gewählt werden. Darüber hinaus gibt es noch die Unterstützung
für die Darstellung von Schwarz-Weiß-Dokumenten. Dabei werden automatisch
alle Elemente, die sonst in der roten Primärfarbe des CD dargestellt werden,
schwarz dargestellt.

\begin{Declaration}
  \Option{cmyk}
\end{Declaration}

Diese Klassenoption erwirkt eine CMYK-Darstellung des Siegelbandlogos und der
CD-Standardfarben. Sie ist für den Druck zu bevorzugen.

\begin{important}
  Bei der Verwendung von \Option{cmyk} ist zu beachten,
  dass das die Farbe Schwarz (\texttt{tuBlack})
  aus den Farbdefinitionen der Mischung 0\%C, 0\%M, 0\%Y, 100\%K entspricht,
  was in der Bildschirmdarstellung allgemein als dunkles grau wirkt. Dies
  gilt somit auch für den Text im Dokument und ist normal.

  Sollte dagegen die Ausgabe aus dem Drucker zu hell erscheinen, so kann dies
  durch einen manuelle Anpassung der Farbe \texttt{tuBlack} korrigiert werden,
  indem zusätzlich die CMY-Farben benutzt werden. Dies kann jedoch auch zu
  schlechteren Druckergebnissen durch übermäßigen Tinteneinsatz führen!
  
  Beispiel zur Umdefinierung:
  \lstinline!\definecolor{tuBlack}{CMYK}{0.5,0.5,0.5,1.0}!
\end{important}

\begin{Declaration}
  \Option{rgb}
\end{Declaration}

Diese Klassenoption erwirkt eine RGB-Darstellung des Siegelbandlogos und der
CD-Standardfarben. Sie ist für Bildschirmdarstellungen zu bevorzugen.

\begin{Declaration}
  \Option{mono}
\end{Declaration}

Diese Klassenoption stellt sowohl das Siegelbandlogo als auch alle in der
roten Primärfarbe (\texttt{tuRed}) des CD definierten Elemente schwarz dar.
Sie entscheidet dabei \emph{nicht} über das verwendete Farbmodell (RGB/CMYK).
\begin{important}
  Bei der Verwendung von \Option{mono} in Verbindung mit \Option{cmyk}
  ist zu beachten, dass das die Farbe Schwarz (\texttt{tuBlack})
  aus den Farbdefinitionen (wie oben beschrieben) nicht dem 'wirklichen' Schwarz
  des Siegelbandlogos entspricht.
  Im Zweifelsfall ist also eine Verwendung des RGB-Farbmodells in Verbindung
  mit \Option{mono} oder eine Neudefinition von \texttt{tuBlack} vorzuziehen.
\end{important}

\subsection{Farbschema}

Das Farbschema des CD gliedert sich grob in zwei Kategorien: Primärfarben
und Sekundärfarben. Während zu den Primärfarben nur das TU-Rot (\texttt{tuRed})
und Schwarz, Weiß, sowie weitere Grautöne gehören, bieten die Sekundärfarben
eine größere Auswahl. Sie sind nochmals in 4 verschiedene Farbklänge
(orange, grün, blau, violett) unterteilt, die jeweils 3 aufeinander
abgestimmte Farben enthalten, die wiederum noch in 20\%-Schritten aufgehellt
werden können.

\begin{Declaration}
  \Option{orange}\\
  \Option{blue}\\
  \Option{green}\\
  \Option{violet}
\end{Declaration}

\begin{sloppypar}
Da oftmals nur mit einem Hauptfarbklang gearbeitet wird, gibt es die Möglichkeit
diesen in den Klassenoptionen zu wählen. Die Farben sind dann unter den
Aliasnamen
\texttt{tuSecondary\textit{<Light//Medium/Dark>}\textit{<100/80/60/40/20}}
verwendbar.
So entspricht \texttt{tuSecondaryDark80} in Verbindung mit Option
\Option{blue} \texttt{tuBlueDark80}, in Verbindung mit Option \Option{green}
jedoch \texttt{tuGreenDark80}.
\end{sloppypar}

Viele der standardmäßig definierte Styles und Vorlagen nutzen ebenfalls diese
Alias-Farben, sodass deren Darstellung durch die Optionen ebenfalls beeinflusst 
werden kann.

\begin{Declaration}
  \Macro{colorshow}\OParameter{Breite}\Parameter{Farbe}\Parameter{Helligkeit}
\end{Declaration}

\begin{sloppypar}
Hilfreicher Befehl für den Design-Prozess. Er zeigt alle Abstufungen
einer \PName{Farbe} (\PValue{Orange/Blue/Green/Violet}) mit gewählter 
\PName{Helligkeit} (\PValue{Light/Medium/Dark}).
Optional kann noch die \PName{Breite} der Darstellung gewählt werden,
ist kein Wert angegeben, so wird \Macro{textwidth} verwendet.sollte
\end{sloppypar}

\begin{figure}
  tuOrange\ldots\\
  \colorshow{Orange}{Light}
  \colorshow{Orange}{Medium}
  \colorshow{Orange}{Dark}\\[-1ex]
  tuBlue\ldots\\
  \colorshow{Blue}{Light}
  \colorshow{Blue}{Medium}
  \colorshow{Blue}{Dark}\\[-1ex]
  tuGreen\ldots\\
  \colorshow{Green}{Light}
  \colorshow{Green}{Medium}
  \colorshow{Green}{Dark}\\[-1ex]
  tuViolet\ldots\\
  \colorshow{Violet}{Light}
  \colorshow{Violet}{Medium}
  \colorshow{Violet}{Dark}
  \caption{Im CD definierte Farben und deren Benennung (Auszug)}
\end{figure}


\paragraph{Hinweis zur Verwendung}

Für Typografie auf farbigen Flächen gilt:
Bis zu einer Abstufung von 60\% kann die Schrift in Weiß auf diesem Farbton stehen.
Ab einem Tonwert von 40\% sollte die Schrift,
im Sinne einer optimalen Lesbarkeit, in Schwarz dargestellt werden. 

Weitere Hinweise zur allgemeinen Verwendung sind bitte der
CD-Toolbox\cite{toolbox} zu entnehmen.

\subsection{hyperref-Anpassung}

\begin{Declaration}
  \Option{hyperref}\\
  \Option{hyperrefdark}
\end{Declaration}

Mit Hilfe der Klassenoption \Option{hyperref} kann ein auf \tubslatex abgestimmtes
Farbschema für das \texttt{hyperref}-Paket geladen werden.
Das Paket selber wird dabei nicht geladen, sondern muss manuell geladen werden.
Somit können auch einzelne Farben und Optionen noch nach Belieben geändert werden.
Die Anweisung lädt ebenfalls die \texttt{hyperref}-Option \Option{colorlinks}.\\
Die Option \Option{hyperrefdark} bewirkt dasselbe wie \Option{hyperref},
lädt jedoch ein deutlich dunkleres und damit dezenteres Farbschema vor.



% TODO... later...
% \section{Arbeiten mit \tubslatex}

% \subsection{Text schreiben}



% TODO: margins, textwidth-Problem, etc.


