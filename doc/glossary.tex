%Befehle für Glossar
\newglossaryentry{glos:siegelbandlogo}{%
  name=Siegelbandlogo,
  description={\tubslogo}
}
\newglossaryentry{glos:gaussraster}{%
  name={Gau\ss-Raster},
  description={Auf der gaußschen Summenformel basierende Unterteilung der Seite
    in Segmente. Benachbarte Segmente können beliebig zusammen gefasst werden.}
}
\newglossaryentry{glos:cmyk}{%
  name={CMYK-Farbmodell},
  description={Das CMYK-Farbmodell ist ein subtraktives Farbmodell,
    das die technische Grundlage für den modernen Vierfarbdruck bildet.
    Die Abkürzung CMYK steht für die drei Farbbestandteile
    \emph{Cyan}, \emph{Magenta}, \emph{Yellow}
    und den Schwarzanteil \emph{Key} als Farbtiefe.}
}
\newglossaryentry{glos:Absenderbereich}{%
  name={Absenderbereich},
  description={Bereich zur Darstellung von Absendern}
}
%Akronyme
\newacronym{CD}{CD}{Corporate Design}
\newacronym{CMYK}{CMYK}{Cyan, Magenta, Yellow, Key}
