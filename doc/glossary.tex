%Befehle für Glossar
\newglossaryentry{glos:siegelbandlogo}{%
  name=Siegelbandlogo,
  description={\tubslogo}
}
\newglossaryentry{glos:gaussraster}{%
  name={Gau\ss raster},
  description={Auf der gaußschen Summenformel basierende Unterteilung der Seite
    in Segmente. Benachbarte Segmente können beliebig zusammen gefasst werden.}
}
\newglossaryentry{glos:cmyk}{%
  name={CMYK-Farbmodell},
  description={Das CMYK-Farbmodell ist ein subtraktives Farbmodell,
    das die technische Grundlage für den modernen Vierfarbdruck bildet.
    Die Abkürzung CMYK steht für die drei Farbbestandteile
    \emph{Cyan}, \emph{Magenta}, \emph{Yellow}
    und den Schwarzanteil \emph{Key} als Farbtiefe.}
}
\newglossaryentry{glos:absenderbereich}{%
  name={Absenderbereich},
  description={Bereich zur Darstellung von Absendern}
}
\newglossaryentry{glos:spaltenraster}{%
  name={Spaltenraster},
  description={Abhängig vom Format kann der Textbereich einer Seite in
    6, 4 oder 2 Grundspalten geteilt werden, welche alle die selbe Breite und
    den selben Abstand zueinander haben.
    Benachbarte Grundspalten können variabel zu einer Darstellungsspalte
    zusammengefasst werden.}
}
\newglossaryentry{glos:bindekorrektur}{%
  name={Bindekorrektur},
  description={}
}
% \newglossaryentry{glos:gaussraster}{%
%   name={Gaußraster},
%   description={Die Darstellung von Seiten basiert im CD-Layout im Prinzip immer
%   auf dem sogenannten Gaußraster. Dies unterteilt die Seite in verschieden große
%   Segmente, deren Höhe durch die gaußsche Summenformel berechnet werden kann.
%   Außerdem könne Seiten ein Reihe von Standardelementen wie das Siegellogo oder
%   ein zusätzliches individuelles Logo aufweisen.}
% }
%Akronyme
\newacronym{CD}{CD}{Corporate Design}
\newacronym{CMYK}{CMYK}{Cyan, Magenta, Yellow, Key}
