%Befehle für Glossar
\newglossaryentry{glos:siegelbandlogo}{%
  name=Siegelbandlogo,
  description={\tubslogo}
}
\newglossaryentry{glos:gaussraster}{%
  name={Gau\ss raster},
  description={Auf der \glslink{glos:summenformel}{gauß'schen Summenformel} basierende Unterteilung der Seite
    in Segmente. Benachbarte Segmente können beliebig zusammen gefasst werden}
}
\newglossaryentry{glos:modulsystem}{%
  name={Modulsystem},
  description={Flexibles Platzierungssystem für wissenschaftliche Plakate.
  Dabei wird der Darstellungsbereich komplett in einzelne Module
  verschiedener Größe aufgeteilt.}
}
\newglossaryentry{glos:cmyk}{%
  name={CMYK-Farbmodell},
  description={Das CMYK-Farbmodell ist ein subtraktives Farbmodell,
    das die technische Grundlage für den modernen Vierfarbdruck bildet.
    Die Abkürzung CMYK steht für die drei Farbbestandteile
    \emph{Cyan}, \emph{Magenta}, \emph{Yellow}
    und den Schwarzanteil \emph{Key} als Farbtiefe}
}
\newglossaryentry{glos:absenderbereich}{%
  name={Absenderbereich},
  description={
    Freier Bereich am oberen oder unteren Blattrand
    zur Darstellung eines Absenders (Institut/zentrale Eintrichtung).
    Ihm schließt sich direkt der \gls{glos:kommunikationsbereich} mit dem Inhalt an.\\
    Die Position des Absenderbereichs kontrolliert beim Gauß-Layout auch die
    Reihenfolge der Segmentaufteilung. Das größte Segment befindet sich immer
    auf der Seite des Absenderbereichs.
    }
}
\newglossaryentry{glos:kommunikationsbereich}{%
  name={Kommunikationsbereich},
  description={Bereich zur Darstellung von Inhalten}
}
\newglossaryentry{glos:spaltenraster}{%
  name={Spaltenraster},
  description={Abhängig vom Format kann der Textbereich einer Seite in
    6, 4 oder 2 Grundspalten geteilt werden, welche alle die selbe Breite und
    den selben Abstand zueinander haben.
    Benachbarte Grundspalten können variabel zu einer Darstellungsspalte
    zusammengefasst werden}
}
\newglossaryentry{glos:bindekorrektur}{%
  name={Bindekorrektur},
  description={}
}
\newglossaryentry{glos:sekundaerfarbklang}{%
  name={Sekund\"arfarbklang},
  description={}
}
\newglossaryentry{glos:mediaevalziffern}{%
  name={Medi\"avalziffern},
  description={Ziffern, die im Gegensatz zu Versalziffern Ober- und Unterlänge
  haben und sich dadurch im Mengentext besser in das Schriftbild einfügen als \gls{glos:versalziffern}}
}
\newglossaryentry{glos:versalziffern}{%
  name={Versalziffern},
  description={Auf der Grundlinie ausgerichtete Ziffern ohne Ober- und Unterlängen.
  Sie fügen sich daher meist schlechter in Schriftbild ein als \gls{glos:mediaevalziffern}}
}
\newglossaryentry{glos:dinlang}{%
  name={DIN lang},
  description={Bezeichnet hier ein Format von $1/3$ der Blattgröße \mbox{DIN\,A4},
  wie es zum Beispiel für Flyer (doppelt gefaltet) häufig verwendet wird.}
}
\newglossaryentry{glos:summenformel}{%
  name={Gauß'sche Summenformel},
  description={Berechnungsgrundlage für die formatübergreifende
  vertikale Aufteilung des Kommunikationsbereichs im CD (siehe \gls{glos:gaussraster}).
  \[1 + 2 + 3 + 4 + \ldots + n = \sum_{k=1}^n k = \frac{n(n+1)}{2}\]}
}

%Akronyme
\newacronym{CD}{CD}{Corporate Design}
\newglossaryentry{CMYK}{%
  type=\acronymtype,
  name={CMYK},
  description={Cyan, Magenta, Yellow, Key (siehe \gls{glos:cmyk})}
}
\newacronym{RGB}{RBG}{Red, Green, Blue}
\newacronym{TDS}{TDS}{TeX Directory Structure}
