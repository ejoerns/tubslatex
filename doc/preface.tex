\chapter*{Vorwort}

Eine klare, wiedererkennbare und dennoch abwechslungsreiche
Repräsentation ist in einem medialen Umfeld wie wir es heutzutage
vorfinden nicht nur für jedes Unternehmen, sondern vor allem auch
für eine Universität von großer Bedeutung.
Sie schafft Identität, fördert Identifikation und behauptet sich zugleich im 
Blickfeld der Öffentlichkeit.

Mit dem Corporate Design der TU Braunschweig ist dieses wichtige Ziel
erfüllt worden. Es verbindet ein ganzheitliches Aussehen mit dem nötigen
Spielraum für Individualität und Kreativität der einzelnen Institute und 
Einrichtungen.\bigskip


Die Vorgaben und Möglichkeiten, die das Corporate Design bietet,
mit einem effektiven und professionellen Textsatz zu verbinden, ist
das Ziel von \tubslatex. Diese Sammlung von Paketen und Klassen
für das Textsatzprogramm \LaTeX\ soll dem Benutzer zum einen das Erstellen und
die Gestaltung von verschiedensten Schriftstücken vereinfachen und zum anderen
die Einhaltung der notwendigen Richtlinien sicherstellen.

Die Einsatzmöglichkeiten sind dabei sehr vielfältig.
Vom Schreiben einfacher Textdokumente und Briefe über die Gestaltung von Postern
und Flyern bis hin zur Erstellung von umfangreichen Präsentationen erstreckt
sich das Spektrum der Möglichkeiten.
Im Mittelpunkt fast jedes Dokumentes stehen die Grundelemente des Corporate
Designs. Dazu zählen das rote Siegelband-Logo, die Hausschrift Nexus,
das markante Gauß-Layout, sowie ein klares Farbschema.\bigskip


Alle Vorlagen wurden nach bestem Wissen erstellt und getestet, was
jedoch den Fehlerteufel erfahrungsgemäß zumeist nicht sonderlich beeindruckt.
Daher bitte ich bei Auftreten eines Fehlers oder Problems Milde walten
zu lassen und diesen zu melden.
Dies ist zum einen mit dem auf der Projektseite\footnote{http://enricojoerns.de/tubslatex}
verlinkten Bugtracker möglich,
oder natürlich auch immer gerne per Mail an \textit{e.joerns@tu-bs.de}.
Dies gilt natürlich auch für Anwendungsfragen, Anregungen und Wünsche.

Ich wünsche viel Erfolg und Produktivität bei der Arbeit mit \tubslatex!

\hfill\textit{Braunschweig, den \today}\\[\bigskipamount]
\noindent Enrico Jörns

