\chapter{Hausschrift Nexus}\label{chap:nexus}

Die im \acs{CD} definiert Hausschrift \emph{Nexus} wird durch das Paket
\newpackage{nexus} bereitgestellt, welches in allen verfügbaren \tubslatex-Klassen
fest eingebunden ist.

\begin{Declaration}
  \Option{nexus}\\
  \Option{aial}
\end{Declaration}

Paket-/Klassenoption zur Auswahl der Schriftfamilie.
Zur Verfügung stehen Nexus (Standard) oder die Ersatzschrift Arial.

\paragraph{Encoding}

Das Paket \texttt{nexus} wählt automatisch das korrekte Font-encoding.
Dies ist \emph{LY1} für Nexus und \emph{T1}, falls Arial verwendet wird.
Ein zusätzliches Laden des Paketes \texttt{fontenc} ist nicht zu empfehlen.

\section*{Serif}

\paragraph{Normal}\hfill\\
{
Lorem ipsum dolor sit amet, consectetur adipisici elit, sed eiusmod tempor
incidunt ut labore et dolore magna aliqua.}
\paragraph{Italic}\hfill\\
{\itshape
Lorem ipsum dolor sit amet, consectetur adipisici elit, sed eiusmod tempor
incidunt ut labore et dolore magna aliqua.}
\paragraph{Slanted}\hfill\\
{\slshape
Lorem ipsum dolor sit amet, consectetur adipisici elit, sed eiusmod tempor
incidunt ut labore et dolore magna aliqua.}
\paragraph{SmallCaps}\hfill\\
{\scshape
Lorem ipsum dolor sit amet, consectetur adipisici elit, sed eiusmod tempor
incidunt ut labore et dolore magna aliqua.}

{\bfseries
\paragraph{Bold Normal}\hfill\\
{
Lorem ipsum dolor sit amet, consectetur adipisici elit, sed eiusmod tempor
incidunt ut labore et dolore magna aliqua.}
\paragraph{Bold Italic}\hfill\\
{\itshape
Lorem ipsum dolor sit amet, consectetur adipisici elit, sed eiusmod tempor
incidunt ut labore et dolore magna aliqua.}
\paragraph{Bold Slanted}\hfill\\
{\slshape
Lorem ipsum dolor sit amet, consectetur adipisici elit, sed eiusmod tempor
incidunt ut labore et dolore magna aliqua.}
\paragraph{Bold SmallCaps}\hfill\\
{\scshape
Lorem ipsum dolor sit amet, consectetur adipisici elit, sed eiusmod tempor
incidunt ut labore et dolore magna aliqua.}
}

\section*{SansSerif}

{\sffamily
\paragraph{Normal}\hfill\\
{
Lorem ipsum dolor sit amet, consectetur adipisici elit, sed eiusmod tempor
incidunt ut labore et dolore magna aliqua.}
\paragraph{Italic}\hfill\\
{\itshape
Lorem ipsum dolor sit amet, consectetur adipisici elit, sed eiusmod tempor
incidunt ut labore et dolore magna aliqua.}
\paragraph{Slanted}\hfill\\
{\slshape
Lorem ipsum dolor sit amet, consectetur adipisici elit, sed eiusmod tempor
incidunt ut labore et dolore magna aliqua.}
\paragraph{SmallCaps}\hfill\\
{\scshape
Lorem ipsum dolor sit amet, consectetur adipisici elit, sed eiusmod tempor
incidunt ut labore et dolore magna aliqua.}

{\bfseries
\paragraph{Bold Normal}\hfill\\
{
Lorem ipsum dolor sit amet, consectetur adipisici elit, sed eiusmod tempor
incidunt ut labore et dolore magna aliqua.}
\paragraph{Bold Italic}\hfill\\
{\itshape
Lorem ipsum dolor sit amet, consectetur adipisici elit, sed eiusmod tempor
incidunt ut labore et dolore magna aliqua.}
\paragraph{Bold Slanted}\hfill\\
{\slshape
Lorem ipsum dolor sit amet, consectetur adipisici elit, sed eiusmod tempor
incidunt ut labore et dolore magna aliqua.}
\paragraph{Bold SmallCaps}\hfill\\
{\scshape
Lorem ipsum dolor sit amet, consectetur adipisici elit, sed eiusmod tempor
incidunt ut labore et dolore magna aliqua.}
}
}

\section*{Typewriter}

Der Typewriter-Font stammt nicht aus Nexus, sondern wird aus dem
\emph{TxFonts}-Satz übernommen.

\paragraph{Normal}\hfill\\
{\ttfamily
Lorem ipsum dolor sit amet, consectetur adipisici elit, sed eiusmod tempor
incidunt ut labore et dolore magna aliqua.
}

\paragraph{Italic}\hfill\\
{\ttfamily\itshape%
Lorem ipsum dolor sit amet, consectetur adipisici elit, sed eiusmod tempor
incidunt ut labore et dolore magna aliqua.
}


\paragraph{Bold Normal}\hfill\\
{\ttfamily\bfseries%
Lorem ipsum dolor sit amet, consectetur adipisici elit, sed eiusmod tempor
incidunt ut labore et dolore magna aliqua.
}

\paragraph{Bold Italic}\hfill\\
{\ttfamily\bfseries\itshape%
Lorem ipsum dolor sit amet, consectetur adipisici elit, sed eiusmod tempor
incidunt ut labore et dolore magna aliqua.
}


\section*{Mediävalziffern/Versalziffern}
\Index{Mediävalziffern}
\Index{Versalziffern}
\Index{Ziffern!Versal-}
\Index{Ziffern!Mediäval-}

Standardmäßig sind in Nexus \gls{glos:mediaevalziffern} definiert, das heißt Ziffern,
die eine Ober- unt Unterlänge besitzen.
Diese passen sich beim Mengentext besser in das Schriftbild ein.\par

Alternativ können auch \gls{glos:versalziffern} verwendet werden, das heißt Ziffern,
welche alle die selbe Höhe haben.
\begin{center}
\noindent\begin{tabular}{lll}
& \bfseries Mediävalziffern & \bfseries Versalziffern\\
\midrule
Serif: & {\rmfamily\oldstylenums{ 1,\,2,\,3,\,4,\,5,\,6,\,7,\,8,\,9,\,0}} &
  {\rmfamily\lnum{1,\,2,\,3,\,4,\,5,\,6,\,7,\,8,\,9,\,0}}\\
Sans-Serif: & {\sffamily\oldstylenums{ 1,\,2,\,3,\,4,\,5,\,6,\,7,\,8,\,9,\,0}} &
  {\sffamily\lnum{1,\,2,\,3,\,4,\,5,\,6,\,7,\,8,\,9,\,0}}
\end{tabular}
\end{center}

Versalziffern können entweder mit der Paket-/Klassenoption \Option{lnum} oder dem Befehl
\Macro{lnum} gesetzt werden.

\begin{Declaration}
  \Option{lnum}
\end{Declaration}

Die Paket-/Klassenoption \Option{lnum} wählt für das Dokument einen Schriftschnitt mit
Versalziffern.

\begin{Declaration}
  \Macro{lnum}\Parameter{Text}\\
  \Macro{oldstylenums}\Parameter{Text}
\end{Declaration}

Der Befehl \Macro{lnum} setzt den \PName{Text} immer mit Versalziffern.
Der Befehl \Macro{oldstylenums} bewirkt das Gegenteil und
setzt den \PName{Text} immer mit Mediävalziffern.
