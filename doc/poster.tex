\chapter{Plakate}

Plakate werden mit Hilfe der Klasse \newdocumentclass{tubsposter} erstellt.
Dabei kann mittels der Option \texttt{style} zwischen der Darstellung als
normales Plakat oder als wissenschaftliches Plakat gewählt werden.

\section{Veranstaltungsplakate}

\begin{Declaration}
  \XMacro{begin}\PParameter{\Environment{tubsposter}}%
    \OParameter{options}\\
  \quad\dots\\
  \XMacro{end}\PParameter{tubsposter}
\end{Declaration}

Ein neues Plakat wird mit der Umgebung \Environment{tubsposter} erstellt.
Der optionale Parameter \PName{options} akzeptiert dabei die
unter \ref{} beschriebenen Optionen.

\begin{Declaration}
  \XMacro{begin}\PParameter{\Environment{posterrow}}%
    \OParameter{cols}\\%
  \quad\dots\\
  \XMacro{end}\PParameter{posterrow}%\\
%   \XMacro{begin}\PParameter{\Environment{posterrow*}}%
%     \OParameter{cols}\\%
%   \quad\dots\\
%   \XMacro{end}\PParameter{posterrow*}
\end{Declaration}

Einzelne Segmente im Gaußraster können mit der Umgebung \Environment{posterrow}
erzeugt werden.

\section{Wissenschaftliche Plakate}

\begin{Declaration}
  \Option{scifiposter}
\end{Declaration}

Paket-Option zum erstellen von wissenschaftlichen Postern.

\begin{Declaration}
  \XMacro{begin}\PParameter{\Environment{tubsposter}}%
    \OParameter{options}%
    \Parameter{rows}\\
  \quad\dots\\
  \XMacro{end}\PParameter{tubsposter}
\end{Declaration}

Für die Erstellung von wissenschaftlichen Plakaten wird ebenfalls die Umgebung
\Environment{tubsposter} verwendet, welche in diesem Fall jedoch einen
zusätzlichen Parameter \PName{rows} erwartet.
Damit wird die Anzahl an Modulzeilen bestimmt. Dies geschieht mittels
einer kommagetrennten Liste, wobei jedes Element entweder eine Länge
oder der Buchstabe 'X' sein kann. Eine Länge legt die Höhe der jeweiligen
Modulzeile genau fest, ein X sorgt dafür, dass alle mit X gekennzeichneten
Zeilen den restlichen zur Verfügung stehenden Platz gleichmäßig untereinander
aufteilen. Dieses Vorgehen ist an die Tabellen-Umgebung \Environment{tabularx}
angelehnt.

\begin{Example}
  \noindent\Macro{begin}\PParameter{\Environment{tubsposter}}
    \Parameter{3cm,X,5cm,X}\par
  \noindent Erzeugt 4 Modulzeilen, wobei die 1. 3cm und die 3. 5cm hoch sind.
  Die Zeilen 2 und 4 nehmen den Restlichen verfügbaren Platz ein
  und sind gleich hoch.
\end{Example}


\begin{Declaration}
  \XMacro{begin}\PParameter{\Environment{posterrow}}%
    \OParameter{cols}\\%
  \quad\dots\\
  \XMacro{end}\PParameter{posterrow}\\
  \XMacro{begin}\PParameter{\Environment{posterrow*}}%
    \OParameter{cols}\\%
  \quad\dots\\
  \XMacro{end}\PParameter{posterrow*}
\end{Declaration}

Die mit \Environment{tubsposter} angelegten Modulzeilen können nun jeweils mit
der Umgebung \Environment{posterrow} mit Inhalt gefüllt werden.
Dabei kann entweder direkt der gewünschte Inhalt geschrieben oder das optionale
Parameter \PName{cols} benutzt werden.
Dieses erlaubt die Definition zusätzlicher Spalten in der aktuellen Modulzeile.
Es wird wieder ein kommagetrennte Liste erwartet,
deren Elemente dieselbe Bedeutung haben wie bereits beschrieben, außer, dass
sie die Breite und nicht die Höhe definieren.

Der normale Abstand des Inhalts vom Rand der Modulbox beträgt halbe
Rahmenbreite. Für das Einfügen von Bildern etwa kann es sinnvoll sein,
diesen Rahmen wegzulassen. Dies geschieht mit der Sternchen-Variante
\Environment{posterrow*}.


\begin{Declaration}
  \XMacro{begin}\PParameter{\Environment{postercol}}%
    \OParameter{rows}\\%
  \quad\dots\\
  \XMacro{end}\PParameter{postercol}\\
  \XMacro{begin}\PParameter{\Environment{postercol*}}%
    \OParameter{rows}\\%
  \quad\dots\\
  \XMacro{end}\PParameter{postercol*}
\end{Declaration}

Die mit \Environment{posterrow} angelegten Spalten können jeweils mit 
\Environment{postercol} mit Inhalt gefüllt oder in neue Unterzeilen aufgeteilt 
werden.

\begin{Declaration}
  \XMacro{begin}\PParameter{\Environment{postersubrow}}\\%
  \quad\dots\\
  \XMacro{end}\PParameter{postersubrow}\\
  \XMacro{begin}\PParameter{\Environment{postersubrow*}}\\%
  \quad\dots\\
  \XMacro{end}\PParameter{postersubrow*}
\end{Declaration}

Die mit \Environment{postercol} angelegten Unterzeilen können jeweils mit 
\Environment{postercol} mit Inhalt gefüllt werden.

\begin{Example}

\end{Example}

\section{Seitenlayout}

% margin
% bcor
% sender=top/bottom


\section{Format}

\section{Farben}

\section{Befehle}

\section{Optionen}
