\chapter{Plakate}

Plakate können in \tubslatex mit Hilfe der Dokumentenklasse \newdocumentclass{tubsposter} erstellt werden.
Dabei werden drei unterschiedliche Plakat-Typen unterstützt,
zwischen denen mit der Klassenoption \KOption{style}\PName{<standard/scifi/bulletin>} gewählt werden kann.

Dies sind zum einen \emph{Veranstaltungsplakate},
welche sich am normalen Gaußraster
orientieren und als Standard-Darstellungstyp voreingestellt sind.

Für die Darstellung \emph{wissenschaftlicher} Inhalte gibt es ein spezielles Layout,
das mehr Platz und Flexibilität für die Unterbringung von Informationen bietet.
Es orientiert sich an einem eigenen Modulsystem und nicht am Gaußraster.

Zusätzlich gibt es noch eine Unterstützung für die Erstellung einfacher
\emph{Aushänge}. Diese weisen nur einen einfachen großen Textbereich auf, der
nach oben durch das Siegelband-Logo mit einer Trennlinie begrenzt ist.

\begin{figure}[htp]\centering
\fboxsep0mm
\begin{minipage}{0.32\textwidth}\centering
  \fbox{\includegraphics[width=0.9\textwidth]{examples/posterstyle_standard.pdf}}
  \subcaption{Veranstaltung (\PValue{standard})}
\end{minipage}
\begin{minipage}{0.32\textwidth}\centering
  \fbox{\includegraphics[width=0.9\textwidth]{examples/posterstyle_scifi.pdf}}
  \subcaption{wissenschaftlich (\PValue{scifi})}
\end{minipage}
\begin{minipage}{0.32\textwidth}\centering
  \fbox{\includegraphics[width=0.9\textwidth]{examples/posterstyle_bulletin.pdf}}
  \subcaption{Aushang (\PValue{bulletin})}
\end{minipage}
\caption{Die verschiedenen Plakat-Typen}
\end{figure}

\begin{Declaration}
  \KOption{style}\PName{<standard/scifi/bulletin>}
\end{Declaration}

Mit \KOption{style}{standard} wird die Darstellung für Veranstaltungsplakate
gewählt. Diese ist bereits voreingestellt und muss daher nicht explizit gewählt
werden.
Mit \KOption{style}{scifi} wir das Layout-System für wissenschaftliche Plakate
geladen, welches sich teilweise grundlegend von dem normalen Layout unterscheidet.
Der Wert \KOption{style}{bulletin} erlaubt die schnelle Erstellung von
Aushängen und lädt eine relative schlichte voreingestellte Darstellung.

\begin{Declaration}
  \KOption{sender}\PName{top/bottom}
\end{Declaration}

Diese Option legt fest, ob der Absenderbereich am oberen oder unteren Blattrand
platziert werden soll. Für weitere Informationen siehe vorausgehende Kapitel.
%TODO: ja? wo?
%TODO: Klasse + Umgebung


\begin{Declaration}
  \Option{a3paper}\\
  \Option{a2paper}\\
  \Option{a1paper}\\
  \Option{a0paper}
\end{Declaration}

Für alle Poster-Varianten stehen vier Papierformate zur Auswahl für die
Schriftgrößen etc. vordefiniert sind und automatisch geladen werden.
Dies sind A3, A2, A1 und A0, die mit den entsprechenden Optionen
\Option{a3paper}, \Option{a2paper}, \Option{a1paper} bzw. \Option{a0paper}
gewählt werden können.
\bigskip

Im folgenden werden die einzelnen Poster-Typen einzeln vorgestellt und beschrieben.

\paragraph{Weitere Klassenoptionen:}

Wie bei allen anderen \tubslatex-Klassen können auch für Poster
die diversen verfügbaren Optionen zur Farbdarstellung oder Schriftartwahl
verwendet werden.
Näheres hierzu findet sich in den vorausgehenden Kapiteln oder detaillierter
in Kapitel~\ref{chap:tubscolors} und \ref{chap:nexus}.

\clearpage
\section{Veranstaltungsplakate}

Veranstaltungsplakate werden im Gaußraster gesetzt und entsprechen somit
im Grunde allen Standarddarstellungen des Corporate Design.

\begin{lstlisting}
\documentclass[style=standard]{tubsposter}
\end{lstlisting}

(Optionale) Klassenoption zur Erstellung von Veranstaltungsplakaten.

Da das Gauß-Layoutsystem zum einen umfangreich, zum anderen auch für verschiedene Zwecke eingesetzt wird und werden kann, ist ihm ein Extra-Kapitel (\ref{chap:gausspage})
gewidmet.
Dies beschreibt die möglichen Befehle und Optionen ausführlich.
Hier werden sie jeweils nur kurz angerissen.

% TODO: landscape?

\begin{Declaration}
  \XMacro{begin}\PParameter{\Environment{tubsposter}}%
    \OParameter{options}\\
  \quad\dots\\
  \XMacro{end}\PParameter{tubsposter}
\end{Declaration}

Ein neues Plakat wird mit der Umgebung \Environment{tubsposter} erstellt.
Der optionale Parameter \PName{options} akzeptiert dabei die
unter \ref{sec:gausspage:bglayout} beschriebenen Optionen.

Der Inhalt des Posters sollte aus einem oder mehreren Segmenten bestehen,
die jeweils aus einem oder mehreren Basis-Segmenten des Gaußrasters so
bestehen, dass alle Basis-Segmente verwendet werden.
%TODO: ref, link, ?
% sender
% bgcolor?

\begin{Declaration}
  \XMacro{begin}\PParameter{\Environment{posterrow}}%
    \OParameter{options}
    \Parameter{Höhe}\\
  \quad\dots\\
  \XMacro{end}\PParameter{posterrow}%\\
%   \XMacro{begin}\PParameter{\Environment{posterrow*}}%
%     \OParameter{cols}\\%
%   \quad\dots\\
%   \XMacro{end}\PParameter{posterrow*}
\end{Declaration}

Einzelne Poster-Segmente im Gaußraster können mit der Umgebung
\Environment{posterrow} erzeugt werden.
Dabei gibt der Parameter \PName{Höhe} die Höhe der
\Environment{posterrow} in Basis-Segmenten des Gaußrasters an.
Der optionale Parameter \PName{options} akzeptiert die
unter \ref{subsec:gausspage:bgelement} beschriebenen Optionen mit
denen zum Beispiel recht einfach eine Hintergrundfarbe oder ein Hintergrundbild
gesetzt werden kann.

\begin{figure}[!ht]
\begin{minipage}{0.65\textwidth}
\begin{lstlisting}
\documentclass{tubsposter}
\begin{document}
\begin{tubsposter}
\showtubslogo
\begin{posterrow}[bgcolor=tuGreen]{3}
\end{posterrow}
\begin{posterrow}[bgcolor=tuOrange]{4}
\end{posterrow}
\begin{posterrow}[bgcolor=tuGreenDark]{1}
\end{posterrow}
\end{tubsposter}
\end{document}
\end{lstlisting}
\end{minipage}
\begin{minipage}{0.35\textwidth}
  \fboxsep0mm
  \fbox{\includegraphics[width=0.98\textwidth]{examples/stdposter_minimal.pdf}}
\end{minipage}
\caption{Minimal-Beispiel zur Verwendung von \Environment{tubsposter} und
  \Environment{posterrow}}
\end{figure}


\clearpage
\section{Wissenschaftliche Plakate}

\begin{lstlisting}
\documentclass[style=scifi]{tubsposter}
\end{lstlisting}

Klassenoption zum Erstellen von wissenschaftlichen Postern.

\begin{Declaration}
  \XMacro{begin}\PParameter{\Environment{tubsposter}}%
    \OParameter{options}%
    \Parameter{rows}\\
  \quad\dots\\
  \XMacro{end}\PParameter{tubsposter}
\end{Declaration}

Für die Erstellung von wissenschaftlichen Plakaten wird ebenfalls die Umgebung
\Environment{tubsposter} verwendet, welche in diesem Fall jedoch einen
zusätzlichen Parameter \PName{rows} erwartet.
Damit wird die Anzahl an Modulzeilen bestimmt. Dies geschieht mittels
einer kommagetrennten Liste, wobei jedes Element entweder eine Länge
oder der Buchstabe 'X' sein kann. Eine Länge legt die Höhe der jeweiligen
Modulzeile genau fest, ein X sorgt dafür, dass alle mit X gekennzeichneten
Zeilen den restlichen zur Verfügung stehenden Platz gleichmäßig untereinander
aufteilen. Dieses Vorgehen ist an die Tabellen-Umgebung \Environment{tabularx}
angelehnt.

\begin{Example}
  \noindent\Macro{begin}\PParameter{\Environment{tubsposter}}
    \Parameter{3cm,X,5cm,X}\par
  \noindent Erzeugt 4 Modulzeilen, wobei die 1. 3cm und die 3. 5cm hoch sind.
  Die Zeilen 2 und 4 nehmen den restlichen verfügbaren Platz ein
  und sind gleich hoch.
\end{Example}


\begin{Declaration}
  \XMacro{begin}\PParameter{\Environment{posterrow}}%
    \OParameter{cols}\\%
  \quad\dots\\
  \XMacro{end}\PParameter{posterrow}\\
  \XMacro{begin}\PParameter{\Environment{posterrow*}}%
    \OParameter{cols}\\%
  \quad\dots\\
  \XMacro{end}\PParameter{posterrow*}
\end{Declaration}

Die mit \Environment{tubsposter} angelegten Modulzeilen können nun jeweils mit
der Umgebung \Environment{posterrow} mit Inhalt gefüllt werden.
Dabei kann entweder direkt der gewünschte Inhalt geschrieben oder der optionale
Parameter \PName{cols} benutzt werden.
Dieses erlaubt die Definition zusätzlicher Spalten in der aktuellen Modulzeile.
Es wird wieder ein kommagetrennte Liste erwartet,
deren Elemente dieselbe Bedeutung haben wie bereits beschrieben, außer, dass
sie die Breite und nicht die Höhe definieren.

Der normale Abstand des Inhalts vom Rand der Modulbox beträgt halbe
Rahmenbreite. Für das Einfügen von Bildern etwa kann es sinnvoll sein,
diesen Rahmen wegzulassen. Dies geschieht mit der Sternchen-Variante
\Environment{posterrow*}.


\begin{Declaration}
  \XMacro{begin}\PParameter{\Environment{postercol}}%
    \OParameter{rows}\\%
  \quad\dots\\
  \XMacro{end}\PParameter{postercol}\\
  \XMacro{begin}\PParameter{\Environment{postercol*}}%
    \OParameter{rows}\\%
  \quad\dots\\
  \XMacro{end}\PParameter{postercol*}
\end{Declaration}

Die mit \Environment{posterrow} angelegten Spalten können jeweils mit 
\Environment{postercol} mit Inhalt gefüllt oder in neue Unterzeilen aufgeteilt 
werden.

\begin{Declaration}
  \XMacro{begin}\PParameter{\Environment{postersubrow}}\\%
  \quad\dots\\
  \XMacro{end}\PParameter{postersubrow}\\
  \XMacro{begin}\PParameter{\Environment{postersubrow*}}\\%
  \quad\dots\\
  \XMacro{end}\PParameter{postersubrow*}
\end{Declaration}

Die mit \Environment{postercol} angelegten Unterzeilen können jeweils mit 
\Environment{postercol} mit Inhalt gefüllt werden.

\begin{Example}

\end{Example}


\clearpage
\section{Aushänge}

\begin{lstlisting}
\documentclass[style=bulletin]{tubsposter}
\end{lstlisting}

% \section{Seitenlayout}

% margin
% bcor
% sender=top/bottom


% \section{Format}

% \section{Farben}

% \section{Befehle}

% \section{Optionen}
