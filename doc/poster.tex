\chapter{Plakate}
\Index{Plakate|indexbf}
\Index{Poster}
% TODO: Note: mehrere Plakate pro Dokument möglich!

Plakate können in \tubslatex mit Hilfe der Dokumentenklasse \newdocumentclass{tubsposter} erstellt werden.
Dabei werden drei unterschiedliche Plakat-Typen unterstützt,
zwischen denen mit der Klassenoption \KOption{style}\PName{<standard/scifi/bulletin>} gewählt werden kann.

Dies sind zum einen \emph{Veranstaltungsplakate},
welche sich am normalen Gaußraster
orientieren und als Standard-Darstellungstyp voreingestellt sind.

Für die Darstellung \emph{wissenschaftlicher} Inhalte gibt es ein spezielles Layout,
das mehr Platz und Flexibilität für die Unterbringung von Informationen bietet.
Es orientiert sich an einem eigenen \gls{glos:modulsystem} und nicht am
\gls{glos:gaussraster}.

Zusätzlich gibt es noch eine Unterstützung für die Erstellung einfacher
\emph{Aushänge}. Diese weisen nur einen einfachen großen Textbereich auf, der
nach oben durch das Siegelband-Logo mit einer Trennlinie begrenzt ist.

\begin{figure}[htp]\centering
\fboxsep0mm
\begin{minipage}{0.32\textwidth}\centering
  \fbox{\includegraphics[width=0.9\textwidth]{examples/posterstyle_standard.pdf}}
  \subcaption{Veranstaltung (\PValue{standard})}
\end{minipage}
\begin{minipage}{0.32\textwidth}\centering
  \fbox{\includegraphics[width=0.9\textwidth]{examples/posterstyle_scifi.pdf}}
  \subcaption{wissenschaftlich (\PValue{scifi})}
\end{minipage}
\begin{minipage}{0.32\textwidth}\centering
  \fbox{\includegraphics[width=0.9\textwidth]{examples/posterstyle_bulletin.pdf}}
  \subcaption{Aushang (\PValue{bulletin})}
\end{minipage}
\caption{Die verschiedenen Plakat-Typen}
\end{figure}

\begin{Declaration}
  \KOption{style}\PName{<standard/scifi/bulletin>}
\end{Declaration}

Mit der Klassenoption \KOption{style}\PName{Typ} kann zwischen
den verschiedenen Plakattypen gewählt werden.
Mit \KOption{style}{standard} wird die Darstellung für Veranstaltungsplakate
gewählt. Diese ist bereits voreingestellt und muss daher nicht explizit gewählt
werden.
Mit \KOption{style}{scifi} wir das Layout-System für wissenschaftliche Plakate
(Modulsystem) geladen, welches sich teilweise grundlegend von
dem normalen (Gaussraster-)Layout unterscheidet.
Der Wert \KOption{style}{bulletin} erlaubt die schnelle Erstellung von
Aushängen und lädt eine relative schlichte, voreingestellte Darstellung.


\section{Format und Darstellung}

Allen Plakatvarianten gleich sind die grundlegenden Varianten in der Formatwahl,
die am Anfang eines jede Plakatdesignprozeses stehen sollten.
Dazu gehören unter anderem:

\begin{itemize}
  \item Papierformat [DIN\;Ax / individuell]
  \item Ausrichtung [Hoch- / Querformat]
  \item Position des Absenderbereichs [oben / unten]
  \item Position des Siegelbandlogos [links / rechts]
\end{itemize}

Daraus ergeben sich eine Vielzahl von Variationsmöglichkeiten.
Wie die einzelnen Optionen gewählt werden können, wird im folgenden
allgemein erläutert.
Auf die Positionierung des Siegelbandlogos gehen die Plakattyp-spezifischen
Kapitel näher ein.% TODO: so ok?

\subsection*{Papierformat}%

Prinzipiell stehen dem Anwender alle DIN-Formate von A0 bis A5 zu Verfügung.
Für Plakate sind jedoch speziell die Formate A0 bis A3 vorgesehen.

% Da auf wissenschaftlichen Plakaten in der Regel mehr Informationen untergebracht
% werden müssen als auf normalen Veranstaltungsplakaten, gibt es hierfür
% speziell angepasste Schriftgrößen für die Formate A0 bis A2.

\begin{Declaration}
  \Option{a4paper}\\
  \Option{a3paper}\\
  \Option{a2paper}\\
  \Option{a1paper}\\
  \Option{a0paper}
\end{Declaration}

Für alle Poster-Varianten stehen vier Papierformate zur Auswahl für die
Schriftgrößen etc. vordefiniert sind und automatisch geladen werden.
Dies sind A4, A3, A2, A1 und A0, die mit den entsprechenden Optionen
\Option{a3paper}, \Option{a2paper}, \Option{a1paper} bzw. \Option{a0paper}
gewählt werden können.
Wird kein Papierformat angegeben, wird automatisch A4 gewählt.

Eine Übersicht über die vordefinierten Schriftgrößen für die einzelnen
Papierformate gibt \tablename~\ref{table:fontsizes}.

\begin{hint}
Es wird generell dazu geraten, \emph{nur} das gewünschte Papierformat
anzugeben und \emph{keine explizite Schriftgrößenwahl} vorzunhemen.
Dies stellt sicher, dass (sofern möglich) die korrekten Schriftgrößen
automatisch passend zum angegebenen Papierformat geladen werden.
\end{hint}

\begin{hint}
  Für wissenschaftliche Plakate gibt es spezielle Schriftgrößen-Definitionen.
  Diese liegen nur für die Formate A2, A1 und A0 vor.
\end{hint}

\paragraph{Individuelle Papierformate}
\Index{Format!inidviduell}
\Index{Papierformat!individuell}

Die Vorlagen unterstützen auch weitgehend die Angabe individueller Papiergrößen.
Dazu kann die Darstellungsgröße mit den folgenden Klassenoptionen
gewählt werden:
\begin{Declaration}
\KOption{paper}\PValue{width:height}\\
\KOption{paperwidth}\PValue{width}\\
\KOption{paperheight}\PValue{height}
\end{Declaration}

Die Maße des Posters werden automatisch CD-konform an das eingestellte
Papierformat angepasst.
Auch das Siegelband wird entsprechend skaliert und gesetzt. 
% TODO: What happens to Text size?

\subsection*{Ausrichtung}

Wenn keine Option angegeben ist, werden Plakate im Hochformat (portrait)
dargestellt.

\begin{Declaration}
  \Option{landscape}
\end{Declaration}

Die Option \Option{landscape} erlaubt die Umschaltung auf eine
querformatige Darstellung.

\begin{hint}
Die Wahl der Ausrichtung hat unter anderem bei Veranstaltungsplakaten
Einfluss auf die Anzahl der verfügbaren Gauß-Segmente.
Währen im Hochformat 8 Segmente vorgesehen sind, sind es im Querformat nur
6. Dies muss bei der Layout-Entwicklung berücksichtigt werden.
\end{hint}

\subsection*{Position des Absenderbereichs}

Der Absenderbereich ist der weiß hinterlegte freie Platz am oberen oder
unteren Seitenende.
Er wird ausschließlich zur Darstellung eines speziellen Absenders
(ein Institut oder eine zentrale Einrichtung der TU) verwendet.

\begin{Declaration}
  \KOption{sender}\PName{top/bottom}
\end{Declaration}

Die Option \Option{sender} legt fest, ob der Absenderbereich am oberen (\PValue{top})
oder unteren (\PValue{bottom} Blattrand platziert werden soll.

Die Position des Absenderbereichs kontrolliert beim Gauß-Layout auch die
Reihenfolge der Segmentaufteilung.
Für weitere Informationen siehe \chaptername~\ref{sec:intro:dascddertubs}:

\subsection*{Schriftauszeichnung}

Für die verschiedenen Gliederungsebenen der Plakate gibt es vordefiniert
Font-Elemente, die sicherstellen, dass sowohl die richtige Schriftgröße
als auch die korrekte Art und Auszeichnung gesetzt wird.

Eine Übesichtsliste über alle definierten Font-Elemente, welche auch
für Plakate gültig ist, findet sich in \tablename~\ref{tbl:documents:komafonts}.
Die Benutzung wird in \chaptername~\ref{par:documents:schriftauszeichnung} erläutert.
% TODO: Modul-Elemente?

% TODO: place where??
\begin{Declaration}
  \Macro{headline}\OParameter{small}\PParameter{Text}\\
  \Macro{subheadline}\OParameter{small}\PParameter{Text}
\end{Declaration}

Für die Darstellung von Überschriften stehen die vereinfachenden
Befehle \Macro{headline} und \Macro{subheadline} zur Verfügung,
die zusätzlich noch eine korrekte Ausrichtung (rechts flatternd) sicherstellen.
Die zusätzliche Option \Option{small} wechselt auf die jeweils alternativ
definierte Version mit kleinerer Schriftgröße.


\paragraph{Weitere Klassenoptionen:}

Wie bei allen anderen \tubslatex-Klassen können auch für Poster
die diversen verfügbaren Optionen zur Farbdarstellung oder Schriftartwahl
verwendet werden.
Näheres hierzu findet sich in den vorausgehenden Kapiteln oder detaillierter
in Kapitel~\ref{chap:tubscolors} und \ref{chap:nexus}.
\bigskip

Im folgenden werden die einzelnen Poster-Typen einzeln vorgestellt und beschrieben.

\clearpage
\section{Veranstaltungsplakate}
\Index{Veranstaltungsplakate}
\Index{Plakate!Veranstaltungs-}

Veranstaltungsplakate werden im Gaußraster gesetzt und entsprechen somit
im Grunde allen Standarddarstellungen des Corporate Design.

Veranstaltungsposter sind die Standarddarstellung, wenn als Klassenoption
kein bestimmter Poster-Stil gewählt wurde. Sie können aber auch explizit
unter Angabe der Optionswahl \KOption{style}\PName{standard} gewählt werden:
\begin{lstlisting}
\documentclass[style=standard]{tubsposter}
\end{lstlisting}

Da das Gauß-Layoutsystem zum einen umfangreich, zum anderen auch für
verschiedenste Dokumentenklassen eingesetzt wird,
ist ihm ein Extra-Kapitel (\ref{chap:gausspage}) gewidmet.
Dies beschreibt die möglichen Befehle und Optionen ausführlich.
Hier werden sie daher jeweils nur verkürzt erläutert.
Eine Lektüre des genannten Kapitels empfiehlt sich daher in jedem Fall.


\begin{Declaration}
  \XMacro{begin}\PParameter{\Environment{tubsposter}}%
    \OParameter{options}\\
  \quad\dots\\
  \XMacro{end}\PParameter{tubsposter}
\end{Declaration}

Ein neues Plakat wird mit der Umgebung \Environment{tubsposter} erstellt.
Der optionale Parameter \PName{options} akzeptiert dabei die
unter \ref{sec:gausspage:bglayout} beschriebenen Optionen.

Der Inhalt des Posters sollte aus einem oder mehreren Segmenten bestehen.
Diese können jeweils aus einem oder mehreren zusammengefassten Basis-Segmenten
des Gaußrasters bestehen. Dabei sollten alle Basis-Segmente verwendet werden.

\begin{sloppypar}
Das Siegelband-Logo kann durch Verwendung des Befehls \Macro{showtubslogo}
innerhalb der \Environment{tubsposter}-Umgebung angezeigt werden.
Das Institutslogo wird analog mit \Macro{showlogo}\Parameter{Inhalt}
dargestellt. Hierzu gibt das Kapitel~\ref{subsec:gausspage:bgelement}
weitere Details.
\end{sloppypar}
%TODO: ref, link, ?
% sender
% bgcolor?

\begin{Declaration}
  \XMacro{begin}\PParameter{\Environment{modrow}}%
    \OParameter{options}
    \Parameter{Höhe}\\
  \quad\dots\\
  \XMacro{end}\PParameter{modrow}%\\
%   \XMacro{begin}\PParameter{\Environment{modrow*}}%
%     \OParameter{cols}\\%
%   \quad\dots\\
%   \XMacro{end}\PParameter{modrow*}
\end{Declaration}

Einzelne Poster-Segmente im Gaußraster können mit der Umgebung
\Environment{modrow} erzeugt werden.
Dabei gibt der Parameter \PName{Höhe} die Höhe des
Poster-Segments in Basis-Segmenten des Gaußrasters an.
Der optionale Parameter \PName{options} akzeptiert die
unter \ref{subsec:gausspage:bgelement} beschriebenen Optionen mit
denen zum Beispiel recht einfach eine Hintergrundfarbe oder ein Hintergrundbild
gesetzt werden kann.

\begin{figure}[!ht]
\begin{minipage}{0.65\textwidth}
\begin{lstlisting}
\documentclass{tubsposter}
\begin{document}
\begin{tubsposter}
\showtubslogo
\begin{modrow}[bgcolor=tuGreen]{3}
\usekomafont{headline}Der Titel
\end{modrow}
\begin{modrow}[bgcolor=tuOrange]{4}
\end{modrow}
\begin{modrow}[bgcolor=tuGreenDark]{1}
\end{modrow}
\end{tubsposter}
\end{document}
\end{lstlisting}
\end{minipage}
\begin{minipage}{0.35\textwidth}
  \fboxsep0mm
  \fbox{\includegraphics[width=0.98\textwidth]{examples/stdposter_minimal.pdf}}
\end{minipage}
\caption{Minimal-Beispiel zur Verwendung von \Environment{tubsposter} und
  \Environment{modrow}}
\end{figure}


\clearpage
\section{Wissenschaftliche Plakate}
\Index{Plakate!wissenschaftlich}
\Index{Wissenschaftliche Plakate}
%TODO: Optionen und Schachtelung angeben

Wissenschaftliche Poster unterscheiden sich im Layout in diversen Punkten
von allen übrigen Vorlagen. Die Darstellung ist dazu optimiert, möglichst
viel Inhalt möglichst flexibel platzieren zu können.

Die Grundaufteilung in einen Absernder- und einen Kommunikationsbereich 
bleibt erhalten.
Jedoch wird der Absenderbereich zu Gunsten eines vergrößerten Kommunikationsbereichs
schmaler dargestellt mit einem auf 90\% (Hochformat) der Standardgröße
skalierten Siegelbandlogo versehen.
Der Kommunikationsbereich ist rahmenlos und komplett einfarbig hinterlegt.

Wissenschaftliche Poster werden immer unter Angabe der \Class{tubsposter}-Klassenoption
\KOption{style}\PName{scifi} erstellt:
\begin{lstlisting}
\documentclass[style=scifi]{tubsposter}
\end{lstlisting}

\subsection{Modulsystem}

Der Inhalt auf wissenschaftlichen Plakaten wird in sogenannten \emph{Modulen}
platziert.
Dies sind Kästen unterschiedlicher Größe, die durch einen schmalen Abstand optisch
voneinander getrennt sind und so die Informationseinheiten klar voneinander
abgrenzen.

Die Platzierung der Module erfolgt in \tubslatex in einem lockeren Rastersystem.
Dazu wird der Kommunikationsbereich zuerst vertikal in mehrere Blöcke (Modulzeilen)
beliebiger Höhe aufgeteilt, die jedoch zusammen den Inhaltsbereich
komplett ausfüllen müssen.
Jeder dieser einzelnen Blöcke kann dann bei Bedarf unabhängig von den anderen
wieder in unterschiedlich große Bereiche (Modulspalten) unterteilt unterteilt werden.
dass die einzelnen Bereiche in Zeilen, Spalten und Unterzeilen teilen kann.
Entsprechend können auch die Modulspalten wieder vertikal geteilt werden.
Eine tiefere Schachtelung ist aktuell leider nicht möglich.


Im Folgenden wird die Benutzung des Modulrasters genauer erläutert.

\begin{Declaration}
  \XMacro{begin}\PParameter{\Environment{tubsposter}}%
    \OParameter{Optionen}%
    \Parameter{Aufteilung}\\
  \quad\dots\\
  \XMacro{end}\PParameter{tubsposter}
\end{Declaration}

Zur Erstellung eines wissenschaftlichen Plakates wird die Umgebung
\Environment{tubsposter} verwendet, welche -- anders als bei Veranstaltungsplakaten --
einen zusätzlichen Parameter \PName{Aufteilung} erwartet.
Mit diesem wird die Anzahl an Modulzeilen bestimmt,
in die der Darstellungsbereich geteilt werden soll.
Dies geschieht mittels einer kommagetrennten Liste nach folgendem Prinzip,
das an die Tabellen-Umgebung \Environment{tabularx} angelehnt ist:
\begin{itemize}
  \item Jedes Element kann entweder eine \emph{Länge} oder der Buchstabe \Option{X} sein
  \item Die Anzahl der Elemente bestimmt die Anzahl der Modulzeilen
  \item Die Reihenfolge der Elemente entspricht der Reihenfolge der Modulzeilen
  \item Eine \emph{Länge} gibt an wie hoch die jeweilige Modulzeile sein soll
  \item Ein \Option{X} teilt den Restplatz gleichmäßig auf alle mit
    \Option{X} gewählten Modulzeilen auf.
\end{itemize}

\begin{hint}
Es empfiehlt sich, immer mindestens ein \Option{X} zu verwenden, da sonst
eine korrekte Aufteilung des verfügbaren Platzes manuell berechnet werden
müsste.
\end{hint}


\begin{minipage}{0.6\textwidth}
\begin{example}~\par
  \noindent\Macro{begin}\PParameter{\Environment{tubsposter}}
    \Parameter{6cm,X,4cm,X}\par
  \quad\dots\\
  \Macro{end}\PParameter{tubsposter}\bigskip\par
  \noindent Definiert 4 Modulzeilen, wobei die erste 6cm und die dritte 4cm hoch sind.
  Die Zeilen 2 und 4 nehmen dann jeweils die Hälfte des restlichen 
  noch zur Verfügung stehenden Platzes ein.
\end{example}
\end{minipage}
\hfill
\begin{minipage}{0.3\textwidth}\centering
  \examplegraphic[width=\textwidth,page=1]{examples/module_split.pdf}
\end{minipage}

Damit die definierten Modulzeilen nun auch wirklich dargestellt werden,
müssen sie mit Inhalt gefüllt werden. Dies geschieht mit der Umgebung
\Environment{modrow}, welche im folgenden erläutert wird.

\begin{Declaration}
  \XMacro{begin}\PParameter{\Environment{modrow}}%
    \OParameter{Optionen,Aufteilung}\\%
  \quad\dots\\
  \XMacro{end}\PParameter{modrow}\\
  \XMacro{begin}\PParameter{\Environment{modrow*}}%
    \OParameter{Optionen,Aufteilung}\\%
  \quad\dots\\
  \XMacro{end}\PParameter{modrow*}
\end{Declaration}

Die Umgebung \Environment{modrow} wird benutzt,
um die mit \Environment{tubsposter} angelegten Modulzeilen mit Inhalt zu füllen.
Die Box kann entweder direkt mit dem gewünschten Text-Inhalt gefüllt werden
oder der optionale Parameter \PName{Aufteilung} wird benutzt, um
eine weitere Aufteilung zu erwirken.

Als \PName{Aufteilung} wird wieder eine kommagetrennte Liste erwartet,
Deren Elemente haben dieselbe Bedeutung wie bereits beschrieben, außer, dass
sie nun zusätzliche \emph{Modulspalten} in der aktuellen Modulzeile definieren.

Wird die Box mit Inhalt gefüllt, beträgt der normale Abstand des Inhalts
vom Rand der Modulbox halbe Rahmenbreite.

Die Sternchen-Variante \Environment{modrow*} verhält sich identisch mit
der normalen Variant, bloß, dass der Inhalt hier ohne Abstand zum Rand der Box
dargestellt wird. 
% Dies ist zum Beispiel für das Einfügen von Bildern etwa kann es sinnvoll sein,
% diesen Rahmen wegzulassen. -> bgimage
% TODO: Optionen beschreiben

\begin{minipage}{0.6\textwidth}
\begin{example}~\par
  \noindent\Macro{begin}\PParameter{\Environment{modrow}}
    \OParameter{X,X}\par
  \quad\dots\\
  \Macro{end}\PParameter{modrow}\bigskip\par
  \noindent Teilt eine Modulzeile in 2 gleich große Spalten auf.
\end{example}
\end{minipage}
\hfill
\begin{minipage}{0.3\textwidth}\centering
  \examplegraphic[width=\textwidth,page=2]{examples/module_split.pdf}
\end{minipage}

\begin{Declaration}
  \XMacro{begin}\PParameter{\Environment{modcol}}%
    \OParameter{Optionen,Aufteilung}\\%
  \quad\dots\\
  \XMacro{end}\PParameter{modcol}\\
  \XMacro{begin}\PParameter{\Environment{modcol*}}%
    \OParameter{Optionen,Aufteilung}\\%
  \quad\dots\\
  \XMacro{end}\PParameter{modcol*}
\end{Declaration}

Die mit \Environment{modrow} angelegten Spalten können jeweils mit 
\Environment{modcol} mit Inhalt gefüllt oder wiederum in neue Unterzeilen
aufgeteilt werden.

Die Sternchen-Variante \Environment{modcol*} erlaubt wieder eine
randlose Darstellung des Inhalts.

\begin{minipage}{0.6\textwidth}
\begin{example}~\par
  \noindent\Macro{begin}\PParameter{\Environment{modcol}}
    \OParameter{3cm,X}\par
  \quad\dots\\
  \Macro{end}\PParameter{modcol}\bigskip\par
  \noindent Teilt eine Modulspalte in 2 Unterzeilen auf.
\end{example}
\end{minipage}
\hfill
\begin{minipage}{0.3\textwidth}\centering
  \examplegraphic[width=\textwidth,page=3]{examples/module_split.pdf}
\end{minipage}


\begin{Declaration}
  \XMacro{begin}\PParameter{\Environment{modsubrow}}\\%
  \quad\dots\\
  \XMacro{end}\PParameter{modsubrow}\\
  \XMacro{begin}\PParameter{\Environment{modsubrow*}}\\%
  \quad\dots\\
  \XMacro{end}\PParameter{modsubrow*}
\end{Declaration}

Die mit \Environment{modcol} angelegten Unterzeilen können jeweils mit 
\Environment{modsubrow} mit Inhalt gefüllt werden.

Es ist aktuell keine weitere Unterteilung möglich.


\clearpage
\section{Aushänge}

Einfache Aushänge sollen möglichst einfach und schnell zu erstellen sein.
Der Poster-Typ für Aushänge besitzt eine einfache voreingestellte Darstellung
mit Siegelband-Logo und einer roten Trennlinie zwischen dem Inhaltsbereich
und Absenderbereich.

Aushänge werden unter Angabe der Optionswahl
\KOption{style}\PName{bulletin} erstellt:
\begin{lstlisting}
\documentclass[style=bulletin]{tubsposter}
\end{lstlisting}

\begin{figure}[!ht]
\begin{minipage}{0.65\textwidth}
\begin{lstlisting}
\documentclass[a3paper,style=bulletin]{tubsposter}
\begin{document}
\begin{tubsposter}
  \vspace*{100pt}
  {\usekomafont{headline}
    Gait, commy nisi!\par\vspace*{1.5ex}}
  {\itshape\huge\color{tuRed}
    Dunt venim dolorerosto do odit
    ametum vulla conum dolore
    co\-nulput in vero od el\par}
  \vfill
  {\Large
    Iduissit Ium do eum digna:
    0531 123-4567}
\end{tubsposter}
\end{document}

\end{lstlisting}
\end{minipage}
\begin{minipage}{0.35\textwidth}
  \fboxsep0mm
  \fbox{\includegraphics[width=0.98\textwidth]{examples/posterstyle_bulletin.pdf}}
\end{minipage}
\caption{Minimal-Beispiel zur Verwendung von \Environment{tubsposter}-Stil 
  \Option{bulletin}}
\end{figure}

% \section{Seitenlayout}

% margin
% bcor
% sender=top/bottom


% \section{Format}

% \section{Farben}

% \section{Befehle}

% \section{Optionen}

\begin{table}\centering
\begin{tabular}{lll}
      & Dokumente/Plakate & wissenschaftl.\\
\midrule
A0    & 40pt  & 35pt  \\
A1    & 25pt  & 25pt  \\
A2    & 18pt  & 18pt  \\
A3    & 13pt  & --    \\
A4    & 9pt, 10pt, 11pt & --\\
A5/lang & 9pt, 10pt, 11pt & --\\
\end{tabular}
\caption{Übersicht über verfügbare Schriftgrößen}
\end{table}
