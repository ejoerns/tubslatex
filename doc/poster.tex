\chapter{Plakate}

Plakate können in \tubslatex mit Hilfe der Dokumentenklasse \newdocumentclass{tubsposter} erstellt werden.
Dabei werden drei unterschiedliche Plakat-Typen unterstützt,
zwischen denen mit der Klassenoption \KOption{style}\PName{<standard/scifi/bulletin>} gewählt werden kann.

Dies sind zum einen \emph{Veranstaltungsplakate},
welche sich am normalen Gaußraster
orientieren und als Standard-Darstellungstyp voreingestellt sind.

Für die Darstellung \emph{wissenschaftlicher} Inhalte gibt es ein spezielles Layout,
das mehr Platz und Flexibilität für die Unterbringung von Informationen bietet.
Es orientiert sich an einem eigenen \gls{glos:modulsystem} und nicht am
\gls{glos:gaussraster}.

Zusätzlich gibt es noch eine Unterstützung für die Erstellung einfacher
\emph{Aushänge}. Diese weisen nur einen einfachen großen Textbereich auf, der
nach oben durch das Siegelband-Logo mit einer Trennlinie begrenzt ist.

\begin{figure}[htp]\centering
\fboxsep0mm
\begin{minipage}{0.32\textwidth}\centering
  \fbox{\includegraphics[width=0.9\textwidth]{examples/posterstyle_standard.pdf}}
  \subcaption{Veranstaltung (\PValue{standard})}
\end{minipage}
\begin{minipage}{0.32\textwidth}\centering
  \fbox{\includegraphics[width=0.9\textwidth]{examples/posterstyle_scifi.pdf}}
  \subcaption{wissenschaftlich (\PValue{scifi})}
\end{minipage}
\begin{minipage}{0.32\textwidth}\centering
  \fbox{\includegraphics[width=0.9\textwidth]{examples/posterstyle_bulletin.pdf}}
  \subcaption{Aushang (\PValue{bulletin})}
\end{minipage}
\caption{Die verschiedenen Plakat-Typen}
\end{figure}

\begin{Declaration}
  \KOption{style}\PName{<standard/scifi/bulletin>}
\end{Declaration}

Mit der Klassenoption \KOption{style}\PName{Typ} kann zwischen
den verschiedenen Plakattypen gewählt werden.
Mit \KOption{style}{standard} wird die Darstellung für Veranstaltungsplakate
gewählt. Diese ist bereits voreingestellt und muss daher nicht explizit gewählt
werden.
Mit \KOption{style}{scifi} wir das Layout-System für wissenschaftliche Plakate
(Modulsystem) geladen, welches sich teilweise grundlegend von
dem normalen (Gaussraster-)Layout unterscheidet.
Der Wert \KOption{style}{bulletin} erlaubt die schnelle Erstellung von
Aushängen und lädt eine relative schlichte, voreingestellte Darstellung.

\begin{Declaration}
  \KOption{sender}\PName{top/bottom}
\end{Declaration}

Diese Option legt fest, ob der Absenderbereich am oberen oder unteren Blattrand
platziert werden soll. Für weitere Informationen siehe vorausgehende Kapitel.
%TODO: ja? wo?
%TODO: Klasse + Umgebung


\subsubsection{Papierformate}% TODO: fix level

Prinzipiell stehen dem Anwender alle DIN-Formate von A0 bis A5 zu Verfügung.
Für Plakate sind jedoch speziell die Formate A0 bis A3 vorgesehen.

Da auf wissenschaftlichen Plakate in der Regel mehr Informationen untergebracht
werden müssen als auf normalen Veranstaltungsplakaten, gibt es hierfür
speziell angepasste Schriftgrößen für die Formate A0 bis A2.

\begin{important}
Es wird generell dazu geraten, \emph{nur} das gewünschte Papierformat
anzugeben und \emph{keine explizite Schriftgrößenwahl} vorzunhemen.
Dies stellt sicher, dass (sofern möglich) die korrekten Schriftgrößen
automatisch passend zum angegebendne Papierformat geladen werden.
\end{important}



\begin{Declaration}
  \Option{a4paper}\\
  \Option{a3paper}\\
  \Option{a2paper}\\
  \Option{a1paper}\\
  \Option{a0paper}
\end{Declaration}

Für alle Poster-Varianten stehen vier Papierformate zur Auswahl für die
Schriftgrößen etc. vordefiniert sind und automatisch geladen werden.
Dies sind A4, A3, A2, A1 und A0, die mit den entsprechenden Optionen
\Option{a3paper}, \Option{a2paper}, \Option{a1paper} bzw. \Option{a0paper}
gewählt werden können.
Wird kein Papierformat angegeben, wird automatisch A4 gewählt.

Eine Übersicht über die vordefinierten Schriftgrößen für die einzelnen
Papierformate gibt Tabelle~\ref{table:fontsizes}.

\begin{important}
  Für wissenschaftliche Plakate gibt es spezielle Schriftgrößen-Definitionen.
  Diese liegen nur für die Formate A3, A2, A1 und A0 vor.
\end{important}

\paragraph{Individuelle Papierformate}
\Index{Format!inidviduell}
\Index{Papierformat!individuell}

Die Vorlagen unterstützen auch weitgehend die Angabe individueller Papiergrößen.
Dazu kann die Darstellungsgröße mit den folgenden Klassenoptionen
gewählt werden:
\begin{Declaration}
\KOption{paper}\PValue{width:height}\\
\KOption{paperwidth}\PValue{width}\\
\KOption{paperheight}\PValue{height}
\end{Declaration}

Die Maße des Posters werden automatisch CD-konform an das eingestellte
Papierformat angepasst.
Auch das Siegelband wird entsprechend skaliert und gesetzt. 
% TODO: What happens to Text size?

\begin{Declaration}
  \Macro{headline}\\
  \Macro{subheadline}
\end{Declaration}


\paragraph{Weitere Klassenoptionen:}

Wie bei allen anderen \tubslatex-Klassen können auch für Poster
die diversen verfügbaren Optionen zur Farbdarstellung oder Schriftartwahl
verwendet werden.
Näheres hierzu findet sich in den vorausgehenden Kapiteln oder detaillierter
in Kapitel~\ref{chap:tubscolors} und \ref{chap:nexus}.
\bigskip

Im folgenden werden die einzelnen Poster-Typen einzeln vorgestellt und beschrieben.

\clearpage
\section{Veranstaltungsplakate}

Veranstaltungsplakate werden im Gaußraster gesetzt und entsprechen somit
im Grunde allen Standarddarstellungen des Corporate Design.

Veranstaltungsposter werden entweder ohne Option oder
unter Angabe der Optionswahl \KOption{style}\PName{standard} erstellt:
\begin{lstlisting}
\documentclass[style=standard]{tubsposter}
\end{lstlisting}

Da das Gauß-Layoutsystem zum einen umfangreich, zum anderen auch für verschiedene Zwecke eingesetzt wird und werden kann, ist ihm ein Extra-Kapitel (\ref{chap:gausspage})
gewidmet.
Dies beschreibt die möglichen Befehle und Optionen ausführlich.
Hier werden sie daher jeweils nur kurz angerissen.

% TODO: landscape?

\begin{Declaration}
  \XMacro{begin}\PParameter{\Environment{tubsposter}}%
    \OParameter{options}\\
  \quad\dots\\
  \XMacro{end}\PParameter{tubsposter}
\end{Declaration}

Ein neues Plakat wird mit der Umgebung \Environment{tubsposter} erstellt.
Der optionale Parameter \PName{options} akzeptiert dabei die
unter \ref{sec:gausspage:bglayout} beschriebenen Optionen.

Der Inhalt des Posters sollte aus einem oder mehreren Segmenten bestehen.
Diese können jeweils aus einem oder mehreren zusammengefassten Basis-Segmenten
des Gaußrasters bestehen. Dabei sollten alle Basis-Segmente verwendet werden.

\begin{sloppypar}
Das Siegelband-Logo kann durch Verwendung des Befehls \Macro{showtubslogo}
innerhalb der \Environment{tubsposter}-Umgebung angezeigt werden.
Das Institutslogo wird analog mit \Macro{showlogo}\Parameter{Inhalt}
dargestellt. Hierzu gibt das Kapitel~\ref{subsec:gausspage:bgelement}
weitere Details.
\end{sloppypar}
%TODO: ref, link, ?
% sender
% bgcolor?

\begin{Declaration}
  \XMacro{begin}\PParameter{\Environment{posterrow}}%
    \OParameter{options}
    \Parameter{Höhe}\\
  \quad\dots\\
  \XMacro{end}\PParameter{posterrow}%\\
%   \XMacro{begin}\PParameter{\Environment{posterrow*}}%
%     \OParameter{cols}\\%
%   \quad\dots\\
%   \XMacro{end}\PParameter{posterrow*}
\end{Declaration}

Einzelne Poster-Segmente im Gaußraster können mit der Umgebung
\Environment{posterrow} erzeugt werden.
Dabei gibt der Parameter \PName{Höhe} die Höhe des
Poster-Segments in Basis-Segmenten des Gaußrasters an.
Der optionale Parameter \PName{options} akzeptiert die
unter \ref{subsec:gausspage:bgelement} beschriebenen Optionen mit
denen zum Beispiel recht einfach eine Hintergrundfarbe oder ein Hintergrundbild
gesetzt werden kann.

\begin{figure}[!ht]
\begin{minipage}{0.65\textwidth}
\begin{lstlisting}
\documentclass{tubsposter}
\begin{document}
\begin{tubsposter}
\showtubslogo
\begin{posterrow}[bgcolor=tuGreen]{3}
\usekomafont{headline}Der Titel
\end{posterrow}
\begin{posterrow}[bgcolor=tuOrange]{4}
\end{posterrow}
\begin{posterrow}[bgcolor=tuGreenDark]{1}
\end{posterrow}
\end{tubsposter}
\end{document}
\end{lstlisting}
\end{minipage}
\begin{minipage}{0.35\textwidth}
  \fboxsep0mm
  \fbox{\includegraphics[width=0.98\textwidth]{examples/stdposter_minimal.pdf}}
\end{minipage}
\caption{Minimal-Beispiel zur Verwendung von \Environment{tubsposter} und
  \Environment{posterrow}}
\end{figure}


\clearpage
\section{Wissenschaftliche Plakate}
%TODO: Optionen und Schachtelung angeben
% - A2, A1, A0

Wissenschaftliche Poster unterscheiden sich im Layout in diversen Punkten
von allen übrigen Vorlagen. Die Darstellung ist dazu optimiert möglichst
viel Inhalt möglichst flexibel platzieren zu können.

Das Grundlayout weist einen schmaleren Absenderbereich mit einem auf 90\%
skalierten Siegelband-Logo auf.
Der Inhaltsbereich ist rahmenlos und komplett einfarbig hinterlegt.

Der Inhalt kann in sogenannten \emph{Modulen} platziert werden.
Dies sind Kästchen, die durch einen schmalen Abstand optisch
voneinander getrennt sind und so die Informationseinheiten klar voneinander
trennen.
Die Platzierung der Module erfolgt in einem lockeren Rastersystem,
dass die einzelnen Bereiche in Zeilen, Spalten und Unterzeilen teilen kann.

Wissenschaftliche Poster werden unter Angabe der Optionswahl
\KOption{style}\PName{scifi} erstellt:
\begin{lstlisting}
\documentclass[style=scifi]{tubsposter}
\end{lstlisting}

\begin{Declaration}
  \XMacro{begin}\PParameter{\Environment{tubsposter}}%
    \OParameter{options}%
    \Parameter{rows}\\
  \quad\dots\\
  \XMacro{end}\PParameter{tubsposter}
\end{Declaration}

Für die Erstellung von wissenschaftlichen Plakaten wird die Umgebung
\Environment{tubsposter} verwendet, welche
(anders als bei Veranstaltungsplakaten) einen
zusätzlichen Parameter \PName{rows} erwartet.
Mit diesem wird die Anzahl an Modulzeilen bestimmt, in die der Darstellungsbereich geteilt werden soll. Dies geschieht mittels einer kommagetrennten Liste,
wobei jedes Element entweder eine Länge oder der Buchstabe '\texttt{X}' sein kann.
Eine Länge legt die Höhe der jeweiligen Modulzeile genau fest,
ein \texttt{X} sorgt dafür, dass alle mit \texttt{X} gekennzeichneten
Zeilen den restlichen zur Verfügung stehenden Platz gleichmäßig untereinander
aufteilen.
Dieses Vorgehen ist an die Tabellen-Umgebung \Environment{tabularx} angelehnt.

\begin{minipage}{0.6\textwidth}
\begin{Example}\par
  \noindent\Macro{begin}\PParameter{\Environment{tubsposter}}
    \Parameter{6cm,X,4cm,X}\par
  \quad\dots\\
  \Macro{end}\PParameter{tubsposter}\bigskip\par
  \noindent Erzeugt 4 Modulzeilen, wobei die erste 6cm und die dritte 4cm hoch sind.
  Die Zeilen 2 und 4 nehmen den restlichen verfügbaren Platz ein
  und sind gleich hoch.
\end{Example}
\end{minipage}
\hfill
\begin{minipage}{0.3\textwidth}\centering
  \examplegraphic[width=\textwidth,page=1]{examples/module_split.pdf}
\end{minipage}

Die angegebenen Modulzeilen müssen dann mit der Umgebung \Environment{posterrow}
befüllt werden.

\begin{Declaration}
  \XMacro{begin}\PParameter{\Environment{posterrow}}%
    \OParameter{cols}\\%
  \quad\dots\\
  \XMacro{end}\PParameter{posterrow}\\
  \XMacro{begin}\PParameter{\Environment{posterrow*}}%
    \OParameter{cols}\\%
  \quad\dots\\
  \XMacro{end}\PParameter{posterrow*}
\end{Declaration}

Die Umgebung \Environment{posterrow} wird benutzt,
um die mit \Environment{tubsposter} angelegten Modulzeilen mit Inhalt zu füllen.
Dabei kann entweder direkt der gewünschte Inhalt geschrieben oder der optionale
Parameter \PName{cols} benutzt werden.
Dieses erlaubt die Definition zusätzlicher Spalten in der aktuellen Modulzeile.
Es wird wieder ein kommagetrennte Liste erwartet,
deren Elemente dieselbe Bedeutung haben wie bereits beschrieben, außer, dass
sie die Breite und nicht die Höhe definieren.

Der normale Abstand des Inhalts vom Rand der Modulbox beträgt halbe
Rahmenbreite. Für das Einfügen von Bildern etwa kann es sinnvoll sein,
diesen Rahmen wegzulassen.
Dies geschieht mit der Sternchen-Variante \Environment{posterrow*}.


\begin{minipage}{0.6\textwidth}
\begin{Example}\par
  \noindent\Macro{begin}\PParameter{\Environment{posterrow}}
    \OParameter{X,X}\par
  \quad\dots\\
  \Macro{end}\PParameter{posterrow}\bigskip\par
  \noindent Teilt eine Modulzeile in 2 gleich große Spalten auf.
\end{Example}
\end{minipage}
\hfill
\begin{minipage}{0.3\textwidth}\centering
  \examplegraphic[width=\textwidth,page=2]{examples/module_split.pdf}
\end{minipage}

\begin{Declaration}
  \XMacro{begin}\PParameter{\Environment{postercol}}%
    \OParameter{rows}\\%
  \quad\dots\\
  \XMacro{end}\PParameter{postercol}\\
  \XMacro{begin}\PParameter{\Environment{postercol*}}%
    \OParameter{rows}\\%
  \quad\dots\\
  \XMacro{end}\PParameter{postercol*}
\end{Declaration}

Die mit \Environment{posterrow} angelegten Spalten können jeweils mit 
\Environment{postercol} mit Inhalt gefüllt oder in neue Unterzeilen aufgeteilt 
werden.

\begin{minipage}{0.6\textwidth}
\begin{Example}\par
  \noindent\Macro{begin}\PParameter{\Environment{postercol}}
    \OParameter{3cm,X}\par
  \quad\dots\\
  \Macro{end}\PParameter{postercol}\bigskip\par
  \noindent Teilt eine Modulspalte in 2 Unterzeilen auf.
\end{Example}
\end{minipage}
\hfill
\begin{minipage}{0.3\textwidth}\centering
  \examplegraphic[width=\textwidth,page=3]{examples/module_split.pdf}
\end{minipage}


\begin{Declaration}
  \XMacro{begin}\PParameter{\Environment{postersubrow}}\\%
  \quad\dots\\
  \XMacro{end}\PParameter{postersubrow}\\
  \XMacro{begin}\PParameter{\Environment{postersubrow*}}\\%
  \quad\dots\\
  \XMacro{end}\PParameter{postersubrow*}
\end{Declaration}

Die mit \Environment{postercol} angelegten Unterzeilen können jeweils mit 
\Environment{postersubrow} mit Inhalt gefüllt werden.


\clearpage
\section{Aushänge}

Einfache Aushänge sollen möglichst einfach und schnell zu erstellen sein.
Der Poster-Typ für Aushänge besitzt eine einfache voreingestellte Darstellung
mit Siegelband-Logo und einer roten Trennlinie zwischen dem Inhaltsbereich
und Absenderbereich.

Aushänge werden unter Angabe der Optionswahl
\KOption{style}\PName{bulletin} erstellt:
\begin{lstlisting}
\documentclass[style=bulletin]{tubsposter}
\end{lstlisting}

\begin{figure}[!ht]
\begin{minipage}{0.65\textwidth}
\begin{lstlisting}
\documentclass[a3paper,style=bulletin]{tubsposter}
\begin{document}
\begin{tubsposter}
  \vspace*{100pt}
  {\usekomafont{headline}
    Gait, commy nisi!\par\vspace*{1.5ex}}
  {\itshape\huge\color{tuRed}
    Dunt venim dolorerosto do odit
    ametum vulla conum dolore
    co\-nulput in vero od el\par}
  \vfill
  {\Large
    Iduissit Ium do eum digna:
    0531 123-4567}
\end{tubsposter}
\end{document}

\end{lstlisting}
\end{minipage}
\begin{minipage}{0.35\textwidth}
  \fboxsep0mm
  \fbox{\includegraphics[width=0.98\textwidth]{examples/posterstyle_bulletin.pdf}}
\end{minipage}
\caption{Minimal-Beispiel zur Verwendung von \Environment{tubsposter}-Stil 
  \Option{bulletin}}
\end{figure}

% \section{Seitenlayout}

% margin
% bcor
% sender=top/bottom


% \section{Format}

% \section{Farben}

% \section{Befehle}

% \section{Optionen}

\begin{table}\centering
\begin{tabular}{lll}
      & Dokumente/Plakate & wissenschaftl.\\
\midrule
A0    & 40pt  & 35pt  \\
A1    & 25pt  & 25pt  \\
A2    & 18pt  & 18pt  \\
A3    & 13pt  & --    \\
A4    & 9pt, 10pt, 11pt & --\\
A5/lang & 9pt, 10pt, 11pt & --\\
\end{tabular}
\caption{Übersicht über verfügbare Schriftgrößen}
\end{table}
