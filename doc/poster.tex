\chapter{Plakate}
\Index{Plakate|indexbf}
\Index{Poster}
% TODO: Note: mehrere Plakate pro Dokument möglich!

Plakate können in \tubslatex mit Hilfe der Dokumentenklasse \newdocumentclass{tubsposter} erstellt werden.
Dabei werden drei unterschiedliche Plakat-Typen unterstützt,
zwischen denen mit der Klassenoption \KOption{style}\PName{<standard/scifi/bulletin>} gewählt werden kann.

Dies sind zum einen \emph{Veranstaltungsplakate},
welche sich am normalen Gaußraster
orientieren und als Standard-Darstellungstyp voreingestellt sind.

Für die Darstellung \emph{wissenschaftlicher} Inhalte gibt es ein spezielles Layout,
das mehr Platz und Flexibilität für die Unterbringung von Informationen bietet.
Es orientiert sich an einem eigenen \gls{glos:modulsystem} und nicht am
\gls{glos:gaussraster}.

Zusätzlich gibt es noch eine Unterstützung für die Erstellung einfacher
\emph{Aushänge}. Diese weisen nur einen einfachen großen Textbereich auf, der
nach oben durch das Siegelband-Logo mit einer Trennlinie begrenzt ist.

\begin{figure}[htp]\centering
\fboxsep0mm
\begin{minipage}{0.32\textwidth}\centering
  \fbox{\includegraphics[width=0.9\textwidth]{examples/posterstyle_standard.pdf}}
  \subcaption{Veranstaltung (\PValue{standard})}
\end{minipage}
\begin{minipage}{0.32\textwidth}\centering
  \fbox{\includegraphics[width=0.9\textwidth]{examples/posterstyle_scifi.pdf}}
  \subcaption{wissenschaftlich (\PValue{scifi})}
\end{minipage}
\begin{minipage}{0.32\textwidth}\centering
  \fbox{\includegraphics[width=0.9\textwidth]{examples/posterstyle_bulletin.pdf}}
  \subcaption{Aushang (\PValue{bulletin})}
\end{minipage}
\caption{Die verschiedenen Plakat-Typen}
\end{figure}

\begin{Declaration}
  \KOption{style}\PName{<standard/scifi/bulletin>}
\end{Declaration}

Mit der Klassenoption \KOption{style}\PName{Typ} kann zwischen
den verschiedenen Plakattypen gewählt werden.
Mit \KOption{style}{standard} wird die Darstellung für Veranstaltungsplakate
gewählt. Diese ist bereits voreingestellt und muss daher nicht explizit gewählt
werden.
Mit \KOption{style}{scifi} wir das Layout-System für wissenschaftliche Plakate
(Modulsystem) geladen, welches sich teilweise grundlegend von
dem normalen (Gaussraster-)Layout unterscheidet.
Der Wert \KOption{style}{bulletin} erlaubt die schnelle Erstellung von
Aushängen und lädt eine relative schlichte, voreingestellte Darstellung.


\section{Format und Darstellung}

Allen Plakatvarianten gleich sind die grundlegenden Varianten in der Formatwahl,
die am Anfang eines jede Plakat-Designprozesses stehen sollten.
Dazu gehören unter anderem:

\begin{itemize}
  \item Papierformat [DIN\;Ax / individuell]
  \item Ausrichtung [Hoch- / Querformat]
  \item Position des Absenderbereichs [oben / unten]
  \item Position des Siegelbandlogos [links / rechts]
\end{itemize}

Daraus ergeben sich eine Vielzahl von Variationsmöglichkeiten.
Wie die einzelnen Optionen gewählt werden können, wird im folgenden
allgemein erläutert.
Auf die Positionierung des Siegelbandlogos gehen die Plakattyp-spezifischen
Kapitel näher ein.% TODO: so ok?

\subsection*{Papierformat}%

Prinzipiell stehen dem Anwender alle DIN-Formate von A0 bis A5 zu Verfügung.
Für Plakate sind jedoch speziell die Formate A0 bis A3 vorgesehen.

% Da auf wissenschaftlichen Plakaten in der Regel mehr Informationen untergebracht
% werden müssen als auf normalen Veranstaltungsplakaten, gibt es hierfür
% speziell angepasste Schriftgrößen für die Formate A0 bis A2.

\begin{Declaration}
  \Option{a4paper}\\
  \Option{a3paper}\\
  \Option{a2paper}\\
  \Option{a1paper}\\
  \Option{a0paper}
\end{Declaration}

Für alle Plakat-Varianten stehen vier Papierformate zur Auswahl für die
Schriftgrößen etc. vordefiniert sind und automatisch geladen werden.
Dies sind A4, A3, A2, A1 und A0, die mit den entsprechenden Optionen
\Option{a3paper}, \Option{a2paper}, \Option{a1paper} bzw. \Option{a0paper}
gewählt werden können.
Wird kein Papierformat angegeben, wird automatisch A4 gewählt.

Eine Übersicht über die vordefinierten Schriftgrößen für die einzelnen
Papierformate gibt \tablename~\ref{table:fontsizes}.

\begin{hint}
Es wird generell dazu geraten, \emph{nur} das gewünschte Papierformat
anzugeben und \emph{keine explizite Schriftgrößenwahl} vorzunhemen.
Dies stellt sicher, dass (sofern möglich) die korrekten Schriftgrößen
automatisch passend zum angegebenen Papierformat geladen werden.
\end{hint}

\begin{hint}
  Für wissenschaftliche Plakate gibt es spezielle Schriftgrößen-Definitionen.
  Diese liegen nur für die Formate A2, A1 und A0 vor.
\end{hint}

\paragraph{Individuelle Papierformate}
\Index{Format!inidviduell}
\Index{Papierformat!individuell}

Die Vorlagen unterstützen auch weitgehend die Angabe individueller Papiergrößen.
Dazu kann die Darstellungsgröße mit den folgenden Klassenoptionen
gewählt werden:
\begin{Declaration}
\KOption{paper}\PValue{width:height}\\
\KOption{paperwidth}\PValue{width}\\
\KOption{paperheight}\PValue{height}
\end{Declaration}

Die Maße des Plakates werden automatisch CD-konform an das eingestellte
Papierformat angepasst.
Auch das Siegelband wird entsprechend skaliert und gesetzt. 
% TODO: What happens to Text size?

\subsection*{Ausrichtung}

Wenn keine Option angegeben ist, werden Plakate im Hochformat (portrait)
dargestellt.

\begin{Declaration}
  \Option{landscape}
\end{Declaration}

Die Option \Option{landscape} erlaubt die Umschaltung auf eine
querformatige Darstellung.

\begin{hint}
Die Wahl der Ausrichtung hat unter anderem bei Veranstaltungsplakaten
Einfluss auf die Anzahl der verfügbaren Gauß-Segmente.
Währen im Hochformat 8 Segmente vorgesehen sind, sind es im Querformat nur
6. Dies muss bei der Layout-Entwicklung berücksichtigt werden.
\end{hint}

\subsection*{Position des Absenderbereichs}\label{sec:poster:sender}

Der Absenderbereich ist der weiß hinterlegte freie Platz am oberen oder
unteren Seitenende.
Er wird ausschließlich zur Darstellung eines speziellen Absenders
(ein Institut oder eine zentrale Einrichtung der TU) verwendet.

\begin{Declaration}
  \KOption{sender}\PName{top/bottom}
\end{Declaration}

Die Option \Option{sender} legt fest, ob der Absenderbereich am oberen (\PValue{top})
oder unteren (\PValue{bottom} Blattrand platziert werden soll.

Die Position des Absenderbereichs kontrolliert beim Gauß-Layout auch die
Reihenfolge der Segmentaufteilung.
Für weitere Informationen siehe \chaptername~\ref{sec:intro:dascddertubs}:

\subsection*{Schriftauszeichnung}\label{sec:poster:schriftauszeichnung}

Für die verschiedenen Gliederungsebenen der Plakate gibt es vordefiniert
Font-Elemente, die sicherstellen, dass sowohl die richtige Schriftgröße
als auch die korrekte Art und Auszeichnung gesetzt wird.

Eine Übersichtsliste über alle definierten Font-Elemente, welche auch
für Plakate gültig ist, findet sich in \tablename~\ref{tbl:documents:komafonts}.
Die Benutzung wird in \chaptername~\ref{par:documents:schriftauszeichnung} erläutert.
% TODO: Modul-Elemente?

% TODO: place where??
\begin{Declaration}
  \Macro{headline}\OParameter{small}\PParameter{Text}\\
  \Macro{subheadline}\OParameter{small}\PParameter{Text}
\end{Declaration}

Für die Darstellung von Überschriften stehen die vereinfachenden
Befehle \Macro{headline} und \Macro{subheadline} zur Verfügung,
die zusätzlich noch eine korrekte Ausrichtung (rechts flatternd) sicherstellen.
Die zusätzliche Option \Option{small} wechselt auf die jeweils alternativ
definierte Version mit kleinerer Schriftgröße.


\paragraph{Weitere Klassenoptionen:}

Wie bei allen anderen \tubslatex-Klassen können auch für Plakate
die diversen verfügbaren Optionen zur Farbdarstellung oder Schriftartwahl
verwendet werden.
Näheres hierzu findet sich in den vorausgehenden Kapiteln oder detaillierter
in \chaptername~\ref{sec:tubscolors} und \ref{sec:nexus}.
\bigskip

Im folgenden werden die einzelnen Plakat-Typen einzeln vorgestellt und beschrieben.

\clearpage
\section{Veranstaltungsplakate}
\Index{Veranstaltungsplakate}
\Index{Plakate!Veranstaltungs-}

Veranstaltungsplakate werden im Gaußraster gesetzt und entsprechen somit
im Grunde allen Standarddarstellungen des Corporate Design.

Veranstaltungsplakate sind die Standarddarstellung, wenn als Klassenoption
kein bestimmter Plakat-Stil gewählt wurde. Sie können aber auch explizit
unter Angabe der Optionswahl \KOption{style}\PName{standard} gewählt werden:
\begin{lstlisting}
\documentclass[style=standard]{tubsposter}
\end{lstlisting}

Da das Gauß-Layoutsystem zum einen umfangreich, zum anderen auch für
verschiedenste Dokumentenklassen eingesetzt wird,
ist ihm ein Extra-Kapitel (\ref{sec:gausslayout}) gewidmet.
Dies beschreibt die möglichen Befehle und Optionen ausführlich.
Hier werden sie daher jeweils nur verkürzt erläutert.
Eine Lektüre des genannten Kapitels empfiehlt sich daher in jedem Fall.


\begin{Declaration}
  \XMacro{begin}\PParameter{\Environment{tubsposter}}%
    \OParameter{Optionen}\\
  \quad\dots\\
  \XMacro{end}\PParameter{tubsposter}
\end{Declaration}

Ein neues Plakat wird mit der Umgebung \Environment{tubsposter} erstellt.
Der optionale Parameter \PName{Optionen} akzeptiert dabei die
unter \ref{sec:gausslayout:bglayout} beschriebenen Optionen.

Der Inhalt des Plakats sollte aus einem oder mehreren Segmenten bestehen.
Diese können jeweils aus einem oder mehreren zusammengefassten Basis-Segmenten
des Gaußrasters bestehen. Dabei sollten alle Basis-Segmente verwendet werden.

\begin{sloppypar}
Das Siegelband-Logo kann durch Verwendung des Befehls \Macro{showtubslogo}
innerhalb der \Environment{tubsposter}-Umgebung angezeigt werden.
Das Institutslogo wird analog mit \Macro{showlogo}\Parameter{Inhalt}
dargestellt. Hierzu gibt das Kapitel~\ref{sec:gausslayout:bgsegment}
weitere Details.
\end{sloppypar}
%TODO: ref, link, ?
% sender
% bgcolor?

\begin{Declaration}
  \XMacro{begin}\PParameter{\Environment{segment}}%
    \OParameter{Optionen}
    \Parameter{Höhe}\\
  \quad\dots\\
  \XMacro{end}\PParameter{segment}%\\
%   \XMacro{begin}\PParameter{\Environment{segment*}}%
%     \OParameter{cols}\\%
%   \quad\dots\\
%   \XMacro{end}\PParameter{segment*}
\end{Declaration}

Einzelne Plakat-Segmente im Gaußraster können mit der Umgebung
\Environment{segment} erzeugt werden.
Dabei gibt der Parameter \PName{Höhe} die Höhe des
Plakat-Segments in Basis-Segmenten des Gaußrasters an.
Der optionale Parameter \PName{Optionen} akzeptiert die
unter \ref{sec:gausslayout:bgsegment} beschriebenen Optionen mit
denen zum Beispiel recht einfach eine Hintergrundfarbe oder ein Hintergrundbild
gesetzt werden kann.

\begin{figure}[!ht]
\begin{minipage}{0.65\textwidth}
\begin{lstlisting}
\documentclass{tubsposter}
\begin{document}
\begin{tubsposter}
\showtubslogo
\begin{segment}[bgcolor=tuGreen]{3}
\usekomafont{headline}Der Titel
\end{segment}
\begin{segment}[bgcolor=tuOrange]{4}
\end{segment}
\begin{segment}[bgcolor=tuGreenDark]{1}
\end{segment}
\end{tubsposter}
\end{document}
\end{lstlisting}
\end{minipage}
\begin{minipage}{0.35\textwidth}
  \fboxsep0mm
  \fbox{\includegraphics[width=0.98\textwidth]{examples/stdposter_minimal.pdf}}
\end{minipage}
\caption{Minimal-Beispiel zur Verwendung von \Environment{tubsposter} und
  \Environment{segment}}
\end{figure}


\clearpage
\section{Wissenschaftliche Plakate}
\Index{Plakate!wissenschaftlich}
\Index{Wissenschaftliche Plakate}
%TODO: Optionen und Schachtelung angeben

Wissenschaftliche Plakate unterscheiden sich im Layout in diversen Punkten
von allen übrigen Vorlagen. Die Darstellung ist dazu optimiert, möglichst
viel Inhalt möglichst flexibel platzieren zu können.

Die Grundaufteilung in einen \glslink{glos:absenderbereich}{Absender-} und
einen \glslink{glos:kommunikationsbereich}{Kommunikationsbereich}
bleibt erhalten.
Jedoch wird der Absenderbereich zu Gunsten eines vergrößerten Kommunikationsbereichs
schmaler dargestellt mit einem auf 90\% (Hochformat) der Standardgröße
skalierten Siegelbandlogo versehen.
Der Kommunikationsbereich ist rahmenlos und komplett mit einer frei
wählbaren Hintergrundfarbe hinterlegt.
Wenn nicht anders festgelegt ist dies die Light-Variante des aktiven
Haupt-Sekundärfarbklanges.% TODO: what?

\begin{sloppypar}
Wissenschaftliche Plakat werden immer unter Angabe der \Class{tubsposter}-Klassenoption
\KOption{style}\PName{scifi} erstellt:
\end{sloppypar}
\begin{lstlisting}
\documentclass[style=scifi]{tubsposter}
\end{lstlisting}

\subsection{Modulsystem}

Der Inhalt auf wissenschaftlichen Plakaten wird in sogenannten \emph{Modulen}
platziert.
Dies sind Kästen unterschiedlicher Größe, die durch einen schmalen Abstand optisch
voneinander getrennt sind und so die Informationseinheiten klar voneinander
abgrenzen.

Die Platzierung der Module erfolgt in \tubslatex in einem lockeren Rastersystem.
Dazu wird der Kommunikationsbereich zuerst vertikal in mehrere Blöcke (Modulzeilen)
beliebiger Höhe aufgeteilt, die jedoch zusammen den Inhaltsbereich
komplett ausfüllen müssen.
Jeder dieser einzelnen Blöcke kann dann bei Bedarf unabhängig von den anderen
wieder in unterschiedlich große Bereiche (Modulspalten) unterteilt unterteilt werden.
dass die einzelnen Bereiche in Zeilen, Spalten und Unterzeilen teilen kann.
Entsprechend können auch die Modulspalten wieder vertikal geteilt werden.
Eine tiefere Schachtelung ist aktuell leider nicht möglich.


Im Folgenden wird die Benutzung des Modulrasters genauer erläutert.

\begin{Declaration}
  \XMacro{begin}\PParameter{\Environment{tubsposter}}%
    \OParameter{Optionen}%
    \Parameter{Aufteilung}\\
  \quad\dots\\
  \XMacro{end}\PParameter{tubsposter}
\end{Declaration}

Zur Erstellung eines wissenschaftlichen Plakates wird die Umgebung
\Environment{tubsposter} verwendet, welche -- anders als bei Veranstaltungsplakaten --
einen zusätzlichen Parameter \PName{Aufteilung} erwartet.
Mit diesem wird die Anzahl an Modulzeilen bestimmt,
in die der Darstellungsbereich geteilt werden soll.
Dies geschieht mittels einer kommagetrennten Liste nach folgendem Prinzip,
das an die Tabellen-Umgebung \Environment{tabularx} angelehnt ist:
\begin{itemize}
  \item Jedes Element kann entweder eine \emph{Länge} oder der Buchstabe \Option{X} sein
  \item Die Anzahl der Elemente bestimmt die Anzahl der Modulzeilen
  \item Die Reihenfolge der Elemente entspricht der Reihenfolge der Modulzeilen
  \item Eine \emph{Länge} gibt an wie hoch die jeweilige Modulzeile sein soll
  \item Ein \Option{X} teilt den Restplatz gleichmäßig auf alle mit
    \Option{X} gewählten Modulzeilen auf.
\end{itemize}

\begin{hint}
Es empfiehlt sich, immer mindestens ein \Option{X} zu verwenden, da sonst
eine korrekte Aufteilung des verfügbaren Platzes manuell berechnet werden
müsste.
\end{hint}


\begin{minipage}{0.6\textwidth}
\begin{example}~\par
  \begin{lstlisting}
\begin{tubsposter}{6cm,X,4cmX}
  ...
\end{tubsposter}
  \end{lstlisting}

  \noindent Definiert 4 Modulzeilen, wobei die erste 6cm und die dritte 4cm hoch ist.
  Die Zeilen 2 und 4 nehmen dann jeweils die Hälfte des restlichen 
  noch zur Verfügung stehenden Platzes ein.
\end{example}
\end{minipage}
\hfill
\begin{minipage}{0.3\textwidth}\centering
  \examplegraphic[width=\textwidth,page=1]{examples/module_split.pdf}
\end{minipage}

\paragraph{Optionen}

Mit dem optionalen Parameter der Umgebung \Environment{tubsposter}
kann zum einen, wie in Abschnitt~\ref{sec:poster:sender} beschrieben,
die Position des Absenderbereichs gewählt werden.

Darüber hinaus kann bei wissenschaftlichen Plakaten noch die
Hintergrundfarbe des Plakates gewählt werden.

\begin{Declaration}
  \KOption{bgcolor}\PName{Farbe}
\end{Declaration}

Mit der Option \Option{bgcolor}, die als Argument alle verfügbaren
Farben akzeptiert (siehe \chaptername~\ref{sec:tubscolors}), kann die
Hintergrundfarbe des Plakates individuell gesetzt werden.

\paragraph{Modulzeilen}~\par

Damit die mit der Umgebung \Environment{tubsposter} definierten Modulzeilen
nun auch wirklich dargestellt werden,
müssen sie mit Inhalt gefüllt werden. Dies geschieht mit der Umgebung
\Environment{modrow}, welche im folgenden erläutert wird.

\begin{Declaration}
  \XMacro{begin}\PParameter{\Environment{modrow}}%
    \OParameter{Optionen,Aufteilung}\\%
  \quad\dots\\
  \XMacro{end}\PParameter{modrow}\\
  \XMacro{begin}\PParameter{\Environment{modrow*}}%
    \OParameter{Optionen,Aufteilung}\\%
  \quad\dots\\
  \XMacro{end}\PParameter{modrow*}
\end{Declaration}

Die Umgebung \Environment{modrow} wird benutzt,
um die mit \Environment{tubsposter} angelegten Modulzeilen mit Inhalt zu füllen.
Dafür wird für jede definierte Modulzeile genau eine
\Environment{modrow}-Umbegbung erwartet.

Die Box kann entweder direkt mit dem gewünschten Text-Inhalt gefüllt werden
oder der optionale Parameter \PName{Aufteilung} wird benutzt, um
eine weitere Aufteilung zu erwirken.
Als \PName{Aufteilung} wird dann wieder eine kommagetrennte Liste erwartet,
deren Elemente dieselbe Bedeutung wie bereits beschrieben haben, außer, dass
sie nun \emph{Modulspalten} in der aktuellen Modulzeile definieren.

Wird die Box mit Inhalt gefüllt, beträgt der normale Abstand des Inhalts
vom Rand der Modulbox halbe Rahmenbreite.

Die Sternchen-Variante \Environment{modrow*} verhält sich identisch mit
der normalen Variant, bloß, dass der Inhalt hier ohne Abstand zum Rand der Box
dargestellt wird. 

\begin{minipage}{0.6\textwidth}
\begin{example}~\par
  \noindent\Macro{begin}\PParameter{\Environment{modrow}}
    \OParameter{X,X}\par
  \quad\dots\\
  \Macro{end}\PParameter{modrow}\bigskip\par
  \noindent Teilt eine Modulzeile in 2 gleich große Spalten auf.
\end{example}
\end{minipage}
\hfill
\begin{minipage}{0.3\textwidth}\centering
  \examplegraphic[width=\textwidth,page=2]{examples/module_split.pdf}
\end{minipage}

\paragraph{Modulspalten}~\par

Genau wie die Modulzeilen müssen auch die ggf. definierten Modulspalten
erst verwendet werden, damit sie dargestellt werden.

\begin{Declaration}
  \XMacro{begin}\PParameter{\Environment{modcol}}%
    \OParameter{Optionen,Aufteilung}\\%
  \quad\dots\\
  \XMacro{end}\PParameter{modcol}\\
  \XMacro{begin}\PParameter{\Environment{modcol*}}%
    \OParameter{Optionen,Aufteilung}\\%
  \quad\dots\\
  \XMacro{end}\PParameter{modcol*}
\end{Declaration}

Die Umgebung \Environment{modcol} erlaubt in \Environment{modrow} angelegten Spalten
jeweils mit Inhalt zu füllen oder wiederum in neue Unterzeilen aufzuteilen.

Die Sternchen-Variante \Environment{modcol*} erlaubt wieder eine
randlose Darstellung des Inhalts.


\begin{minipage}{0.6\textwidth}
\begin{example}~\par
  \noindent\Macro{begin}\PParameter{\Environment{modcol}}
    \OParameter{3cm,X}\par
  \quad\dots\\
  \Macro{end}\PParameter{modcol}\bigskip\par
  \noindent Teilt eine Modulspalte in 2 Unterzeilen auf.
\end{example}
\end{minipage}
\hfill
\begin{minipage}{0.3\textwidth}\centering
  \examplegraphic[width=\textwidth,page=3]{examples/module_split.pdf}
\end{minipage}

\paragraph{Modulsubzeilen}~\par

\begin{Declaration}
  \XMacro{begin}\PParameter{\Environment{modsubrow}}\\%
  \quad\dots\\
  \XMacro{end}\PParameter{modsubrow}\\
  \XMacro{begin}\PParameter{\Environment{modsubrow*}}\\%
  \quad\dots\\
  \XMacro{end}\PParameter{modsubrow*}
\end{Declaration}

Die mit \Environment{modcol} angelegten Unterzeilen können jeweils mit 
\Environment{modsubrow} mit Inhalt gefüllt werden.

Es ist aktuell keine weitere Unterteilung möglich.

\subsection{Inhaltsdarstellung}

Innerhalb der Module können Informationen klar gegliedert dargestellt werden.
Module sollten allgemein weiß hinterlegt sein. Lediglich das Modul, welches
den Plakattitel enthält (Absendermodul), kann farbig hinterlegt werden.

Zur Auszeichnung der verschiedenen Gliederungsebenen wie Plakattitel, 
Untertitel, Kurzinformation etc. wird empfohlen die vordefiniert KOMA-Fonts
wie in \chaptername~\ref{sec:poster:schriftauszeichnung} beschrieben
zu werden.

\subsubsection{Moduloptionen}

Alle Modulumgebungen unterstützen eine Reihe von Optionen, die eine
einfache Anpassung der Darstellung von Inhalt und Hintergrund der Module
erlauben.
Sie sind im folgenden aufgeführt.

\begin{hint}
  Alle in einem Modulblock gesetzten Optionen gelten für alle innerhalb
  der Moduleben definierten Untermodule, solange sie dort nicht explizit
  überschrieben werden. Siehe dazu auch das Beispiel in
  \figurename~\ref{fig:poster:module_option_hierarchy}.
\end{hint}

% \begin{example}\hfill
% \begin{minipage}{0.5\textwidth}\small
\begin{figure}[!ht]
\small
% \hspace*{0.1\textwidth}
\centering
\begin{minipage}{0.7\textwidth}
\begin{lstlisting}
\begin{tubsposter}{X,X}
\begin{modrow}[bgcolor=tuVioletLight80,X,10cm]
  \begin{modcol}
  \end{modcol}
  \begin{modcol}
  \end{modcol}
\end{modrow}
\begin{modrow}[bgcolor=tuVioletLight80,10cm,X]
  \begin{modcol}
  \end{modcol}
  \begin{modcol}[bgcolor=tuGreenLight80]
  \end{modcol}
\end{modrow}
\end{tubsposter}
\end{lstlisting}
\end{minipage}
% \end{minipage}%
% \begin{minipage}{0.5\textwidth}
\fboxsep0mm\centering
\fbox{\includegraphics[width=0.7\textwidth]{examples/module_option_hierarchy}}
% \end{minipage}
% \end{example}
\caption{Beispiel zur Verschachtelung von Modulebenen und der Hierarchie der Optionswahl}
\label{fig:poster:module_option_hierarchy}
\end{figure}

% TODO: Hintergrundfarbe Plakat setzen. Standardfarbe

\paragraph{Vertikale Ausrichtung}\hfill

Es stehen verschiedene Optionen für die vertikale Ausrichtung des Textes
innerhalb der Box zu Verfügung. Wird keine Option angegeben, wird
der Text oben ausgerichtet.

\begin{Declaration}
  \Option{t}\\
  \Option{c}\\
  \Option{b}
\end{Declaration}

Mit der Option \Option{t} wird der Inhalt oben ausgerichtet.
Mit der Option \Option{c} wird der Inhalt vertikal zentriert eingerichtet.
Mit der Option \Option{b} wird der Inhalt unten ausgerichtet.

\paragraph{Farbdarstellung}\hfill

Die Module unterstützen sowohl die Wahl einer Hintergrundfarbe mit der das
gesamte Modul randlos eingefärbt wird, als auch die Wahl einer
Vordergrundfarbe, die die Darstellung des Textes innerhalb der Box beeinflusst.

\begin{Declaration}
  \KOption{bgcolor}\PName{Farbe}\\
  \KOption{fgcolor}\PName{Farbe}
\end{Declaration}

Mit der Option \Option{bgcolor} kann eine Hintergrundfarbe für das
jeweilige Modul festgelegt werden. Als Argument können alle in \tubslatex
definierten Farben übergeben werden.
Eine allgemeine Übersicht über die Farbdarstellung ist in
\chaptername~\ref{sec:tubscolors} zu finden.

Die Vordergrundfarbe wird analog zur Hintergrundfarbe mit der
Option \Option{fgcolor} gewählt.
Die Vordergrundfarbe beeinflusst unter anderem die Farbe des Textes.

\paragraph{Bilder}\hfill

Bilder können in Modulen natürlich wie gewohnt mit \Macro{includegraphics}
eingebunden werden.
Leichter ist es allerdings, die verfügbaren Bilddarstellungs-Optionen
von \tubslatex zu verwenden, die eine automatische Einpassung in den
Darstellungsbereich bieten.

\begin{Declaration}
  \KOption{bgimage}\PName{Farbe}\\
  \KOption{imagefit}\PName{Farbe}
\end{Declaration}

Mit der Option \Option{bgimage} kann eine Bilddatei angegeben werden,
die vollflächig in den Hintergrund des Moduls eingefügt wird.
Standardmäßig wird das Bild in eine Richtung (horizontal/vertikal) skaliert
und in die andere Richtung beschnitten,
sodass es randlos in den Darstellungsbereich passt.

Weitere Einpassungsvarianten können mit der Option \Option{imagefit}
gewählt werden.
Die Option \Option{imagefit} akzeptiert die Werte, die in
\tablename~\ref{tbl:gausslayout:imagefitoptions} aufgeführt und
\figurename~\ref{fig:gausslayout:imagefitexample} bildlich dargestellt sind.

\clearpage
\section{Aushänge}

Einfache Aushänge sollen möglichst schnell und mit wenig Aufwand zu erstellen sein.
Der Plakat-Typ für Aushänge besitzt eine einfache voreingestellte Darstellung
mit Siegelbandlogo und einer roten Trennlinie zwischen dem Inhaltsbereich
und Absenderbereich.

Aushänge werden unter Angabe der Optionswahl
\KOption{style}\PName{bulletin} erstellt:
\begin{lstlisting}
\documentclass[style=bulletin]{tubsposter}
\end{lstlisting}

\begin{Declaration}
  \XMacro{begin}\PParameter{\Environment{tubsposter}}%
%     \OParameter{Optionen}\\
  \\
  \quad\dots\\
  \XMacro{end}\PParameter{tubsposter}
\end{Declaration}
% TODO: noch keine Optionen möglich!?!

Mit der Umgebung \Environment{tubsposter} wird in diesem Fall direkt
der Textbereich eingeleitet.
Im Gegensatz zu Veranstaltungsplakaten müssen/können keine Segmente 
definiert werden.

Die Verwendung von Schriften unterscheidet sich nicht von den in
\chaptername~\ref{sec:poster:schriftauszeichnung} erläuterten 
Möglichkeiten.

\begin{figure}[!ht]
\begin{minipage}{0.65\textwidth}
\begin{lstlisting}
\documentclass[a3paper,style=bulletin]{tubsposter}
\begin{document}
\begin{tubsposter}
  \vspace*{100pt}
  {\usekomafont{headline}
    Gait, commy nisi!\par\vspace*{1.5ex}}
  {\itshape\huge\color{tuRed}
    Dunt venim dolorerosto do odit
    ametum vulla conum dolore
    co\-nulput in vero od el\par}
  \vfill
  {\Large
    Iduissit Ium do eum digna:
    0531 123-4567}
\end{tubsposter}
\end{document}

\end{lstlisting}
\end{minipage}
\begin{minipage}{0.35\textwidth}
  \fboxsep0mm
  \fbox{\includegraphics[width=0.98\textwidth]{examples/posterstyle_bulletin.pdf}}
\end{minipage}
\caption{Minimal-Beispiel zur Verwendung von \Environment{tubsposter}-Stil 
  \Option{bulletin}}
\end{figure}


% \begin{table}\centering
% \begin{tabular}{lll}
%       & Dokumente/Plakate & wissenschaftl.\\
% \midrule
% A0    & 40pt  & 35pt  \\
% A1    & 25pt  & 25pt  \\
% A2    & 18pt  & 18pt  \\
% A3    & 13pt  & --    \\
% A4    & 9pt, 10pt, 11pt & --\\
% A5/lang & 9pt, 10pt, 11pt & --\\
% \end{tabular}
% \caption{Übersicht über verfügbare Schriftgrößen}
% \end{table}
