\chapter{Plakate}

Plakate können in \tubslatex mit Hilfe der Dokumentenklasse \newdocumentclass{tubsposter} erstellt werden.
Dabei werden drei unterschiedliche Plakat-Typen unterstützt,
zwischen denen mit der Paketoption \KOption{style}\PName{<standard/scifi/bulletin>} gewählt werden kann.

Dies sind zum einen Veranstaltungsplakate, welche sich am normalen Gaußraster
orientieren und als Standard-Darstellungstyp voreingestellt sind.

Für die Darstellung wissenschaftlicher Inhalte gibt es ein spezielles Layout,
das mehr Platt und Flexible für die Unterbringung von Informationen bietet.
Es orientiert sich an einem eigenen Modulsystem und nicht am Gaußraster.

Zusätzlich gibt es noch eine Unterstützung für die Erstellung einfacher
Aushänge. Diese weisen nur einen einfachen großen Textbereich auf, der
nach oben durch das Siegelbandlogo mit einer Trennlinie begrenzt ist.

\begin{figure}[htp]
\fboxsep0mm
\begin{minipage}{0.3\textwidth}\centering
  \fbox{\includegraphics[width=0.9\textwidth]{examples/posterstyle_standard.pdf}}
  \subcaption{Veranstaltung (\PValue{standard})}
\end{minipage}
\begin{minipage}{0.3\textwidth}\centering
  \fbox{\includegraphics[width=0.9\textwidth]{examples/posterstyle_scifi.pdf}}
  \subcaption{wissenschaftlich (\PValue{scifi})}
\end{minipage}
\begin{minipage}{0.3\textwidth}\centering
  \fbox{\includegraphics[width=0.9\textwidth]{examples/posterstyle_bulletin.pdf}}
  \subcaption{Aushang (\PValue{bulletin})}
\end{minipage}
\caption{Die verschiedenen Plakat-Stile}
\end{figure}

\begin{Declaration}
  \KOption{style}\PName{<standard/scifi/bulletin>}
\end{Declaration}

Mit \KOption{style}{standard} wird die Darstellung für Veranstaltungsplakate
gewählt. Diese ist bereits voreingestellt und muss daher nicht explizit gewählt
werden.
Mit \KOption{style}{scifi} wir das Layout-System für wissenschaftliche Plakate
geladen, was sich teilweise grundlegend von dem normalen Layout unterscheidet.
Der Wert \KOption{style}{bulleting} erlaubt die schnelle Erstellung von
Aushängen und lädt eine relative schlichte voreingestellte Darstellung.

\begin{Declaration}
  \KOption{sender}\PName{top/bottom}
\end{Declaration}

Diese Option legt fest, ob der Absenderbereich am oberen oder unteren Blattrand
platziert werden soll. Für weitere Informationen siehe vorausgehende Kapitel.


\begin{Declaration}
  \Option{a3paper}\\
  \Option{a2paper}\\
  \Option{a1paper}\\
  \Option{a0paper}
\end{Declaration}

Für alle Poster-Variante stehen vier Papiertformate zur Auswahl für die
Schriftgrößen etc. vordefiniert sind und automatisch geladen werden.
Dies sind A3, A2, A1 und A0, die mit den entsprechenden Optionen
\Option{a3paper}, \Option{a2paper}, \Option{a1paper} bzw. \Option{a0paper}
gewählt werden können.
\bigskip

\paragraph{Weitere Paket-Optionen:}

Wie bei allen anderen \tubslatex-Klassen könnnen auch für Poster
die diversen verfügbaren Optionen zur Farbdarstellung verwendet werden.
Näheres hierzu findet sich in den vorausgehenden Kapiteln oder detaillierter
in Kapitel~\ref{chap:tubscolors}.

\section{Veranstaltungsplakate}

Veranstaltungsplakate werden im Gaußraster gesetzt und entsprechen somit
im Grunde allen Standarddarstellungen des Corporate Design.

\begin{Declaration}
  \KOption{style}{standard}
\end{Declaration}

(Optionale) Paketoption zum Erstellung von Veranstaltungsplakaten.

\begin{Declaration}
  \XMacro{begin}\PParameter{\Environment{tubsposter}}%
    \OParameter{options}\\
  \quad\dots\\
  \XMacro{end}\PParameter{tubsposter}
\end{Declaration}

Ein neues Plakat wird mit der Umgebung \Environment{tubsposter} erstellt.
Der optionale Parameter \PName{options} akzeptiert dabei die
unter \ref{} beschriebenen Optionen.

% sender
% bgcolor?

\begin{Declaration}
  \XMacro{begin}\PParameter{\Environment{posterrow}}%
    \OParameter{cols}\\%
  \quad\dots\\
  \XMacro{end}\PParameter{posterrow}%\\
%   \XMacro{begin}\PParameter{\Environment{posterrow*}}%
%     \OParameter{cols}\\%
%   \quad\dots\\
%   \XMacro{end}\PParameter{posterrow*}
\end{Declaration}

Einzelne Segmente im Gaußraster können mit der Umgebung \Environment{posterrow}
erzeugt werden.

\section{Wissenschaftliche Plakate}

\begin{Declaration}
  \KOption{style}{scifi}
\end{Declaration}

Paketoption zum Erstellen von wissenschaftlichen Postern.

\begin{Declaration}
  \XMacro{begin}\PParameter{\Environment{tubsposter}}%
    \OParameter{options}%
    \Parameter{rows}\\
  \quad\dots\\
  \XMacro{end}\PParameter{tubsposter}
\end{Declaration}

Für die Erstellung von wissenschaftlichen Plakaten wird ebenfalls die Umgebung
\Environment{tubsposter} verwendet, welche in diesem Fall jedoch einen
zusätzlichen Parameter \PName{rows} erwartet.
Damit wird die Anzahl an Modulzeilen bestimmt. Dies geschieht mittels
einer kommagetrennten Liste, wobei jedes Element entweder eine Länge
oder der Buchstabe 'X' sein kann. Eine Länge legt die Höhe der jeweiligen
Modulzeile genau fest, ein X sorgt dafür, dass alle mit X gekennzeichneten
Zeilen den restlichen zur Verfügung stehenden Platz gleichmäßig untereinander
aufteilen. Dieses Vorgehen ist an die Tabellen-Umgebung \Environment{tabularx}
angelehnt.

\begin{Example}
  \noindent\Macro{begin}\PParameter{\Environment{tubsposter}}
    \Parameter{3cm,X,5cm,X}\par
  \noindent Erzeugt 4 Modulzeilen, wobei die 1. 3cm und die 3. 5cm hoch sind.
  Die Zeilen 2 und 4 nehmen den Restlichen verfügbaren Platz ein
  und sind gleich hoch.
\end{Example}


\begin{Declaration}
  \XMacro{begin}\PParameter{\Environment{posterrow}}%
    \OParameter{cols}\\%
  \quad\dots\\
  \XMacro{end}\PParameter{posterrow}\\
  \XMacro{begin}\PParameter{\Environment{posterrow*}}%
    \OParameter{cols}\\%
  \quad\dots\\
  \XMacro{end}\PParameter{posterrow*}
\end{Declaration}

Die mit \Environment{tubsposter} angelegten Modulzeilen können nun jeweils mit
der Umgebung \Environment{posterrow} mit Inhalt gefüllt werden.
Dabei kann entweder direkt der gewünschte Inhalt geschrieben oder das optionale
Parameter \PName{cols} benutzt werden.
Dieses erlaubt die Definition zusätzlicher Spalten in der aktuellen Modulzeile.
Es wird wieder ein kommagetrennte Liste erwartet,
deren Elemente dieselbe Bedeutung haben wie bereits beschrieben, außer, dass
sie die Breite und nicht die Höhe definieren.

Der normale Abstand des Inhalts vom Rand der Modulbox beträgt halbe
Rahmenbreite. Für das Einfügen von Bildern etwa kann es sinnvoll sein,
diesen Rahmen wegzulassen. Dies geschieht mit der Sternchen-Variante
\Environment{posterrow*}.


\begin{Declaration}
  \XMacro{begin}\PParameter{\Environment{postercol}}%
    \OParameter{rows}\\%
  \quad\dots\\
  \XMacro{end}\PParameter{postercol}\\
  \XMacro{begin}\PParameter{\Environment{postercol*}}%
    \OParameter{rows}\\%
  \quad\dots\\
  \XMacro{end}\PParameter{postercol*}
\end{Declaration}

Die mit \Environment{posterrow} angelegten Spalten können jeweils mit 
\Environment{postercol} mit Inhalt gefüllt oder in neue Unterzeilen aufgeteilt 
werden.

\begin{Declaration}
  \XMacro{begin}\PParameter{\Environment{postersubrow}}\\%
  \quad\dots\\
  \XMacro{end}\PParameter{postersubrow}\\
  \XMacro{begin}\PParameter{\Environment{postersubrow*}}\\%
  \quad\dots\\
  \XMacro{end}\PParameter{postersubrow*}
\end{Declaration}

Die mit \Environment{postercol} angelegten Unterzeilen können jeweils mit 
\Environment{postercol} mit Inhalt gefüllt werden.

\begin{Example}

\end{Example}

\section{Aushänge}

% \section{Seitenlayout}

% margin
% bcor
% sender=top/bottom


% \section{Format}

% \section{Farben}

% \section{Befehle}

% \section{Optionen}
