\chapter{Installation}

Aktuell wird \tubslatex in drei verschiedenen Varianten angeboten.
Zum einen als gepackter tex-Paketaum, der sowohl lokal als auch systemweit
entpackt werden kann. Allerdings müssen hier alle notwendigen Schritte zur 
Installation der Schriften und Anmelden der neuen Datein manuell gemacht werden.
Für Windows-Benutzer, die MikTeX verwenden, gibt es einen automatischen
Installer, der alle nötigen Schritte automatisch ausführt.
Ebenso gibt es für Debian-basierte Systeme, die texlive verwenden ein
\texttt{.deb}-Paket, das alle Abhängigkeiten korrekt auflöst und die Vorlagen
automatisch installiert.

\section{Automatische Installation}

\subsection{Windows -- MikTeX}

Bei Verwendung von MikTeX unter Windows kann auf den verfügbaren Installer
zurück gegriffen werden. Um eine fehlerfreie Installation zu ermöglichen
sollte MiKTeX in der Version 2.9 vorhanden sein.
Ältere Versionen werden nur bedingt unterstützt.

Der Installer kopiert alle benötigten Dateien in ein frei wählbares Verzeichnis
auf der Festplatte und registriert dies automatisch als neuen texmf-Baum.
Auch die Schriftart Nexus wird standardmäßig installiert und in 
Font-Verwaltung registriert.

\subsection{Ubuntu/Debian -- TexLive}

Für debian-basierte Systeme wie Ubuntu oder Debian selber steht ein
\texttt{.deb}-Paket zur Verfügung, dass einfach mit Hilfe eines Paketmanagers
installiert werden kann.
Genutz werden können dafür die diversen graphischen Tools oder einfach das
Kommandozeilenprogramm \texttt{gdebi}. Diese Programme lösen alle Abhängigkeiten
auf und installieren sie automatisch mit.

Die Installation der Dateien erfolgt in das Verzeichnis
\lstinline{/usr/share/texmf-texlive/}. Sowohl die neuen Dateien als auch 
die Schrift Nexus werden automatisch registriert.

\section{Manuelle Installation}

Zur manuellen Installation liegen die LaTeX-Dateien als gepacktes
zip-Verzeichnis vor, in dem die komplette tex-Pfadstruktur oberhalb der
texmf-Ebene enthalten ist. Konkret sind dies die Ordner
\lstinline{doc}, \lstinline{tex} und \lstinline{fonts}.

Diese können entweder lokal oder global entpackt werden.
Für erste Variante kann der Inahlt einfach in das Verzeichnis
\lstinline{/home/benutzername/texmf} hinein geschrieben werden.

% TODO...
