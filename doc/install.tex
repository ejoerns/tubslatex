\chapter{Installation}\label{chap:install}

Aktuell wird \tubslatex in drei verschiedenen Varianten angeboten.
Zum einen als gepackter tex-Paketaum, der sowohl lokal als auch systemweit
entpackt werden kann. Allerdings müssen hier alle notwendigen Schritte zur 
Installation der Schriften und Anmelden der neuen Datein manuell gemacht werden.
Für Windows-Benutzer, die MikTeX verwenden, gibt es einen automatischen
Installer, der alle nötigen Schritte automatisch ausführt.
Ebenso gibt es für Debian-basierte Systeme, die texlive verwenden ein
\texttt{.deb}-Paket, das alle Abhängigkeiten korrekt auflöst und die Vorlagen
automatisch installiert.


\paragraph{Manuelle Installation}

Zur manuellen Installation liegen die LaTeX-Dateien als gepacktes
zip-Verzeichnis vor, in dem die komplette tex-Pfadstruktur oberhalb der
texmf-Ebene enthalten ist. Konkret sind dies die Ordner
\lstinline{doc}, \lstinline{tex} und \lstinline{fonts}.

Diese können entweder lokal oder global entpackt werden.
Für erste Variante kann der Inahlt einfach in das Verzeichnis
\lstinline{/home/benutzername/texmf} hinein geschrieben werden.

Es sei darauf hingewiesen, dass alle weiteren benötigten Pakete der
Tex-Distribution ggf. manuell installiert werden müssen.
MikTeX macht dies normalerweise automatisch.
Bei texlive können unter Linux zum Beispiel die entsprechenden System-Pakete
installiert werden.

\clearpage
\section{Windows -- MiKTeX}

\subsection{Automatische Installation}

% \subsection{Windows -- MikTeX}

Bei Verwendung von MikTeX unter Windows kann auf den verfügbaren Installer
zurück gegriffen werden. Um eine fehlerfreie Installation zu ermöglichen
sollte MiKTeX in der Version 2.9 vorhanden sein.
Ältere Versionen werden nur bedingt unterstützt.

Der Installer kopiert alle benötigten Dateien in ein frei wählbares Verzeichnis
auf der Festplatte und registriert dies automatisch als neuen texmf-Baum.
Auch die Schriftart Nexus wird standardmäßig installiert und in 
Font-Verwaltung registriert.

% \subsection{Ubuntu/Debian -- TexLive}



\subsection{Manuelle Installation}

\begin{description}
  \item[1. Dateien Kopieren]
  
    Alle zu installierende Dateien sollten entweder in ein bestehendes lokales
    texmf-Verzeichnis kopiert werden oder in ein neu angelegtes Verzeichnis
    (z.\,B. \lstinline{C:\tubslatex}).
    
    In dieses sind die Ordner \lstinline{tex}, \lstinline{doc} und 
    \lstinline{fonts} zu kopieren.
    
    \begin{enumerate}[a)]
      \item {\bfseries }
        
        Für den Fall, dass ein neues Verzeichnis angelegt wurde,
        muss dies MiKTeX noch bekannt gemacht werden.
        Dazu sind die MiKTeX-Einstellungen\\ (Start$\to$Programme$\to$MiKTeX 2.8 
        $\to$Maintenance$\to$Settings) aufzurufen.
        Im Reiter \glqq Roots\grqq\ kann der neue Pfad hinzugefügt werden.
    \end{enumerate}

  \item[2. Dateien registrieren]

    Anschließend ist es noch ratsam den Button \glqq Refresh FNDB\grqq\
    zu drücken, um die Dateidatenbank zu aktualisieren. Alternativ kann auch 
    der Konsolenbefehl \lstinline{initexmf -u} verwendet werden.

  \item[3. Schriften registrieren]
    Um die Schrift Nexus benutzen zu können, sind zunächst wie oben beschrieben
    die benötigten Dateien zu kopieren.

    \begin{enumerate}

      \item {\bfseries map-Dateien bekannt machen}

      Mit dem Konsolen-Befehl \lstinline{initexmf --admin --edit-config-file updmap}
      wird ein Editor geöffnet in den folgender Text einzutragen ist:

      \begin{lstlisting}
Map NexusProSans.map
Map NexusProSerif.map
      \end{lstlisting}

      \item {\bfseries Font maps updaten}

        Danach ist ein Update der Font-Datenbank erforderlich. Dies geschieht
        mittels\\ \lstinline{initexmf --admin --mkmaps}. Die Konsole muss dazu
        mit Administratorrechten gestartet werden.

        \paragraph{Hinweis:} Es ist auch möglich, die Datenbank im 
          Nicht-Administrator-Modus zu erneu\-ern. Dies ist jedoch nur bei einer
          Einzelbenutzer-Installation von MiKTeX sinnvoll. Wird
          \lstinline{initexmf} ohne die Option \lstinline{--admin} aufgerufen,
          so wird eine lokale map-Liste angelegt, auf die MiKTeX anschließen für
          den aktuellen Benutzer ausschließlich zugreift.
          Eine benutzerweite Installation von Fonts ist dann noch 
          möglich, jedoch muss jedes mal manuell \lstinline{initexmf --mkmaps}
          aufgerufen werden.
%           Welche map-Liste aktuell verwendet wird, lässt sich mit dem Befehl
    \end{enumerate}

    Danach sollte die Installation abgeschlossen sein und alle Pakete verwendbar 
    sein.
\end{description}
\begin{description}
  \item[1. Dateien Kopieren]
  \item[2. Dateien registrieren]
  \item[3. Schriften registrieren]
\end{description}


\clearpage
\section{Ubuntu/Debian -- TexLive}

\subsection{Automatische Installation}

Für debian-basierte Systeme wie Ubuntu oder Debian selber steht ein
\texttt{.deb}-Paket zur Verfügung, dass einfach mit Hilfe eines Paketmanagers
installiert werden kann.
Genutz werden können dafür die diversen graphischen Tools oder einfach das
Kommandozeilenprogramm \texttt{gdebi}. Diese Programme lösen alle Abhängigkeiten
auf und installieren sie automatisch mit.

Die Installation der Dateien erfolgt in das Verzeichnis
\lstinline{/usr/share/texmf-texlive/}. Sowohl die neuen Dateien als auch 
die Schrift Nexus werden automatisch registriert.

\subsection{Manuelle Installation}

\paragraph{Lokale/systemweite Installation}

Bei der Installation sollte man sich zuerst zwischen einer Benutzer-Installation
und einer System-Installation entscheiden.
Bei einer \emph{Benutzer-Installation} werden alle Dateien im lokalen Benutzerverzeichnis installiert.
Dies kann damit auch auf Systemen geschehen auf denen man nur
eingeschränkte Rechte besitzt.
Die Vorlagen sind dann aber auch nur für den aktuellen Nutzer benutzbar.
Darüber hinaus ist zu beachten, dass durch eine lokale Installation auch
lokale Datenbanken angelegt werden, die fortan benutzt werden.
Daher würden zum Beispiel Schriften, die anschließend global installiert werden,
nicht gefunden werden, da sie nur für die globale Datenbank registriert worden wären.

Dies Problem stellt sich bei einer \emph{systemweiten Installation} nicht,
welche aber auch nur mit entsprechenden Rechten durchgeführt werden kann.
Sind diese jedoch vorhanden, so ist eine \emph{systemweite Installation im
Allgemeinen einer lokalen Installation vorzuziehen}.

\begin{important}
Die Beschreibung nennt trotz der genannten Empfehlung jeweils zuerst die
Schritte für die lokale Installationsvariante, da systemweite Installation
oft nur ein paar zusätzliche Suffixe bei den Befehlsnamen mit sich zieht.
Jeder Absatz sollte daher zunächst durchgelesen werden bevor die entsprechenden
Kommados angewendet werden.
\end{important}


\begin{description}
  \item[1. Dateien Kopieren] Zuerst müssen alle in der Zip-Datei enthaltenen
    Dateien entpackt werden.
    Diese liegen dort schon in der korrekten Ordner-Struktur vor, sodass
    sie lediglich in das Ziel-texmf-Verzeichnis \emph{hinein}kopiert werden
    müssen.
    
    Grundsätzlich kann man sich bei der Installation entscheiden,
    ob die Vorlagen nur Lokal für den aktuellen Benutzer oder systemweit
    installiert werden sollen.
    Es sei darauf hingewiesen, dass eine lokale Installation ein paar
    Fallstricke mit sich bringt, die evtl. zu Problemen bei der späteren
    Installation weiterer Pakete führen können.
    
    Entpacken als Benutzer ins lokale texmf-Verzeichnis:
    \begin{lstlisting}
$ unzip -d ~/texmf/ tubslatex_0.3-alpha2.zip
    \end{lstlisting}
    Oder als Root ins systemweite texmf-Verzeichnis:
    \begin{lstlisting}
# unzip -d /usr/share/texmf/ tubslatex_0.3-alpha2.zip
    \end{lstlisting}
  \item[2. Dateien registrieren]
    Wurden die Dateien in das lokale texmf-Verzeichnis kopiert, so ist
    in diesem Verzeichnis der Befehl
    \begin{lstlisting}
$ mktexlsr .
    \end{lstlisting}
    auszuführen.

    Bei einer systemweiten Installation ist einfach
    \begin{lstlisting}
# mktexlsr
    \end{lstlisting}
    aufzurufen.

  \item[3. Schriften registrieren]\hfill

    \begin{lstlisting}
$ /usr/bin/updmap --nomkmap --nohash --enable Map=NexusProSans.map
$ /usr/bin/updmap --nomkmap --nohash --enable Map=NexusProSerif.map
$ /usr/bin/updmap
    \end{lstlisting}
    
    Für eine systemweite Installation muss den Befehlen jeweils der Zustatz
    \texttt{-sys} angefügt werden (\lstinline{updmap} $\to$ \lstinline{updmap-sys}).
  
  \item[4. Arial installieren]\hfill
    Falls die Schrift 'arial' noch nicht installiert wurde, so sollte dies
    nachgeholt werden, da sie für einige Vorlagen als Standard-Schrift
    eingestellt ist.
    
    Die Installation erfolgt mit Hilfe des Programms \lstinline{getnonfreefonts},
    dass die benötigten Dateien automatisch herunterlädt und alle zur Installation
    nötigen Schritte durchführt.
    
    Eine Lokale Installation von Arial geschieht mit Hilfe des Befehls
    \begin{lstlisting}
$ getnonfreefonts --verbose arial-urw
    \end{lstlisting}
    
    Die System-Installation ist dabei wieder mit dem Zusatz \lstinline{-sys}
    möglich.



\end{description}


\newenvironment{knownissue}[1]{%
  \paragraph{#1}\hfill
  \newcommand{\solution}[1]{\noindent{\itshape ##1}}
}{%
}

\subsection{Fehlerbehebung}


\begin{knownissue}{%
  updmap bricht plötzlich ab (Fehlercode 2) / getnonfreefonts bricht ab
}

Dieses Problem ist unter anderem von texlive2009 bekannt.
Dort hat sich ein Problem eingeschlichen, das leider die Funktionalität von
updmap teilweise lahmlegt.

\solution{Mögliche Lösung:}
\begin{itemize}
  \item Die Datei 10local.cfg (falls sie noch nicht vorhanden ist) manuell
    erstellen (z.B. mit \texttt{touch}).
  \item Nicht die Option \texttt{--quiet} verwenden.
\end{itemize}

Alternativ kann auch die Zeile \lstinline{set -e} am Anfang des Skripts einfach auskommentiert werden.
\end{knownissue}


\begin{knownissue}{
  \texttt{mktexpk: don't know how to create bitmap font for \ldots}\newline
  nach Schriften-Installation mit \lstinline{updmap-sys}}

Dieses Problem kann auftauchen, wenn auf dem System vom aktuellen Benutzer
zuvor schon einmal \lstinline{updmap} (ohne -sys) ausgeführt wurde.
Dann legt texlive eine lokale Map-Datei an und verwendet fortan diese,
auch wenn die System-Map mit \lstinline{updmap-sys} upgedated wurde.

\noindent\textit{Überprüfung:}

Die Ausgabe von
\begin{lstlisting}
kpsewhich pdftex.map
\end{lstlisting}
sollte einen anderen Pfad als
\lstinline{/var/lib/texmf/fonts/map/pdftex/updmap/} liefern.

\solution{Lösung 1:}

Entfernen der aktuell verwendeten (lokalen) Map-Datei.

\solution{Lösung 2:}

Installation weiterer Schriften mit \lstinline{updmap} statt
\lstinline{updmap-sys}.

\end{knownissue}


\clearpage
\section{Mac OS -- TexLive}

\subsection{Manuelle Installation}

Diese Beschreibung erklärt zur Zeit nur die lokale Installation.

\begin{description}
  \item[1. Dateien Kopieren] Zuerst müssen alle in der Zip-Datei enthaltenen
    Dateien entpackt werden.
    Diese liegen dort schon in der korrekten Ordner-Sktruktur vor, sodass
    sie lediglich in das Ziel-texmf-Verzeichnis \emph{hinein}kopiert werden
    müssen.
    Dies sollte im Allgemeinen \lstinline{~/Library/texmf} sein.
%   \item[2. Dateien registrieren]

  \item[3. Schriften registrieren]\hfill
    \begin{lstlisting}
# /usr/bin/updmap --nomkmap --nohash --enable Map=NexusProSans.map
# /usr/bin/updmap --nomkmap --nohash --enable Map=NexusProSerif.map
# /usr/bin/updmap
    \end{lstlisting}


  \item[4. Arial installieren]\hfill
    Falls die Schrift 'arial' noch nicht installiert wurde, so sollte dies
    nachgeholt werden, da sie für einige Vorlagen als Standard-Schrift
    eingestellt ist.
    
    Die Installation erfolgt mit Hilfe des Programms \lstinline{getnonfreefonts},
    dass die benötigten Dateien automatisch herunterlädt und alle zur Installation
    nötigen Schritte durchführt.
    \begin{enumerate}[a)]
      \item Falls nicht vorhanden kann ein Installer unter folgendem Link
        bezogen werden:
        \url{http://tug.org/fonts/getnonfreefonts/install-getnonfreefonts}.
    
        Die Installation von \lstinline{getnonfreefonts} erfolgt dann mit dem
        Befehl\footnote{Weitere Informationen finden sich hier:\\
          \url{http://www.golatex.de/vollautomatischen-installation-einiger-nicht-freier-fonts-t5386.html}}
        \begin{lstlisting}
# texlua install-getnonfreefonts
        \end{lstlisting}
    \end{enumerate}

    Eine lokale Installation von Arial geschieht mit Hilfe des Befehls
    \begin{lstlisting}
$ getnonfreefonts arial-urw
    \end{lstlisting}
    
    Eine systemweite Installation ist dabei mit dem Zusatz \lstinline{-sys}
    möglich.
    \begin{lstlisting}
$ getnonfreefonts-sys arial-urw
    \end{lstlisting}

\end{description}



% TODO...

\def\foonote{%
kpathsea: Running mktexpk --mfmode / --bdpi 600 --mag 1+240/600 --dpi 840 NexusProSans-Bold-Regular--base
mktexpk: don't know how to create bitmap font for NexusProSans-Bold-Regular--base.

Reproduce:
  - sudo updmap
  - sudo aptitude install texlive-tubs
Test:
  - wenn 'kpsewhich pdftex.map' nicht /var/lib/.. ausgibt, liegt der Fehler vor
Beschreibung:
  - Dieses Problem kann unter anderem auftreten, wenn vor der Installation
    irgendwann einmal 'updmap' aufgerufen wurde.
Begründung:
  - Installation greift auf updmap-sys zurück, die Damit generierte Datei
    wird aber nicht mehr verwendet.
Lösung 1:
  - updmap manuell aufrufen.
Lösung 2:
  - Datei, die bei 'kpsewhich pdftex.map' angezeigt wird, löschen
    (z.B. /root/.texmf-var/fonts/map/pdftex/updmap/pdftex.map)
  - 'sudo updmap-sys' aufrufen
}
