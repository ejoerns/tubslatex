\chapter{Installation}\label{chap:install}
\Index{Installation}

Aktuell wird \tubslatex in drei verschiedenen Varianten angeboten.

\begin{description}
  \item[zip-Archiv]
    Das einfache zip-Archiv enthält einen kompletten \TeX-Paketbaum mit
    allen zu \tubslatex gehörenden Dateien.
    Dieser kann universell zur manuellen Installation auf verschiedensten
    Systemen eingesetzt werden.
    Diese Variante ist eher für erfahrenere \TeX-Benutzer oder bei
    Fehlen eines automatischen Installers zu empfehlen, da Pakete
    und Schriften manuell registriert werdens müssen.
  \item[Windows-Installer]
    Der für Windows vorgesehene Installer kann bei Verwendung der
    populären \LaTeX-Distribution MiKTeX eingesetzt werden.
    Er führt alle notwendigen Schritte automatisch aus und spart somit
    viel Zeit und Probleme.
  \item[deb-Paket]
    Für debian-basierte Systeme, die texLive verwenden gibt es
    ein komplettes deb-Paket, das alle Vorlagendateien enthält,
    alle Abhängigkeiten korrekt auflöst und die notwendigen
    Installationsschritte automatisch durchführt.
\end{description}

Im Nachfolgenden werden automatische und manuelle Installation für
die Betriebssysteme Windows, Linux und Mac in Kombination mit der
jeweils gängigsten \LaTeX-Distribution noch einmal ausführlicher
beschrieben.

Für bekannte Probleme und Fehlermeldungen werden mögliche Lösungen beschrieben.

Generell kann man unter jedem System die Installation entweder global, 
also für alle Benutzer, oder lokal nur für den aktuellen Benutzer vornehmen.
Näheres ist jeweils den systemspezifischen Abschnitten zu entnehmen.


\clearpage
\section{Windows -- MiKTeX}
\Index{MiKTeX}
\Index{Windows}
\Index{Installation!MiKTeX}
\Index{Installation!Windows}

\subsection{Automatische Installation}

Bei Verwendung von MiKTeX unter Windows kann auf den verfügbaren Installer
zurück gegriffen werden. Um eine fehlerfreie Installation zu ermöglichen,
sollte MiKTeX in der Version 2.9 vorhanden sein.
Ältere Versionen werden nur bedingt unterstützt.

Der Installer bietet die Wahl zwischen einer lokalen und systemweiten
Installation.
\begin{important}
Im Zweifel sollte immer eine \emph{systemweite Installation}
vorgezogen werden, da bei der lokalen Installation auch eine lokale
Datenbank angelegt wird auf die das System fortan zugreift.
Dies kann dazu führen, dass später installierte Pakete nicht automatisch
erkannt werden, sondern manuell registriert werden müssen.
\end{important}

Der Installer kopiert alle benötigten Dateien in ein frei wählbares Verzeichnis
auf der Festplatte und registriert dies automatisch als neuen texmf-Verzeichnisbaum.
Auch die Dokumentation und die Schriftart Nexus wird standardmäßig installiert
und in der Font-Verwaltung registriert.
Die Installation erlaubt, einzelne Komponenten von der Installation auszuschließen,
die ist jedoch im Allgemeinen nicht zu empfehlen.\par

Die saubere Deinstallation kann einfach über die Windows-Softwareverwaltung 
durchgeführt werden, in der bei der Installation ein entsprechender Eintrag
angelegt wird.

\newcommand\localinstall[1]{%
  {\leavevmode\color{tuGray80}(#1)}%
}

\subsection{Manuelle Installation} %

Beschrieben wird die \emph{systemweite Installation}.
Die entsprechenden Befehle für die lokale Installation werden wenn nötig
in Klammern und grauer Farbe mit erwähnt.

Für die Installation ist die Eingabe von Kommandozeilenbefehlen nötig.
Dazu sollte der Kommandozeileninterpreter von Windows gestartet
werden (\lstinline{cmd.exe}). Für die systemweite Installation ist es
darüber hinaus notwendig, dass der Interpreter mit
Administratorrechten gestartet wird. % TODO: how?

\begin{description}
  \item[1. Dateien Kopieren]
  
    Alle zu installierenden Dateien sollten entweder in ein bestehendes globales
    \localinstall{lokales}
    texmf-Verzeichnis kopiert werden oder in ein neu angelegtes Verzeichnis
    (z.\,B. \lstinline{C:\tubslatex}).
    
    In dieses sind die Ordner \lstinline{tex}, \lstinline{doc} und 
    \lstinline{fonts} zu kopieren.
    
    \begin{enumerate}[a)]
      \item {\bfseries }
        
        Für den Fall, dass ein neues Verzeichnis angelegt wurde,
        muss dies MiKTeX noch bekannt gemacht werden.
        Dazu sind die MiKTeX-Einstellungen\\
        {\sffamily(Start$\to$Programme$\to$MiKTeX 2.x$\to$Maintenance (Admin)$\to$Settings)\\
        \localinstall{Start$\to$Programme$\to$MiKTeX 2.x$\to$Maintenance$\to$Settings}}
        aufzurufen.
        Im Reiter \glqq Roots\grqq\ kann der neue Pfad hinzugefügt werden.
    \end{enumerate}

  \item[2. Dateien registrieren]

    Anschließend ist es noch ratsam, den Button \glqq Refresh FNDB\grqq\
    zu drücken, um die Dateidatenbank zu aktualisieren. Alternativ kann auch 
    der Kommandozeilenbefehl
    \begin{lstlisting}[style=cmd]
initexmf --admin -u
    \end{lstlisting}
    \localinstall{\lstinline{initexmf -u}} verwendet werden.

  \item[3. Schriften registrieren]
    Nachdem, wie oben beschrieben, die Font-Dateien kopiert wurden, müssen
    diese noch registriert werden.

    \begin{enumerate}

      \item {\bfseries map-Dateien bekannt machen}

      Mit dem Konsolen-Befehl
      \begin{lstlisting}[style=cmd]
initexmf --admin --edit-config-file updmap
      \end{lstlisting}
      \localinstall{\lstinline{initexmf --edit-config-file updmap}}
      wird ein Editor geöffnet in den folgende zwei Zeilen einzutragen sind:

      \begin{lstlisting}[style=file]
Map NexusProSans.map
Map NexusProSerif.map
      \end{lstlisting}

      \item {\bfseries Font-Maps updaten}

        Danach ist ein Update der Font-Datenbank erforderlich. Dies geschieht
        mittels
        \begin{lstlisting}[style=cmd]
initexmf --admin --mkmaps
        \end{lstlisting}
        \localinstall{initexmf --mkmaps}.
%         Die Konsole muss dazu mit Administratorrechten gestartet werden
%         \localinstall{gilt nicht bei lokaler Installation}.% TODO: general note.

%         \paragraph{Hinweis:} Es ist auch möglich, die Datenbank im 
%           Nicht-Administrator-Modus zu erneu\-ern. Dies ist jedoch nur bei einer
%           Einzelbenutzer-Installation von MiKTeX sinnvoll. Wird
%           \lstinline{initexmf} ohne die Option \lstinline{--admin} aufgerufen,
%           so wird eine lokale map-Liste angelegt, auf die MiKTeX anschließen für
%           den aktuellen Benutzer ausschließlich zugreift.
%           Eine benutzerweite Installation von Fonts ist dann noch 
%           möglich, jedoch muss jedes mal manuell \lstinline{initexmf --mkmaps}
%           aufgerufen werden.
%           Welche map-Liste aktuell verwendet wird, lässt sich mit dem Befehl
    \end{enumerate}

    Danach sollte die Installation abgeschlossen und alle Pakete verwendbar 
    sein.
\end{description}


\clearpage
\section{Ubuntu/Debian -- TexLive}
\Index{Debian}
\Index{Ubuntu}
\Index{TexLive}
\Index{Installation!Debian}
\Index{Installation!Ubuntu}
\Index{Installation!TexLive}

\subsection{Automatische Installation}

Für debian-basierte Systeme wie Ubuntu oder Debian selber steht ein
\texttt{.deb}-Paket zur Verfügung, das einfach mit Hilfe eines Paketmanagers
installiert werden kann.
Genutzt werden können dafür die diversen graphischen Tools oder einfach das
Kommandozeilenprogramm \texttt{gdebi}. Diese Programme lösen alle Abhängigkeiten
auf und installieren sie automatisch mit.

Die Installation der Dateien erfolgt in das Verzeichnis
\lstinline{/usr/share/texmf-texlive/}. Sowohl die neuen Dateien als auch 
die Schrift Nexus werden automatisch registriert.

\subsection{Manuelle Installation}
\Index{updmap}
% \paragraph{Lokale/systemweite Installation}

Bei der Installation sollte man sich zuerst zwischen einer Benutzer-Installation
und einer System-Installation entscheiden.
Bei einer \emph{Benutzer-Installation} werden alle Dateien im lokalen Benutzerverzeichnis installiert.
Dies kann damit auch auf Systemen geschehen auf denen man nur
eingeschränkte Rechte besitzt.
Die Vorlagen sind dann aber auch nur für den aktuellen Nutzer benutzbar.
Darüber hinaus ist zu beachten, dass durch eine lokale Installation auch
lokale Datenbanken angelegt werden, die fortan benutzt werden.
Daher würden zum Beispiel Schriften, die anschließend global installiert werden,
nicht gefunden werden, da sie nur für die globale Datenbank registriert worden wären.

Dieses Problem stellt sich bei einer \emph{systemweiten Installation} nicht,
welche aber auch nur mit entsprechenden Rechten durchgeführt werden kann.
Sind diese jedoch vorhanden, so ist eine \emph{systemweite Installation im
Allgemeinen einer lokalen Installation vorzuziehen}.

\subsubsection{Benötigte Pakete}

Die Vorlagen setzen ein Vorhandensein der Folgenden Pakete voraus:

\begin{itemize}
  \item texlive-base
  \item texlive-latex-extra
  \item texlive-fonts-extra
  \item latex-beamer
  \item texlive-lang-german
\end{itemize}

Falls diese nicht auf dem System installiert sind, müssen sie nachinstalliert
werden, da sonst die Vorlagen oder Teile der Vorlagen nicht benutzbar sind.

Die Installation aller Pakete kann mit folgendem Befehl durchgeführt werden:
\begin{lstlisting}[style=cmd]
sudo aptitude install texlive-base texlive-latex-extra \
  texlive-fonts-extra latex-beamer texlive-lang-german
\end{lstlisting}

Im Folgenden werden sowohl systemweite als auch lokale Installation einzeln
beschrieben.

\subsubsection{Systemweite Installation}

\begin{important}
  Hierfür werden root-Rechte benötigt.
\end{important}

\begin{description}
  \item[1. Dateien Kopieren] Zuerst müssen alle in der Zip-Datei enthaltenen
    Dateien entpackt werden.
    Diese liegen dort schon in der korrekten Ordner-Struktur vor, sodass
    sie lediglich in das Ziel-texmf-Verzeichnis \emph{hinein}kopiert werden
    müssen.
    \begin{lstlisting}[style=cmd]
# unzip -d /usr/share/texmf/ tubslatex_0.3-beta2.zip
    \end{lstlisting}

  \item[2. Dateien registrieren]
    Dies geschieht mit dem Befehl
    \begin{lstlisting}[style=cmd]
# mktexlsr
    \end{lstlisting}

  \item[3. Schriften registrieren]\hfill

    \begin{lstlisting}[style=cmd]
# updmap-sys --nomkmap --nohash --enable Map=NexusProSans.map
# updmap-sys --nomkmap --nohash --enable Map=NexusProSerif.map
# updmap-sys
    \end{lstlisting}
    
  \item[4. Arial installieren]
    Falls die Schrift 'arial' noch nicht installiert wurde, so sollte dies
    nachgeholt werden, da sie für einige Vorlagen als Standard-Schrift
    eingestellt ist.
    
    Die Installation erfolgt mit Hilfe des Programms \lstinline{getnonfreefonts},
    dass die benötigten Dateien automatisch herunterlädt und alle zur Installation
    nötigen Schritte durchführt.
    
    Eine systemweite Installation von Arial geschieht mit Hilfe des Befehls
    \begin{lstlisting}[style=cmd]
# getnonfreefonts-sys --verbose arial-urw
    \end{lstlisting}
    
\end{description}

\subsubsection{Lokale Installation}

\begin{description}
  \item[1. Dateien Kopieren] Zuerst müssen alle in der Zip-Datei enthaltenen
    Dateien entpackt werden.
    Diese liegen dort schon in der korrekten Ordner-Struktur vor, sodass
    sie lediglich in das Ziel-texmf-Verzeichnis \emph{hinein}kopiert werden
    müssen.
    \begin{lstlisting}[style=cmd]
$ unzip -d ~/texmf/ tubslatex_0.3-beta2.zip
    \end{lstlisting}

  \item[2. Dateien registrieren]
    Dies geschieht mit dem Befehl
    \begin{lstlisting}[style=cmd]
$ mktexlsr ~/texmf
    \end{lstlisting}

  \item[3. Schriften registrieren]\hfill

    \begin{lstlisting}[style=cmd]
$ updmap --nomkmap --nohash --enable Map=NexusProSans.map
$ updmap --nomkmap --nohash --enable Map=NexusProSerif.map
$ updmap
    \end{lstlisting}
    
  \item[4. Arial installieren]
    Falls die Schrift 'arial' noch nicht installiert wurde, so sollte dies
    nachgeholt werden, da sie für einige Vorlagen als Standard-Schrift
    eingestellt ist.
    
    Die Installation erfolgt mit Hilfe des Programms \lstinline{getnonfreefonts},
    dass die benötigten Dateien automatisch herunterlädt und alle zur Installation
    nötigen Schritte durchführt.
    
    Eine Lokale Installation von Arial geschieht mit Hilfe des Befehls
    \begin{lstlisting}[style=cmd]
$ getnonfreefonts --verbose arial-urw
    \end{lstlisting}
    
\end{description}



\newenvironment{knownissue}[1]{%
  \item {\sffamily\bfseries#1}\hfill
  \newcommand{\solution}[1]{\noindent{\itshape ##1}}
}{%
}

\subsection{Fehlerbehebung}

\begin{itemize}

\begin{knownissue}{%
  updmap bricht plötzlich ab (Fehlercode 2) / getnonfreefonts bricht ab
}

Dieses Problem ist unter anderem von texlive2009 bekannt.
Dort hat sich ein Problem eingeschlichen, das leider die Funktionalität von
updmap teilweise lahmlegt.

\solution{Mögliche Lösung:}
\begin{itemize}
  \item Die Datei 10local.cfg (falls sie noch nicht vorhanden ist) manuell
    erstellen (z.B. mit \texttt{touch}).
  \item Nicht die Option \texttt{--quiet} verwenden.
\end{itemize}

Alternativ kann auch die Zeile \lstinline{set -e} am Anfang des Skripts einfach auskommentiert werden.
\end{knownissue}


\begin{knownissue}{
  \texttt{mktexpk: don't know how to create bitmap font for \ldots}\newline
  nach Schriften-Installation mit \lstinline{updmap-sys}}

Dieses Problem kann auftauchen, wenn auf dem System vom aktuellen Benutzer
zuvor schon einmal \lstinline{updmap} (ohne -sys) ausgeführt wurde.
Dann legt texlive eine lokale Map-Datei an und verwendet fortan diese,
auch wenn die System-Map mit \lstinline{updmap-sys} upgedated wurde.

\noindent\textit{Überprüfung:}

Die Ausgabe von
\begin{lstlisting}[style=cmd]
kpsewhich pdftex.map
\end{lstlisting}
sollte einen anderen Pfad als
\lstinline{/var/lib/texmf/fonts/map/pdftex/updmap/} liefern.

\solution{Lösung 1:}

Entfernen der aktuell verwendeten (lokalen) Map-Datei.

\solution{Lösung 2:}

Installation weiterer Schriften mit \lstinline{updmap} statt
\lstinline{updmap-sys}.

\end{knownissue}

\end{itemize}

\clearpage
\section{Mac OS -- TexLive}%TODO: TexLive/MacTex?
\Index{Mac OS}
\Index{Installation!Mac OS}

\subsection{Manuelle Installation}

Diese Beschreibung erklärt zur Zeit nur die lokale Installation.

\begin{description}
  \item[1. Dateien Kopieren] Zuerst müssen alle in der Zip-Datei enthaltenen
    Dateien entpackt werden.
    Diese liegen dort schon in der korrekten Ordner-Struktur vor, sodass
    sie lediglich in das Ziel-texmf-Verzeichnis \emph{hinein}kopiert werden
    müssen.
    Dies sollte im Allgemeinen \lstinline{~/Library/texmf} sein.
%   \item[2. Dateien registrieren]

  \item[2. Schriften registrieren]
    In einem Terminal sind dazu folgende 3 Befehle einzugeben:
    \begin{lstlisting}[style=cmd]
$ updmap --nomkmap --nohash --enable Map=NexusProSans.map
$ updmap --nomkmap --nohash --enable Map=NexusProSerif.map
$ updmap
    \end{lstlisting}


  \item[3. Arial installieren]
    Falls die Schrift 'arial' noch nicht installiert wurde, so sollte dies
    nachgeholt werden, da sie für einige Vorlagen als Standard-Schrift
    eingestellt ist.
    
    Die Installation erfolgt mit Hilfe des Programms \lstinline{getnonfreefonts},
    dass die benötigten Dateien automatisch herunterlädt und alle zur Installation
    nötigen Schritte durchführt.
    \begin{enumerate}[a)]
      \item Falls nicht vorhanden, kann ein Installer unter folgendem Link
        bezogen werden:
        \url{http://tug.org/fonts/getnonfreefonts/install-getnonfreefonts}
    
        Die Installation von \lstinline{getnonfreefonts} erfolgt dann mit dem
        Befehl\footnote{Weitere Informationen dazu gibt es hier:\\
          \url{http://www.golatex.de/vollautomatischen-installation-einiger-nicht-freier-fonts-t5386.html}}
        \begin{lstlisting}[style=cmd]
$ texlua install-getnonfreefonts
        \end{lstlisting}
        
      \item Eine lokale Installation von Arial geschieht mit Hilfe des Befehls
    \begin{lstlisting}[style=cmd]
$ getnonfreefonts arial-urw
    \end{lstlisting}
    
    Eine systemweite Installation ist \emph{alternativ} mit dem Zusatz \lstinline{-sys}
    möglich\\ (\lstinline{getnonfreefonts-sys arial-urw}).
    \end{enumerate}

\end{description}



% TODO...

\def\foonote{%
kpathsea: Running mktexpk --mfmode / --bdpi 600 --mag 1+240/600 --dpi 840 NexusProSans-Bold-Regular--base
mktexpk: don't know how to create bitmap font for NexusProSans-Bold-Regular--base.

Reproduce:
  - sudo updmap
  - sudo aptitude install texlive-tubs
Test:
  - wenn 'kpsewhich pdftex.map' nicht /var/lib/.. ausgibt, liegt der Fehler vor
Beschreibung:
  - Dieses Problem kann unter anderem auftreten, wenn vor der Installation
    irgendwann einmal 'updmap' aufgerufen wurde.
Begründung:
  - Installation greift auf updmap-sys zurück, die Damit generierte Datei
    wird aber nicht mehr verwendet.
Lösung 1:
  - updmap manuell aufrufen.
Lösung 2:
  - Datei, die bei 'kpsewhich pdftex.map' angezeigt wird, löschen
    (z.B. /root/.texmf-var/fonts/map/pdftex/updmap/pdftex.map)
  - 'sudo updmap-sys' aufrufen
}
