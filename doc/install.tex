\chapter{Installation}\label{chap:install}

Aktuell wird \tubslatex in drei verschiedenen Varianten angeboten.
Zum einen als gepackter tex-Paketaum, der sowohl lokal als auch systemweit
entpackt werden kann. Allerdings müssen hier alle notwendigen Schritte zur 
Installation der Schriften und Anmelden der neuen Datein manuell gemacht werden.
Für Windows-Benutzer, die MikTeX verwenden, gibt es einen automatischen
Installer, der alle nötigen Schritte automatisch ausführt.
Ebenso gibt es für Debian-basierte Systeme, die texlive verwenden ein
\texttt{.deb}-Paket, das alle Abhängigkeiten korrekt auflöst und die Vorlagen
automatisch installiert.

\section{Automatische Installation}

\subsection{Windows -- MikTeX}

Bei Verwendung von MikTeX unter Windows kann auf den verfügbaren Installer
zurück gegriffen werden. Um eine fehlerfreie Installation zu ermöglichen
sollte MiKTeX in der Version 2.9 vorhanden sein.
Ältere Versionen werden nur bedingt unterstützt.

Der Installer kopiert alle benötigten Dateien in ein frei wählbares Verzeichnis
auf der Festplatte und registriert dies automatisch als neuen texmf-Baum.
Auch die Schriftart Nexus wird standardmäßig installiert und in 
Font-Verwaltung registriert.

\subsection{Ubuntu/Debian -- TexLive}

Für debian-basierte Systeme wie Ubuntu oder Debian selber steht ein
\texttt{.deb}-Paket zur Verfügung, dass einfach mit Hilfe eines Paketmanagers
installiert werden kann.
Genutz werden können dafür die diversen graphischen Tools oder einfach das
Kommandozeilenprogramm \texttt{gdebi}. Diese Programme lösen alle Abhängigkeiten
auf und installieren sie automatisch mit.

Die Installation der Dateien erfolgt in das Verzeichnis
\lstinline{/usr/share/texmf-texlive/}. Sowohl die neuen Dateien als auch 
die Schrift Nexus werden automatisch registriert.

\section{Manuelle Installation}

Zur manuellen Installation liegen die LaTeX-Dateien als gepacktes
zip-Verzeichnis vor, in dem die komplette tex-Pfadstruktur oberhalb der
texmf-Ebene enthalten ist. Konkret sind dies die Ordner
\lstinline{doc}, \lstinline{tex} und \lstinline{fonts}.

Diese können entweder lokal oder global entpackt werden.
Für erste Variante kann der Inahlt einfach in das Verzeichnis
\lstinline{/home/benutzername/texmf} hinein geschrieben werden.

Es sei darauf hingewiesen, dass alle weiteren benötigten Pakete der
Tex-Distribution ggf. manuell installiert werden müssen.
MikTeX macht dies normalerweise automatisch.
Bei texlive können unter Linux zum Beispiel die entsprechenden System-Pakete
installiert werden.


\subsection{MikTeX}

\begin{description}
  \item[1. Dateien Kopieren]
  \item[2. Dateien registrieren]
  \item[3. Schriften registrieren]
\end{description}


\subsection{TexLive}

\begin{description}
  \item[1. Dateien Kopieren] Zuerst müssen alle in der Zip-Datei enthaltenen
    Dateien entpackt werden.
    Diese liegen dort schon in der korrekten Ordner-Sktruktur vor, sodass
    sie lediglich in das Ziel-texmf-Verzeichnis \emph{hinein}kopiert werden
    müssen.
    
    Grundsätzlich kann man sich bei der Installation entscheiden,
    ob die Vorlagen nur Lokal für den aktuellen Benutzer oder systemweit
    installiert werden sollen.
    Es sei darauf hingewiesen, dass eine lokale Installation ein paar
    Fallstricke mit sich bringt, die evtl. zu Problemen bei der späteren
    Installation weiterer Pakete führen können.
    
    Entpacken als Benutzer ins lokale texmf-Verzeichnis:
    \begin{lstlisting}
$ unzip -d ~/texmf/ tubslatex_0.3-alpha2.zip
    \end{lstlisting}
    Oder als Root ins systemweite texmf-Verzeichnis:
    \begin{lstlisting}
# unzip -d /usr/share/texmf/ tubslatex_0.3-alpha2.zip
    \end{lstlisting}

  \item[2. Dateien registrieren]
  \item[3. Schriften registrieren]\hfill

    \begin{lstlisting}
# /usr/bin/updmap-sys --nomkmap --nohash --enable Map=NexusProSans.map
# /usr/bin/updmap-sys --nomkmap --nohash --enable Map=NexusProSerif.map
# /usr/bin/updmap-sys
    \end{lstlisting}
    
    Für eine lokale Installation muss jeweils der Zustatz \texttt{-sys}
    weggelassen werden.

\end{description}


\newenvironment{knownissue}[1]{%
  \paragraph{#1}\hfill
  \newcommand{\solution}[1]{\noindent{\itshape ##1}}
}{%
}

\section{Fehlerbehebung}

\subsection{MikTeX}

\subsection{texlive}

\begin{knownissue}{%
  updmap bricht plötzlich ab (Fehlercode 2) / getnonfreefonts bricht ab
}

Dieses Problem ist unter anderem von texlive2009 bekannt.
Dort hat sich ein Problem eingeschlichen, das leider die Funktionalität von
updmap teilweise lahmlegt.

\solution{Mögliche Lösung:}
\begin{itemize}
  \item Die Datei 10local.cfg (falls sie noch nicht vorhanden ist) manuell
    erstellen (z.B. mit \texttt{touch}).
  \item Nicht die Option \texttt{--quiet} verwenden.
\end{itemize}

Alternativ kann auch die Zeile \lstinline{set -e} am Anfang des Skripts einfach auskommentiert werden.
\end{knownissue}


\begin{knownissue}{
  \texttt{mktexpk: don't know how to create bitmap font for \ldots}\newline
  nach Schriften-Installation mit \lstinline{updmap-sys}}

Dieses Problem kann auftauchen, wenn auf dem System vom aktuellen Benutzer
zuvor schon einmal \lstinline{updmap} (ohne -sys) ausgeführt wurde.
Dann legt texlive eine lokale Map-Datei an und verwendet fortan diese,
auch wenn die System-Map mit \lstinline{updmap-sys} upgedated wurde.

\noindent\textit{Überprüfung:}

Die Ausgabe von
\begin{lstlisting}
kpsewhich pdftex.map
\end{lstlisting}
sollte einen anderen Pfad als
\lstinline{/var/lib/texmf/fonts/map/pdftex/updmap/} liefern.

\solution{Lösung 1:}

Entfernen der aktuell verwendeten (lokalen) Map-Datei.

\solution{Lösung 2:}

Installation weiterer Schriften mit \lstinline{updmap} statt
\lstinline{updmap-sys}.

\end{knownissue}




% TODO...

\def\foonote{%
kpathsea: Running mktexpk --mfmode / --bdpi 600 --mag 1+240/600 --dpi 840 NexusProSans-Bold-Regular--base
mktexpk: don't know how to create bitmap font for NexusProSans-Bold-Regular--base.

Reproduce:
  - sudo updmap
  - sudo aptitude install texlive-tubs
Test:
  - wenn 'kpsewhich pdftex.map' nicht /var/lib/.. ausgibt, liegt der Fehler vor
Beschreibung:
  - Dieses Problem kann unter anderem auftreten, wenn vor der Installation
    irgendwann einmal 'updmap' aufgerufen wurde.
Begründung:
  - Installation greift auf updmap-sys zurück, die Damit generierte Datei
    wird aber nicht mehr verwendet.
Lösung 1:
  - updmap manuell aufrufen.
Lösung 2:
  - Datei, die bei 'kpsewhich pdftex.map' angezeigt wird, löschen
    (z.B. /root/.texmf-var/fonts/map/pdftex/updmap/pdftex.map)
  - 'sudo updmap-sys' aufrufen
}
