\chapter{Gauß-Layout}\label{chap:gausspage}

Dieses Kapitel beschreibt das allgemeine Layoutsystem zur Erstellung von
Inhalten und Hintergründen im Gaußraster des Corporate Designs.


\begin{figure}\centering
  \fboxsep0mm
  \begin{minipage}{0.35\textwidth}
    \fbox{\includegraphics[width=\textwidth,page=1]{examples/gausssegments.pdf}}
    \subcaption{Absenderbereicht oben}\label{fig:gausspage:topsender}
  \end{minipage}
  \quad
  \begin{minipage}{0.35\textwidth}
    \fbox{\includegraphics[width=\textwidth,page=2]{examples/gausssegments.pdf}}
    \subcaption{Absenderbereicht unten}\label{fig:gausspage:bottomsender}
  \end{minipage}
  \caption{Gaußraster mit möglichen Logo-Positionen}\label{fig:gausspage}
\end{figure}



\begin{Declaration}
  \Macro{showtubslogo}\OParameter{Position}
\end{Declaration}

Bewirkt Darstellung des TU-Siegelbandlogos im aktiven Layout.
Die Option \PName{Position} erlaubt die Angabe der Darstellungsseite
(links/rechts). Standardmäßig wird das Logo links bzw. innen dargestellt.

\begin{Declaration}
  \Macro{showlogo}\PParameter{Logo}
\end{Declaration}

Bewirkt Darstellung eines Individuellen Logos im aktuellen Layout.
\PName{Logo} kann dabei entweder einfacher Text oder auch ein 
mit \Macro{includegraphics} eingebundenes Bild sein.

Die Positionierung wird automatisch an die Positionierung des Siegelbandlogos
angepasst. Wird dies links bzw. innen platziert, so steht das individuelle
Logo rechts bzw. außen.

\begin{Declaration}
  \Macro{showtopline}
\end{Declaration}

Bewirkt Darstellung einer Trennlinie zwischen Absender und Kommunikationsbereich
im aktuellen Layout.

\begin{Declaration}
  \Macro{showdesignhelper}
\end{Declaration}

\begin{Declaration}
  \Macro{bgelement}\OParameter{Darstellung}\PParameter{Höhe}
\end{Declaration}

Erstellt ein Hintergrundelement im Gaußraster mit angegebener \PName{Höhe}.
Der Parameter \PName{Darstellung} kann die folgenden Einstellungen verarbeiten:

\begin{Declaration}
  \KOption{bgcolor}\PName{Farbe}\\
  \KOption{bgimage}\PName{Bild-Datei}\\
  \KOption{imagefit}\PName{Darstellungsoption}
\end{Declaration}

Mit \OptionValue{bgcolor}{Farbe} wird das Hintergrundelement mit der angegebenen
Farbe gefüllt.

Die Option \OptionValue{bgimage}{Bild-Datei} erlaubt dagegen die Darstellung
eines Hintergrundbildes im Element.
Da der Darstellungsbereich fest vorgegeben ist, muss das eingebundene Bild
in diesen Bereich eingepasst werden. Dies geschieht automatisch, die Art
der Einpassung lässt sich aber mit der Option \Option{imagefit} kontrollieren.
Sie erlaubt folgende Einstellungen:


\begin{desctable}
\entry{\PValue{cropped}}{%
  Automatisches Abschneiden. Dies ist die Standardeinstellung.
}
\entry{\PValue{cropx}}{%
  Abschnitt horizontal.
}
\entry{\PValue{cropy}}{%
  Abschnitt vertikal.
}
\entry{\PValue{scaled}}{%
  Horizontale \emph{und} vertikale Skalierung.
}
\end{desctable}
