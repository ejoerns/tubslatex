\chapter{Gauß-Layout-System}\label{sec:gausslayout}

Dieses Kapitel beschreibt das allgemeine Layoutsystem zur Erstellung von
Inhalten und Hintergründen im Gaußraster des Corporate Designs.
Es findet als Basis unter anderem bei der Darstellung von Titelseiten oder
Veranstaltungsplakaten Anwendung. Die hier beschriebenen Befehle und Optionen
können daher in weiten Teilen auf alle Konstrukte in \tubslatex angewendet
werden, die im Gauß-Layout dargestellt werden.

Eine Grundkenntnis über den allgemeinen Aufbau von Seiten im Corporate
Design bzw. im Gaußraster ist für das Verständnis hilfreich.
Eine kurze allgemeine Einleitung ist in \chaptername~\ref{sec:intro:dascddertubs}
zusammengestellt. Für einen umfassenden Überblick sei hier auf \cite{toolbox}
verwiesen.

Es gibt grundlegend drei verschiedene Möglichkeiten das Gauß-Layout-System
zu nutzen:
\begin{itemize}
  \item Definition einer Hintergrunddarstellung,
  \item Verwendung von Textboxen,
  \item Kombination von Hintergrundelementen und Textboxen,
\end{itemize}
wobei die Letztgenannte den Regelfall darstellt und daher im folgenden Kapitel
zuerst beschrieben wird.


% TODO: note \thispagestyle{}?

\clearpage
\section{Gaußraster-Seiten}\label{sec:gausslayout:gausspage}

Die hier vorgestellte Umgebung zum Erstellen von Seiten im Gaußraster des
Corporate Design ist eine Kombination aus den in den Folgekapiteln
erläuterten Möglichkeiten Hintergrundlayouts und Gaußraster-Textboxen 
zu erstellen. In den Beschreibungen einiger Elemente wird daher für eine
detailliertere Erklärung auf die folgenden Kapitel verwiesen.

Die vorgestellte Umgebung wird in vielen Teilen der Vorlagen eingesetzt
und bildet eines der Basis-Elemente der Vorlagen
Sie bildet zum Beispiel unter anderem die Basis für die Erstellung von
Veranstaltungsplakaten und wird in den automatischen Titelseiten für Dokumenten
genutzt.

\begin{Declaration}
  \XMacro{begin}\PParameter{\Environment{gausspage}}%
    \OParameter{Optionen}\\
  \quad\dots\\
  \XMacro{end}\PParameter{gausspage}
\end{Declaration}

Jede \Environment{gausspage}-Umgebung leitet eine komplette eigenständige
Seite ein innerhalb derer ein am Gaußraster orientiertes Seitenlyout
definiert werden kann.

Als \PName{Optionen} können alle in \chaptername~\ref{sec:gausslayout:bglayout:options}
für \Macro{bglayout} definierten Optionen verwendet werden.
Die Option \Option{pages} ist standardmäßig auf \PValue{single}
voreingestellt.

Innerhalb der Umgebung können alle Standardelemente
des CD dargestellt werden. Dazu gehören das Siegelbandlogo,
ggf. eine rote Trennlinie zwischen Absender- und Kommunikationsbereich,
sowie ein mögliches Sekunär-/Institutslogo.
Der eigentliche Seiteninhalt wird mit der Umgebung \Environment{segment} gesetzt.


\begin{Declaration}
  \Macro{showtubslogo}\OParameter{Optionen}\\
  \Macro{showlogo}\PParameter{Logo}\\
  \Macro{showtopline}
\end{Declaration}

Mit dem Befehl \Macro{showtubslogo} wird die Darstellung des Siegelbandlogos
auf der aktuellen Seite bewirkt. Ein optionaler Parameter kontrolliert
Darstellungsseite und Darstellungsart.

Der Befehl \Macro{showlogo} wird verwendet um sein sekundäres Logo
eines Instituts oder einer zentralen Einrichtung im Absenderbereich
darzustellen. Dies kann sowohl eine Grafik als auch ein
(bei zentralen obligatorischer) einfacher Text sein.

Mit \Macro{showtopline} wird eine horizontal rote Trennlinie zwischen
Absender- und Kommunikationsbereich eingefügt.

Zu allen drei Befehlen ist eine ausführlichere Beschreibung
in \chaptername~\ref{sec:gausslayout:bglayout:elemente} zu finden.

\begin{Declaration}
  \XMacro{begin}\PParameter{\Environment{segment}}%
    \OParameter{Optionen}
    \Parameter{Höhe}\\
  \quad\dots\\
  \XMacro{end}\PParameter{segment}
\end{Declaration}

Mit der Umgebung \Environment{segment} könnnen die einzelnen
Segmente gesetzt werden.
Diese gehen jeweils über die ganze Textbreite.
Der Parameter \PName{Höhe} gibt die Anzahl der Basis-Segmente an, die
das Segment umfassen soll.
Die Basis-Segmente werden dabei von den \Environment{segment}-Umgebungen
von oben nach unten belegt.
Illustriert wird dies in \figurename~\ref{fig:gausslayout:segment:beispiel}.

\begin{figure}[!ht]
\begin{minipage}[b]{0.3\textwidth}\fboxsep0mm
\begin{lstlisting}[escapechar=@]
\begin{gausspage}
  \begin{segment}{@\textcolor{tubsRed}{3}@}
    ...
  \end{segment}
  \begin{segment}{@\textcolor{tubsRed}{4}@}
    ...
  \end{segment}
  \begin{segment}{@\textcolor{tubsRed}{1}@}
    ...
  \end{segment}
\end{gausspage}
\end{lstlisting}%
\subcaption{Aufteilung}
\label{fig:gausslayout:segment:beispiel:code}
\end{minipage}\hfill
\begin{minipage}[b]{0.32\textwidth}\fboxsep0mm
  \fbox{\includegraphics[width=\textwidth,page=3]{examples/gausssegments.pdf}}
%     \subcaption{Absenderbereich oben}\label{fig:gausslayout:topsender}
  \subcaption{Basis-Segmente}
  \label{fig:gausslayout:segment:beispiel:basis}
\end{minipage}\hfill
\begin{minipage}[b]{0.32\textwidth}\fboxsep0mm
  \fbox{\includegraphics[width=\textwidth,page=4]{examples/gausssegments.pdf}}
%     \subcaption{Absenderbereich oben}\label{fig:gausslayout:topsender}
  \subcaption{Segment-Aufteilung}
  \label{fig:gausslayout:segment:beispiel:aufteilung}
\end{minipage}
\caption{Beispiel zur Aufteilung der Basis-Segmente (b)
  mit der \Environment{segment}-Umgebung (a)
  in Ausgabe-Segmente (c)}
\label{fig:gausslayout:segment:beispiel}
\end{figure}


% Als \PName{Optionen} können sowohl alle in \chaptername~\ref{sec:gausslayout:bgsegment}
% für \Macro{bgsegment} definierten Optionen als auch alle in \chaptername~\ref{sec:gausslayout:gaussbox} für
% die \Environment{gaussbox}-Umgebung definierten Optionen verwendet werden.

\paragraph{Vorder-/Hintergrundfarbe}\hfill

\begin{Declaration}
  \KOption{bgcolor}\PName{Farbe}\\
  \KOption{fgcolor}\PName{Farbe}
\end{Declaration}

Mit der Option \Option{bgcolor} kann eine Hintergrundfarbe für das
jeweilige Segment festgelegt werden. Als Argument können alle in \tubslatex
definierten Farben übergeben werden.
Eine allgemeine Übersicht über die Farbdarstellung ist in
\chaptername~\ref{sec:tubscolors} zu finden.

Die Vordergrundfarbe wird analog zur Hintergrundfarbe mit der
Option \Option{fgcolor} gewählt.
Die Vordergrundfarbe beeinflusst unter anderem die Farbe des Textes.


\paragraph{Hintergrundbild}\hfill

\begin{Declaration}
  \KOption{bgimage}\PName{Bild-Datei}\\
  \KOption{imagefit}\PName{Darstellungsoption}
\end{Declaration}

% TODO: just a text copy
Die Option \OptionValue{bgimage}{Bild-Datei} erlaubt die Darstellung
eines Hintergrundbildes im aktuellen Element.
Da der Darstellungsbereich fest vorgegeben ist, muss das eingebundene Bild
in diesen Bereich eingepasst werden.
Dazu ist jedoch kein mühsames manuelles Ausprobieren notwendig.
Die Vorlagen verfügen über einen Algorithmus, der die Einpassung vollautomatisch
übernimmt.
Die Art der Einpassung lässt sich dabei mit der Option \Option{imagefit} kontrollieren.
%
Mögliche Werte sind in \tablename~\ref{tbl:gausslayout:imagefitoptions} erläutert.

\figurename~\ref{fig:gausslayout:imagefitexample} zeigt beispielhaft die Funktionalität
der automatischen Einpassung.


\paragraph{Vertikale Textausrichtung}\hfill

\begin{Declaration}
  \Option{t}\\
  \Option{c}\\
  \Option{b}
\end{Declaration}

% TODO: just a text copy
Mit diesen Optionen lässt sich die vertikale Ausrichtung des Textes innerhalb
der Box festlegen. Die Option \Option{t} entspricht dem Standardverhalten und richtet
den Inhalt am \emph{oberen} Rand des Inhaltsbereichs der Box aus.
Die Option \Option{c} bewirkt eine vertikale Zentrierung des Inhaltes in des Box.
Mir der Option \Option{b} wird der Inhalt am \emph{unteren} Rand des
Inhaltsbereichs der Box ausgerichtet.


% \begin{lstlisting}[captionpos=b,caption={Beispiel-Nutzung von gausspage}]
% \begin{gausspage}{%
%   \showtubslogo
%   \begin{segment}[bgimage=mypic.jpg]{2}
%     // Inhalt
%   \end{segment}
%   \begin{segment}[bgcolor=tuGreen]{3}
%     // Inhalt
%   \end{segment}
%   \begin{segment}[bgcolor=tuGreenDark]{3}
%     // Inhalt
%   \end{segment}
% }
% \end{lstlisting}

% TODO: place? ref?

\clearpage
\section{Hintergrund-Layout}\label{sec:gausslayout:bglayout}

Ein Hintergrund-Layout kann auf zwei verschiedene Weisen definiert bzw.
gesetzt werden.
Zum einen kann ein Layout so gesetzt werden, dass es sofort angewendet wird,
zum anderen kann ein Layout zuerst definiert und unter einem Namen
abgespeichert werden, um dann später dieses Layout mit Hilfe des angegebenen 
Namen anzuwenden. Letzteres ist zum Beispiel für individuelle Vorlagen
nützlich.

\begin{Declaration}
  \Macro{bglayout}\OParameter{Optionen}\PParameter{Layout-Anweisungen}
\end{Declaration}

Mit Hilfe des Befehls \Macro{bglayout} kann das Hintergrundlayout
im Gaußraster direkt gesetzt werden.
Die darzustellenden Komponenten und Segmente werdem im Parameter
\PName{Layout-Anweisungen} übergeben.
Die Benutzung von Segmenten ist in \chaptername~\ref{sec:gausslayout:bgsegment}
erläutert.
Die Art der Darstellung kann über den Parameter \PName{Optionen} angepasst werden.
Gültige Optionen sind weiter unten aufgelistet.

Alle gesetzten Komponenten werden im Hintergrund und unabhängig vom Inhalt der Seite
dargestellt. Dies hat in etwa dieselbe Funktionalität wie vorbedrucktes Papier.
Ein Beispiel zur Verwendung von \Macro{bglayout} liefert auch
\lstlistingname~\ref{lst:bglayout}.


\begin{lstlisting}[captionpos=b,caption={Beispiel-Nutzung von bglayout},label=lst:bglayout]
\bglayout[pages=single]{%
  \showtubslogo
  \bgsegment[bgimage=mypic.jpg]{2}
  \bgsegment[bgcolor=tuGreen]{3}
  \bgsegment[bgcolor=tuGreenDark]{3}
}
\end{lstlisting}


\begin{Declaration}
  \Macro{defbglayout}\OParameter{Optionen}\Parameter{Name}\Parameter{Layout-Anweisungen}\\
  \Macro{usebglayout}\Parameter{Name}
\end{Declaration}

\begin{sloppypar}
Mit dem Befehl \Macro{defbglayout} kann ein Hintergrundlayout vordefiniert werden.
Es wird unter der im Parameter \PName{Name} angegebenen Bezeichnung
abgespeichert.
Der Befehl \Macro{usebglayout} dient dann dazu, ein unter \PName{Name} gespeichertes
Layout anzuwenden.
\end{sloppypar}

\subsubsection{Optionen}\label{sec:gausslayout:bglayout:options}

Für \Macro{bglayout} wie auch für \Macro{defbglayout} stehen folgende
\PName{Optionen} zur Verfügung:

\begin{Declaration}
  \KOption{sender}\PName{top/bottom}
\end{Declaration}

Steuert die Positionierung des \glslink{glos:absenderbereich}{Absenderbereichs}.
Der Wert \PValue{top} platziert den Absenderbereich am oberen,
der Wert \PValue{bottom} am unteren Ende des Blattes.

\begin{hint}
  Die Position des Absenderbereichs hat Einfluss auf die Darstellungsreihenfolge
  der Gauß-Segmente. Das jeweils größte Segment befindet sich immer 
  benachbart zum Absenderbereich.
  Wird der Absenderbereich nach unten gesetzt, ist das kleinste Segment oben
  und das größte Segment unten.
\end{hint}

\begin{figure}[!ht]\centering
  \fboxsep0mm
  \begin{minipage}{0.35\textwidth}
    \fbox{\includegraphics[width=\textwidth,page=1]{examples/gausssegments.pdf}}
    \subcaption{Absenderbereich oben\\ (\KOption{sender}\PName{top})}\label{fig:gausslayout:topsender}
  \end{minipage}
  \quad
  \begin{minipage}{0.35\textwidth}
    \fbox{\includegraphics[width=\textwidth,page=2]{examples/gausssegments.pdf}}
    \subcaption{Absenderbereich unten\\ (\KOption{sender}\PName{bottom})}\label{fig:gausslayout:bottomsender}
  \end{minipage}
  \caption{Gaußraster für Absenderbereich oben bzw. unten mit Basis-Segmenten
  und möglichen Logo-Positionen}\label{fig:gausspage}
\end{figure}

\begin{Declaration}
  \KOption{pages}\PName{all/single}
\end{Declaration}

Legt fest für welche Seiten die aktuelle Einstellung gelten soll.
Der Wert \PValue{all} besagt, dass es für alle folgenden Seiten gelten soll,
während mit \PValue{single} die Darstellung nur auf der aktuellen Seite geändert 
wird.

\begin{Declaration}
  \Option{designhelper}
\end{Declaration}

Besonders für den Designprozess ist die Option \Option{designhelper}
hilfreich. Er bewirkt eine Darstellung der verfügbaren Segmente und
Siegelbandlogo-Positionen durch schwarze Rahmen,
abhängig von der Platzierung des Absenderbereiches.
Dies kann als Hilfe für die Segmentaufteilung benutzt werden.


\subsection{Segmente}\label{sec:gausslayout:bgsegment}

\begin{Declaration}
  \Macro{bgsegment}\OParameter{Darstellung}\PParameter{Höhe}
\end{Declaration}

Erstellt ein Hintergrundelement im Gaußraster mit angegebener \PName{Höhe}.
Diese gibt an wieviele Segmente des Gauß-Layouts für das aktuelle Element
verwendet werden sollen.
Die Segmente werden dabei von oben nach unten belegt.
Abhängig von der Position des Absenderbereiches werden die Segmente
entweder nach unten kleiner (Absender oben) oder größer (Absender unten).

\noindent\begin{minipage}{0.4\textwidth}
Pro Seite stehen allgemein maximal 8 (Hochformat) bzw. 6 (Querformat)
Segmente zur Verfügung. Werden mehr belegt als verfügbar,
so kommt es zu einer Fehlermeldung.
\end{minipage}
\begin{minipage}{0.6\textwidth}\centering
\begin{minipage}[b]{0.4\textwidth}\centering
\begin{tikzpicture}[scale=1.5]
  \def\tikzpageheight{2.97}
  \def\tikzpagewidth{2.1}
  \draw (0,0) rectangle (\tikzpagewidth, \tikzpageheight);
  \foreach \i in {1,3,6,10,15,21,28}{%
    \draw (0, \i*\tikzpageheight/36) -- (\tikzpagewidth, \i*\tikzpageheight/36);
  }
\end{tikzpicture}\\
Hochformat\\ (8 Segmente)
\end{minipage}
\begin{minipage}[b]{0.45\textwidth}\centering
\begin{tikzpicture}[scale=1.5]
  \def\tikzpageheight{2.1}
  \def\tikzpagewidth{2.97}
  \draw (0,0) rectangle (\tikzpagewidth, \tikzpageheight);
  \foreach \i in {1,3,6,10,15}{%
    \draw (0, \i*\tikzpageheight/21) -- (\tikzpagewidth, \i*\tikzpageheight/21);
  }
\end{tikzpicture}\\
Querformat\\ (6 Segmente)
\end{minipage}
\end{minipage}
\vspace*{0pt}

Der optionale Parameter \PName{Darstellung} kann die folgenden Einstellungen verarbeiten:

\subsubsection{Hintergrundfarbe}

\vspace{-6mm}
\begin{Declaration}
  \KOption{bgcolor}\PName{Farbe}
\end{Declaration}

Mit \OptionValue{bgcolor}{Farbe} wird das Hintergrundelement mit der angegebenen
Farbe gefüllt. Die im Corporate Design vordefiniert Farben können
Kapitel~\ref{sec:tubscolors} entnommen werden.

\subsubsection{Hintergrundbild}

\vspace{-6mm}
\begin{Declaration}
  \KOption{bgimage}\PName{Bild-Datei}\\
  \KOption{imagefit}\PName{Darstellungsoption}
\end{Declaration}

Die Option \OptionValue{bgimage}{Bild-Datei} erlaubt die Darstellung
eines Hintergrundbildes im aktuellen Element.
Da der Darstellungsbereich fest vorgegeben ist, muss das eingebundene Bild
in diesen Bereich eingepasst werden.
Dazu ist jedoch kein mühsames manuelles Ausprobieren notwendig.
Die Vorlagen verfügen über einen Algorithmus, der die Einpassung vollautomatisch
übernimmt.
Die Art der Einpassung lässt sich dabei mit der Option \Option{imagefit} kontrollieren.
Sie erlaubt folgende Einstellungen, die auch in \figurename~\ref{fig:gausslayout:imagefitexample}
dargestellt sind:

\begin{desctable}
\entry{\PValue{clipped}}{%
  Automatisches randloses Zuschneiden. Dies ist die Standardeinstellung.
  Wählt abhängig von Seitenverhälntissen automatisch zwischen
  \PValue{hclip} und \PValue{vclip}.
}
\entry{\PValue{hclip}/\PValue{fitheight}}{%
  Vertikale Skalierung, horizontaler Zuschnitt.
  Je nach Seitenverhälntissen von Bild und Darstellungsbereich können
  Teile des Bildes weggeschnitten werden.
}
\entry{\PValue{vclip}/\PValue{fitwidth}}{%
  Horizontale Skalierung, vertikaler Zuschnitt
  Je nach Seitenverhälntissen von Bild und Darstellungsbereich können
  Teile des Bildes weggeschnitten werden.
}
\entry{\PValue{scaled}}{%
  Horizontale \emph{und} vertikale Skalierung.
  Es wird kein Teil des Bildes abgeschnitten.
  Dabei kann jedoch das Seitenverhältnis stark verändert und somit 
  das Bild verzerrt werden.
}
\end{desctable}\label{tbl:gausslayout:imagefitoptions}

\makeatletter
{%
\fboxsep0mm
\newcommand\imgscaleexample[3]{%
%   \def\tubs@sb@imagefit{#1}%
  \setkeys{tubsbox}{imagefit=#3}%
  \tubs@sb@calc@autoscale{examples/infozentrum}{#1}{#2}%
  \fbox{\parbox[t][#2]{#1}{%
    \expandafter\includegraphics\expandafter[\@img@scale@param]{examples/infozentrum}}%
  }
}
\begin{figure}[ht]
\begin{minipage}[b]{0.3\textwidth}\centering
  \imgscaleexample{\textwidth}{2cm}{autoclip}% TODO!!!!
  \subcaption{\KOption{imagefit}\PValue{autoclip}}%
\end{minipage}\hfill
\begin{minipage}[b]{0.3\textwidth}\centering
  \imgscaleexample{\textwidth}{2cm}{scaled}%
  \subcaption{\KOption{imagefit}\PValue{scaled}}%
\end{minipage}\hfill
\begin{minipage}[b]{0.3\textwidth}\centering
  \imgscaleexample{\textwidth}{2cm}{hclip}% TODO!!!!
  \subcaption{\KOption{imagefit}\PValue{hclip}}%
\end{minipage}\\[1ex]
\begin{minipage}[b]{0.3\textwidth}\centering
  \imgscaleexample{2cm}{\textwidth}{autoclip}% TODO!!!!
  \subcaption{\KOption{imagefit}\PValue{autoclip}}%
\end{minipage}\hfill
\begin{minipage}[b]{0.3\textwidth}\centering
  \imgscaleexample{2cm}{\textwidth}{scaled}%
  \subcaption{\KOption{imagefit}\PValue{scaled}}%
\end{minipage}\hfill
\begin{minipage}[b]{0.3\textwidth}\centering
  \imgscaleexample{2cm}{\textwidth}{vclip}% TODO!!!!
  \subcaption{\KOption{imagefit}\PValue{vclip}}%
\end{minipage}
\caption{Beispiele zur Verwendung der automatischen Einpassung in \tubslatex}
\label{fig:gausslayout:imagefitexample}
\end{figure}

}
\makeatother


\subsection{Darstellungselemente}\label{sec:gausslayout:bglayout:elemente}
\subsubsection{TU-Logo}


\vspace*{-5mm}
\begin{Declaration}
  \Macro{showtubslogo}\OParameter{Optionen}
\end{Declaration}%
\vspace*{-2.6cm}\hfill\tubslogoAbs{5.38cm}

Bewirkt Darstellung des TU-Siegelbandlogos im aktiven Layout.
Die \PName{Optionen} erlauben unter anderem die Angabe der Darstellungsseite
(left/right).

\begin{Declaration}
  \Option{left}/\Option{inside}\\
  \Option{right}/\Option{outside}
\end{Declaration}

Mit der Siegelbandlogo-Option \Option{left} bzw. \Option{inside} wird das Siegelband
auf der Seite \emph{links} bzw. \emph{innen} platziert.
Dies entspricht der Standardeinstellung.
Es ist zu beachten, dass bei zweiseitiger Darstellung (\Option{twoside})
das Siegelband auf ungeraden Seiten links und auf geraden Seiten rechts platziert wird
(also immer \emph{innenseitig}!).
Mit der Option \Option{right} bzw. \Option{outside} wird das Siegelband auf der \emph{rechten}
Seite bzw. \emph{außen} dargestellt.

Die Position des Siegelbandlogos bestimmt gleichzeitig auch die Position
des Sekundär-/Institutslogos.
Dieses wird immer auf der entgegengesetzten Seite platziert.

\begin{hint}
Diese Optionen beeinflussen ebenfalls direkt die Darstellung aller möglicherweise
folgender Logos.
Diese werden entsprechend der jeweils letzten Seitenwahl gesetzt.
Wird das erste Logo also rechts, bzw. außen platziert, so werden alle folgenden
Logos ebenfalls rechts (bzw. im \Option{twoside}-Modus außen) platziert.
\end{hint}

\vspace*{1cm}
\begin{Declaration}
  \Option{plain}
\end{Declaration}
{\fboxsep0mm
  \vspace*{-2.6cm}\hfill\colorbox{tubsRed}{\parbox[b][2cm]{5.38cm}{~}}%
}

Die Siegelbandlogo-Option \Option{plain} bewirkt, dass nicht das Logo selber,
sondern stattdessen eine gleichgroße einfarbige Fläche in der Farbe
des Siegelbandlogos dargestellt wird.
Dies ist vor allem für Rückseiten sinnvoll.

\begin{Declaration}
  \Option{inbcorr}
\end{Declaration}

Mit der Siegelbandlogo-Option \Option{inbcorr} wird die Hintergrundfarbe
des Siegelbandes in den Bereich der Bindekorrektur weiter geführt,
sodass es nicht zu unschönen weißen Kanten bei zu schmaler Bindung kommen kann.
Auch lässt sich das Logo so praktisch nahtlos um eine Broschüre von vorne
nach hinten herum führen.

\subsubsection{Instituts-Logo}

\vspace*{-5mm}
\begin{Declaration}
  \Macro{showlogo}\PParameter{Logo}
\end{Declaration}

Bewirkt Darstellung eines individuellen Logos im aktuellen Layout.
\PName{Logo} kann dabei entweder einfacher Text oder auch ein
mit \Macro{includegraphics} eingebundenes Bild sein.
In diesem Fall wird eine eingebundene Grafik automatisch auf die korrekte
Höhe skaliert (solange in den Optionen nicht anders angegeben).

Die Positionierung wird automatisch an die Position des Absenderbereichs und
die Positionierung des Siegelbandlogos angepasst.
Wird dies links bzw. innen platziert, so steht das individuelle
Logo rechts bzw. außen.

\subsubsection{Trennlinie}

\vspace*{-5mm}
\begin{Declaration}
  \Macro{showtopline}
\end{Declaration}

Bewirkt Darstellung einer roten Trennlinie zwischen Absender- und Kommunikationsbereich.

\begin{hint}
  Die Trennlinie sollte aus optischen Aspekten nur in Verbindung
  mit einem weißen Hintergrund dargestellt werden.
  Farbige Segmente bewirken schon von sich aus eine sichtbare Trennung zwischen 
  Absender- und Kommunikationsbereich.
\end{hint}


\clearpage
\section{Text-Boxen}\label{sec:gausslayout:gaussbox}

\begin{Declaration}
  \XMacro{begin}\PParameter{\Environment{gaussbox}}%
          \OParameter{Optionen}%
          \Parameter{hPos}%
          \Parameter{vPos}%
          \Parameter{Breite}%
          \Parameter{Höhe}\\
  \quad\dots\\
  \XMacro{end}\PParameter{gaussbox}
\end{Declaration}

Mit \Environment{gaussbox} können einfache Boxen im Gaußraster gesetzt werden,
die unabhängig vom Seiteninhalt an der festgelegten Position gesetzt werden.
Neben der vertikalen Größe kann auch eine horizontale Größe im
\gls{glos:spaltenraster}
definiert werden, also mit einer Unterteilung in 6 Spalten.

Der Wert \PName{vPos} gibt dabei das Start-Segment im Gaußraster an
($[1\ldots 8]$ bzw. $[1\ldots6]$),
der Wert \PName{hPos} die Start-Spalte ($[1\ldots6]$).
Mit \PName{Höhe} wird angegeben wieviele Gauß-Segmente die Box umfassen soll,
mit \PName{Breite} wieviel Spalten.

\begin{example}\hfill
  \begin{lstlisting}
\begin{gaussbox}{1}{1}{3}{2}
  // Inhalt
\end{gaussbox}
  \end{lstlisting}
  Erzeugt eine Text-Box, beginnend in der ersten Zeile und Spalte,
  mit halber Textbreite (3 Spalten) und einer Höhe von 2 Segmenten.
\end{example}

% \subsection{Inhaltsdarstellungs-Optionen}

\subsubsection{Vertikale Text-Positionierung}

\begin{Declaration}
  \Option{t}\\
  \Option{c}\\
  \Option{b}
\end{Declaration}

Mit diesen Optionen lässt sich die vertikale Ausrichtung des Textes innerhalb
der Box festlegen. Die Option \Option{t} entspricht dem Standardverhalten und richtet
den Inhalt am \emph{oberen} Rand des Inhaltsbereichs der Box aus.
Die Option \Option{c} bewirkt eine vertikale Zentrierung des Inhaltes in des Box.
Mir der Option \Option{b} wird der Inhalt am \emph{unteren} Rand des
Inhaltsbereichs der Box ausgerichtet.

\begin{Declaration}
  \KOption{fgcolor}\PName{Farbe}\\
  \KOption{frame}\PName{Box}
\end{Declaration}

Mit \KOption{fgcolor}\PName{Farbe} kann die Vordergrund- bzw. Textfarbe
innerhalb der Gauss-Box ausgewählt werden.
Die Option \KOption{frame}\PName{Box} fügt einen Rahmen um den Textbereich
ein.% none, fbox, ..?


\subsubsection{Bilder}

\begin{Declaration}
  \KOption{imagefit}\PName{Option}
\end{Declaration}

Bestimmt die automatische Einpassung für Bilder.
Bei Verwendung des Befehls \Macro{includegraphics} ohne optionale
Argumente wird eine automatische Einpassung in den Darstellungsbereich
erwirkt.
Die Art der Einpassung wird mit der Option \Option{imagefit} kontrolliert.
Gültige Werte sind der \tablename~\ref{tbl:gausslayout:imagefitoptions}
zu entnehmen.


\subsubsection{Darstellungsbereich anpassen}

%TODO: Rahmenbreite?
%TODO: Spaltenraster
%TODO: Stegbreite?
Normalerweise werden Gauss-Boxen so gesetzt, dass der Textbereich
einen Abstand zum Kommunikationsbereich von Rahmenbreite hat.
Vertikal haben die Boxen zum Rand des Gauss-Segments ebenfalls einen
Abstand von Rahmenbreite.
Horizontal zwischen den einzelnen Boxen beträgt der Abstand Stegbreite.

Diese Einstellungen sind für die meisten Anwendungsfälle passend.
Für eine freiere Positionierung, vor allem für Bildelemente sinnvoll, gibt
es eine Reihe von Optionen, um die Innenabstände in Boxen flexibel
kontrollieren zu können.

\begin{Declaration}
  \KOption{padding}\PName{Option}\\
  \KOption{innerpadding}\PName{Option}\\
  \KOption{outerpadding}\PName{Option}
\end{Declaration}

Die meistgebrauchten Anpassungen können über die Option \Option{padding}
vorgenommen werden.
Detailliertere Feinjustierung ist mit den Optionen \Option{innerpadding}
und \Option{outerpadding} möglich,
wobei \Option{innerpadding} die Abstände \emph{zwischen} Gauss-Boxen
und \Option{outerpadding} die Abstände zum \emph{Seitenrand} kontrolliert.

Die möglichen Werte für die Option \Option{padding} sind in 
\tablename~\ref{tbl:gausslayout:padding} aufgeführt.

\begin{table}[!ht]
\begin{desctable}
\entry{\PValue{default}}{%
  Dies entspricht der erläuterten Standard-Darstellung.
}
\entry{\PValue{minimal}}{%
  Abstände von Gauss-Boxen zum Rand des Kommunikationsbereich sind 0
  und Abstände zwischen Gauss-Boxen sind sowohl horizontal als aus vertikal
  Stegbreite.
}
\entry{\PValue{none}}{%
  Setzt sowohl Abstände zum Kommunikationsbereich
  als auch zwischen den Gauss-Boxen auf 0.
}
\end{desctable}% padding
\caption[{Mögliche Werte für Option \Option{padding}}]%
  {Mögliche Werte für Option \Option{padding} zur
    Kontrolle des Abstandes von Gauss-Boxen}
  \label{tbl:gausslayout:padding}
\end{table}

Die möglichen Werte für \Option{innerpadding} und \Option{outerpadding}
stehen in \tablename~\ref{tbl:gausslayout:innerpadding} bzw.
\tablename~\ref{tbl:gausslayout:outerpadding}.

\begin{table}[!ht]
\begin{desctable}
\entry{\PValue{default}}{%
  Standardeinstellung. Horizontaler Abstand zwischen Gauss-Boxen ist Stegbreite,
  vertikaler Abstand zum Segmentrand ist Rahmenbreite.
}
\entry{\PValue{columnsep}}{%
  Sowohl horizontaler als auch vertikaler Abstand zwischen Gauss-Boxen
  ist Stegbreite.
}
\entry{\PValue{none}}{%
  Gauss-Boxen haben keinerlei Abstand zueinander.
}
\entry{\PValue{vnone}}{%
  Gauss-Boxen haben keinen vertikalen Abstand zueinander.
}
\entry{\PValue{hnone}}{%
  Gauss-Boxen haben keinen horizontalen Abstand zueinander.
}
\end{desctable}% innerpadding
\caption[{Mögliche Werte für Option \Option{innerpadding}}]%
  {Mögliche Werte für Option \Option{innerpadding} zur
    Kontrolle des Abstandes zwischen Gauss-Boxen}
  \label{tbl:gausslayout:innerpadding}
\end{table}

\begin{table}[!ht]
\begin{desctable}
\entry{\PValue{default}}{%
  Standardeinstellung. Der Abstand zum Kommunikationsbereich ist Rahmenbreite.
}
\entry{\PValue{none}}{%
  Der Abstand zum Kommunikationsbereich ist immer 0.
}
\entry{\PValue{vnone}}{%
  Der horizontale Abstand zum Kommunikationsbereich ist 0.
}
\entry{\PValue{hnone}}{%
  Der vertikale Abstand zum Kommunikationsbereich ist 0.
}
\end{desctable}% outerpadding
\caption[{Mögliche Werte für Option \Option{outerpadding}}]%
  {Mögliche Werte für Option \Option{outerpadding} zur
    Kontrolle des Abstandes von Gauss-Boxen zum Kommunikationsbereich}
  \label{tbl:gausslayout:outerpadding}
\end{table}

\begin{Declaration}
  \Option{logosep}
\end{Declaration}

Auf Seiten auf denen ein Sigelbandlogo dargestellt ist, muss der Abstand
des Textes zum oberen Rand des Kommunikationsbereich mindestens
dreifache Rahmenbreite betragen. Dies Option \Option{logosep} bewirkt,
dass der obere Abstand der Gauss-Box zum Segmentrand 3 Rahmenbreiten beträgt.

