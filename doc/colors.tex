\chapter{Farben}\label{chap:tubscolors}
\Index{Farben|indexbf}

\newcommand{\classoptionitem}[1][ ]{
  \item[\mdseries{\ttfamily%
    \textbackslash usepackage%
    {[{\color{tuRed}#1}]}%
    \{tubslogo\}}]\hfill\\
}

Die Farbdefinitionen in \tubslatex werden vom Paket \newpackage{tubscolors}
zur Verfügung gestellt. Die folgende Beschreibung bezieht sich
auf den Funktionsumfang dieses Paketes.
Da das Paket aber in allen verfügbaren \tubslatex-Klassen fest eingebunden ist,
gelten die Erklärungen auch allgemein.

\newcommand{\rainbow}[2][\relax]{{\noindent\sffamily\footnotesize%
\ifx#1\relax\colorlet{fglbg}{black}\else\colorlet{fglbg}{#1}\fi
\colorbox{#2100}{\hbox to 0.188\textwidth{%
  \color{fglbg}\vphantom{Fg}#2{}100\hfill}}% 
\colorbox{#280}{\hbox to 0.188\textwidth{%
  \color{fglbg}\vphantom{Fg}#2{}80\hfill}}% 
\colorbox{#260}{\hbox to 0.188\textwidth{\vphantom{Fg}#2{}60\hfill}}% 
\colorbox{#240}{\hbox to 0.188\textwidth{\vphantom{Fg}#2{}40\hfill}}% 
\colorbox{#220}{\hbox to 0.188\textwidth{\vphantom{Fg}#2{}20\hfill}}\\% 
}}

\section{Verfügbare Farben}

Der Farbklang der TU-Braunschweig ist in eine Primär- und einen
Sekundärfarbbereich aufgeteilt.

Die Primärfarben bilden dabei Rot, Schwarz und Weiß, sowie in
20-Prozent-Schritten abgestufte Grautöne. Die Primärfarben dienen
vor allem zur Auszeichnung von Hintergrund, Textfarbe und dem TU-Logo.
Zur individuellen Gestaltung von Dokumenten ist der Sekundärfarbbereich
vorgesehen.

Die Sekundärfarben setzen sich aus 12 weiteren aufeinander abgestimmten
Farben zusammen, die 
in 4 Farbklänge (Gelb-Orange, Grün, Blau und Violett) mit je 3 Basisfarben
aufgeteilt sind.
Alle Sekundärfarben können in 20-Prozent-Schritten aufgehellt werden.

\subsection{Benennungsschema}
\Index{Farben!Benennungsschema}

Alle Farben tragen das Präfix \texttt{tubs} vor Ihrem Namen.
Der eigentliche Farbname setzt sich zusammen aus dem Namen
des Sekundärfarbklangs
(\texttt{Orange}, \texttt{Blue}, \texttt{Green}, \texttt{Violet}) 
und einer Farbvariante
(\texttt{Light}, \texttt{Medium}, \texttt{Dark})
gefolgt von der Prozentzahl ihrer Helligkeitsstufe
(\texttt{20}, \texttt{40}, \texttt{60}, \texttt{80}, \texttt{100}).

Die Farben können sowoh in Präfix-Notation
(\Color{tubs\textsf{<Variante>}\textsf{<Klang>}80})
als auch in Suffix-Notation (\Color{tubs\textmd{<Klang>}\textmd{<Variante>}80})
angegeben werden.

\begin{example}
  Die helle Variante des Blau in 60\%iger Helligkeit kann also
  über den Namen \Color{tubsBlueLight60} als auch
  \Color{tubsLightBlue60} gewählt werden.
\end{example}

Vereinfachte Varianten der Benennung gibt es unter anderem für die 100\%-Varianten.
Hier kann die Prozentzahl weggelassen werden.
\begin{important}
Die Primärfarbe \Color{tubsRed} ist \emph{nicht} identisch mit \Color{tubsRed100}
aus dem gelb-orange-Sekundärfarbbereich.
In diesem speziellen Fall handelt es sich um verschiedene Farben.
\end{important}
Auch für die \texttt{Medium}-Varianten ist kein Präfix/Suffix notwendig.
\begin{example}
  Die Medium-Variante des Grün in 100\%iger Helligkeit kann vereinfacht
  mit \Color{tubsGreen} bezeichnet werden.
\end{example}

Die drei Primärfarben rot, schwarz und weiß sind unter den Namen
\Color{tubsRed}, \Color{tubsBlack} und \Color{tubsWhite} verfügbar.

Im Folgenden werden die zu Verfügung stehenden Farben aufgelistet.
Die Standardnamen über die die einzelnen Farben angesprochen werden können,
sind in den Beispielfeldern angegeben.

\subsection{Primärfarben}
\Index{Primärfarben}
\Index{Farben!Primär-}

{\sffamily\footnotesize%
\colorbox{tubsRed}{\hbox to 0.188\textwidth{%
  \vphantom{Fg}tubsRed\hfill}}%
\colorbox{tubsBlack}{\hbox to 0.188\textwidth{%
  \color{white}\vphantom{Fg}tubsBlack\hfill}}%
\fcolorbox{tubsBlack}{tuWhite}{\hbox to 0.188\textwidth{%
  \vphantom{Fg}tubsWhite\hfill}}\\%
}

\rainbow[tuWhite]{tubsGray}

Zur Vereinfachung sind noch die Farben \Color{tubsGray} und
\Color{tubsLightGrey} definiert, die den Farben \Color{tubsGray60} und
\Color{tubsGray20} entsprechen.

Alle Graytöne sind darüber hinaus auch in britischer Schreibweise nutzbar
(\Color{tubsGrey}).


\pagebreak
\subsection{Sekundärfarben}
\Index{Sekundärfarben}
\Index{Farben!Sekundär-}

  tubsOrange\ldots\\
  \colorshow{Orange}{Light}
  \colorshow{Orange}{Medium}
  \colorshow{Orange}{Dark}\\[-1ex]
  tubsBlue\ldots\\
  \colorshow{Blue}{Light}
  \colorshow{Blue}{Medium}
  \colorshow{Blue}{Dark}\\[-1ex]
  tubsGreen\ldots\\
  \colorshow{Green}{Light}
  \colorshow{Green}{Medium}
  \colorshow{Green}{Dark}\\[-1ex]
  tubsViolet\ldots\\
  \colorshow{Violet}{Light}
  \colorshow{Violet}{Medium}
  \colorshow{Violet}{Dark}
%   \caption{Im CD definierte Farben und deren Benennung (Auszug)}


% \begin{example}
% In folgendem Beispiel wurde \lstinline{blau} als Farbklang ausgewählt:
% 
% tubsSecondary\ldots\\
% \colorshow{Secondary}{Light}
% \colorshow{Secondary}{Medium}
% \colorshow{Secondary}{Dark}\\[-1ex]
% \end{example}

\subsection{Farbmodelle}
\Index{Farbmodelle}

Die Farben sind für 3 Farbmodelle (\acrshort{RGB}, RGB beamer-optimiert, \acrshort{CMYK}) definiert.
Das gewünschte Farmodell kann durch Klassen- bzw. Paketoptionen gewählt werden.
Die einzelnen Klassen haben bereits das jeweils passende Farbmodell voreingestellt.

Zusätzlich kann eine Monochrom-Variante gewählt werden, die die Farbe
\Color{tusbRed} automatisch als Schwarz darstellt.


\clearpage
\section{Verwendung/Farbwahl}
\sloppy

% TODO: hier ist noch etwas Bedarf...

Die TU-Farben können nach dem oben erläuterten Schema abgerufen und verwendet
werden wie jede andere Farbe auch.

\paragraph{Haupt-Sekundärfarbklang}\label{sec:secondary}

Die Farben des aktuell gewählten Haupt-Sekundärfarbklangs können zusätzlich
über die Namen \Color{tuSecondaryLight},
\Color{tuSecondaryMedium}, \Color{tuSecondaryDark}, sowie die
entsprechenden Prozentualwert \Color{tuSecondaryLight20},
\Color{tuSecondaryLight40}, \ldots) angesprochen werden können.
Dies erlaubt eine flexible Verwendung der 4 Sekundärfarbklänge.

Die Wahl der Haupt-Sekundärfarbklangs erfolgt entweder über die weiter
unten beschriebenen Klassen-/Paketoptionen oder innerhalb des Dokumentes auch
mit dem Befehl \Macro{selectsecondary}.

\paragraph{Fettes Schwarz}\label{sec:richblack}
\Index{rich black|indexbf}
\Index{Schwarz!fettes|indexbf}

Bei der Verwendung des CMYK-Farbraums ist zu beachten,
dass das die Farbe Schwarz (\Color{tubsBlack})
der Mischung 0\%C, 0\%M, 0\%Y, 100\%K entspricht,
was in der Bildschirmdarstellung meist nur als sehr dunkles grau wahrgenommen wird.
Dies gilt somit auch für den Text im Dokument und ist normal.
Für Bildschirmdarstellung wird immer das RGB-Farbprofil empfohlen.

\begin{figure}[!ht]
\centering
{\selectcolormodel{cmyk}
\colorbox[cmyk]{0,0,0,1}{\parbox[t][2cm]{3cm}{\vfill\centering\sffamily\color{tubsWhite} tubsBlack\vfill}}%
\qquad
\colorbox[cmyk]{0.75,0.68,0.67,0.9}{\parbox[t][2cm]{3cm}{\vfill\centering\sffamily\color{tubsWhite} tubsRichBlack\vfill}}%
}
\caption{Vergleich der Farben \Color{tubsBlack} und \Color{tubsRichBlack} im CMYK-Farbraum}
\end{figure}


Im CMYK-Druck dagegen wird es mit der Key-Patrone gedruckt und daher korrekt
als Schwarz ausgegeben.
Sollte trotz dessen die Ausgabe aus dem Drucker zu hell erscheinen,
so kann dies entweder durch Verwendung der Klassen-/Paketoption \Option{richblack}
(siehe folgendes Kapitel) oder durch einen manuelle Anpassung der Farbe
\Color{tubsBlack} korrigiert werden, indem ein sogenanntes
\emph{Fettes Schwarz} (rich black) verwendet wird.
Dabei werden zusätzlich die CMY-Farben benutzt, um ein möglichst dunkles
Schwarz zu mischen. Die Standarddefinition der Option \Option{richblack}
entspricht dies  einer Mischung von 75\%C, 68\%M, 67\%Y, 90\%K.
Alternativ kann auch die definierte Farbe \Color{tubsRichBlack} verwendet werden.

Zu beachten ist jedoch, dass diese Farbe im Druck ggf. zu unsaubereren
Ergebnissen durch den übermäßigen Tinteneinsatz führen kann!
  
\begin{example}
  Manuelle Umdefinierung von \Color{tubsBlack}:\\
  \lstinline!\definecolor{tubsBlack}{CMYK}{0.5,0.5,0.5,1.0}!
\end{example}


\subsection{Paket-/Klassenoptionen und Befehle}
\Index{RGB}
\Index{RGB!Beamer}
\Index{CMYK}
\Index{monochrom}

\begin{Declaration}
  \Option{rgb}\\
  \Option{rgbbeamer}\\
  \Option{cmyk}\\[1ex]
  \Option{mono}
\end{Declaration}

Die Option \Option{rgb} bewirkt eine Darstellung der Farben im vordefinierten
RGB-Farbprofil.
Die Option \Option{rgbbeamer} lädt ein Beamer-optimierts RGB-Farbprofil.
Mit der Option \Option{cmyk} wird das CMYK-Farbprofil geladen.
Die Zusatz-Option \Option{mono} bewirkt eine Darstellung der Farbe
\Color{tubsRed} als \Color{tubsBlack} im aktuell gewählten Farbprofil.
Dies kann insbesondere für schwarz-weiß Kopiervorlagen in Verbindung
mit der schwarzen Siegelbandlogo-Variante sinnvoll sein.


\begin{Declaration}
  \Option{richblack}
\end{Declaration}

Definiert die Farbe \Color{tubsBlack} bei Verwendung des CMYK-Farbprofils
in ein sog. "`fettes Schwarz"' um. Siehe dazu auch Abschnitt~\ref{sec:richblack}.

\begin{Declaration}
  \Option{orange}\\
  \Option{green}\\
  \Option{blue}\\
  \Option{violet}
\end{Declaration}

Mit diesen Optionen kann der Standard-Sekundärfarbklang des Dokumentes gewählt
werden. Der jeweils geladene Sekundärfarbklang ergibt sich aus dem Namen
der Option.
Die Farben des Sekundärfarbklangs können wie in Abschnitt~\ref{sec:secondary}
beschrieben benutzt werden.

\begin{Declaration}
  \Macro{selectsecondary}\Parameter{Farbklang}
\end{Declaration}

Befehl zur Wahl des Haupt-Sekundärfarbklangs. Als Option für \PName{Farbklang}
sind die Werte \PValue{orange}, \PValue{green}, \PValue{blue} und \PValue{violet}
erlaubt.

\begin{Declaration}
  \Macro{tubscolorshow}\Parameter{Farbe}\Parameter{Variante}
\end{Declaration}

Zeigt farbige Boxen mit allen Helligkeitsstufen der angegebenen Farbklang-Variante.
\begin{example}
 Die Ausgabe von \lstinline!\tubscolorshow{Blue}{Light}! sieht beispielsweise
 wie folgt aus:\\
 \hspace*{-1cm}\tubscolorshow{Blue}{Light}
\end{example}

\fussy
