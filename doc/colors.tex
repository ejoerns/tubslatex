\chapter{Farben}\label{chap:tubscolors}

\newcommand{\classoptionitem}[1][ ]{
  \item[\mdseries{\ttfamily%
    \textbackslash usepackage%
    {[{\color{tuRed}#1}]}%
    \{tubslogo\}}]\hfill\\
}

Die Farbdefinitionen in \tubslatex werden vom Paket \newpackage{tubscolors}
zur Verfügung gestellt. Die folgende Beschreibung bezieht sich
auf den Funktionsumfang des Paketes. Das Paket ist in allen verfügbaren
Klassen bereits korrekt eingebunden.

\newcommand{\rainbow}[2][\relax]{{\noindent\sffamily\footnotesize%
\ifx#1\relax\colorlet{fglbg}{black}\else\colorlet{fglbg}{#1}\fi
\colorbox{#2100}{\hbox to 0.188\textwidth{%
  \color{fglbg}\vphantom{Fg}#2{}100\hfill}}% 
\colorbox{#280}{\hbox to 0.188\textwidth{%
  \color{fglbg}\vphantom{Fg}#2{}80\hfill}}% 
\colorbox{#260}{\hbox to 0.188\textwidth{\vphantom{Fg}#2{}60\hfill}}% 
\colorbox{#240}{\hbox to 0.188\textwidth{\vphantom{Fg}#2{}40\hfill}}% 
\colorbox{#220}{\hbox to 0.188\textwidth{\vphantom{Fg}#2{}20\hfill}}\\% 
}}

\section{Verfügbare Farben}

Der Farbklang der TU-Braunschweig ist in eine Primär- und einen
Sekundärfarbbereich aufgeteilt.

Die Primärfarben bilden dabei Rot, Schwarz und Weiß, sowie in
20-Prozent-Schritten abgestufte Grautöne. Die Primärfarben dienen
vor allem zur Auszeichnung von Hintergrund, Textfarbe und dem TU-Logo.
Zur individuellen Gestaltung von Dokumenten ist der Sekundärfarbbereich
vorgesehen.

Die Sekundärfarben setzen sich aus 12 weiteren aufeinander abgestimmten
Farben zusammen, die 
in 4 Farbklänge (Gelb-Orange, Grün, Blau und Violett) mit je 3 Basisfarben
aufgeteilt sind.
Alle Sekundärfarben können in 20-Prozent-Schritten aufgehellt werden.

Die Namen über die die einzelnen Farben angesprochen werden können, sind in den
Beispielfeldern angegeben.

\subsection{Primärfarben}

{\sffamily\footnotesize%
\colorbox{tuRed}{\hbox to 0.188\textwidth{%
  \vphantom{Fg}tuRed\hfill}}%
\colorbox{tuBlack}{\hbox to 0.188\textwidth{%
  \color{white}\vphantom{Fg}tuBlack\hfill}}%
\fcolorbox{tuBlack}{tuWhite}{\hbox to 0.188\textwidth{%
  \vphantom{Fg}tuWhite\hfill}}\\%
}

\rainbow[tuWhite]{tuGray}

Zur Vereinfachung sind noch die Farben \lstinline{tuGray} und
\lstinline{tuLightGrey} definiert, die den Farben \lstinline{tuGray60} und
\lstinline{tuGray20} entsprechen.

Alle Graytöne sind darüber hinaus auch in britischer Schreibweise nutzbar
(\lstinline{tuGrey}).

\paragraph{Hinweis:}
\lstinline{tuRed} ist nicht zu verwechseln mit \lstinline{tuRed100} aus dem
Sekundärfarbbereich. Es handelt sich dabei um eine komplett andere Farbe.

\pagebreak
\subsection{Sekundärfarben}
  tuOrange\ldots\\
  \colorshow{Orange}{Light}
  \colorshow{Orange}{Medium}
  \colorshow{Orange}{Dark}\\[-1ex]
  tuBlue\ldots\\
  \colorshow{Blue}{Light}
  \colorshow{Blue}{Medium}
  \colorshow{Blue}{Dark}\\[-1ex]
  tuGreen\ldots\\
  \colorshow{Green}{Light}
  \colorshow{Green}{Medium}
  \colorshow{Green}{Dark}\\[-1ex]
  tuViolet\ldots\\
  \colorshow{Violet}{Light}
  \colorshow{Violet}{Medium}
  \colorshow{Violet}{Dark}
%   \caption{Im CD definierte Farben und deren Benennung (Auszug)}


Zusätzlich kann als Paketoption ein Farbklang ausgewählt werden, dessen Farben
dann über die Werte  \\\lstinline{tuSecondaryLight},
\lstinline{tuSecondaryMedium}, \lstinline{tuSecondaryDark}, sowie die
entsprechenden Prozentualwert \\(\lstinline{tuSecondaryLight20},
\lstinline{tuSecondaryLight40}, \ldots) angesprochen werden können.
Dies erlaubt eine flexible Verwendung der 4 Sekundärfarbklänge.

In folgendem Beispiel wurde \lstinline{blau} als Farbklang ausgewählt:

\colorshow{Secondary}{Light}
\colorshow{Secondary}{Medium}
\colorshow{Secondary}{Dark}\\[-1ex]

\paragraph{Hinweise:}
Die Farben des gelb-orange-Farbklangs können entsprechend der anderen
Farbmodelle auch noch einheitlich über die Alternativnamen
\lstinline{tuOrangeLight}, \lstinline{tuOrange},
\lstinline{tuOrangeDark}, sowie de entsprechenden Prozentwerte aufgerufen
werden.

Außerdem können jeweils die mittleren Farbwerte der Farbklänge auch über den
Zusatz \lstinline{Medium} angesprochen werden (statt \lstinline{tuGreen100} auch
\lstinline{tuGreenMedium100}).

Bei allen 100-Prozent-Farben (außer \lstinline{tuRed}) kann die
Zahl weggelassen werden (statt \lstinline{tuGreenLight100} auch 
\lstinline{tuGreenLight}).

\subsection{Farbmodelle}

In den Paketoptionen kann zwischen 3 Farbmodellen gewählt werden.
Das Standardmodell stellt die CD-konformen RGB-Farbwerte zur Verfügung.

Für die Ausgabe in CMYK-Farben steht die Option \lstinline!cmyk! zur Verfügung.

Darüber hinaus gibt es noch ein RGB-Farbschema, das für die Ausgabe auf
Beamern optimiert ist. Es kann über die Option \lstinline!rgbbeamer! geladen
werden.
