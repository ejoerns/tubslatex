\documentclass{scrartcl}

% \usepackage[green,rgbprint]{TU-CDcolors}
\usepackage[T1]{fontenc}
\usepackage{Nexus}
\usepackage[utf8]{inputenc}
\usepackage[ngerman]{babel}
\usepackage{color}
\usepackage{graphicx}
\usepackage[a4paper]{TU-logo}
\usepackage{TU-CDcolors}

\usepackage{listings}
\lstset{basicstyle=\ttfamily}

\newcommand{\classoptionitem}[1][ ]{
  \item[\mdseries{\ttfamily%
    \textbackslash usepackage%
    {[{\color{tuRed}#1}]}%
    \{TU-CDcolors\}}]\hfill\\
}

\parindent0mm
\parskip\medskipamount

\title{TU-logo}
\subtitle{Paket-Dokumentation}
\author{Dummy Author \and Enrico Jörns}

\begin{document}

\maketitle

\section{Beschreibung}

Dieses Paket ermöglicht das einfache Einbinden des TU-logos in \LaTeX-Dokumente.
Dabei kann zwischen einer Variante im rgb-Farbmodell, einer im
cmyk-Farbmodell oder (für Ausnahmefälle) einer monochromen Variante gewählt
werden. Die Auwahl erfolgt dabei über die Paketoptionen
(siehe Abschnitt \ref{options:color}).

\begin{minipage}{0.5\textwidth}
  \centering
  \includegraphics[width=\tulogoWidth]{TUBraunschweig-rgb}
  {\sffamily Logo in Standardgröße (rgb)}
\end{minipage}
\begin{minipage}{0.5\textwidth}
  \centering
  \includegraphics[width=\tulogoWidth]{TUBraunschweig-cmyk}
  {\sffamily Logo in Standardgröße (cmyk)}
\end{minipage}

\begin{minipage}{0.5\textwidth}
  \centering
  \includegraphics[width=\tulogoWidth]{TUBraunschweig-mono}
  {\sffamily Logo in Standardgröße (schwarz)}
\end{minipage}

Es werden 2 Befehle zum Einfügen des Logos in Dokumente bereit gestellt.
Mit dem Befehl {\color{tuRed}\lstinline{\tulogo}} wird ein Logo in
standardkonformer Größe der verwendeten Papiergröße entsprechend skaliert
eingefügt. Die verwendete Papiergröße kann dabei als Paketoption übergeben
werden und ist standardmäßig auf \lstinline{a4paper} voreingestellt
(siehe Abschnitt \ref{options:papersize}.
Also optionales Argument kann auch ein frei gewählter Skalierungsfaktor
vorgegeben werden (siehe Abschnitt \ref{cmd:tulogo}).

Mit {\color{tuRed}\lstinline!\tulogoAbs{}!} kann dagegen zusätzlich eine
individuell gewählte absolute Breite angegeben werden.

% die sowohl das einfache Einfügen in
% den Standardbreiten als auch in individuell relativ oder absolut
% angegebener Breite ermöglichen.

\clearpage
\section{Paket-Optionen}

\subsection{Farbmodell}\label{options:color}

\begin{description}
  \classoptionitem[rgb]
    Verwendung der rgb-Druckfarben. Dies ist die Standardeinstellung.
  \classoptionitem[cmyk]
    Verwendung der cmyk-Farben
  \classoptionitem[rgbprint]
    Verwendung der rgb-Druckfarben. Dies ist die Standardeinstellung.
    Entspricht Option \lstinline{rgb}.
  \classoptionitem[mono]
    Verwendung einer schwarz-weiß-Version des Logos.
\end{description}

\subsection{Papiergröße}\label{options:papersize}

\begin{description}
  \classoptionitem[a6paper]
    Skaliert das Logo passend für A6-Dokumente auf 60\% der Standardgröße.
  \classoptionitem[a5paper]
    Skaliert das Logo passend für A5-Dokumente auf 70\% der Standardgröße.
  \classoptionitem[a4paper]
    Skaliert das Logo passend für A4-Dokumente auf Standardgröße.
  \classoptionitem[a3paper]
    Skaliert das Logo passend für A3-Dokumente auf 140\% der Standardgröße.
  \classoptionitem[a2paper]
    Skaliert das Logo passend für A2-Dokumente auf 200\% der Standardgröße.
  \classoptionitem[a1paper]
    Skaliert das Logo passend für A1-Dokumente auf 280\% der Standardgröße.
  \classoptionitem[a0paper]
    Skaliert das Logo passend für A0-Dokumente auf 400\% der Standardgröße.
\end{description}


\clearpage
\section{Befehle}

\begin{description}\label{cmd:tulogo}
  \item[\color{tuRed}\mdseries\ttfamily \textbackslash tulogo%
    {[\textcolor{tuGreenDark}{\sffamily\itshape $<$Breite rel.$>$}]}]
    Erzeugt ein Logo, dass entsprechend des durch Angabe der Papiergröße
    gewählten Skalierungsfaktors skaliert wird.

    Wird das optionale Argument {\sffamily\itshape $<$Breite rel.$>$}
    verwendet, so kann ein freier Skalierungsfaktor ausgehend von der
    Standardgröße für DINA4-Dokumente ($70$mm$\times 26$mm) übergeben werden.

    \lstinline!\tulogo[0.6]! erzeugt beispielsweise ein auf 60\% der 
    Standardgröße für DINA4-Dokumente skaliertes Logo.
    
  \item[\color{tuRed}\ttfamily \textbackslash tulogoAbs%
    \{\textcolor{tuGreenDark}{Breite abs.}\}]
    Dieser Befehl erlaubt die Erzeugung eines Logos mit absolut vorgegebener
    Breite.\footnote{Generell ist die Verwendung von standardkonformen Größen
    einer individuellen Skalierung vorzuziehen!}
    
    \lstinline!\tulogoAbs{5cm}! erzeugt beispielsweise ein 5cm breites Logo.
\end{description}


\section{Längen}
  \begin{description}
    \item[\mdseries\ttfamily \textbackslash tulogoWidth]
      Standardbreite des Logos (70mm).
    \item[\mdseries\ttfamily \textbackslash tulogoHeight]
      Standardhöhe des Logos (26mm).
  \end{description}


\end{document}
