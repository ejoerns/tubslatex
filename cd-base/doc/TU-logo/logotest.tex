\documentclass{scrartcl}

% \usepackage[green,rgbprint]{TU-CDcolors}
\usepackage[T1]{fontenc}
\usepackage{Nexus}
\usepackage[utf8]{inputenc}
\usepackage[ngerman]{babel}
\usepackage{color}
\usepackage{graphicx}
\usepackage{TU-logo}

\usepackage{listings}
\lstset{basicstyle=\ttfamily}

\newcommand{\classoptionitem}[1][ ]{
  \item[\mdseries{\ttfamily%
    \textbackslash usepackage%
    {[{\color{red}#1}]}%
    \{TU-CDcolors\}}]\hfill\\
}

\parindent0mm
\parskip\medskipamount

\title{TU-logo}
\subtitle{Paket-Dokumentation}
\author{Martin Bäker \and \normalsize Enrico Jörns}

\begin{document}

\maketitle

\section{Beschreibung}

Dieses Paket ermöglicht das einfache Einbinden des TU-logos in \LaTeX-Dokumente.
Dabei kann zwischen einer Variante im rgb-Farbmodell und einer im
cmyk-Farbmodell gewählt werden.

Es werden 2 Befehle bereit gestellt, die sowohl das einfache Einfügen in
Standard-Breite%??
als auch in relativ oder absolut angegebener Breite ermöglichen.


\begin{minipage}{0.5\textwidth}
  \centering
  \includegraphics[width=100\tulogoWidth]{TUBraunschweig-rgb}
  {\sffamily Logo in Standardgröße (rgb)}
\end{minipage}
\begin{minipage}{0.5\textwidth}
  \centering
  \includegraphics[width=100\tulogoWidth]{TUBraunschweig-cmyk}
  {\sffamily Logo in Standardgröße (cmyk)}
\end{minipage}

\begin{minipage}{0.5\textwidth}
  \centering
  \includegraphics[width=100\tulogoWidth]{TUBraunschweig-mono}
  {\sffamily Logo in Standardgröße (schwarz)}
\end{minipage}

\section{Optionen}

\begin{description}
  \classoptionitem[rgb]
    Verwendung der rgb-Druckfarben. Dies ist die Standardeinstellung.
  \classoptionitem[cmyk]
    Verwendung der cmyk-Farben
  \classoptionitem[rgbprint]
    Verwendung der rgb-Druckfarben. Dies ist die Standardeinstellung.
    Entspricht Option \lstinline{rgb}.
  \classoptionitem[mono]
    Verwendung einer schwarz-weiß-Version des Logos.
\end{description}



\section{Befehle}

\begin{description}
  \item[\mdseries\ttfamily \textbackslash tulogo%
    {[\textcolor{red}{\sffamily\itshape $<$Breite rel.$>$}]}]
  \item[\ttfamily \textbackslash tulogoAbs\{\textcolor{red}{Breite abs.}\}]
\end{description}

\section{Interna}

Ein Prozent der Standardbreite (70mm) ist definiert in der Länge
\lstinline{\tulogoWidth}.

\end{document}
