\documentclass{scrartcl}

\usepackage[green]{TU-CDcolors}
\usepackage[T1]{fontenc}
\usepackage{Nexus}
\usepackage[utf8]{inputenc}
\usepackage[ngerman]{babel}

\usepackage{listings}
\lstset{basicstyle=\ttfamily}

\def\rainbow#1{{\sffamily\footnotesize%
\colorbox{#1100}{\hbox to 0.18\textwidth{#1{}100}}% 
\colorbox{#180}{\hbox to 0.18\textwidth{#1{}80}}% 
\colorbox{#160}{\hbox to 0.18\textwidth{#1{}60}}% 
\colorbox{#140}{\hbox to 0.18\textwidth{#1{}40}}% 
\colorbox{#120}{\hbox to 0.18\textwidth{#1{}20}}\\% 
}}

\newcommand{\classoptionitem}[1][ ]{
  \item[\mdseries{\ttfamily%
    \textbackslash usepackage%
    {[{\color{red}#1}]}%
    \{TU-CDcolors\}}]\hfill\\
}

\parindent0mm
\parskip\medskipamount

\title{TU-CDcolors}
\subtitle{Paket-Dokumentation}
\author{Martin Bäker \and \normalsize Enrico Jörns}

\begin{document}

\maketitle

\section{Verfügbare Farben}

Der Farbklang der TU-Braunschweig ist in eine Primär- und einen
Sekundärfarbbereich aufgeteilt.

Die Primärfarben bilden dabei Rot, Schwarz und Weiß, sowie in
20-Prozent-Schritten abgestufte Grautöne. Die Primärfarben dienen
vor allem Hintergrund, Textfarbe und dem TU-Logo.
Zu in individuellen Dokumente-Gestaltung ist der Sekundärfarbbereich vorgesehen.

Die Sekundärfarben setzen sich aus 12 weiteren aufeinander abgestimmten
Farben zusammen, die 
in 4 Farbklänge (Gelb-Orange, Grün, Blau und Violett) mit je 3 Grundfarben,
aufgeteilt sind.
Alle Sekundärfarben können in 20-Prozent-Schritten aufgehellt werden.

Die Namen über die die einzelnen Farben angesprochen werden können, sind in den
Beispielfeldern angegeben.

\subsection{Primärfarben}

{\sffamily\footnotesize%
\colorbox{tuRed}{\hbox to 0.18\textwidth{tuRed}}%
\colorbox{tuBlack}{\hbox to 0.18\textwidth{\color{white}tuBlack}}%
\colorbox{tuWhite}{\hbox to 0.18\textwidth{tuWhite}}\\%
}

\rainbow{tuGrey}

\paragraph{Hinweis:}
\lstinline{tuRed} ist nicht zu verwechseln mit \lstinline{tuRed100} aus dem
Sekundärfarbbereich. Es handelt sich dabei um eine komplett andere Farbe.

\subsection{Sekundärfarben}

\rainbow{tuYellow}
\rainbow{tuOrange}
\rainbow{tuRed}

\rainbow{tuGreenLight}
\rainbow{tuGreen}
\rainbow{tuGreenDark}

\rainbow{tuBlueLight}
\rainbow{tuBlue}
\rainbow{tuBlueDark}

\rainbow{tuVioletLight}
\rainbow{tuViolet}
\rainbow{tuVioletDark}

Zusätzlich kann als Paketoption ein Farbklang ausgewählt werden, dessen Farben
dann über die Werte \lstinline{tuSecondaryLight},
\lstinline{tuSecondaryMedium}, \lstinline{tuSecondaryDark}, sowie die
entsprechenden Prozentualwert (\lstinline{tuSecondaryLight20},
\lstinline{tuSecondaryLight40}, \ldots) angsprochen werden können.
Dies erlaubt eine flexible Verwendung der 4 Sekundärfarbklänge.

In folgendem Beispiel wurde \lstinline{green} als Farbklang ausgewählt:

\rainbow{tuSecondaryLight}
\rainbow{tuSecondary}
\rainbow{tuSecondaryDark}

\paragraph{Hinweise:}
Die Farben des gelb-orange-Farbklangs können entsprechend der anderen
Farbmodelle auch noch einheitlich über die Alternativnamen
\lstinline{tuOrangeLight}, \lstinline{tuOrange},\\
\lstinline{tuOrangeDark}, sowie de entsprechenden Prozentwerte aufgerufen
werden.

Außerdem können alle mittleren Farbwerte auch über den Zusatz
\lstinline{Medium} angesprochen werden (statt \lstinline{tuGreen100} auch
\lstinline{tuGreenMedium100}).

Bei allen 100er-Farben (außer \lstinline{tuRed}) kann darüber hinaus die
Zahl weggelassen werden (statt \lstinline{tuGreenLight100} auch 
\lstinline{tuGreenLight}).

\section{Optionen}

\begin{description}
  \classoptionitem[cmyk]
    Verwendung der cmyk-Farben
  \classoptionitem[rgbprint]
    Verwendung der rgb-Druckfarben. Dies ist die Standardeinstellung.
\end{description}


\begin{description}
  \classoptionitem[\textit{\sffamily$<$farbe$>$}]
    Legt den verwendeten Sekundärfarbklang fest.

    Die Farben können dann über die Werte \lstinline{tuSecondaryLight},
    \lstinline{tuSecondaryMedium}, \lstinline{tuSecondaryDark} angsprochen
    werden.

    Mögliche Werte für \textit{farbe}:
    \begin{itemize}
      \item orange
      \item green
      \item blue
      \item violet
    \end{itemize}
\end{description}

\section{Befehle}

\begin{description}
  \item[\ttfamily\mdseries\textbackslash selectsecondary]
    Wählt eine Farbe als Sekundärfarbe.
\end{description}



\end{document}
