% \iffalse meta-comment
%
% Copyright (C) 2011 by Enrico Jörns
% -----------------------------------
%
% This file may be distributed and/or modified under the
% conditions of the LaTeX Project Public License, either version 1.2
% of this license or (at your option) any later version.
% The latest version of this license is in:
%
%   http://www.latex-project.org/lppl.txt
%
% and version 1.2 or later is part of all distributions of LaTeX
% version 1999/12/01 or later.
%
% \fi
%
% \CheckSum{0}
%
% \CharacterTable
%  {Upper-case    \A\B\C\D\E\F\G\H\I\J\K\L\M\N\O\P\Q\R\S\T\U\V\W\X\Y\Z
%   Lower-case    \a\b\c\d\e\f\g\h\i\j\k\l\m\n\o\p\q\r\s\t\u\v\w\x\y\z
%   Digits        \0\1\2\3\4\5\6\7\8\9
%   Exclamation   \!     Double quote  \"     Hash (number) \#
%   Dollar        \$     Percent       \%     Ampersand     \&
%   Acute accent  \'     Left paren    \(     Right paren   \)
%   Asterisk      \*     Plus          \+     Comma         \,
%   Minus         \-     Point         \.     Solidus       \/
%   Colon         \:     Semicolon     \;     Less than     \<
%   Equals        \=     Greater than  \>     Question mark \?
%   Commercial at \@     Left bracket  \[     Backslash     \\
%   Right bracket \]     Circumflex    \^     Underscore    \_
%   Grave accent  \`     Left brace    \{     Vertical bar  \|
%   Right brace   \}     Tilde         \~}
%
% \iffalse
%
%<*driver>
\documentclass{ltxdoc}
\usepackage[ngerman,english]{babel}
\usepackage[utf8]{inputenc}
\RequirePackage{xkeyval}
\usepackage[colorlinks, linkcolor=blue]{hyperref}
\EnableCrossrefs
\CodelineIndex
\RecordChanges
\begin{document}
  \DocInput{tubslogo.dtx}
\end{document}
%</driver>
% \fi
%
% \newenvironment{key}[2]{\expandafter\macro\expandafter{`#2'}}{\endmacro}
% \newenvironment{Options}%
%  {\begin{list}{}{%
%   \renewcommand{\makelabel}[1]{\texttt{##1}\hfil}%
%   \setlength{\itemsep}{-.5\parsep}
%   \settowidth{\labelwidth}{\texttt{xxxxxxxxxxx\space}}%
%   \setlength{\leftmargin}{\labelwidth}%
%   \addtolength{\leftmargin}{\labelsep}}%
%   \raggedright}
%  {\end{list}}
%
%
% \changes{v1.02}{ 2011 / 09 / 18 }{Initial version}
%
% \changes{v1.1}{ 2011 / 09 / 28 }{%
%   Optionsverarbeitung für Farbraum komplett überarbeitet}
%
% \changes{v1.2}{ 2011 / 11 / 22 }{%
%   Logo in beamer-RGB-Modell hinzugefuegt}
%
% \GetFileInfo{tubcolors.sty}
%
% \DoNotIndex{ list of control sequences }
%
% \title{\textsf{tubslogo} -- Siegelbandlogo für \emph{tubslatex}%
%   \thanks{This document corresponds to \textsf{tubslogo}~\fileversion,
%   dated \filedate.}}
% \author{Enrico Jörns \\ \texttt{e dot joerns at tu minus bs dot de}}
%
% \maketitle
%
% \StopEventually{\PrintIndex}
%
% \section{Implementierung}
%
%    \begin{macrocode}
%<*package>
\NeedsTeXFormat{LaTeX2e}
\ProvidesPackage{tubslogo}[\tubslatexVersion, file v1.2 CD-Logo der TU Braunschweig]
%    \end{macrocode}
%
% Lade benötigte Pakete
%    \begin{macrocode}
\RequirePackage{graphicx}
\RequirePackage{ifthen}
\RequirePackage{xkeyval}
\RequirePackage{calc}
%    \end{macrocode}
%
% Farb-Variable
%    \begin{macrocode}
\def\tubs@logo@file{TUBraunschweig_RGB}
%    \end{macrocode}
%
%
%    \begin{macro}{\logo@papersize}
%    \begin{macro}{\tubslogo@relscale}
% Speicher-Makros für Papiergrößen und relative Skalierung
%    \begin{macrocode}
\def\logo@papersize{}
\def\tubslogo@relscale{}
%    \end{macrocode}
%    \end{macro}\end{macro}
%
% \subsection{Optionen}
%    \begin{key}{}{rgb}
%    \begin{key}{}{rgbbeamer}
%    \begin{key}{}{hks}
%    \begin{key}{}{cmyk}
%    \begin{key}{}{mono}
% Farbmodell-Optionen
%    \begin{macrocode}
\DeclareOptionX{rgb}{\def\tubs@logo@file{TUBraunschweig_RGB}}
\DeclareOptionX{rgbbeamer}{\def\tubs@logo@file{TUBraunschweig_RGB_beamer}}
\DeclareOptionX{hks}{\def\tubs@logo@file{TUBraunschweig_SC}}
\DeclareOptionX{cmyk}{\def\tubs@logo@file{TUBraunschweig_4C}}
\DeclareOptionX{mono}{\def\tubs@logo@file{TUBraunschweig_B}}
%    \end{macrocode}
%    \end{key}\end{key}\end{key}\end{key}\end{key}
%
%    \begin{key}{}{a6paper}
%    \begin{key}{}{langpaper}
%    \begin{key}{}{a5paper}
%    \begin{key}{}{a4paper}
%    \begin{key}{}{a3paper}
%    \begin{key}{}{a2paper}
%    \begin{key}{}{a1paper}
%    \begin{key}{}{a0paper}
% Papiergrößen-Optionen
%    \begin{macrocode}
\DeclareOptionX{a6paper}{\def\logo@papersize{6}}
\DeclareOptionX{langpaper}{\def\logo@papersize{6}}
\DeclareOptionX{a5paper}{\def\logo@papersize{5}}
\DeclareOptionX{a4paper}{\def\logo@papersize{4}}
\DeclareOptionX{a3paper}{\def\logo@papersize{3}}
\DeclareOptionX{a2paper}{\def\logo@papersize{2}}
\DeclareOptionX{a1paper}{\def\logo@papersize{1}}
\DeclareOptionX{a0paper}{\def\logo@papersize{0}}
\DeclareOptionX{custompaper}{\def\logo@papersize{c}}
%    \end{macrocode}
%    \end{key}\end{key}\end{key}\end{key}\end{key}\end{key}\end{key}\end{key}
%
%    \begin{key}{}{relscale}
% Relative Skalierung [0.0 -- 1.0]
%    \begin{macrocode}
\DeclareOptionX{relscale}[1.0]{%
  \def\tubslogo@relscale{#1}
}
%    \end{macrocode}
%    \end{key}
%
% Fehler bei unbekannter Option
%    \begin{macrocode}
\DeclareOptionX*{%
  \PackageWarning{tubslogo}{Unknown option `\CurrentOption'}{}%
}
%    \end{macrocode}
%
% Optionen behandlen
%    \begin{macrocode}
\ExecuteOptionsX{rgb}%
\ProcessOptionsX*\relax% Behandelt auch Optionen der Dokumentenklasse!
%    \end{macrocode}
%
% \subsection{Bearbeitung}
%
%    \begin{macro}{\tubslogo@width}
% Logo-Dimensionen
%    \begin{macrocode}
\newlength{\tubslogo@width}
%    \end{macrocode}
%    \end{macro}
%    \begin{macro}{\tubslogoBaseWidth}
%    \begin{macro}{\tubslogoBaseHeight}
% Grunddimensionen des Siegelbandlogos (A4, unskaliert)
%    \begin{macrocode}
\newlength{\tubslogoBaseWidth}
\setlength{\tubslogoBaseWidth}{70mm}
\newlength{\tubslogoBaseHeight}
\setlength{\tubslogoBaseHeight}{26mm}
%    \end{macrocode}
%    \end{macro}\end{macro}
%    \begin{macro}{\tubslogoWidth}
%    \begin{macro}{\tubslogoHeight}
% Dimensionen des skalierten Logos
%    \begin{macrocode}
\newlength{\tubslogoWidth}
\newlength{\tubslogoHeight}
%    \end{macrocode}
%    \end{macro}\end{macro}
%
%    \begin{macrocode}
\ifthenelse{\equal{\logo@papersize}{6}}{\def\tubslogo@scale{0.6}}{%  60%
\ifthenelse{\equal{\logo@papersize}{5}}{\def\tubslogo@scale{0.7}}{%  70%
\ifthenelse{\equal{\logo@papersize}{4}}{\def\tubslogo@scale{1.0}}{% 100%
\ifthenelse{\equal{\logo@papersize}{3}}{\def\tubslogo@scale{1.4}}{% 140%
\ifthenelse{\equal{\logo@papersize}{2}}{\def\tubslogo@scale{2.0}}{% 200%
\ifthenelse{\equal{\logo@papersize}{1}}{\def\tubslogo@scale{2.8}}{% 280%
\ifthenelse{\equal{\logo@papersize}{0}}{\def\tubslogo@scale{4.0}}{% 400%
\ifthenelse{\equal{\logo@papersize}{c}}{% custom
  % calculate scaling factor
  \newdimen\temp@
  \ifthenelse{\boolean{tubspage@landscape}}{%
    \setlength{\temp@}{1pt*\ratio{0.3333\paperheight}{\tubslogoBaseWidth}}%
  }{%
    \setlength{\temp@}{1pt*\ratio{0.3333\paperwidth}{\tubslogoBaseWidth}}%
  }%
  \edef\tubslogo@scale{\strip@pt\temp@}% default
}{% 400%
  \def\tubslogo@scale{1.0}% default
}}}}}}}}
%    \end{macrocode}
%
% paper format scale
%    \begin{macrocode}
\setlength{\tubslogoWidth}{\tubslogo@scale\tubslogoBaseWidth}
\setlength{\tubslogoHeight}{\tubslogo@scale\tubslogoBaseHeight}
%    \end{macrocode}
% relative scale
%    \begin{macrocode}
\setlength{\tubslogoWidth}{\tubslogo@relscale\tubslogoWidth}
\setlength{\tubslogoHeight}{\tubslogo@relscale\tubslogoHeight}
%    \end{macrocode}
%
%    \begin{macro}{\tubslogo}
% \oarg{scale}\par
% Fügt TU-Logo als Grafik ein. Dimensionen ergeben sich aus Paketoptionen.
% Parameter |scale| erlaubt zusätzliche Skalierung.
%    \begin{macrocode}
\newcommand{\tubslogo}[1][\relax]{%
  \ifx#1\relax\setlength\tubslogo@width{\tubslogoWidth}%
  \else\setlength\tubslogo@width{#1\tubslogoBaseWidth}\fi%
  \includegraphics[width=\tubslogo@width]{\tubs@logo@file}}
%    \end{macrocode}
%    \end{macro}
%
%    \begin{macro}{\tubslogoAbs}
% \marg{width}\par
% Fügt TU-Logo als Grafik ein. Parameter |width| bestimmt absolute Breite.
%    \begin{macrocode}
\newcommand{\tubslogoAbs}[1]{\includegraphics[width=#1]{\tubs@logo@file}}
%    \end{macrocode}
%    \end{macro}
%
%    \begin{macrocode}
%</package>
%    \end{macrocode}
%
% \Finale
\endinput
