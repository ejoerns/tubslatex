\documentclass[cmyk,a4paper,colorscheme=green]{tubsreprt}

\usepackage[latin1]{inputenc}
\usepackage[ngerman]{babel}
\usepackage{listings}
\lstset{basicstyle=\ttfamily,columns=fixed}
\usepackage{xcolor}
\usepackage[colorlinks=true]{hyperref}
\usepackage{tikz}
\usepackage{booktabs}

%\input{tubslttr2ug-macros}

\author{Tobias Rad}
\title{Briefe mit \LaTeX\ im Corporate Design}
\institute{TU-Braunschweig}
\subject{Anleitung und Dokumentation}

\begin{document}

\begin{TUBStitlepage}
	\TUBSGaussGridImageBox{3}{8}{0}{6}{./Samuel_van_Hoogstraten.jpg}
	\TUBSGaussGridTextBox{1}{3}{0}{4}{\sffamily \huge Briefe mit \LaTeX\ im Corporate Design\\Anleitung und Dokumentation}{tuBlack}{tuSecondaryLight40}{top}
	\TUBSGaussGridTextBox{1}{3}{4}{6}{\begin{flushright}\sffamily \LARGE Tobias Rad\end{flushright}}{tuBlack}{tuSecondaryLight60}{bottom}
	\TUBSGaussGridTextBox{0}{1}{0}{6}{}{tuWhite}{tuSecondaryDark60}{center}
\end{TUBStitlepage}

\begin{abstract}
Ist Microsoft Word zum Schreiben von Briefen geeignet?
Diese Frage mag jeder f�r sich beantworten, doch wer sie mit nein beantwortet braucht eine Alternative.
Aus diesem Grund wurden auch die Vorgaben f�r Briefkorrespondenz mit Hilfe der KOMA-Skript-Klasse scrlttr2 umgesetzt.

Wie bereits bei den �brigen Vorlagen lag der Fokus auf der Beibehaltung des Standardverhaltens und der sinnvollen Erweiterung um Notwendiges.
Auch eine Integration mit den bereits bestehenden Paketen wurde weitestgehend angestrebt.

Die Basis f�r diese Vorlage stammt von Tobias Rad.
Ebenfalls daran beteiligt waren Mr X, Mrs Y und Enrico J�rns.%TODO...
\bigskip

Wir w�nschen viel Erfolg und Freude bei der Arbeit mit der Vorlage.
\bigskip

{\hfill Braunschweig, \today}
\vfill
\footnotesize{Titelbild (Selbstport�t von Samuel van Hoogstraten):\\
\url{http://commons.wikimedia.org/wiki/File:Samuel_van_Hoogstraten_-_Self-Portrait_-_WGA11724.jpg}}
\end{abstract}

\clearpage
\tableofcontents

\chapter{Installation}
\section{Miktex}
Unter Version 2.9 sollte es keine Probleme bei der Verwendung der Vorlage geben.
Alle ben�tigten Pakete werden automatisch nachgeladen.

\section{Texlive}


\chapter{Grunds�tzliche Verwendung}


\chapter{lco-Dateien}
\section{Institut}
\lstinputlisting{./examples/musterinstitut.lco}
\section{Mitarbeiter}
\lstinputlisting{./examples/mustermann.lco}

\end{document}