\RequirePackage{graphicx}
\RequirePackage{tikz}

\usetikzlibrary{fit,patterns}

%% Switch logo position and layout
\ifthenelse{\equal{\TUBSLogoPosition}{top}}{%
\def\DirectionSign{1}%
\def\DirectionOrigin{current page.south west}%
\def\RightSign{-1}%
}{}
\ifthenelse{\equal{\TUBSLogoPosition}{topright}}{%
\def\DirectionSign{1}
\def\DirectionOrigin{current page.south west}
\def\RightSign{1}%
}{}

\ifthenelse{\equal{\TUBSLogoPosition}{bottom}}{%
\def\DirectionSign{-1}
\def\DirectionOrigin{current page.north west}
\def\RightSign{-1}%
}{}
\ifthenelse{\equal{\TUBSLogoPosition}{bottomright}}{%
\def\DirectionSign{-1}%
\def\DirectionOrigin{current page.north west}%
\def\RightSign{1}%
}{}


% Include dimensions from the TUBS Design
%% Important dimensions and definitions from the TU Braunschweig Corporate Design
% The unit for length ist mm!

%% Calculate DIN Ax paper dimensions
% Fits DIN at least up to A9
% Page orientation
\ifthenelse{\equal{\landscape}{false}}
	{
	\pgfmathparse{(round(1189/2^(( \papersize + 1 )/2)-.2))}
	\edef\ThePaperWidth{\pgfmathresult}
	\pgfmathparse{(round(1189/2^(( \papersize)/2)-.2))}
	\edef\ThePaperHeight{\pgfmathresult}
	}{
	\pgfmathparse{(round(1189/2^(( \papersize)/2)-.2))}
	\edef\ThePaperWidth{\pgfmathresult}
	\pgfmathparse{(round(1189/2^(( \papersize + 1 )/2)-.2))}
	\edef\ThePaperHeight{\pgfmathresult}
	}

% Number of Segments. 8 in portrait and 6 in landscape,
% Sum over Segments 1+2+3+...+8
\ifthenelse{\equal{\landscape}{false}}
	{
	\def\TUBSGaussGridSegments{8}
	\def\TUBSGaussGridSum{36}
	}{
	\def\TUBSGaussGridSegments{6}
	\def\TUBSGaussGridSum{21}
	}

% The width of the white margin around the page:
\ifthenelse{\equal{\papersize}{0}}{\def\TUBSFrameWidth{32}}{}
\ifthenelse{\equal{\papersize}{1}}{\def\TUBSFrameWidth{22}}{}
\ifthenelse{\equal{\papersize}{2}}{\def\TUBSFrameWidth{16}}{}
\ifthenelse{\equal{\papersize}{3}}{\def\TUBSFrameWidth{11}}{}
\ifthenelse{\equal{\papersize}{4}}{\def\TUBSFrameWidth{8}}{}
\ifthenelse{\equal{\papersize}{5}}{\def\TUBSFrameWidth{5.5}}{}
\ifthenelse{\equal{\papersize}{6}}{\def\TUBSFrameWidth{5}}{}

% The margin in height and width direction:
\def\TUBSLeftMargin{\TUBSFrameWidth}
\def\TUBSBottomMargin{\DirectionSign * \TUBSFrameWidth}

% The width of the area of the main content. \ThePaperWidth - 2 * \TUBSFrameWidth
\pgfmathparse{\ThePaperWidth - 2 * \TUBSFrameWidth}
\edef\TUBSCommunicationWidth{\pgfmathresult}

% The height of the area of the adresser.
\ifthenelse{\equal{\papersize}{0}}{\def\TUBSAdresserHeight{160}}{}
\ifthenelse{\equal{\papersize}{1}}{\def\TUBSAdresserHeight{120}}{}
\ifthenelse{\equal{\papersize}{2}}{\def\TUBSAdresserHeight{80}}{}
\ifthenelse{\equal{\papersize}{3}}{\def\TUBSAdresserHeight{60}}{}
\ifthenelse{\equal{\papersize}{4}}{\def\TUBSAdresserHeight{40}}{}
\ifthenelse{\equal{\papersize}{5}}{\def\TUBSAdresserHeight{30}}{}
\ifthenelse{\equal{\papersize}{6}}{\def\TUBSAdresserHeight{25}}{}

% The height of the area of the main content. \ThePaperHeight - \TUBSAdresserHeight - \TUBSFrameWidth
\pgfmathparse{\ThePaperHeight - \TUBSFrameWidth - \TUBSAdresserHeight}
\edef\TUBSCommunicationHeightABS{\pgfmathresult}
\def\TUBSCommunicationHeight{\DirectionSign * \TUBSCommunicationHeightABS}

% The height of one gaussgrid segment.
\def\TUBSGaussRatio{\TUBSCommunicationHeight / \TUBSGaussGridSum}

% The distance between an institute logo and its name 
\def\TUBSDistanceBetweenInstituteLogoAndName{3}

% Coefficient to determine the logo size, relative to the original 70mm x 26 mm 
\ifthenelse{\equal{\papersize}{0}}{\def\TUBSLogoScaleCoefficient{4}}{}
\ifthenelse{\equal{\papersize}{1}}{\def\TUBSLogoScaleCoefficient{2.8}}{}
\ifthenelse{\equal{\papersize}{2}}{\def\TUBSLogoScaleCoefficient{2}}{}
\ifthenelse{\equal{\papersize}{3}}{\def\TUBSLogoScaleCoefficient{1.4}}{}
\ifthenelse{\equal{\papersize}{4}}{\def\TUBSLogoScaleCoefficient{1}}{}
\ifthenelse{\equal{\papersize}{5}}{\def\TUBSLogoScaleCoefficient{0.7}}{}
\ifthenelse{\equal{\papersize}{6}}{\def\TUBSLogoScaleCoefficient{0.6}}{}

% Actual dimensions of the TUBS Logo
\pgfmathparse{26*\TUBSLogoScaleCoefficient}
\edef\TUBSLogoHeight{\pgfmathresult}
\pgfmathparse{70*\TUBSLogoScaleCoefficient}
\edef\TUBSLogoWidth{\pgfmathresult}

% Institute logo area height
\pgfmathparse{abs(\TUBSAdresserHeight)- 2 * \TUBSFrameWidth}
\edef\TUBSInstituteLogoHeight{\pgfmathresult}
%% Standard variables for the use in the layout
\gdef\TUBSauthor{\makeatletter\@author\makeatother}
\gdef\TUBStitle{\makeatletter\@title\makeatother}
\gdef\TUBSsubject{\makeatletter\@subject\makeatother}
\gdef\TUBSdate{\makeatletter\@date\makeatother}
\def\institute#1{\gdef\@institute{#1}}
\gdef\TUBSinstitute{\makeatletter\@institute\makeatother}
\def\address#1{\gdef\@address{#1}}
\gdef\TUBSaddress{\makeatletter\@address\makeatother}

%% Non-standard variables
\gdef\@tubsinstlogo{}
\gdef\TUBSInstituteLogo#1{\makeatletter\gdef\@tubsinstlogo{#1}\makeatother}
\gdef\@tubsinstname{}
\gdef\TUBSInstituteName#1{\makeatletter\gdef\@tubsinstname{#1}\makeatother}

%% Gauss addition series at each iteration [parameter: \thegaussgrid]
% Parameters:
% 1. beam into a TikZ environment after each iteration 
% 2. beam into a TikZ environment after the eighth iteration
\newcounter{gaussgrid}
\newcommand{\TUBSGaussGrid}[2]{%
  \begin{tikzpicture}[remember picture,overlay]
  \setcounter{gaussgrid}{0}
  \foreach \y in {0,...,\TUBSGaussGridSegments}{%
    \addtocounter{gaussgrid}{\y}
    #1
    }
  #2
  \end{tikzpicture}
}

%% Schow debugging information: grid with its own coordninate system and dimensions of each cell.
% Parameter:
% none
\newcounter{cell}
\newcommand{\TUBSShowGaussGrid}{%
	\TUBSGaussGrid{%
		\draw(\DirectionOrigin)++(0mm,\TUBSGaussRatio * \thegaussgrid mm +
\TUBSBottomMargin mm) -- ++(\TUBSCommunicationWidth mm + 2 * \TUBSFrameWidth
mm,0mm)node[xshift=-\TUBSFrameWidth mm /2]{\y};
		\foreach \x in {0,...,6}{%
			\ifthenelse{\equal{0}{\y}}{%
				\draw(\DirectionOrigin)++(\TUBSLeftMargin mm + \TUBSCommunicationWidth
mm/6*\x,0)node[name=a]{}node[yshift=\DirectionSign * \TUBSFrameWidth mm /2]{\x}
-- ++(0mm,\TUBSCommunicationHeight mm + \TUBSBottomMargin mm)node[name=b]{};
				\ifthenelse{\x > 0}{%
					\node[xshift={\TUBSLeftMargin mm + \TUBSCommunicationWidth mm/6*(\x
-1)},yshift={\TUBSGaussRatio * \thecell mm + \TUBSBottomMargin mm}] (a) at
(\DirectionOrigin){};
					\node[xshift={\TUBSLeftMargin mm + \TUBSCommunicationWidth mm/6*\x},yshift={\TUBSGaussRatio * \thegaussgrid mm + \TUBSBottomMargin mm}] (b) at (\DirectionOrigin){};
					\pgfmathparse{\TUBSCommunicationWidth /6}
					\edef\cellwidth{\pgfmathresult}
					\node[fit=(a)(b),below]{\small \bfseries \cellwidth mm};
					}{}
				}{%
				\ifthenelse{\x = 1}{%
					\node[xshift={\TUBSLeftMargin mm + \TUBSCommunicationWidth mm/6*(\x
-1)},yshift={\TUBSGaussRatio * \thecell mm + \TUBSBottomMargin mm}] (a) at
(\DirectionOrigin){};
					\node[xshift={\TUBSLeftMargin mm + \TUBSCommunicationWidth mm/6*\x},yshift={\TUBSGaussRatio * \thegaussgrid mm + \TUBSBottomMargin mm}] (b) at (\DirectionOrigin){};
					\pgfmathparse{\TUBSGaussRatio*abs(\thegaussgrid-\thecell)}
					\edef\cellheight{\pgfmathresult}
					\node[fit=(a)(b)](tmpnode){};
					\node[at=(tmpnode.west),right,xshift=-\TUBSLeftMargin mm /2]{\small \bfseries \cellheight mm};
				}{}
			}
		}
		\setcounter{cell}{\thegaussgrid}
	}{}
}

%% A colored box on the TUBS gauss grid.
% Parameter:
% 1. Start gauss row
% 2. End gauss row
% 3. Start gaus column
% 4. End gauss column
% 5. Content of box
% 6. Color of the font
% 7. Color of the box (background)
% 8. Vertical text position: 'top', 'bottom', 'center'
\newcounter{firstnode}
\newcounter{secondnode}
\newcommand{\TUBSGaussGridTextBox}[8]{%
	\ifthenelse{#1 > #2}{%
		\def\numone{#2}
		\def\numtwo{#1}
	}{%
		\def\numone{#1}
		\def\numtwo{#2}
	}
	\ifthenelse{#3 > #4}{%
		\def\numthree{#4}
		\def\numfour{#3}
	}{%
		\def\numthree{#3}
		\def\numfour{#4}
	}
	\TUBSGaussGrid{%
		\ifthenelse{\equal{\numone}{\y}}{%
			\setcounter{firstnode}{\thegaussgrid}
			}{}
		\ifthenelse{\equal{\numtwo}{\y}}{%
			\setcounter{secondnode}{\thegaussgrid}
			}{}
		}{%
		\pgfmathparse{\TUBSLeftMargin + \numthree * \TUBSCommunicationWidth /6}
		\edef\cellax{\pgfmathresult mm}
		\pgfmathparse{\TUBSBottomMargin + \thefirstnode * \TUBSGaussRatio}
		\edef\cellay{\pgfmathresult mm}
		\pgfmathparse{\TUBSLeftMargin + \numfour * \TUBSCommunicationWidth /6}
		\edef\cellbx{\pgfmathresult mm}
		\pgfmathparse{\TUBSBottomMargin + \thesecondnode *\TUBSGaussRatio}
		\edef\cellby{\pgfmathresult mm}
			
		\node[	inner sep=0,outer sep=0,line width=0,
			xshift=\cellax,
			yshift=\cellay,
			] (firstnode) at (\DirectionOrigin){};
		\node[	inner sep=0,outer sep=0,line width=0,
			xshift=\cellbx,
			yshift=\cellby,
			] (secondnode) at (\DirectionOrigin){};

		\fill[#7] (firstnode) rectangle (secondnode);
		\ifthenelse{\equal{\TUBSDesignHelper}{true}}{\fill[#7,pattern=north west lines] (firstnode) rectangle (secondnode);}{}

		\node[	inner sep=0,outer sep=0,line width=0,
			xshift=\cellax+\TUBSFrameWidth mm,
			yshift=\cellay+\DirectionSign * \TUBSFrameWidth mm,
			] (firstnode) at (\DirectionOrigin){};
		\node[	inner sep=0,outer sep=0,line width=0,
			xshift=\cellbx-\TUBSFrameWidth mm,
			yshift=\cellby-\DirectionSign * \TUBSFrameWidth mm,
			] (secondnode) at (\DirectionOrigin){};

		\pgfmathparse{abs(\thefirstnode - \thesecondnode)*(abs(\TUBSGaussRatio))};
		\edef\TUBSBoxHeight{\pgfmathresult}
		\pgfmathparse{abs(#3 - #4)/6*\TUBSCommunicationWidth - 2*abs(\TUBSFrameWidth) };
		\edef\TUBSBoxWidth{\pgfmathresult}
		
		\node[name=Box,outer sep=0mm,inner sep=0,fit=(firstnode)(secondnode)]{};
		\ifthenelse{\equal{\TUBSDesignHelper}{true}}{\fill[#7] (firstnode) rectangle (secondnode);}{}
		
		\def\textposition{north west} %default
		\ifthenelse{\equal{#8}{top}}{%
			\def\textposition{north west}
		}{}
		\ifthenelse{\equal{#8}{bottom}}{%
			\def\textposition{south west}
		}{}
		\ifthenelse{\equal{#8}{center}}{%
			\def\textposition{west}
		}{}

		\node[at=(Box.\textposition),inner sep=0,outer sep=0,text width=\TUBSBoxWidth mm,anchor=\textposition,text ragged,#6]{#5};


%		\node[fit=(firstnode),above]{#5};
		}
	}

%% A box on the TUBS gauss grid with an scaled image. If the aspect ratio
%% of the given image does not fit to the dimensions of the box, the image will be deformed!
%Parameter:
% 1. Start gauss row
% 2. End gauss row
% 3. Start gaus column
% 4. End gauss column
% 5. Path to the image file
\newcommand{\TUBSGaussGridImageBox}[5]{%
	\TUBSGaussGrid{%
		\ifthenelse{\equal{#1}{\y}}{%
			\setcounter{firstnode}{\thegaussgrid}
			}{}
		\ifthenelse{\equal{#2}{\y}}{%
			\setcounter{secondnode}{\thegaussgrid}
			}{}
	}{
		\ifthenelse{#1 > #2}{%
			\def\numone{#2}
			\def\numtwo{#1}
		}{%
			\def\numone{#1}
			\def\numtwo{#2}
		}
		\ifthenelse{#3 > #4}{%
			\def\numthree{#4}
			\def\numfour{#3}
		}{%
			\def\numthree{#3}
			\def\numfour{#4}
		}

		\pgfmathparse{\TUBSLeftMargin + \numthree * \TUBSCommunicationWidth /6}
		\edef\cellax{\pgfmathresult mm}
		\pgfmathparse{\TUBSBottomMargin + \thefirstnode * \TUBSGaussRatio}
		\edef\cellay{\pgfmathresult mm}
		\pgfmathparse{\TUBSLeftMargin + \numfour * \TUBSCommunicationWidth /6}
		\edef\cellbx{\pgfmathresult mm}
		\pgfmathparse{\TUBSBottomMargin + \thesecondnode * \TUBSGaussRatio}
		\edef\cellby{\pgfmathresult mm}
		
		\node[	inner sep=0,outer sep=0,line width=0,
			xshift=\cellax,
			yshift=\cellay,
			] (firstnode) at (\DirectionOrigin){};
		\node[	inner sep=0,outer sep=0,line width=0,
			xshift=\cellbx,
			yshift=\cellby,
			] (secondnode) at (\DirectionOrigin){};

		\pgfmathparse{abs(#3 - #4)/6*\TUBSCommunicationWidth};
		\edef\TUBSImageWidth{\pgfmathresult}
		\pgfmathparse{abs(\thefirstnode - \thesecondnode)*(abs(\TUBSGaussRatio))};
		\edef\TUBSImageHeight{\pgfmathresult}
		% quick-hack, else texlive and miktex would display something different
		\node[inner sep=0,fit=(firstnode)(secondnode),inner sep=0mm](firstandsecond){};
		\node (final) at ++(firstandsecond) {\includegraphics[width=\TUBSImageWidth mm, height=\TUBSImageHeight mm]{#5}};
		%quick-hack end
		\ifthenelse{\equal{\TUBSDesignHelper}{true}}{\node[inner sep=0,fit=(firstnode)(secondnode),inner sep=0]{\Large \bfseries Width: \TUBSImageWidth mm \\ Height: \TUBSImageHeight mm \\ Aspect ratio: \pgfmathparse{ \TUBSImageWidth / \TUBSImageHeight}\pgfmathresult};}{}
	}
}
