% \iffalse meta-comment
%
% Copyright (C) 2011 by Enrico Jörns
% -----------------------------------
%
% This file may be distributed and/or modified under the
% conditions of the LaTeX Project Public License, either version 1.2
% of this license or (at your option) any later version.
% The latest version of this license is in:
%
%   http://www.latex-project.org/lppl.txt
%
% and version 1.2 or later is part of all distributions of LaTeX
% version 1999/12/01 or later.
%
% \fi
%
% \CheckSum{0}
%
% \CharacterTable
%  {Upper-case    \A\B\C\D\E\F\G\H\I\J\K\L\M\N\O\P\Q\R\S\T\U\V\W\X\Y\Z
%   Lower-case    \a\b\c\d\e\f\g\h\i\j\k\l\m\n\o\p\q\r\s\t\u\v\w\x\y\z
%   Digits        \0\1\2\3\4\5\6\7\8\9
%   Exclamation   \!     Double quote  \"     Hash (number) \#
%   Dollar        \$     Percent       \%     Ampersand     \&
%   Acute accent  \'     Left paren    \(     Right paren   \)
%   Asterisk      \*     Plus          \+     Comma         \,
%   Minus         \-     Point         \.     Solidus       \/
%   Colon         \:     Semicolon     \;     Less than     \<
%   Equals        \=     Greater than  \>     Question mark \?
%   Commercial at \@     Left bracket  \[     Backslash     \\
%   Right bracket \]     Circumflex    \^     Underscore    \_
%   Grave accent  \`     Left brace    \{     Vertical bar  \|
%   Right brace   \}     Tilde         \~}
%
% \iffalse
%
%<*driver>
\documentclass[11pt]{ltxdoc}
\usepackage[ngerman,english]{babel}
\usepackage[utf8]{inputenc}
\RequirePackage{xkeyval}
\usepackage[colorlinks, linkcolor=blue]{hyperref}
\EnableCrossrefs
\CodelineIndex
\RecordChanges
\begin{document}
  \DocInput{tubsfont.dtx}
\end{document}
%</driver>
% \fi
%
%
% \changes{v1.0}{ YYYY / MM / DD }{Initial version}
%
% \GetFileInfo{tubsodc.sty}
%
% \DoNotIndex{ list of control sequences }
%
% \title{\textsf{tubsfont} -- 
%   font definitions for tubslatex\thanks{This document
%   corresponds to \textsf{tubsfont}~\fileversion,
%   dated \filedate.}}
% \author{Enrico Jörns \\ \texttt{e dot joerns at tu minus bs dot de}}
%
% \maketitle
%
% \begin{abstract}
%   Put text here.
% \end{abstract}
%
% \section{Introduction}
%
% Put text here.
%
% \section{Usage}
%
%
% \StopEventually{\PrintIndex}
%
% \section{Implementation}
%
%    \begin{macrocode}
%<*class>
%    \end{macrocode}
%
%
%    \begin{macrocode}
%</class>
%    \end{macrocode}
% 
%
% \subsection{Options}
% Es werden zur Zeit keine Optionen zur Verfügung gestellt
%    \begin{macrocode}
%<*option>
%    \end{macrocode}
%    \begin{macrocode}
%</option>
%    \end{macrocode}
%
%    \begin{macrocode}
%<*body>
%    \end{macrocode}
% 
% Lade Nexus-Schrift
%    \begin{macrocode}
\RequirePackage{nexus}
%    \end{macrocode}
% Eliminiert den standardmäßigen Abstand über |\chapter|, das dieser aufgrund
% des tief hängenden Textbereichs überflüssig ist.
%    \begin{macrocode}
\RequirePackage[compact]{titlesec}
\titleformat{\chapter}[block]
{\normalfont\huge\sffamily\bfseries}{\thechapter}{10pt}{}
\titlespacing*{\chapter}{0pt}{0pt}{30pt}
%    \end{macrocode}
% 
%    \begin{macrocode}
%</body>
%    \end{macrocode}
% 
% Font-Definitionen teilweise aus memoir geklaut.
% Schriftgrößen sind an den im CD festgelegten orientiert.
% Die genau definierten sind dasbei |Huge|, |huge|, |Large|, |large|, 
% |normalsize| und |footnotesize|.
% Die verbleibenden Zwischengrößen sind linear interpoliert.
% 
%    \begin{macrocode}
%<*clo>
%    \end{macrocode}
% 
% Definition der Größen nach Vorgabe
%    \begin{macrocode}
%<*a4|a5|din>
\renewcommand{\normalsize}{%
   \@setfontsize\normalsize\@ixpt\@xiipt
   \abovedisplayskip 9\p@ \@plus 2\p@ \@minus 4.5\p@
   \abovedisplayshortskip \z@ \@plus 3\p@
   \belowdisplayshortskip 5.5\p@ \@plus 2.5\p@ \@minus 3\p@
   \belowdisplayskip \abovedisplayskip
   \let\@listi\@listI}
% 
\renewcommand{\footnotesize}{%
    \@setfontsize\footnotesize\@viipt{10}%
    \abovedisplayskip 6\p@ \@plus 2\p@ \@minus 4\p@
    \abovedisplayshortskip \z@ \@plus 2\p@
    \belowdisplayshortskip 4\p@ \@plus 2\p@ \@minus 2\p@
    \def\@listi{\leftmargin\leftmargini
                \topsep 2\p@ \@plus 2\p@ \@minus 2\p@
                \parsep 1\p@ \@plus\p@ \@minus\p@
                \itemsep \parsep
%%                \itemindent\z@
               }%
   \belowdisplayskip \abovedisplayskip
}
%</a4|a5|din>
% Definition der Papierformatabhänigen Größen (A4)
%<*a4>
\def\large{\@setfontsize\large{14pt}{18}}
\def\Large{\@setfontsize\Large{18pt}{23}}
\def\huge{\@setfontsize\huge{36pt}{43}}
\def\Huge{\@setfontsize\Huge{45pt}{54}}
%    \end{macrocode}
% Definition der interpolierten Größen (A4)
%    \begin{macrocode}
\def\LARGE{\@setfontsize\LARGE{27pt}{35}}
\def\tiny{\@setfontsize\tiny\@vpt{6}}
\def\scriptsize{\@setfontsize\scriptsize\@vipt{7}}
%</a4>
%    \end{macrocode}
%
% Definition der Papierformatabhänigen Größen (A5/DIN lang)
%    \begin{macrocode}
%<*a5|din>
\def\large{\@setfontsize\large{10pt}{12}}
\def\Large{\@setfontsize\Large{13pt}{17}}
\def\LARGE{\@setfontsize\LARGE{16pt}{21}}
\def\huge{\@setfontsize\huge{24pt}{29}}
\def\Huge{\@setfontsize\Huge{30pt}{36}}
%    \end{macrocode}
% Definition der interpolierten Größen (A5/DIN lang)
%    \begin{macrocode}
\def\tiny{\@setfontsize\tiny\@vpt{6}}
\def\scriptsize{\@setfontsize\scriptsize\@vipt{7}}
%</a5|din>
%    \end{macrocode}
%    \begin{macrocode}
%</clo>
%    \end{macrocode}
% 

% 
% \Finale
\endinput
