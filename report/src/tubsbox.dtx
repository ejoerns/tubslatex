
% \iffalse meta-comment
%
% Copyright (C) 2011 by Enrico Jörns
% -----------------------------------
%
% This file may be distributed and/or modified under the
% conditions of the LaTeX Project Public License, either version 1.2
% of this license or (at your option) any later version.
% The latest version of this license is in:
%
%   http://www.latex-project.org/lppl.txt
%
% and version 1.2 or later is part of all distributions of LaTeX
% version 1999/12/01 or later.
%
% \fi
%
% \CheckSum{0}
%
% \CharacterTable
%  {Upper-case    \A\B\C\D\E\F\G\H\I\J\K\L\M\N\O\P\Q\R\S\T\U\V\W\X\Y\Z
%   Lower-case    \a\b\c\d\e\f\g\h\i\j\k\l\m\n\o\p\q\r\s\t\u\v\w\x\y\z
%   Digits        \0\1\2\3\4\5\6\7\8\9
%   Exclamation   \!     Double quote  \"     Hash (number) \#
%   Dollar        \$     Percent       \%     Ampersand     \&
%   Acute accent  \'     Left paren    \(     Right paren   \)
%   Asterisk      \*     Plus          \+     Comma         \,
%   Minus         \-     Point         \.     Solidus       \/
%   Colon         \:     Semicolon     \;     Less than     \<
%   Equals        \=     Greater than  \>     Question mark \?
%   Commercial at \@     Left bracket  \[     Backslash     \\
%   Right bracket \]     Circumflex    \^     Underscore    \_
%   Grave accent  \`     Left brace    \{     Vertical bar  \|
%   Right brace   \}     Tilde         \~}
%
% \iffalse
%
%<*driver>
\documentclass{tubsltxdoc}
\usepackage[ngerman]{babel}
\usepackage[utf8]{inputenc}
\usepackage{nexus}
\usepackage[colorlinks, linkcolor=blue]{hyperref}
\usepackage{tabularx}
\EnableCrossrefs
\CodelineIndex
\RecordChanges
\begin{document}
  \DocInput{tubsbox.dtx}
\end{document}
%</driver>
% \fi
%
% \changes{v1.0}{ 2011 / 08 / 23 }{Initial version}
% \changes{v1.1}{ 2011 / 08 / 23 }{%
%   Fehler bei Darstellung von modsubrow am Zeilenende behoben,
%   Hintergrundfarbe als Option nutzbar
% }
%
% \GetFileInfo{tubsbox.sty}
%
% \DoNotIndex{ list of control sequences }
%
% \title{\textsf{tubsbox} -- 
%   Box-Definitionen für \emph{tubslatex}\thanks{This document
%   corresponds to \textsf{tubsbox}~\fileversion,
%   dated \filedate.}}
% \author{Enrico Jörns \\ \texttt{e dot joerns at tu minus bs dot de}}
%
% \maketitle
%
% \begin{abstract}
%   Diese Datei stellt die Umgebung |gaussbox| für Darstellungen im Gaußraster 
%   und die Umgebungen |modulebox| für Darstellungen im Modulraster zur Verfügung.
% \end{abstract}
%
% \tableofcontents
%
% \section{Benutzung}
%
% \subsection{Gaußraster}
%
% Das Gaußraster wird zur Inhalts-Darstellung auf Postern und Titelseiten 
% verwendet.
%
% \DescribeEnv{gaussbox} Box im Gauß-Raster
%
% Syntax: |\gaussbox|\oarg{options}\marg{}
%
%
% \subsection{Modulsystem}
%
% Das Modulsystem wird ausschließlich für wissenschaftliche Plakate verwendet.
%
% \StopEventually{\PrintIndex}
%
% \section{Implementierung}
%
%
%    \begin{macrocode}
%<*package>
%    \end{macrocode}
%
% Es werden die folgenden Pakete geladen: |ifthen| und |xkeyval| werden
% allgemein zum Definieren und Auswerten von Optionen etc. benötigt.
% |etoolbox| und |forloop| werden für die Auswertung von kommagetrennten
% Listen benutzt. Aus |tubstypearea| werden alle benötigten Längen entnommen,
% um die Boxen mit |texpos| korrekt positionieren zu können.
%    \begin{macrocode}
\RequirePackage{ifthen}
\RequirePackage{xkeyval}
\RequirePackage{etoolbox}
\RequirePackage{forloop}
\RequirePackage{tubsdebug}
\RequirePackage{tubstypearea}
\RequirePackage[absolute,overlay]{textpos}
%    \end{macrocode}
%
%%%%%%%%%%%%%%%%%%%%%%%%%%%%%%%%%%%%%%%%%%%%%%%%%%%%%%%%%%%%%%%%%%%%%%%%%%%%%%%%
% \subsection{Paketoptionen}\label{sec:options}
%%%%%%%%%%%%%%%%%%%%%%%%%%%%%%%%%%%%%%%%%%%%%%%%%%%%%%%%%%%%%%%%%%%%%%%%%%%%%%%%
%
% Alle hier beschriebenen Optionen sind sowohl für die |gaussbox| als
% auch für die |modulebox| verwendbar.
%
%    \begin{macro}{tubsbox@bottomsender}
% Merker für Option '|sender|'.
%    \begin{macrocode}
\newboolean{tubsbox@bottomsender}\setboolean{tubsbox@bottomsender}{false}
%    \end{macrocode}
%    \end{macro}
%
%    \begin{macro}{\tubsbox@framebox}
% Speichert ggf. Kommando mit dem ein Rahmen um die jeweilige Box
% gezeichnet werden soll.
%    \begin{macrocode}
\def\tubsbox@framebox{\relax}
%    \end{macrocode}
%    \end{macro}
%
%    \begin{macro}{\tubsbox@colorbox}
% Speichert ggf. colorbox-Kommando mit dem die Hintergrundfarbe der jeweiligen
% Box gezeichnet werden soll.
%    \begin{macrocode}
\newcommand{\tubsbox@colorbox}{\relax}
%    \end{macrocode}
%    \end{macro}
%
%    \begin{macro}{\if@tubs@oddpage}
% \marg{odd cmd}\marg{even cmd}\par
% Switch um Kommandos zu definieren, die für gerade und ungerade Seiten
% unterschiedlich arbeiten.
% Ist nur aktiv, wenn Option |twoside| benutzt wird.
%    \begin{macrocode}
\newcommand{\if@tubs@oddpage}[2]{%
  \ifthenelse{\boolean{tubsstyle@twoside}}{%
    \ifthispageodd{#1}{#2}%
  }{#1}%
}
%    \end{macrocode}
%    \end{macro}
%
%
%    \begin{macro}{\modulesetup}
% \oarg{option}\par
% Setzt keys für die |modulebox|.
% Die keys werden dabei an die Familien |tubsbox.sty|, |tubsbox| und |modulebox| delegiert.
% TODO: Genaue Verwendung noch unklar.
%    \begin{macrocode}
\newcommand*{\modulesetup}[1][]{%
  \setkeys*{tubsbox.sty}{#1}%
  \setrmkeys*{tubsbox}%
  \setrmkeys{modulebox}%
}
%    \end{macrocode}
%    \end{macro}
%
%    \begin{macro}{\gausspagesetup}
% \oarg{option}\par
% Setzt Paketoptionen der familie tubsbox.sty
%    \begin{macrocode}
\newcommand*{\gausspagesetup}[1][]{%
  \setkeys*{tubsbox.sty}{#1}%
}
%    \end{macrocode}
%    \end{macro}
%
%
% \subsection{Paket-Optionen}
%
% Dummy-Option, die aktuell nur aus Designschwäche existiert.
%    \begin{macrocode}
\define@key{tubsbox.sty}{bgcolor}[]{}
%    \end{macrocode}
%
% \begin{key}{tubsbox.sty}{sender}
% Mögliche Werte: |bottom| oder |top|.\\
% Schaltet zwischen den beiden Layoutvarianten mit Absenderbereich oben
% und unten auf der Seite um, damit die Boxen korrekt platziert werden.\\
% Standardmäßig ist der Absenderbereich oben.
%    \begin{macrocode}
\define@choicekey{tubsbox.sty}{sender}[\var\nr]{top,bottom}{%
  \ifcase\nr\relax
    \setboolean{tubsbox@bottomsender}{false}
  \or
    \setboolean{tubsbox@bottomsender}{true}
  \fi
}
%    \end{macrocode}
%    \end{key}
%
%
%    \begin{key}{tubsbox.sty}{nosep}
% Kein Abstand zwischen Boxen (erzeugt |\if@gb@nosep|)
%    \begin{macrocode}
\define@boolkey{tubsbox.sty}[@gb@]{nosep}[true]{}
%    \end{macrocode}
%    \end{key}
%
% Paket-Optionen auswerten,
% Initialisierung mit Werten der Paket-Optionen.
%    \begin{macrocode}
\ProcessOptionsX*\relax%
%    \end{macrocode}%
%
%%%%%%%%%%%%%%%%%%%%%%%%%%%%%%%%%%%%%%%%%%%%%%%%%%%%%%%%%%%%%%%%%%%%%%%%%%%%%%%%
% \subsection{Gemeinsame Box-Optionen}
%%%%%%%%%%%%%%%%%%%%%%%%%%%%%%%%%%%%%%%%%%%%%%%%%%%%%%%%%%%%%%%%%%%%%%%%%%%%%%%%
%
% Optionen der Famillie |tubsbox| sind sowohl für das Gaussraster (|gaussbox|)
% als auch für das Modulsystem (|modulebox|) verfügbar
%    \begin{key}{tubsbox.sty}{bgcolor}
%   Standard-Hintergrundfarbe der Boxen festlegen.\\
%   Mögliche Werte sind |none|, sowie die jeweilige gewünschte Farbe.\\
%   Standardmäßig haben die Boxen keine Hintergrundfarbe.%TODO?
%    \begin{macrocode}
\define@key{tubsbox}{bgcolor}{%
  \ifthenelse{\equal{#1}{none}}{%
    \renewcommand\tubsbox@colorbox{\relax}
  }{%
    \def\bgcolor{#1}
    \renewcommand\tubsbox@colorbox{\colorbox{\bgcolor}}
  }
}
%    \end{macrocode}
%    \end{key}
%
%    \begin{key}{tubsbox}{fgcolor}
% Legt Vordergrund- bzw. Schriftfarbe der Box fest
%    \begin{macrocode}
\define@key{tubsbox}{fgcolor}{
  \ifthenelse{\equal{#1}{none}}{%
  }{%
    \color{#1}%
  }
}
%    \end{macrocode}
%    \end{key}
%
% Hierin werden vfill-Kommandos gespeichert, um die vertikale Positionierung
% des Box-Inhaltes über Optionen steuern zu können.
%    \begin{macrocode}
\newcommand\tb@topfill{\relax}
\newcommand\tb@bottomfill{\relax}
%    \end{macrocode}
%
%    \begin{key}{tubsbox}{t}
%  Inhalt in Boxen oben ausrichten.
%    \begin{macrocode}
\define@key{tubsbox}{t}[true]{%
  \renewcommand\tb@topfill{\relax}
  \renewcommand\tb@bottomfill{\relax}
}
%    \end{macrocode}
%    \end{key}
%
%    \begin{key}{tubsbox}{c}
%  Inhalt in Boxen vertikal zentrieren.
%    \begin{macrocode}
\define@key{tubsbox}{c}[true]{%
  \renewcommand\tb@topfill{\vfil}
  \renewcommand\tb@bottomfill{\vfil}
}
%    \end{macrocode}
%    \end{key}
%
%    \begin{key}{tubsbox}{b}
%  Inhalt in Boxen unten ausrichten.
%    \begin{macrocode}
\define@key{tubsbox}{b}[true]{%
  \renewcommand\tb@topfill{\vfill}
  \renewcommand\tb@bottomfill{\relax}
}
%    \end{macrocode}
%    \end{key}
%
%    \begin{key}{tubsbox}{frame}
% Zeichnet wenn gewünscht einen Rahmen um die jeweiligen Boxen.\\
% Mögliche Werte: |none| - Kein Rahmen, |fbox| - einfacher Rahmen.\\
% Standardmäßig haben die Boxen keinen Rahmen.
%    \begin{macrocode}
\define@choicekey{tubsbox}{frame}[\val\nr]{none,fbox}{%
  \ifcase\nr\relax
    \def\tubsbox@framebox{\relax}
  \or
    \def\tubsbox@framebox{\fbox}
  \fi
}
%    \end{macrocode}
%    \end{key}
%
%%%%%%%%%%%%%%%%%%%%%%%%%%%%%%%%%%%%%%%%%%%%%%%%%%%%%%%%%%%%%%%%%%%%%%%%%%%%%%%%
% \subsection{Gaussbox}
%%%%%%%%%%%%%%%%%%%%%%%%%%%%%%%%%%%%%%%%%%%%%%%%%%%%%%%%%%%%%%%%%%%%%%%%%%%%%%%%
%
%
%    \begin{key}{gaussbox}{nosep}
% Kein Abstand zwischen Boxen
%    \begin{macrocode}
% \define@boolkey{gaussbox}[@gb@]{nosep}[true]{}
%    \end{macrocode}
%    \end{key}
%
% Erlaubt zusätzlichen Abstand zum oberen Boxrand.
% Dies wird unter anderem für den 3fachen Abstand in der ersten Box verwendet.
%    \begin{macrocode}
\newcommand*\tubs@gb@addtoppadding{0mm}
%    \end{macrocode}
%
% \subsubsection{Optionen}
%
%    \begin{key}{gaussbox}{logosep}
% Fügt zusätzlichen Top-Innenabstand hinzu. Wenn kein Wert angegeben, wird
% doppelte Rahmenstärke hinzugefügt. (Dies ist die Standardeinstellung für
% die erste tubsbox auf einer Seite mit Siegelbandlogo.)
%    \begin{macrocode}
\define@boolkey{gaussbox}{logosep}[true]{%
  \def\tubs@gb@toppadding@outer{3\tubsborderwidth}
}
%    \end{macrocode}
%    \end{key}
%
%    \begin{key}{gaussbox}{outerpadding}
% Setzt äußeres Padding der gaussbox
%    \begin{macrocode}
\define@choicekey{gaussbox}{outerpadding}[\val\nr]{%
        default,none,vnone,hnone}{%
  \ifcase\nr\relax% default
  \or% kein Außen-Padding
    \def\tubs@gb@toppadding@outer{0mm}%
    \def\tubs@gb@bottompadding@outer{0mm}%
    \def\tubs@gb@leftpadding@outer{0mm}%
    \def\tubs@gb@rightpadding@outer{0mm}%
  \or% keine Abstände zum Blattrand
    \def\tubs@gb@toppadding@outer{0mm}%
    \def\tubs@gb@bottompadding@outer{0mm}%
  \or% Innenabstände auf columnsep
    \def\tubs@gb@leftpadding@outer{0mm}%
    \def\tubs@gb@rightpadding@outer{0mm}%
  \fi
}
%    \end{macrocode}
%    \end{key}
%
%    \begin{key}{gaussbox}{innerpadding}
% Setzt inneres Padding der gaussbox
%    \begin{macrocode}
\define@choicekey{gaussbox}{innerpadding}[\val\nr]{%
        default,columnsep,none,vnone,hnone}{%
  \ifcase\nr\relax% default
  \or% Innenabstände auf columnsep
    \def\tubs@gb@toppadding@inner{0.5\tubscolumnsep}%
    \def\tubs@gb@bottompadding@inner{0.5\tubscolumnsep}%
  \or% Alle Innenabstände auf 0
    \def\tubs@gb@toppadding@inner{0mm}%
    \def\tubs@gb@bottompadding@inner{0mm}%
    \def\tubs@gb@leftpadding@inner{0mm}
    \def\tubs@gb@rightpadding@inner{0mm}
  \or% Alle vertikalen Innenabstände auf 0
    \def\tubs@gb@toppadding@inner{0mm}%
    \def\tubs@gb@bottompadding@inner{0mm}%
  \or% Alle horizontalen Innenabstände auf 0
    \def\tubs@gb@leftpadding@inner{0mm}
    \def\tubs@gb@rightpadding@inner{0mm}
  \fi
}
%    \end{macrocode}
%    \end{key}
%
%    \begin{key}{gaussbox}{padding}
% Setzt Padding der gaussbox
%    \begin{macrocode}
\define@choicekey{gaussbox}{padding}[\val\nr]{%
        default,minimal,none}{%
  \ifcase\nr\relax% default
  \or% Abstände normal minimal
    \setkeys{gaussbox}{outerpadding=none}%
    \setkeys{gaussbox}{innerpadding=columnsep}%
  \or% Alle Abstände auf 0
    \setkeys{gaussbox}{outerpadding=none}%
    \setkeys{gaussbox}{innerpadding=none}%
  \fi
}
%    \end{macrocode}
%    \end{key}
%
%
%%%%%%%%%%%%%%%%%%%%%%%%%%%%%%%%%%%%%%%%%%%%%%%%%%%%%%%%%%%%%%%%%%%%%%%%%%%%%%%%
% \subsubsection{Box}
%%%%%%%%%%%%%%%%%%%%%%%%%%%%%%%%%%%%%%%%%%%%%%%%%%%%%%%%%%%%%%%%%%%%%%%%%%%%%%%%
%
%
%    \begin{macro}{\tb@gb@setorig}
% Berechnet den Ursprung des von gaussbox benutzten Koordinatensystems (neu).\\
% Der Ursprung wird dabei horiz. auf den Beginn des Textbereichs gesetzt,
% um mit |\TPHorizModule| das Spaltenlayout realisieren zu können.
%    \begin{macrocode}
\providecommand\tb@gb@setorig{%
  \ifthenelse{\boolean{tubsbox@bottomsender}}{%
    \if@tubs@oddpage{%
      \textblockorigin{\tubsborderwidth+\tubspage@bcor}{%
        \tubsborderwidth}%
    }{%
      \textblockorigin{\tubsborderwidth}{%
        \tubsborderwidth}%
    }%
  }{%
    \if@tubs@oddpage{%
      \textblockorigin{\tubsborderwidth+\tubspage@bcor}{%
        \tubssenderheight}%
    }{%
      \textblockorigin{\tubsborderwidth}{%
        \tubssenderheight}%
    }
  }
}
%    \end{macrocode}
%    \end{macro}
%
%
% Lege Rastermaße fest.
% |\TPHorizModule| entspricht der Breite des Kommunikationsbereichs, abzüglich
% der Breite des Rahmens. TODO: needs fix for bcor??
%    \begin{macrocode}
\setlength{\TPHorizModule}{%
  (\paperwidth-2\tubsborderwidth-\tubspage@bcor)*\ratio{1mm}{\tubspage@xsegments mm}}
\setlength{\TPVertModule}{%
  (\tubscommunicationheight)*\ratio{1mm}{\value{tubspage@gausssum} mm}}
%    \end{macrocode}
%
%    \begin{macro}{\tubs@gb@toppadding}
%    \begin{macro}{\tubs@gb@toppadding}
%    \begin{macro}{\tubs@gb@toppadding}
%    \begin{macro}{\tubs@gb@toppadding}
% Innenabstände der Box, d.h. Abstand zwischen Boxrand und Textbereich
%    \begin{macrocode}
\newlength{\tubs@gb@toppadding}
\newlength{\tubs@gb@bottompadding}
\newlength{\tubs@gb@leftpadding}
\newlength{\tubs@gb@rightpadding}
%    \end{macrocode}
%    \end{macro}\end{macro}\end{macro}\end{macro}
%
%    \begin{macro}{\tubs@gb@storebox}
% Savebox zum Speichern der Box-Inhalte.
%    \begin{macrocode}
\newsavebox{\tubs@gb@storebox}
%    \end{macrocode}
%    \end{macro}
%
%    \begin{macro}{tubs@gb@lastelement}
%    \begin{macro}{tubs@gb@calcypos}
%    \begin{macro}{tubs@gb@calcheight}
% Definition einiger benötigter Counter.
%    \begin{macrocode}
\newcounter{tubs@gb@lastelement}
\newcounter{tubs@gb@calcypos}
\newcounter{tubs@gb@calcheight}
%    \end{macrocode}
%    \end{macro}\end{macro}\end{macro}
%
%    \begin{macro}{gaussbox}
% \oarg{options}\marg{xpos}\marg{ypos}\marg{width}\marg{height}\par
% \begin{tabularx}{\textwidth}{lX}
%   |options| & Siehe Abschnit \ref{sec:options}  \\
%   |xpos|    & Horizontaler Startpunt der Box,
%               gemessen im Spalten-Raster,
%               Standard-Wertebereich: [1-6]  \\
%   |ypos|    & Vertikaler Startpunkt der Box,
%               gemessen im Gauß-Raster
%               Standard-Wertebereich: [1-6] (Querformat), [1-8] (Hochformat)\\
%   |width|   & Breite der Box in Spalten,
%               Wertebereich: [1-6] \\
%   |height|  & Höhre der Box in Gaußraster-Elementen,
%               Wertebereich: [1-6] / [1-8]
% \end{tabularx}
%    \begin{macrocode}
\newenvironment{gaussbox}[5][]{%
% Innere Padding-Werte (durch Optionen anpassbar)
\def\tubs@gb@toppadding@inner{\tubsborderwidth}%
\def\tubs@gb@bottompadding@inner{\tubsborderwidth}%
\def\tubs@gb@leftpadding@inner{0.5\tubscolumnsep}%
\def\tubs@gb@rightpadding@inner{0.5\tubscolumnsep}%
% Äußere Padding-Werte (durch Optionen anpassbar)
\def\tubs@gb@toppadding@outer{\tubsborderwidth}%
\def\tubs@gb@bottompadding@outer{\tubsborderwidth}%
\def\tubs@gb@leftpadding@outer{\tubsborderwidth}%
\def\tubs@gb@rightpadding@outer{\tubsborderwidth}%
%
\setkeys*{tubsbox}{#1}%
\setrmkeys{gaussbox}%
\tb@gb@setorig
%    \end{macrocode}
% Berechnung der linken und rechten Ränder
%    \begin{macrocode}
% Berechne Positionen
\ifnum#2=1% Segment beginnt bei erster column?
  \setlength\tubs@gb@leftpadding{\tubs@gb@leftpadding@outer}%
\else
  \setlength\tubs@gb@leftpadding{\tubs@gb@leftpadding@inner}%
\fi
%
\setcounter{tubs@gb@lastelement}{#2-1}%
\edef\tubs@gb@xpos{\thetubs@gb@lastelement}%TODO: etwas unschön...
\addtocounter{tubs@gb@lastelement}{#4}%
\ifnum\value{tubs@gb@lastelement}=\tubspage@xsegments% Segment endet bei letzter column?
  \setlength{\tubs@gb@rightpadding}{\tubs@gb@rightpadding@outer}%
\else
  \setlength\tubs@gb@rightpadding{\tubs@gb@rightpadding@inner}%
\fi
%    \end{macrocode}
% Berechnung der oberen und unteren Ränder
%    \begin{macrocode}
\ifnum#3=1%
  \setlength\tubs@gb@toppadding{\tubs@gb@toppadding@outer}% Erstes Segment?
%   \addtolength\tubs@gb@toppadding\tubs@gb@addtoppadding
\else
  \setlength\tubs@gb@toppadding{\tubs@gb@toppadding@inner}%
\fi
%
\setcounter{tubs@gb@lastelement}{#3+#5-1}%
% Prüfe, ob Werte korrekt
\ifnum\value{tubs@gb@lastelement}>\tubspage@ysegments%
  \PackageError{tubstypearea}{Invalid segment number}{}%
\fi
%
\ifnum\value{tubs@gb@lastelement}=\tubspage@ysegments% Letztes Segment?
  \setlength{\tubs@gb@bottompadding}{\tubs@gb@bottompadding@outer}%
\else
  \setlength{\tubs@gb@bottompadding}{\tubs@gb@bottompadding@inner}%
\fi
%
% Makro |\@inv@arg| wird benutzt, um Argument 'inverted' zu übergeben
\def\@inv@arg{\relax}%
\ifthenelse{\boolean{tubsbox@bottomsender}}{%
  \def\@inv@arg{inverted}%
}{}%
\calc@gauss@elementpos[\@inv@arg]{tubs@gb@calcypos}{#3}%
\def\tubs@gb@ypos{\thetubs@gb@calcypos}%
% 
\calc@gauss@elementpos[\@inv@arg]{tubs@gb@calcheight}{#3+#5}%
\addtocounter{tubs@gb@calcheight}{-\thetubs@gb@calcypos}%
% 
\def\tubs@gb@width{#4}%
\def\tubs@gb@height{\thetubs@gb@calcheight}%
%TODO: Option to set margins of boxes to 0
  \begin{lrbox}{\tubs@gb@storebox}%
  % Store parskip and parindent to restore them in minipage
  \newdimen\tubs@restore@parskip
  \newdimen\tubs@restore@parindent
  \tubs@restore@parskip\parskip
  \tubs@restore@parindent\parindent
  \begin{minipage}%
        [t]%
        [\tubs@gb@height\TPVertModule-\tubs@gb@toppadding-\tubs@gb@bottompadding]%
        {\tubs@gb@width\TPHorizModule-\tubs@gb@leftpadding-\tubs@gb@rightpadding}% boxwidth minus text padding
  % restore par-lengths, avoid initial parskip
  \parskip\tubs@restore@parskip%
  \parindent\tubs@restore@parindent%
  \vspace*{-\parskip}%
  % Standard-Version von |\includegraphics| mit imagefit-Version ersetzen, wenn keine Argumente angegeben
  \let\oldincludegraphics\includegraphics
  \renewcommand\includegraphics[2][\relax]{%
    \ifx##1\relax
      \tubs@sb@calc@autoscale{##2}{\tubs@gb@width\TPHorizModule-\tubs@gb@leftpadding-\tubs@gb@rightpadding}{\tubs@gb@height\TPVertModule-\tubs@gb@toppadding-\tubs@gb@bottompadding}%
      \expandafter\oldincludegraphics\expandafter[\@img@scale@param]{##2}%
    \else
      \oldincludegraphics[##1]{##2}%
    \fi
  }%
  \tb@topfill
  \noindent\ignorespaces
}{%
  \ignorespacesafterend
  \tb@bottomfill
  \par%
  \end{minipage}%
  \end{lrbox}%
  \setlength{\fboxsep}{0cm}% needed here??
  \begin{textblock}{\tubs@gb@width}(\tubs@gb@xpos,\tubs@gb@ypos)%
  \vspace*{-\parskip}%
  \vspace*{\tubs@gb@toppadding}%
  \newdimen\temp@
  \newdimen\temp@b
  \setlength\temp@{\tubs@gb@height\TPVertModule}%
  \setlength\temp@b{\tubs@gb@width\TPHorizModule}%
  \noindent
  \fboxsep0mm%
  \tubsbox@framebox{\tubsbox@colorbox{% inner paddings are set *only* here!
    \hbox to \temp@b{%
      \hspace*{\tubs@gb@leftpadding}%
      \usebox{\tubs@gb@storebox}%
      \hspace*{\tubs@gb@rightpadding}%
    }%
  }}%
  \end{textblock}%
}
%    \end{macrocode}
%    \end{macro}
%
%
%%%%%%%%%%%%%%%%%%%%%%%%%%%%%%%%%%%%%%%%%%%%%%%%%%%%%%%%%%%%%%%%%%%%%%%%%%%%%%%%
% \subsection{Modul-System}
%%%%%%%%%%%%%%%%%%%%%%%%%%%%%%%%%%%%%%%%%%%%%%%%%%%%%%%%%%%%%%%%%%%%%%%%%%%%%%%%
%
%
%    \begin{macro}{\tb@sb@setorig}
% Berechnet den Ursprung des von modulebox benutzten Koordinatensystems (neu).
%    \begin{macrocode}
\providecommand\tb@sb@setorig{%
  \ifthenelse{\boolean{tubsbox@bottomsender}}{%
    \if@tubs@oddpage{%
      \textblockorigin{\tubsborderwidth+\tubspage@bcor}{%
        \tubsborderwidth}%
    }{%
      \textblockorigin{\tubsborderwidth}{%
        \tubsborderwidth}%
    }%
  }{%
    \if@tubs@oddpage{%
      \textblockorigin{\tubsborderwidth+\tubspage@bcor}{%
        0.1429\paperheight}%
    }{%
      \textblockorigin{\tubsborderwidth}{%
        0.1429\paperheight}% TODO: Kompatibilität mit Querformat
    }
  }
}
%    \end{macrocode}
%    \end{macro}
%
%
%    \begin{macrocode}
\newsavebox{\tubs@sb@storebox}
%    \end{macrocode}
%
%    \begin{macrocode}
\newsavebox{\tubs@sb@bgimagebox}
%    \end{macrocode}
%
%    \begin{macro}{tubs@sb@elementcount}
% Counter zählt Anzahl Zeilen für Box-Layout.
%    \begin{macrocode}
\newcounter{tubs@sb@elementcount}
%    \end{macrocode}
%    \end{macro}
%
%    \begin{macro}{tubs@sb@xcount}
% Counter zählt Anzahl Xen
%    \begin{macrocode}
\newcounter{tubs@sb@xcount}
%    \end{macrocode}
%    \end{macro}
%
%    \begin{macro}{\tubs@sb@xfreespace}
% Länge speichert aktuellen freien Platz für Xe
%    \begin{macrocode}
\newlength{\tubs@sb@xfreespace}
%    \end{macrocode}
%    \end{macro}
%
%    \begin{macro}{\tubs@sb@xlength}
% Länge eines X-Elements
%    \begin{macrocode}
\newlength{\tubs@sb@xlength}
%    \end{macrocode}
%    \end{macro}
%
%    \begin{macro}{tubs@cnt}
% Counter für forloop-Schleife
%    \begin{macrocode}
\newcounter{tubs@cnt}
%    \end{macrocode}
%    \end{macro}
%
% Boolean für Hintergrundbild
%    \begin{macrocode}
\newboolean{modulebox@showbgimage}
\setboolean{modulebox@showbgimage}{false}
\newcommand{\@bgimage@to@insert}{dummy}
%    \end{macrocode}
%
% \subsubsection{Optionen}
%
%    \begin{key}{modulebox}{bgimage}
% Setze Hintergrundbild.
%    \begin{macrocode}
\define@key{modulebox}{bgimage}[none]{%
  \ifthenelse{\equal{#1}{none}}{%
    \setboolean{modulebox@showbgimage}{false}%
  }{%
    \setkeys{tubsbox}{bgcolor=none}%TODO?
    \renewcommand{\@bgimage@to@insert}{#1}%
    \setboolean{modulebox@showbgimage}{true}%
  }
}
%    \end{macrocode}
%    \end{key}
%
%
%    \begin{key}{modulebox}{imagefit}
% Einstellung der Bildeinpassung in den Darstellungsbereich.\par
% \noindent\begin{tabularx}{\textwidth}{lX}
%   |scaled| & Bild wird horiz. und vert. passend skaliert\\
%   |autoclip| & Bild wird entweder horiz. skaliert und vert. beschnitten
%     oder vert. skaliert und horiz. beschnitten. Jeweils so, dass der
%     Darstellungsbereich komplett ausgefüllt wird\\
%   |clipx| & Bild wird vert. skaliert und horiz. beschnitten\\
%   |clipy| & Bild wird horiz. skaliert und vert. beschnitten\\
% \end{tabularx}
%
% NOTE: |crop|-Options are only kept for compatiblity reasons\par
% TODO: rename internal cmds\\
% TODO: add vclip, hclip
%    \begin{macrocode}
\define@choicekey{tubsbox}{imagefit}[\val\nr]{%
        default,scaled,autoclip,cropped,hclip,cropx,vclip,cropy,fitheight,fitwidth}{%
  \ifcase\nr\relax% default
    \PackageInfo{tubsstyle}{%
      Option 'imagefit' not set. Using standard value 'cropped'.
    }
    \renewcommand{\tubs@sb@imagefit}{cropped}
  \or% scaled
    \renewcommand{\tubs@sb@imagefit}{scaled}
  \or% autoclip
    \renewcommand{\tubs@sb@imagefit}{cropped}
  \or% cropped
    \renewcommand{\tubs@sb@imagefit}{cropped}
  \or% clipx
    \renewcommand{\tubs@sb@imagefit}{cropx}
  \or% cropx
    \renewcommand{\tubs@sb@imagefit}{cropx}
  \or% clipy
    \renewcommand{\tubs@sb@imagefit}{cropy}
  \or% cropy
    \renewcommand{\tubs@sb@imagefit}{cropy}
  \or% fitheight
    \renewcommand{\tubs@sb@imagefit}{cropx}
  \or% fitwidth
    \renewcommand{\tubs@sb@imagefit}{cropy}
  \fi
}
%    \end{macrocode}
%    \end{key}
%
%
%    \begin{macrocode}
\newcommand\tubs@sb@imagefit{cropped}
\newcommand\@img@scale@param{}%
%    \end{macrocode}
%
%    \begin{macro}{\tubs@sb@calc@autoscale}
% Setzt |\@img@scale@param| als Parameter für |\includegraphics|\par
% \marg{imagefile}\marg{viewwidth}\marg{viewheight}
% TODO: generalize with other packages?
%    \begin{macrocode}
\newcommand\tubs@sb@calc@autoscale[3]{%
  % Temporäre Längen für die Berechnung
  \newdimen{\@sb@image@xorig}
  \newdimen{\@sb@image@xcalc}
  \newdimen{\@sb@image@yorig}
  \newdimen{\@sb@image@ycalc}
  % store dimension arguments (width, height)
  \newdimen\@view@width%
  \newdimen\@view@height%
  \setlength\@view@width{#2}%
  \setlength\@view@height{#3}%
  \ifthenelse{\equal{\tubs@sb@imagefit}{scaled}}{%
    \renewcommand\@img@scale@param{clip,%
        height=\@view@height,
        width=\@view@width}
  }{\ifthenelse{\equal{\tubs@sb@imagefit}{cropped}}{%
      % Ermittelt, ob das Bild an den Seiten oder oben und unten beschnitten
      % werden muss, um in den Darstellungsbereich zu passen
      % Dazu wird die Höhe des auf korrekte Breite skalierten Bildes
      % mit der Höhe des Darstellungsbereichs verglichen und entsprechend
      % eine crop-Option gesetzt.
      \settoheight{\@sb@image@ycalc}{%
        \includegraphics[clip,width=\@view@width]{#1}}
      \ifthenelse{\lengthtest{\@sb@image@ycalc>\@view@height}}{%
        \renewcommand{\tubs@sb@imagefit}{cropy}
      }{%
        \renewcommand{\tubs@sb@imagefit}{cropx}
      }
    }{}
    \ifthenelse{\equal{\tubs@sb@imagefit}{cropy}}{%
      % Berechne abzuschneidende Ränder (oben+unten)
      % Dazu wird die Differenz zwischen Darstellungsbereich und Höhe des
      % korrekt auf die Breite skalierten Bildes berechnet und mit dem
      % ermittelten Skalierungsfaktor multipliziert, sowie durch 2 geteilt.
      % Das Ergebnis wir dann einmal am oberen und einmal am unteren Teil
      % des (Original-)Bildes mit Hilfe der 'trim'-Option abgeschnitten.
      \settoheight{\@sb@image@yorig}{%
        \includegraphics[clip]{#1}}%
      \settoheight{\@sb@image@ycalc}{%
        \includegraphics[clip,width=\@view@width]{#1}}%
      \setlength{\@sb@image@ycalc}{(\@sb@image@ycalc-\@view@height)*\ratio{\@sb@image@yorig}{\@sb@image@ycalc}}
      \setlength{\@sb@image@ycalc}{0.5\@sb@image@ycalc}%
      \renewcommand\@img@scale@param{clip,%
          width=\@view@width,
          trim=0pt {\@sb@image@ycalc} 0pt {\@sb@image@ycalc}}%
    }{\ifthenelse{\equal{\tubs@sb@imagefit}{cropx}}{%
      \settowidth{\@sb@image@xorig}{%
        \includegraphics[clip]{#1}}%
      \settowidth{\@sb@image@xcalc}{%
        \includegraphics[clip,height=\@view@height]{#1}}%
      \setlength{\@sb@image@xcalc}{(\@sb@image@xcalc-(\@view@width))*\ratio{\@sb@image@xorig}{\@sb@image@xcalc}}
      \setlength{\@sb@image@xcalc}{0.5\@sb@image@xcalc}%
      \renewcommand\@img@scale@param{clip,%
          height=\@view@height,
          trim={\@sb@image@xcalc} 0pt {\@sb@image@xcalc} 0pt}%
    }{\ifthenelse{\equal{\tubs@sb@imagefit}{keepsize}}{%
      \renewcommand\@img@scale@param{}%
    }{}}}%
  }%
}
%    \end{macrocode}
%    \end{macro}
%
%
%    \begin{macro}{\tubs@sb@setelements}
% \marg{list}\par
% Erwartet als Parameter eine komma-getrennte Liste, deren Elemente
% entweder Längen oder der Buchstabe X ist.
% Elemente mit Buchstaben X teilen den restlichen zur Verfügung stehenden Platz
% gleichmäßig untereinander auf.
%    \begin{macrocode}
\newcommand\tubs@sb@setelements[2]{%
  \setcounter{tubs@sb@elementcount}{0}
  \setcounter{tubs@sb@xcount}{0}
  % Iteriere über Liste
  \let\do\tubs@sb@parsenext
  \docsvlist{#1}
  % X-Länge setzen
  \setlength\tubs@sb@xlength{\tubs@sb@xfreespace/\thetubs@sb@xcount}
  % Iteriere über alle Zeilen und ersetze X-Zeilen durch die entsprechende Länge
  \stepcounter{tubs@sb@elementcount}% only for forloop!
  \forloop[1]{tubs@cnt}{1}{\value{tubs@cnt}<\thetubs@sb@elementcount}{%
    \expandafter\def\expandafter\tubs@sb@current@row\expandafter{%
      \csname tubs@sb@tmplength@\thetubs@cnt\endcsname%
    }
    % Schreibe nun gespeicherte Längen und X-Längen in endgültiges Makro
    \ifthenelse{\equal{\tubs@sb@current@row}{X}}{%
      \writetubslog{replace X with \the\tubs@sb@xlength}%
      \expandafter\edef\csname #2\thetubs@cnt\endcsname{%
        \the\tubs@sb@xlength%
      }%
    }{%
      \expandafter\edef\csname #2\thetubs@cnt\endcsname{%
        \tubs@sb@current@row%
      }%
    }
  }
  \addtocounter{tubs@sb@elementcount}{-1}% only for forloop!
}
%    \end{macrocode}
%    \end{macro}
%
%
%    \begin{macro}{\tubs@sb@setrows}
% \marg{element}\par
% Prüft, ob das übergebene Element ein X ist, ansonsten wird davon
% ausgegangen, dass es sich um eine Länge handelt und es wird versucht
% diese zu speichern.
% TODO: Optionen 'XX', 'XXX', um Verhältnisse wiederzugeben\\
%       X : XX : XXX = 1 : 2 : 3
%    \begin{macrocode}
\newcommand*{\tubs@sb@parsenext}[1]{%
  \stepcounter{tubs@sb@elementcount}
  \ifthenelse{\equal{#1}{X}}{% X
    \stepcounter{tubs@sb@xcount}%
    \expandafter\def\csname tubs@sb@tmplength@\thetubs@sb@elementcount\endcsname{X}
  }{% Laenge
    % Ziehe Länge von xfreespace ab und speichere sie.
    \addtolength\tubs@sb@xfreespace{-#1}
    \expandafter\def\csname% 
      tubs@sb@tmplength@\thetubs@sb@elementcount\endcsname{#1}%
  }
  % Ziehe zusätzlich Stegbreite ab.
  \addtolength{\tubs@sb@xfreespace}{-0.5\tubsborderwidth}
}
%    \end{macrocode}
%    \end{macro}
%
%    \begin{environment}{modcontainer}
% \oarg{options}\marg{rows}\par
% Extra-Umgebung für wissenschaftliche Poster mit Modulsystem.
% Als Argument ist die gewünschte Anzahl an Modul-Zeilen anzugeben.
% Dies geschieht mit einer durch Kommas getrennten Liste.
% Jedes Element muss dabei entweder eine Längenangabe oder der Buchstabe 'X'
% sein. Jedes Element steht für eine Modul-Zeile und legt gleichzeitig deren
% Höhe fest. Der Buchstabe 'X' sorgt dabei dafür, dass der restliche zur
% Verfügung stehende Platz gleichmäßig auf X-Elemente aufgeteilt wird.
% Die Verwendung von mindestens einem 'X' pro Layout ist sinnvoll, da sonst
% der Darstellungsraum mühsam von Hand auf die richtige Größe gebracht werden 
% muss.\par
% Jede angelegte Modul-Zeile muss dann mit der Umgebung |modrow| mit Inhalt
% gefüllt werden.\par
% Wertet Argument-Liste aus und speichert ermittelte Längen in Makros mit Präfix
% |tubs@sb@rowlength|.
%    \begin{macrocode}
\newenvironment{modcontainer}[2][]{%
  \writetubslog{modcontainer: #1, #2}%
%   \modulesetup[bgcolor=tuWhite,#1]%
  \setkeys{tubsbox.sty}{#1}%
  \setkeys{tubsbox}{t,bgcolor=tuWhite,fgcolor=tuBlack}{}% TODO: check!
  % reset modrow counter
  \setcounter{tubs@sb@current@row}{0}%
  \setlength{\tubs@sb@xfreespace}{\textheight+1.5\tubsborderwidth}%
  \writetubslog{page: xfree: \the\tubs@sb@xfreespace}%
  \tubs@sb@setelements{#2}{tubs@sb@rowlength@}%
}{%
}
%    \end{macrocode}
%    \end{environment}
%
%    \begin{macro}{tubs@sb@current@col}
%    \begin{macro}{tubs@sb@current@row}
%    \begin{macro}{tubs@sb@current@subrow}
% Zähler für die jeweils aktuell gesetzte Modul-Zeile/-Spalte.
%    \begin{macrocode}
\newcounter{tubs@sb@current@col} % aktuelle Spalte
\newcounter{tubs@sb@current@row} % aktuelle Zeile
\newcounter{tubs@sb@current@subrow}  % aktuelle Subspalte
%    \end{macrocode}
%    \end{macro}\end{macro}\end{macro}
%
%    \begin{macrocode}
\newboolean{tubs@sb@borderless}\setboolean{tubs@sb@borderless}{false}
%    \end{macrocode}
%
% \subsubsection{Modulzeile (modrow)}
%
%    \begin{environment}{@modrow}
% Implementierung von |modrow|.
%    \begin{macrocode}
\newenvironment{@modrow}[1][X]{%
  % setze keys
  \setkeys*{tubsbox}{#1}%
  \setrmkeys*{modulebox}%
  \edef\@remaining@keys{\XKV@rm}%
  % Ersetze mit X, falls keine Länge angegeben
  \ifthenelse{\equal{\@remaining@keys}{}}{%
    \edef\@remaining@keys{X}}{}%
  % reset modcol counter
  \setcounter{tubs@sb@current@col}{0}%
  % Setze Länge auf Textbreite + linker und rechter Rand + Stegkorrektur
  \setlength{\tubs@sb@xfreespace}{\textwidth+1.5\tubsborderwidth}%
  % Parse Komma-Liste und speichere Längen mit Präfix 'tubs@sb@collength@'
  \expandafter\tubs@sb@setelements\expandafter{\@remaining@keys}{tubs@sb@collength@}%
  \stepcounter{tubs@sb@current@row}%
  % Wenn nur eine Spalte, dann direkt modulebox setzen
  \ifthenelse{\expandafter\equal\expandafter{\@remaining@keys}{X}}{%
    \stepcounter{tubs@sb@current@col}%
    \expandafter\edef\expandafter\@argI\expandafter{%
      \csname tubs@sb@rowlength@\thetubs@sb@current@row\endcsname}%
    \expandafter\edef\expandafter\@argII\expandafter{%
      \csname tubs@sb@collength@\thetubs@sb@current@col\endcsname}%
    \def\tubs@sb@box@cmd{\@modulebox{\@argI}{\@argII}}%
    \def\tubs@sb@box@endcmd{\end@modulebox}%
  }{%
    \def\tubs@sb@box@cmd{\relax}%
    \def\tubs@sb@box@endcmd{\relax}%
  }
  \writetubslog{modrow: xfree: \the\tubs@sb@xfreespace}%
  \tubs@sb@box@cmd%
}{%
  \tubs@sb@box@endcmd%
}
%    \end{macrocode}
%    \end{environment}
%

%    \begin{environment}{modrow}
%    \begin{environment}{modrow*}
% \oarg{cols}\par
% Umgebung mit der Modulzeilen mit Inhalt gefüllt oder
% zusätzliche Spaltenbereiche angelegt werden können.
% Das optionale Argument \emph{rows} hat dabei dieselbe Funktion wie
% in |modcontainer| beschrieben, allerdings definiert sie nun
% \emph{Spalten}breiten.\par
% Wenn nur eine einzelne (X)-Spalte definiert wurde, kann der Inhalt
% direkt in die Umgebung geschrieben werden,
% ansonsten müssen die einzelnen Spalteninhalte mittels der Umgebung
% |modcol| definiert werden.\par
% Wertet Argument-Liste aus und speichert ermittelte Längen in Makros
% mit Präfix |tubs@sb@collength@|. Wenn nur eine einzelne (X)-Spalte
% definiert wurde, bzw. das optionale Argument leer gelassen, so wird
% direkt eine modulebox erstellt.
%    \begin{macrocode}
\newcommand{\@@modrow}{%
  \@ifnextchar[{%
  }{%
  }%
  \@modrow%
}
%
\newenvironment{modrow}{%
  \setboolean{tubs@sb@borderless}{false}%
  \@@modrow%
}{%
  \end@modrow%
}
%    \end{macrocode}
%    \begin{macrocode}
\newenvironment{modrow*}{%
  \setboolean{tubs@sb@borderless}{true}%
  \@@modrow%
}{%
  \end@modrow%
}
%    \end{macrocode}
%    \end{environment}\end{environment}
%
%
% \subsubsection{Modulspalte (modcol)}
%
%
%    \begin{environment}{@modcol}
% Implementierung von |modcol|.
%    \begin{macrocode}
\newenvironment{@modcol}[1][]{%
  % setze keys
%   \modulebox@setdefault@opts
  \setkeys*{tubsbox}{#1}
  \setrmkeys*{modulebox}
  \edef\@remaining@keys{\XKV@rm}%
  % Ersetze mit X, falls keine Länge angegeben
  \ifthenelse{\equal{\@remaining@keys}{}}{%
    \edef\@remaining@keys{X}}{}%
  % reset modsubrow counter
  \setcounter{tubs@sb@current@subrow}{0}%
  % Setze Länge auf Zeilenhöhe + Trennstrich
  \expandafter\setlength\expandafter\tubs@sb@xfreespace\expandafter{\csname tubs@sb@rowlength@\thetubs@sb@current@row\endcsname+0.5\tubsborderwidth}%
  % Parse Komma-Liste und speichere Längen mit Präfix 'tubs@sb@subrowlength@'
  \expandafter\tubs@sb@setelements\expandafter{\@remaining@keys}{tubs@sb@subrowlength@}%
  \stepcounter{tubs@sb@current@col}%
  % Wenn nur eine Spalte, dann direkt modulebox setzen
  \ifthenelse{\expandafter\equal\expandafter{\@remaining@keys}{X}}{%
    \stepcounter{tubs@sb@current@subrow}%
    \expandafter\edef\expandafter\@argI\expandafter{%
      \csname tubs@sb@subrowlength@\thetubs@sb@current@subrow\endcsname}%
    \expandafter\edef\expandafter\@argII\expandafter{%
      \csname tubs@sb@collength@\thetubs@sb@current@col\endcsname}%
    \def\tubs@sb@box@cmd{\@modulebox{\@argI}{\@argII}}%
    \def\tubs@sb@box@endcmd{\end@modulebox}%
  }{%
    \def\tubs@sb@box@cmd{\relax}%
    \def\tubs@sb@box@endcmd{\relax}%
  }
  \tubs@sb@box@cmd%
}{%
  \tubs@sb@box@endcmd%
}
%    \end{macrocode}
%    \end{environment}
%
%    \begin{environment}{modcol}
%    \begin{environment}{modcol*}
% \oarg{subrows,settings}\par
% Umgebung mit der Modulspalten mit Inhalt gefüllt oder
% zusätzliche Unterzeilen angelegt werden können.\par
% Die Sternchen-Version sorgt dafür, dass der innere Rand weggelassen
% wird, wie dies zum Beispiel für das Einfügen von Grafiken sinnvoll ist.
%    \begin{macrocode}
\newenvironment{modcol}{%
  \setboolean{tubs@sb@borderless}{false}%
  \@modcol%
}{%
  \end@modcol%
}
%    \end{macrocode}
%    \begin{macrocode}
\newenvironment{modcol*}{%
  \setboolean{tubs@sb@borderless}{true}%
  \@modcol%
}{%
  \end@modcol%
}
%    \end{macrocode}
%    \end{environment}\end{environment}
%
%
% \subsubsection{Modulunterzeile (modsubrow)}
%
%
%    \begin{environment}{modsubrow}
%    \begin{environment}{modsubrow*}
% Umgebung mit der Modul-Unterzeilen mit Inhalt gefüllt werden können.
%    \begin{macrocode}
\newenvironment{modsubrow}{%
  \setboolean{tubs@sb@borderless}{false}%
  \@modsubrow%
}{%
  \end@modsubrow%
}
%    \end{macrocode}
%
%    \begin{macrocode}
\newenvironment{modsubrow*}{%
  \setboolean{tubs@sb@borderless}{true}%
  \@modsubrow%
}{%
  \end@modsubrow%
}
%    \end{macrocode}
%    \end{environment}\end{environment}
%
%    \begin{environment}{@modsubrow}
% Implementierung von |modsubrow|.
%    \begin{macrocode}
\newenvironment{@modsubrow}[1][]{%
  % setze keys
  \setkeys*{tubsbox}{#1}%
  \setrmkeys{modulebox}%
  \stepcounter{tubs@sb@current@subrow}%
  \expandafter\edef\expandafter\@oarg\expandafter{%
    \csname tubs@sb@rowlength@\thetubs@sb@current@row\endcsname}%
  \expandafter\edef\expandafter\@argI\expandafter{%
    \csname tubs@sb@subrowlength@\thetubs@sb@current@subrow\endcsname}%
  \expandafter\edef\expandafter\@argII\expandafter{%
    \csname tubs@sb@collength@\thetubs@sb@current@col\endcsname}%
  \@modulebox[][\@oarg]{\@argI}{\@argII}%
}{%
  \end@modulebox%
}
%    \end{macrocode}
%    \end{environment}
%
%
% \subsubsection{Box}
%
%
% Benutzer-Längen für Höhe und Breite des akutell verwendeten Moduls
%    \begin{macrocode}
\newlength{\modulewidth}
\newlength{\moduleheight}
\newdimen\@box@tmp% Generelle temp-Var.
%    \end{macrocode}
%
%    \begin{macro}{\tubs@sb@box@xpos}
%    \begin{macro}{\tubs@sb@box@ypos}
%    \begin{macro}{\tubs@sb@box@rowheight}
%    \begin{macro}{\tubs@sb@box@splitrow}
% Von |@modulebox| verwendete Längen.
%    \begin{macrocode}
\newlength{\tubs@sb@box@xpos}% x-Position
\newlength{\tubs@sb@box@ypos}% y-Position
\newlength{\tubs@sb@box@rowheight}% Höhe einer Zeile, sofern als opt. Arg. gegeben
\newlength{\tubs@sb@box@splitrow}% Speichert benutze Länge von Splitrow für späteren Reset
\newlength{\tubs@sb@padding}% Abstand Text zu Boxrand
%    \end{macrocode}
%    \end{macro}\end{macro}\end{macro}\end{macro}
%
%    \begin{macro}{@modulebox}
% \oarg{rowheight}\marg{height}\marg{width}\par
% Box für Inhalt auf wissenschaftlichen Postern (Modulsystem).
% Dies ist unabhängig vom Gaußraster und hat schmalere Ränder als die gaussbox.
% Das optionale Argument rowheight muss gesetzt werden, wenn in einer Zeile mehrere Boxen untergebracht werden sollen.
%    \begin{macrocode}
\newenvironment{@@modulebox}[3][\relax]{%
  % Argumente sichern
  \ifx#1\relax%
    \def\@current@rowheight{#2}%
  \else%
    \def\@current@rowheight{#1}%
  \fi%
  \def\@current@height{(#2)}%
  \def\@current@width{(#3)}% Breite der weißen Box, ohne Ränder!
  \ifthenelse{\boolean{tubs@sb@borderless}}{%
    \setlength\tubs@sb@padding{0mm}%
  }{%
    \setlength\tubs@sb@padding{2\tubspage@module@padding}%
  }%
  \newdimen\@temp@width
  \newdimen\@temp@height
  \setlength\@temp@height{#2}%
  \@temp@width#3%
  \tb@sb@setorig%
  \newdimen\tubs@restore@parskip%
  \newdimen\tubs@restore@parindent%
  \tubs@restore@parskip\parskip%
  \tubs@restore@parindent\parindent%
  % Speichere Inhalt in minipage-Box.
  \begin{lrbox}{\tubs@sb@storebox}%
    \begin{minipage}[t][\@current@height-\tubs@sb@padding]{\@current@width-\tubs@sb@padding}%
      \setlength{\moduleheight}{\@current@height}%
      \setlength{\modulewidth}{\@current@width}%
      \parskip\tubs@restore@parskip%
      \parindent\tubs@restore@parindent%
      \tb@topfill
      \noindent\ignorespaces
}{%
      \ignorespacesafterend
      \tb@bottomfill
    \end{minipage}%
  \end{lrbox}%
  \ifthenelse{\boolean{modulebox@showbgimage}}{%
    \tubs@sb@calc@autoscale{\@bgimage@to@insert}%
          {\@current@width}%
          {\@current@height}%
    \begin{lrbox}{\tubs@sb@bgimagebox}%
      \expandafter\includegraphics\expandafter[\@img@scale@param]{%
              \@bgimage@to@insert}%
    \end{lrbox}%
  }{}%
  % Setze Box mit Hintergrundbild/-farbe
  \begin{textblock*}{\@current@width}(\tubs@sb@box@xpos,\tubs@sb@box@ypos)%
    \fboxsep0mm%
    \tubsbox@framebox{\tubsbox@colorbox{%
      \vbox to \@temp@height{\usebox{\tubs@sb@bgimagebox}}%
    }}%
  \end{textblock*}%
  % Definiere und platziere Textblock mit gespeichertem Inhalt.
  \begin{textblock*}{\@current@width}(\tubs@sb@box@xpos,\tubs@sb@box@ypos)%
    \fboxsep0mm%
    \vbox to \@temp@height{%
      \vspace*{0.5\tubs@sb@padding}%
      \hbox to \@temp@width{%
        \hspace*{0.5\tubs@sb@padding}%
        \usebox{\tubs@sb@storebox}%
        \hspace*{0.5\tubs@sb@padding}%
      }%
      \vspace*{0.5\tubs@sb@padding}%
    }%
  \end{textblock*}%
  \writetubslog{modulebox @(\the\tubs@sb@box@xpos, \the\tubs@sb@box@ypos)}%
  \if@tubs@dbg% DEBUGGING output!!
  \begin{textblock*}{\@current@width}(\tubs@sb@box@xpos,\tubs@sb@box@ypos)%
    \colorbox{tuWhite}{\the\tubs@sb@box@xpos, \the\tubs@sb@box@ypos}%
  \end{textblock*}%
  \fi%
  % Prüfe, ob Zeile gesplittet werden soll (opt. Arg. != \relax)
  \ifthenelse{\equal{\@current@rowheight}{\relax}}{%
    \addtolength\tubs@sb@box@xpos{\@current@width+0.5\tubsborderwidth}%
  }{%
    \setlength\tubs@sb@box@rowheight{\@current@rowheight}
    \addtolength\tubs@sb@box@ypos{\@current@height+0.5\tubsborderwidth}%
    \addtolength\tubs@sb@box@splitrow{\@current@height+0.5\tubsborderwidth}%
    % Wenn Zeilenhöhe ausgefüllt, normal weiter machen...
    \ifthenelse{\lengthtest{\tubs@sb@box@splitrow > \tubs@sb@box@rowheight}}{%
      \addtolength\tubs@sb@box@xpos{\@current@width+0.5\tubsborderwidth}%
      \addtolength\tubs@sb@box@ypos{-\tubs@sb@box@splitrow}%
      \setlength\tubs@sb@box@splitrow{0mm}%
    }{}
  }
  % Wenn Zeile gefüllt ist, springen in nächste Zeile
  \setlength\@box@tmp{\textwidth+\tubsborderwidth}% Texthöhe + Rahmenhöhe
  \writetubslog{width: is \the\tubs@sb@box@xpos > \the\@box@tmp ?}%
  \ifthenelse{\lengthtest{\tubs@sb@box@xpos > \@box@tmp}}{%
    \addtolength\tubs@sb@box@ypos{\@current@rowheight+0.5\tubsborderwidth}%
    \setlength\tubs@sb@box@xpos{0mm}%
    % Wenn darüber hinaus Seite gefüllt ist, fange neue an!
    \setlength\@box@tmp{\textheight+\tubsborderwidth}% Texthöhe + Rahmenhöhe
    \writetubslog{height: is \the\tubs@sb@box@ypos > \the\@box@tmp ?}%
    \ifthenelse{\lengthtest{\tubs@sb@box@ypos > \@box@tmp}}{%
      \setlength\tubs@sb@box@ypos{0mm}%
    }{}
  }{}
  % speichere global!
  \global\tubs@sb@box@xpos=\tubs@sb@box@xpos%
  \global\tubs@sb@box@ypos=\tubs@sb@box@ypos%
  \global\tubs@sb@box@splitrow=\tubs@sb@box@splitrow%
}
% Umgebungsende
\let\end@modulebox\end@@modulebox
%
\newcommand\modulebox@setdefault@opts{%
  \setkeys{tubsbox}{bgcolor=tuWhite,fgcolor=tuRed}%TODO
}
%    \end{macrocode}
%
% Behandlung von erstem optionalen Argument (keyvals)
%    \begin{macrocode}
\def\@modulebox@oparam[#1]{%
  \setkeys*{modulebox}{#1}%
  \setkeys*{gaussbox}{#1}%
  \@@modulebox%
}
%    \end{macrocode}
%
% Test nach erstem optionalen Argument (keyvals)
%    \begin{macrocode}
\def\@modulebox{%
  \@ifnextchar[{%
    \@modulebox@oparam%
  }{%
    \@@modulebox%
  }%
}
%    \end{macrocode}
%    \end{macro}
%
%    \begin{macrocode}
%</package>
%    \end{macrocode}
%
% \Finale
\endinput
%
% Präfixe:
% tubs  - allgemeiner Präfix
% gb    - Präfix für gaussbox
% sb    - Präfix für modulebox
%