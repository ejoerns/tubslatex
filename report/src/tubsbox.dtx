
% \iffalse meta-comment
%
% Copyright (C) 2011 by Enrico Jörns
% -----------------------------------
%
% This file may be distributed and/or modified under the
% conditions of the LaTeX Project Public License, either version 1.2
% of this license or (at your option) any later version.
% The latest version of this license is in:
%
%   http://www.latex-project.org/lppl.txt
%
% and version 1.2 or later is part of all distributions of LaTeX
% version 1999/12/01 or later.
%
% \fi
%
% \CheckSum{0}
%
% \CharacterTable
%  {Upper-case    \A\B\C\D\E\F\G\H\I\J\K\L\M\N\O\P\Q\R\S\T\U\V\W\X\Y\Z
%   Lower-case    \a\b\c\d\e\f\g\h\i\j\k\l\m\n\o\p\q\r\s\t\u\v\w\x\y\z
%   Digits        \0\1\2\3\4\5\6\7\8\9
%   Exclamation   \!     Double quote  \"     Hash (number) \#
%   Dollar        \$     Percent       \%     Ampersand     \&
%   Acute accent  \'     Left paren    \(     Right paren   \)
%   Asterisk      \*     Plus          \+     Comma         \,
%   Minus         \-     Point         \.     Solidus       \/
%   Colon         \:     Semicolon     \;     Less than     \<
%   Equals        \=     Greater than  \>     Question mark \?
%   Commercial at \@     Left bracket  \[     Backslash     \\
%   Right bracket \]     Circumflex    \^     Underscore    \_
%   Grave accent  \`     Left brace    \{     Vertical bar  \|
%   Right brace   \}     Tilde         \~}
%
% \iffalse
%
%<*driver>
\documentclass{ltxdoc}
\usepackage[ngerman,english]{babel}
\usepackage[utf8]{inputenc}
\RequirePackage{xkeyval}
\usepackage[colorlinks, linkcolor=blue]{hyperref}
\EnableCrossrefs
\CodelineIndex
\RecordChanges
\begin{document}
  \DocInput{tubsbox.dtx}
\end{document}
%</driver>
% \fi
%
%
% \changes{v1.0}{ YYYY / MM / DD }{Initial version}
%
% \GetFileInfo{tubshead.sty}
%
% \DoNotIndex{ list of control sequences }
%
% \title{\textsf{tubsbox} -- 
%   box definitions for tubslatex\thanks{This document
%   corresponds to \textsf{tubsbox}~\fileversion,
%   dated \filedate.}}
% \author{Enrico Jörns \\ \texttt{e dot joerns at tu minus bs dot de}}
%
% \maketitle
%
% \begin{abstract}
%   Put text here.
% \end{abstract}
%
% \section{Introduction}
%
% Put text here.
%
% \section{Usage}
%
% \DescribeMacro{\YOURMACRO}
% Put description of |\YOURMACRO| here.
%
% \DescribeEnv{YOURENV}
% Put description of |YOURENV| here.
%
% \StopEventually{\PrintIndex}
%
% \section{Implementation}
%
%    \begin{macrocode}
%<*class>
%    \end{macrocode}
%
%
%    \begin{macrocode}
%</class>
%    \end{macrocode}
% 
%
% \subsection{Options}
% Es werden zur Zeit keine Optionen zur Verfügung gestellt
%    \begin{macrocode}
%<*option>
%    \end{macrocode}
%    \begin{macrocode}
%</option>
%    \end{macrocode}
%
% 
%    \begin{macrocode}
%<*package>
%    \end{macrocode}
%
%
%    \begin{macrocode}
\RequirePackage{ifthen}
\RequirePackage{xkeyval}
\RequirePackage{tubstypearea}
\RequirePackage[absolute]{textpos}
%    \end{macrocode}
%
%    \begin{macrocode}
\newboolean{tubsbox@bottomsender}\setboolean{tubsbox@bottomsender}{false}
%    \end{macrocode}
%
%    \begin{macrocode}
\define@key{tubsbox.sty}{sender}{%
  \ifthenelse{\equal{#1}{bottom}}{%
    \setboolean{tubsbox@bottomsender}{true}
  }{%
    \setboolean{tubsbox@bottomsender}{false}
  }
}
%
\ProcessOptionsX\relax
%    \end{macrocode}
%
%    \begin{macrocode}
\newcommand\tubsbox@setorig{%
  % TODO: this seems to be a bug, only one borderwidth should be needed
  \ifthenelse{\boolean{tubsbox@bottomsender}}{%
    \textblockorigin{2\tubspage@borderwidth}{%
      2\tubspage@borderwidth}
  }{%
    \textblockorigin{2\tubspage@borderwidth}{%
      \tubspage@senderheight+\tubspage@borderwidth}
  }
}
%    \end{macrocode}
%
%    \begin{macrocode}
\newcommand{\tubsboxsetup}[1][]{%
  \setkeys{tubsbox.sty}{#1}
  \tubsbox@setorig
}
%    \end{macrocode}
%
%    \begin{macrocode}
\tubsboxsetup % call once to init
%    \end{macrocode}
%
% Lege Rastermaße fest.
%    \begin{macrocode}
\setlength{\TPHorizModule}{%
  (\textwidth)*\ratio{1mm}{\tubspage@xsegments mm}}
\setlength{\TPVertModule}{%
  (\tubspage@communicationheight)*\ratio{1mm}{\value{tubspage@gausssum} mm}}
%    \end{macrocode}
%
%    \begin{macrocode}
\newlength{\tubsbox@leftmargin}
\newlength{\tubsbox@rightmargin}
\newlength{\tubsbox@leftsep}
\newlength{\tubsbox@rightsep}
\newlength{\tubsbox@topmargin}
\newlength{\tubsbox@toppadding}
\newlength{\tubsbox@bottommargin}
\newsavebox{\storebox}
\newcounter{tubsbox@lastelement}
\newcounter{tubsbox@calcypos}
\newcounter{tubsbox@calcheight}
%    \end{macrocode}
%
%    \begin{macro}{tubsbox}
%    \begin{macrocode}
% params:
% 1 - color
% 2 - xpos
% 3 - ypos
% 4 - width [1-6]
% 5 - height [1-6] / [1-8]
\newenvironment{tubsbox}[5][tuWhite]{%
% \tubspage@columnsep\tubspage@borderwidth
% Berechnung der linken und rechten Ränder
\ifnum#2=0
\setlength{\tubsbox@leftmargin}{\tubspage@borderwidth}
\setlength{\tubsbox@rightmargin}{0mm}
\setlength{\tubsbox@leftsep}{0mm}
% \setlength{\tubsbox@rightsep}{0.5\tubspage@columnsep}
\else
\setlength{\tubsbox@leftsep}{0.5\tubspage@columnsep}
\fi
%
\setcounter{tubsbox@lastelement}{#2}\addtocounter{tubsbox@lastelement}{#4}
\ifnum\value{tubsbox@lastelement}=\tubspage@xsegments
  % \setlength{\tubsbox@leftmargin}{0mm}
  \setlength{\tubsbox@rightmargin}{\tubspage@borderwidth}
  % \setlength{\tubsbox@leftsep}{0.5\tubspage@columnsep}
  \setlength{\tubsbox@rightsep}{0mm}
\else
  % \setlength{\tubsbox@leftmargin}{0mm}
  \setlength{\tubsbox@rightmargin}{0mm}
  \setlength{\tubsbox@rightsep}{0.5\tubspage@columnsep}
  % \setlength{\tubsbox@leftsep}{0.5\tubspage@columnsep}
\fi
% \fi
%
% Berechnung der oberen und unteren Ränder
\ifnum#3=1
  \setlength{\tubsbox@topmargin}{\tubspage@borderwidth}
  \setlength{\tubsbox@toppadding}{3\tubspage@borderwidth}
  \setlength{\tubsbox@bottommargin}{0mm}
\else
  \setlength{\tubsbox@toppadding}{\tubspage@borderwidth}
  % \addtocounter{tubsbox@lastelement}{#5}\addtocounter{tubsbox@lastelement}{-1}
\fi
\setcounter{tubsbox@lastelement}{#3+#5-1}
\ifnum\value{tubsbox@lastelement}=\tubspage@ysegments
  % \setlength{\tubsbox@topmargin}{0mm}
  \setlength{\tubsbox@bottommargin}{\tubspage@borderwidth}
\else
  % \setlength{\tubsbox@topmargin}{0mm}
  \setlength{\tubsbox@bottommargin}{0mm}
\fi
% \fi
%
\def\bgcolor{#1}
\def\tubsbox@xpos{#2}
% Makro |\@inv@arg| wird benutzt, um Argument 'inverted' an
% |calc@gauss@elementpos| zu übergeben, falls bottomsender-Layout gewählt wurde.
\def\@inv@arg{\relax}
\ifthenelse{\boolean{tubsbox@bottomsender}}{%
  \def\@inv@arg{inverted}%
}{}
% \setcounter{tubsbox@calcypos}{%
%   \value{tubspage@gausssum}-((\tubspage@ysegments+1-#3)*(\tubspage@ysegments-#3+2))/2}
\calc@gauss@elementpos[\@inv@arg]{tubsbox@calcypos}{#3}
\def\tubsbox@ypos{\thetubsbox@calcypos}
% 
% \setcounter{tubsbox@calcheight}{%
%   \value{tubspage@gausssum}-((\tubspage@ysegments+1-#3-#5)*(\tubspage@ysegments-#3-#5+2))/2}
\calc@gauss@elementpos[\@inv@arg]{tubsbox@calcheight}{#3+#5}
\addtocounter{tubsbox@calcheight}{%
% -\value{tubspage@gausssum}+((\tubspage@ysegments+1-#3)*(\tubspage@ysegments-#3+2))/2}
  -\thetubsbox@calcypos}
% 
\def\tubsbox@width{#4}
\def\tubsbox@height{\thetubsbox@calcheight}
\begin{lrbox}{\storebox}
\begin{minipage}[t]%
  [\tubsbox@height\TPVertModule]%
  {\tubsbox@width\TPHorizModule-\tubsbox@leftsep-\tubsbox@rightsep}%
  \vspace*{\tubsbox@toppadding}%
}{%
\vspace*{\tubspage@borderwidth}%
\end{minipage}
\end{lrbox}
\setlength{\fboxsep}{0cm}%
\begin{textblock}{\tubsbox@width}(\tubsbox@xpos,\tubsbox@ypos)%
  \hspace*{\tubsbox@leftsep}%
  \hspace*{-\tubsbox@leftmargin}%
  % \vspace*{\tubsbox@topmargin}%
  \raisebox{\tubspage@borderwidth}[0cm]{%
    \fbox{%
      \hspace*{\tubsbox@leftmargin}%
      \usebox{\storebox}%
      \hspace*{\tubsbox@rightmargin}%
      }%
    }%
  \hspace*{\tubsbox@rightsep}%
\end{textblock}
}
%    \end{macrocode}
%    \end{macro}
%
%
%    \begin{macrocode}
%</package>
%    \end{macrocode}
%
% \Finale
\endinput
