\documentclass[a4paper,11pt]{tubsreprt}

\usepackage[utf8x]{inputenc}
\usepackage{xspace}
% \usepackage[T1]{fontenc}
% \usepackage{ae}
% \usepackage[a4paper]{tubsdoc}

\usepackage{scrpage2}
\usepackage{amsmath}
\usepackage{tabularx}
\usepackage{booktabs}
\usepackage{listings}
\lstset{basicstyle=\ttfamily}
\usepackage[%
  colorlinks=true,
  linkcolor=tuRed100,
  citecolor=tuGreenDark]{hyperref}
\usepackage[ngerman]{babel}
\usepackage{caption}
\usepackage{subcaption}
\usepackage{longtable}

% glossaries
\usepackage[ngerman]{translator}
%Paket laden
\usepackage[
nonumberlist, %keine Seitenzahlen anzeigen
acronym,      %ein Abkürzungsverzeichnis erstellen
toc,          %Einträge im Inhaltsverzeichnis
section]      %im Inhaltsverzeichnis auf section-Ebene erscheinen
{glossaries}
%Glossar-Befehle anschalten
\makeglossaries

% Kopiert von KOMA
\makeatletter
\providecommand\marg[1]{%
  {\ttfamily\char`\{}\meta{#1}{\ttfamily\char`\}}}
\providecommand\oarg[1]{%
  {\ttfamily[}\meta{#1}{\ttfamily]}}
\def\cmd#1{\cs{\expandafter\cmd@to@cs\string#1}}
\def\cmd@to@cs#1#2{\char\number`#2\relax}
\DeclareRobustCommand\cs[1]{\texttt{\char`\\#1}}

\newenvironment{Declaration}{%
%    \end{macrocode}
% \begin{macro}{\new@element}
%   Help macro to define new Declaration elements.
%    \begin{macrocode}
  \newcommand*{\new@element}[1]{%
    \expandafter\newcommand\expandafter*\csname X##1\endcsname{}%
    \expandafter\let\csname X##1\expandafter\endcsname
    \csname ##1\endcsname
    \expandafter\newcommand\expandafter*\csname new##1\endcsname[1]{%
%      \begingroup
%        \let\ensuremath\@firstofone
%        \let\textit\@firstofone
%        \lowercase{\def\@tempa{##1}}%
%        \pdfstringdef\@tempb{\label@base.\@tempa.####1}%
%        \xdef\@currentHref{\@tempb}%
%        \Hy@raisedlink{\hyper@anchorstart{\@currentHref}\hyper@anchorend}%
%        \label{desc:\label@base.\@tempa.####1}%
%      \endgroup
      \csname X##1\endcsname{####1}\ignorespaces
    }%
    \expandafter\let\csname ##1\expandafter\endcsname\csname new##1\endcsname
  }%
  \newcommand*{\new@xelement}[2]{%
    \expandafter\newcommand\expandafter*\csname X##1\endcsname{}%
    \expandafter\let\csname X##1\expandafter\endcsname
    \csname ##1\endcsname
    \expandafter\newcommand\expandafter*\csname new##1\endcsname[2]{%
%      \begingroup
%        \let\ensuremath\@firstofone
%        \let\textit\@firstofone
%        \lowercase{\def\@tempa{##1}}%
%        \pdfstringdef\@tempb{\label@base.\@tempa.####1.####2}%
%        \xdef\@currentHref{\@tempb}%
%        \Hy@raisedlink{\hyper@anchorstart{\@currentHref}\hyper@anchorend}%
%        \label{desc:\label@base.\@tempa.####1.####2}%
%      \endgroup
      \csname X##1\endcsname{####1}{##2{####2}}\ignorespaces
    }%
    \expandafter\let\csname ##1\expandafter\endcsname\csname new##1\endcsname
  }%
%    \end{macrocode}
%    \begin{macrocode}
  \new@element{Option}%
  \new@element{Macro}%
  \new@element{Environment}%
  \new@element{Counter}%
  \new@element{FloatStyle}%
  \new@element{PLength}%
  \new@element{Variable}%
  \new@xelement{OptionValue}{\PValue}%
%    \end{macrocode}
% \end{macro}
%    \begin{macrocode}
  \ifvmode\else\par\fi\addvspace{2\baselineskip}%
  \vspace{-\baselineskip}%
  \vspace{\z@ plus \baselineskip}%
  \noindent
  \start@Declaration
  \tabular{|l|}\hline\ignorespaces
}{%
  \\\hline\endtabular\nobreak\after@Declaration\nobreak\par\nobreak
  \vspace{1.5\baselineskip}\nobreak\vspace{-\baselineskip}\nobreak%
  \vspace{0pt minus .5\baselineskip}\nobreak%
  \aftergroup\@afterindentfalse\aftergroup\@afterheading
}
\newcommand*{\start@Declaration}{\hspace{-1em}}
\newcommand*{\after@Declaration}{}
% \begin{macro}{\Macro}
% \begin{macro}{\Option}
% \begin{macro}{\KOption}
% \begin{macro}{\OptionValue}
% \begin{macro}{\Environment}
% \begin{macro}{\Counter}
% \begin{macro}{\Length}
% \begin{macro}{\PLength}
% \begin{macro}{\FloatStyle}
% \begin{macro}{\Pagestyle}
% \begin{macro}{\Variable}
% \begin{macro}{\FontElement}
% \begin{macro}{\PName}
% \begin{macro}{\PValue}
% \begin{macro}{\Parameter}
% \begin{macro}{\OParameter}
% \begin{macro}{\AParameter}
% \begin{macro}{\PParameter}
% \begin{macro}{\POParameter}
%   \begin{description}
%   \item[\cs{Macro}] \LaTeX{} or \TeX{} macro
%   \item[\cs{Option}] class or package option
%   \item[\cs{KOption}] |\KOMAoptions| option
%   \item[\cs{Environment}] \LaTeX{} environment
%   \item[\cs{Counter}] \LaTeX{} counter
%   \item[\cs{Length}] \LaTeX{} length
%   \item[\cs{PLength}] \KOMAScript{} pseudo length
%   \item[\cs{Variable}] \KOMAScript{} variable
%   \item[\cs{FontElement}] \KOMAScript{} element that has its own font
%     selection
%   \item[\cs{PName}] name of a parameter of a macro or environment
%   \item[\cs{PValue}] value of a parameter of a macro or environment
%   \item[\cs{Parameter}] the mandatory parameter of a macro or environment
%   \item[\cs{OParameter}] the optional parameter of a macro or environment
%   \item[\cs{AParameter}] the alternativ parameter of a macro or environment
%   \item[\cs{PParameter}] the part-of-command parameter of a macro or
%     environment
%   \end{description}
%    \begin{macrocode}
\DeclareRobustCommand*{\Macro}[1]{\mbox{\texttt{\char`\\#1}}}
\DeclareRobustCommand*{\Option}[1]{\mbox{\texttt{#1}}}
\DeclareRobustCommand*{\KOption}[1]{\mbox{\Option{#1}\texttt=}}
\DeclareRobustCommand*{\OptionValue}[2]{\mbox{\texttt{#1=#2}}}
\DeclareRobustCommand*{\FloatStyle}[1]{\mbox{\texttt{#1}}}
\DeclareRobustCommand*{\Pagestyle}[1]{\mbox{\texttt{#1}}}
\DeclareRobustCommand*{\Environment}[1]{\mbox{\texttt{#1}}}
\DeclareRobustCommand*{\Counter}[1]{\mbox{\texttt{#1}}}
\DeclareRobustCommand*{\Length}[1]{\mbox{\texttt{\char`\\#1}}}
\DeclareRobustCommand*{\PLength}[1]{\mbox{\PValue{#1}}}
\DeclareRobustCommand*{\Variable}[1]{\mbox{\PValue{#1}}}
\DeclareRobustCommand*{\FontElement}[1]{\PValue{#1}}
\DeclareRobustCommand*{\PName}[1]{\texttt{\textit{#1}}}
\DeclareRobustCommand*{\PValue}[1]{\texttt{#1}}
\DeclareRobustCommand*{\Parameter}[1]{\texttt{\{}\PName{#1}\texttt{\}}}
\DeclareRobustCommand*{\OParameter}[1]{%
  \texttt{[%]
  }\PName{#1}\texttt{%[
    ]}}
\DeclareRobustCommand*{\AParameter}[1]{%
  \texttt{(%)
  }\PName{#1}\texttt{%(
    )}}
\DeclareRobustCommand*{\PParameter}[1]{\texttt{\{#1\}}}
\DeclareRobustCommand*{\POParameter}[1]{\texttt{[#1]}}
%    \end{macrocode}
% \end{macro}
% \end{macro}
% \end{macro}
% \end{macro}
% \end{macro}
% \end{macro}
% \end{macro}
% \end{macro}
% \end{macro}
% \end{macro}
% \end{macro}
% \end{macro}
% \end{macro}
% \end{macro}
% \end{macro}
% \end{macro}
% \end{macro}
% \end{macro}
% \end{macro}
% NOTE: taken from scrguide.cls

% \begin{environment}{desctable}
%   This is almost the same like \texttt{desctabular} but it uses a longtable
%   to allow page breaks.
%    \begin{macrocode}
\newenvironment{desctable}[1][2em]{%
  \onelinecaptionsfalse
  \start@desctab{#1}%
  \newcommand{\Endfirsthead}{\toprule\endfirsthead}%
  \newcommand{\Endhead}{\midrule\endhead}%
  \newcommand*{\standardfoot}{%
    \addlinespace[-.5\normalbaselineskip]\midrule
    \multicolumn{2}{r@{}}{\dots}\\
    \endfoot
    \addlinespace[-.5\normalbaselineskip]\bottomrule
    \endlastfoot
  }%
  \longtable{lp{\descwidth}}%
}{%
  \endlongtable
}
%    \end{macrocode}
% \end{environment}

% \begin{length}{\descwidth}
%   I need a length of local usage. I could have used |\@tempdima| or
%   another local length from kernel. But I've decided not to try to find a
%   unused length at \texttt{tabular} environment.
%    \begin{macrocode}
\newlength{\descwidth}
%    \end{macrocode}
% \end{length}

% \begin{macro}{\start@desctab}
%   This is the \emph{worker} macro of \texttt{desctable} and
%   \texttt{desctabular}. It does the complete calculations and definition of
%   the entry (something like |\item|) commands.
%    \begin{macrocode}
\newcommand*{\start@desctab}[1]{%
  \setlength{\descwidth}{\linewidth}%
  \addtolength{\descwidth}{-4\tabcolsep}%
  \addtolength{\descwidth}{-#1}%
  \setlength{\labelwidth}{\linewidth}%
  \addtolength{\labelwidth}{-2\tabcolsep}%
  \newcommand{\nentry}[2]{%
    \multicolumn{2}{p{\labelwidth}}{\raggedright##1}\\*%
    \hspace*{#1} & ##2\tabularnewline%
  }%
  \newcommand{\entry}[2]{\nentry{##1}{##2}[.5\baselineskip]}%
  \newcommand*{\pentry}[1]{%
    \entry{\PLength{##1}\IndexPLength[indexmain]{##1}}}%
  \newcommand*{\pventry}[1]{\entry{\PValue{##1}}}%
  \newcommand*{\mentry}[1]{\entry{\Macro{##1}}}%
  \newcommand*{\ventry}[1]{%
    \entry{\Variable{##1}%\IndexVariable[indexmain]{##1}%
    }%
  }%
  \newcommand*{\feentry}[1]{%
    \entry{\FontElement{##1}\IndexFontElement[indexmain]{##1}}%
  }%
  \newcommand*{\oentry}[1]{%
    \entry{\Option{##1}\IndexOption[indexmain]{##1}}%
  }%
}
% \end{macro}

\makeatother

\newcommand{\tubslatex}{\emph{tubslatex}\xspace}

\def\example{\par\smallskip\noindent\textit{Beispiel: }}

% Schreibt 'Beispiel vor den folgenden Inhalt und rückt alles nach dem ersten
% Absatz um 2em ein.
\newenvironment{Example}{%
\begingroup
\leftskip2em
\par\smallskip\noindent\hspace*{-2em}\textit{Beispiel: }
}{%
\par\endgroup
}

% Umgebung 'Wichtig:' ...
\newenvironment{important}{%
  \begin{description}
    \item[\itshape\mdseries\rmfamily Wichtig:]
}{%
  \end{description}
}

\newcommand{\zB}{\mbox{z.\,B.}\xspace}


%opening
\title{Das CorporateDesign in \LaTeX}
\subtitle{Anleitung und Dokumentation}
\author{Enrico Jörns}
\publishers{Institut fuer Lorem Ipsum} % to logo?


\begin{document}

\maketitle
\pagestyle{scrheadings}
\tableofcontents


\newcommand{\newdocumentclass}[1]{\textcolor{tuRed}{\lstinline{#1}}}

\chapter{Schnellstart}

\paragraph{Dokumente}
Die schnellste Methode ein Dokument im Corporate Design zu erstellen ist
das Laden einer der zur Verfügung stehenden Dokumentenklassen.
Für die Erstellung von Textdokumenten sind dies \newdocumentclass{tubsartcl},
\newdocumentclass{tubsreprt} und \newdocumentclass{tubsbook}.

\paragraph{Poster}
Die Klasse \newdocumentclass{tubsposter} kann zum Erstellen von Postern verwendet werden.
% tubsflyer?

\paragraph{Papierformat}
Die zu verwendende Papierformat sollte dabei als optionales Argument mit
übergeben werden. Für ein Dokument in DIN A4 ist dies \texttt{a4paper}.
Zur Verfügung stehen alle Papierformate A0 bis A6.

Standardmäßig wird die Titelseite in Dokumenten einfach mit dem Logo und einer
roten Trennlinie zwischen Kommunikations- und Absenderbereich versehen.

\paragraph{Präsentationen}
Um Präsentation zu erstellen existiert ein Style für \LaTeX-Beamer.
Dieser wird einfach mit \lstinline!\usetheme{tubs}! geladen.


\chapter{Dokumente}

Einfache Textdokumente können mit den Klassen \newdocumentclass{tubsartcl},
\newdocumentclass{tubsreprt} und \newdocumentclass{tubsbook} erstellt werden.

% \section{Bindekorrektur und Marginalen}

Da Dokumente normalerweise gedruckt werden und ggf. auch gebunden, werden
sie standardmäßig mit einer kleinen Bindekorrektur gesetzt, sodass das
Logo beim Drucken oder Abheften nicht abgeschnitten wird.
Diese Bindekorrektur kann mittels der Paketoption \texttt{bcor} angepasst werden.
Mit \texttt{bcor=0mm} wird sie beispielsweise deaktiviert.



\section{Titelseite}

Titelseiten können bei \LaTeX\ generell auf zwei verschiedene Arten erstellt
werden; entweder mit Hilfe des Befehls \lstinline{\maketitle} oder mit
der Umgebung \lstinline{titlepage}. Beide Varianten werden von tubslatex
unterstützt und leicht modifiziert.

\begin{Declaration}
  \Macro{maketitle}\OParameter{style}
\end{Declaration}

Die einfache Verwendung von \lstinline{\maketitle} erzeugt eine Titelseite
mit dem TU-Logo und einer roten Trennlinie zwischen Absender- und
Kommunikationsbereich.

Mit Hilfe des optionalen Arguments \PName{style} kann die Darstellung
der Titelseite geändert werden, indem aus einer Reihe vordefinierter Styles
ausgewählt wird.\bigskip

Neben den standardmäßig definierten Elementen für Haupttitelseiten werden
in \tubslatex noch ein paar zusätzliche definiert.

\begin{Declaration}
  \Macro{logo}\Parameter{logo}\\
  \Macro{titlepicture}\Parameter{file}\\
  \Macro{titleabstract}\Parameter{text}
\end{Declaration}

\Macro{logo} dient zur Darstellung eines zusätzlichen Absenders als Schrift
oder Bild. Es wird in allen Stilen im Absenderbereich auf der dem
TU-Logo gegenüberliegenden Seite dargestellt.
Mit \Macro{titlepicture} kann eine Bilddatei angegeben werden, die bei
Verwendung eines entsprechenden Styles auf der Titelseite dargestellt werden.
\Macro{titeabstract} erlaubt die Darstellung eines kurzen zusammenfassenden
Textes auf der Titelseite, sofern der gewählte Stil dies unterstützt.

\begin{Declaration}
  \XMacro{begin}\PParameter{\Environment{titlepage}}\\
  \quad\dots\\
  \XMacro{end}\PParameter{titlepage}
\end{Declaration}

Mit der \Environment{titlepage}-Umgebung können, wie von den Standardklassen
gewohnt, Titelseiten definiert werden.
% TODO: \showtubslogo  etc.

\begin{Declaration}
  \XMacro{begin}\PParameter{\Environment{titlerow}}%
    \OParameter{options}%
    \Parameter{gaussheight}\\
  \quad\dots\\
  \XMacro{end}\PParameter{titlerow}
\end{Declaration}

Die Umgebung \Environment{titlerow} erlaubt es dabei, die Titelelemente im 
Gaußraster anzulegen. Der Parameter \PName{gaussheight} gibt dabei
die Höhe des jeweiligen Elements in Segmenten an. Die Position der Elemente
ergibt sich aus der Reihenfolge der Definition.
Mit dem optionalen Parameter \PName{options} können Einstellung wie die Hintergrundfarbe oder ein Hintergrundbild übergeben werden.

\subsection{Vordefiniert Titel-Styles}

Es sind 3 einfache Styles vordefiniert.

\begin{center}
  \fboxsep0mm
  \begin{minipage}[t]{0.33\textwidth}
    \centering\sffamily
    \fbox{%
      \includegraphics[width=0.95\textwidth]{examples/article1.pdf}}
    [default]
  \end{minipage}%
  \begin{minipage}[t]{0.33\textwidth}
    \centering\sffamily
    \fbox{%
      \includegraphics[width=0.95\textwidth]{examples/titlestyle_image.pdf}}
    [image]
  \end{minipage}%
  \begin{minipage}[t]{0.33\textwidth}
    \centering\sffamily
    \fbox{%
      \includegraphics[width=0.95\textwidth]{examples/titlestyle_imagetext.pdf}}
    [imagetext]
  \end{minipage}

\end{center}

\section{Kopf-/ Fußzeile}

Standardmäßig wird die Fußzeile komplett leer gelassen und die Kopfzeile
wird mit Seitennummer rechts und Kapitelname links gesetzt. Bei zweiseitigem
Layout gilt dies für die ungeraden Seiten, gerade Seiten werden entsprechend
mit Seitennummer links und Kapitelname rechts gesetzt.

\begin{Declaration}
  \Macro{ihead}\Parameter{innen}\\
  \Macro{ohead}\Parameter{außen}
\end{Declaration}

Bei Bedarf können die Kopfzeilen individuell angepasst werden.
Dies geschieht mit den Befehlen \Macro{ihead} und \Macro{ohead}, welche
auf den gleichnamigen Kommandos aus dem Koma-Skript aufbauen und für eine
korrekte vertikale Positionierung sorgen.

\begin{Declaration}
  \Macro{headsepline}
\end{Declaration}


Die standardmäßig gesetzten kurzen Linien am jeweils oberen äußeren Ende des
Kopfbereiches sind im Makro \Macro{headsepline} definiert.
Dies kann bei Bedarf überschrieben werden.


\section{Seitenlayout}

Eine Seite im Corporate Design unterteilt sich grundlegend in 2 Bereiche.
Der Absenderbereich und der Kommunikationsbereich. Der Absenderbereich wird zur
Darstellung des TU-Siegelbandlogos und des Logos eines speziellen Instituts
oder einer Abteilung verwendet. Im Textteil dient er teilweise als Bereich für
Inhalte einer Kopf-/Fußzeile.

Der Absenderbereich kann entweder am Anfang oder am Ende der Seite platziert
werden. Dies beeinflusst auch das Gaußraster, es beginnt jeweils mit dem
größten Segment am Absenderbereich.

\begin{Declaration}
  \Option{a0paper}\\
  \Option{a1paper}\\
  \Option{a2paper}\\
  \Option{a3paper}\\
  \Option{a4paper}\\
  \Option{a5paper}\\
  \Option{a6paper}
\end{Declaration}

Für die Auswahl des verwendeten Papierformats stehen alle DIN A-Größen von
0 bis 6 zur Verfügung. Standardeinstellung ist \Option{a4paper}.

\begin{Declaration}
  \Option{landscape}
\end{Declaration}

Schaltet das Dokument in Querformat-Darstellung.

\begin{minipage}{0.45\textwidth}\sffamily\centering
  {\fboxsep0mm\fbox{\includegraphics[width=\textwidth]{examples/article_landscape.pdf}}}\\
  Dokument im Querformat
\end{minipage}


\begin{important}
  Im Querformat hat das Gaußraster eine abweichende
  Segmentanzahl. Für alle gängigen Formate sind im Querformat \emph{6 Segmente}
  definiert. Daher lösen Layouts, die für hochformatige Dokumente
  geschrieben wurden, im Querformat einen Fehler aus.
\end{important}


\begin{Declaration}
  \KOption{sender}\PName{Position}
\end{Declaration}

Mit der Option \Option{sender} kann die Position des Absenderbereiches
festgelegt werden. Mit der Einstellung \OptionValue{sender}{top} wird
der Absenderbereich am oberen Seitenende dargestellt. Dies ist auch die
Standardeinstellung. Wählt man dagegen \OptionValue{sender}{bottom}, so
wird der Absenderbereich am unteren Ende der Seite platziert

\begin{important}
  Die Positionierung des Absenderbereichs beeinflusst die Orientierung des
  Gaußrasters. Das größte Segment wird dabei immer so gesetzt, dass es direkt
  an den Absenderbereich anschließt und alle folgende von absteigender Höhe sind.
\end{important}



\begin{Declaration}
  \KOption{bcor}\PName{Wert}
\end{Declaration}

Die Bindekorrektur beschreibt einen zusätzlichen Abstand des eigentlichen
Darstellungsbereichs vom inneren Formatrand. Die Bindekorrektur
ist bei Textdokumenten standardmäßig auf Rahmenbreite voreingestellt.

Sinnvoll ist eine Bindekorrektur selbstverständlich zum einen für Bindungen,
wo sie der Breite des durch die Bindung verdeckten Bereiches entsprechen sollte.
Zum anderen ist sie aber auch für den Druckvorgang sinnvoll, da normale Drucker
keinen randlosen Druck ermöglichen. Die Bindekorrektur verhindert so \zB ein 'Abschneiden' des Siegellogos.
Soll das Dokument ohne Bindekorrektur dargestellt werden, so ist dies
mit \OptionValue{bcor}{0mm} möglich.

\begin{Declaration}
  \Option{twoside}
\end{Declaration}

Für das Setzen von zweiseitige Dokumente ist die Option \Option{twoside}
vorgesehen. Mit ihr wird von einseitigem Layout auf zweiseitiges Layout
umgeschaltet, was bedeutet, dass die Innenseite eines Dokuments abwechselnd
auf der linken und rechten Seite definiert ist. Die hat unter anderem Einfluss
auf definierte Ränder (Bindungskorrektur, Marginale).


\begin{Declaration}
  \Option{marginleft}\\
  \Option{marginright}
\end{Declaration}

Das setzen einer Marginale wird durch die Befehle \PName{marginleft} und
\PName{marginright} vereinfacht. Diese setzen jeweils auf der linken (inneren)
bzw. rechten (äußeren) Seite des Dokumentes eine Marginale, deren Breite
einem Element im Spaltenraster entspricht.


\section{Schrift}

Dokumente im Corporate Design werden allgemein in der Schrift \emph{Nexus}
gesetzt.
Diese wird standardmäßig bei allen Dokumenten geladen. Auf eine explizite
Angabe der Schriftgröße in den Dokumentenoptionen sollte nach Möglichkeit
verzichtet werden, da für die jeweiligen Papierformate spezielle
Schriftgrößen vordefiniert sind und geladen werden.
Das Laden anderer Schriftgrößen führt im Vergleich dazu teilweise zu stark abweichenden Ergebnissen.

Für wichtige Standard-Elemente gibt es des weiteren vordefinierte Koma-Fonts.


\begin{desctable}
\entry{\PValue{headline}}{%
  Einfache Überschrift (groß)
}
\entry{\PValue{subheadline}}{%
  Unterüberschrift
}
\entry{\PValue{institute}}{%
  Institutsname im Logo-Bereich des Absenders.
}
% \entry{\PValue{infotext}}{%
% }
% \entry{\PValue{infotextitem}}{%
% }
\entry{\PValue{copytext}}{%
  Mengentext
}
\end{desctable}


\subsection{Farben}

Für eine Übersicht über die verfügbaren Farben siehe \ref{sec:colors}.%TODO...

\begin{Declaration}
  \Option{mono}
\end{Declaration}

Schwarz-Weiß-Darstellung der Standardelemente

\begin{Declaration}
  \Option{cmyk}
\end{Declaration}

CMYK-Darstellung der Standardelemente


% \noindent\hspace*{-\parindent}\fbox{\cmd{\maketitle}\oarg{asdf}\marg{asdf}}
\chapter{Plakate}

Plakate werden mit Hilfe der Klasse \newdocumentclass{tubsposter} erstellt.
Dabei kann Mittels der Option \texttt{style} zwischen der Darstellung als
normales Plakat oder als wissenschaftliches Plakat gewählt werden.

\section{Veranstaltungsplakate}

\begin{Declaration}
  \XMacro{begin}\PParameter{\Environment{tubsposter}}%
    \OParameter{options}\\
  \quad\dots\\
  \XMacro{end}\PParameter{tubsposter}
\end{Declaration}

Ein neues Plakat wird mit der Umgebung \Environment{tubsposter} erstellt.
Der optionale Parameter \PName{options} akzeptiert dabei die
unter \ref{} beschriebenen Optionen.

\begin{Declaration}
  \XMacro{begin}\PParameter{\Environment{posterrow}}%
    \OParameter{cols}\\%
  \quad\dots\\
  \XMacro{end}\PParameter{posterrow}%\\
%   \XMacro{begin}\PParameter{\Environment{posterrow*}}%
%     \OParameter{cols}\\%
%   \quad\dots\\
%   \XMacro{end}\PParameter{posterrow*}
\end{Declaration}

Einzelne Segmente im Gaußraster können mit der Umgebung \Environment{posterrow}
erzeugt werden.

\section{Wissenschaftliche Plakate}

\begin{Declaration}
  \XMacro{begin}\PParameter{\Environment{tubsposter}}%
    \OParameter{options}%
    \Parameter{rows}\\
  \quad\dots\\
  \XMacro{end}\PParameter{tubsposter}
\end{Declaration}

Für die Erstellung von wissenschaftlichen Plakaten wird ebenfalls die Umgebung
\Environment{tubsposter} verwendet, welche in diesem Fall jedoch einen
zusätzlichen Parameter \PName{rows} erwartet.
Damit wird die Anzahl an Modulzeilen bestimmt. Dies geschieht mittels
einer kommagetrennten Liste, wobei jedes Element entweder eine Länge
oder der Buchstabe 'X' sein kann. Eine Länge legt die Höhe der jeweiligen
Modulzeile genau fest, ein X sorgt dafür, dass alle mit X gekennzeichneten
Zeilen den restlichen zur Verfügung stehenden Platz gleichmäßig untereinander
aufteilen. Dieses Vorgehen ist an die Tabellen-Umgebung \Environment{tabularx}
angelehnt.

\begin{Example}
  \noindent\Macro{begin}\PParameter{\Environment{tubsposter}}
    \Parameter{3cm,X,5cm,X}\par
  \noindent Erzeugt 4 Modulzeilen, wobei die 1. 3cm und die 3. 5cm hoch sind.
  Die Zeilen 2 und 4 nehmen den Restlichen verfügbaren Platz ein
  und sind gleich hoch.
\end{Example}


\begin{Declaration}
  \XMacro{begin}\PParameter{\Environment{posterrow}}%
    \OParameter{cols}\\%
  \quad\dots\\
  \XMacro{end}\PParameter{posterrow}\\
  \XMacro{begin}\PParameter{\Environment{posterrow*}}%
    \OParameter{cols}\\%
  \quad\dots\\
  \XMacro{end}\PParameter{posterrow*}
\end{Declaration}

Die mit \Environment{tubsposter} angelegten Modulzeilen können nun jeweils mit
der Umgebung \Environment{posterrow} mit Inhalt gefüllt werden.
Dabei kann entweder direkt der gewünschte Inhalt geschrieben oder das optionale
Parameter \PName{cols} benutzt werden.
Dieses erlaubt die Definition zusätzlicher Spalten in der aktuellen Modulzeile.
Es wird wieder ein kommagetrennte Liste erwartet,
deren Elemente dieselbe Bedeutung haben wie bereits beschrieben, außer, dass
sie die Breite und nicht die Höhe definieren.

Der normale Abstand des Inhalts vom Rand der Modulbox beträgt halbe
Rahmenbreite. Für das Einfügen von Bildern etwa kann es sinnvoll sein,
diesen Rahmen wegzulassen. Dies geschieht mit der Sternchen-Variante
\Environment{posterrow*}.


\begin{Declaration}
  \XMacro{begin}\PParameter{\Environment{postercol}}%
    \OParameter{rows}\\%
  \quad\dots\\
  \XMacro{end}\PParameter{postercol}\\
  \XMacro{begin}\PParameter{\Environment{postercol*}}%
    \OParameter{rows}\\%
  \quad\dots\\
  \XMacro{end}\PParameter{postercol*}
\end{Declaration}

Die mit \Environment{posterrow} angelegten Spalten können jeweils mit 
\Environment{postercol} mit Inhalt gefüllt oder in neue Unterzeilen aufgeteilt 
werden.

\begin{Declaration}
  \XMacro{begin}\PParameter{\Environment{postersubrow}}\\%
  \quad\dots\\
  \XMacro{end}\PParameter{postersubrow}\\
  \XMacro{begin}\PParameter{\Environment{postersubrow*}}\\%
  \quad\dots\\
  \XMacro{end}\PParameter{postersubrow*}
\end{Declaration}

Die mit \Environment{postercol} angelegten Unterzeilen können jeweils mit 
\Environment{postercol} mit Inhalt gefüllt werden.

\begin{Example}

\end{Example}

\section{Seitenlayout}

% margin
% bcor
% sender=top/bottom


\section{Format}

\section{Farben}

\section{Befehle}

\section{Optionen}


\chapter{Briefe}

\chapter{Präsentationen}
\Index{Präsentation}
\Index{Beamer}

% \section{Anwendung und Folienaufbau}

Grundsätzlich handelt es sich bei der Vorlage für Präsentationen lediglich um
ein \emph{theme} für das \emph{beamer}-Paket.
Für allgemeine Fragen zur Präsentationserstellung wir daher auf die
entsprechende Dokumentation\cite{beamer} verwiesen.

Beschrieben werden hier alle Besonderheiten des Corporate-Design-Themes.
Außerdem sollen einige damit verbundene allgemeine Hinweise gegeben werden.

Die Vorlage wird mit dem Beamer-Befehl \lstinline!\usetheme{tubs}! geladen.

\begin{hint}\sloppy
  Alle tubaltex-spezifischen Template-Argumente sollten nicht der Dokumentenklasse,
  sondern bevorzugt beim Laden des entsprechendes Themes übergeben werden,
  um eine klare und wiederverwendbare Struktur zu erhalten.
  
  Zum Beispiel: \lstinline!\usetheme[<Optionen>]{tubs}!
\end{hint}
% TODO: Template argumente von Macro-Argumenten unterscheidbar machen?
  

\section{Titelfolie}\label{sec:titelfolie}

Die Titelfolie ist im Absender-/Kommunikations-Bereich-Layout mit
\gls{glos:siegelbandlogo} im Sinne der allgemeinen Gestaltungsprinzipien des
Corporate Design gehaltens.

Die Kommunikationsfläche ist nach Vorlage des \gls{glos:gaussraster}s
in drei Bereiche aufgeteilt:
Ein \emph{Bildbereich}, der ein Foto oder eine Grafik als Blickfang enthalten
sollte, darunter der \emph{Titelbereich}, der Präsentationstitel,
sowie alle relevanten Informationen trägt
und zum Abschluss ein einfarbiger (roter) Streifen.

Zusätzlich kann in der rechten oberen Ecke des Absenderbereichs ein
Instituts-Logo platziert werden.

\begin{minipage}{0.5\textwidth}
\begin{verbatim}
\title{Corporate Design}
\subtitle{Jetzt mit \LaTeX}
\author{Max Mustermann}

\begin{frame}[plain]
\titlepage
\end{frame}
\end{verbatim}
\end{minipage}
\fboxsep0mm
\begin{minipage}{0.5\textwidth}
\fbox{\includegraphics[width=0.9\textwidth]{examples/titelseite.pdf}}
\end{minipage}

\paragraph{Standardbefehle}

Es können die Standardbefehle zur Titelseitenerstellung, wie
\lstinline{\title},
\lstinline{\subtitle},
\lstinline{\author},
\lstinline{\institute},
\lstinline{\titlegraphic}
und \lstinline{\logo} verwendet werden.

Die Titelseite wird normal mit \lstinline{\titlepage} erzeugt:

\begin{lstlisting}
\begin{frame}[plain]
  \titlepage
\end{frame}
\end{lstlisting}

\emph{Wichtig:} Der \lstinline{frame}-Umgebung muss die Option
\lstinline{[plain]} übergeben werden,
sonst kommt es zu einer falschen Darstellung!

\begin{Declaration}
  \Option{colorfoot}
\end{Declaration}

Mit der Beamer-Option \Option{colorfoot} wird die untere einfarbige Linie
der Titelseite statt in rot in der Farbe des benutzten Sekundärfarbklangs
dargestellt. Weitere Informationen zur Farbanpassung liefert
Kapitel~\ref{beamer:sec:color}.


\subsection{Titelgrafik}
\Index{Titelgrafik!Beamer}
\CommandIndex{titlegraphic}

\begin{Declaration}
  \Macro{titlegraphic}\OParameter{Optionen}\Parameter{Inhalt}
\end{Declaration}

\begin{sloppypar}
Dieser von \LaTeX-beamer bereitgestellte Befehl erlaubt in \tubslatex
die Darstellung von Inhalten im Bildbereich der Titel-Folie.
Der \PName{Inhalt} kann dabei ein beliebiges \LaTeX-Konstrukt sein
und wird am unteren Rande des Bildbereiches ausgerichtet dargestellt.

Im Normalfall wird als Titelgrafik allerdings eine Bild-Datei mittels
\lstinline{\includegraphics} eingebunden werden.
Diese kann (im Rahmen der allgemeinen Corporate Design-Richtlinien)% TODO:cite?
frei gewählt werden.
Das eingebundene Bild wird dabei automatisch randlos
auf den Darstellungsbereich skaliert und zugeschnitten, sofern keine
Standard-Optionen an \lstinline{\includegraphics} übergeben werden oder
manuell eine andere Einpassungsart gewählt wurde.

Innerhalb von \lstinline!\titlegraphic{}! stehen folgende
Optionen zur Einpassung des Bildes zur Verfügung:
\end{sloppypar}

%TODO: float-table + Link?
\begin{desctable}
\toprule
\entry{\PValue{clipped}}{%
  Automatisches Abschneiden. Dies ist die \emph{Standardeinstellung}.
  Das Bild wird dabei optimal in den Darstellungsbereich eingepasst
  (automatische Wahl von \PValue{hclip}/\PValue{vclip}).
}
\entry{\PValue{hclip}}{%
  Das Bild wird vertikal auf Höhe des Bildbereiches skaliert und (falls nötig)
  horizontal zugeschnitten.
}
\entry{\PValue{vclip}}{%
  Das Bild wird horizontal auf Breite des Bildbereiches skaliert und (falls nötig)
  vertikal zugeschnitten.
}
\entry{\PValue{scaled}}{%
  Horizontale \emph{und} vertikale Skalierung. Das Seitenverhältnis des
  Bildes kann dabei verändert und das Bild somit verzerrt werden.
}
\entry{\PValue{keepsize}}{%
  Es wird keinerlei Skalierung oder Beschneidung durchgeführt.
}
\bottomrule
\caption{Mögliche Parameter für \Macro{titlegraphic} zur
automatische Einpassung von Titelgrafik}
\end{desctable}

\begin{Example}
  \begin{lstlisting}
\titlegraphic[scaled]{\includegraphics{myimage.jpg}}
  \end{lstlisting}
  Fügt \textit{myimage.jpg} passend in den Bildbereich ein, indem es das Bild
  horizontal und vertikal skaliert.
\end{Example}

Im Prinzip ist das Seitenverhältnis des Quellbildes durch die automatische
Einpassung nicht besonders kritisch. Jedoch wird durch ein korrektes Verhältnis
sicher gestellt, dass keine wichtigen Teile des Bildes abgeschnitten werden.
In Tabelle \ref{tab:picratio} sind daher Bild-Seiten\-verhältnisse für
verschiedene Seitenverhältnisse der Präsentation aufgeführt.

\begin{table}[ht]
\centering
\begin{tabular}{ll}
\toprule
\bfseries Präsentation  & \bfseries  Bild  \\
\midrule
$3:4$   & $1:3{,}15$ \\
% Seitenverhältnis für 5:4 einfügen
$16:9$  & $1:4{,}30$ \\
$16:10$ & $1:3{,}83$ \\
\bottomrule
\end{tabular}
\caption{Bild-Seitenverhältnisse}
\label{tab:picratio}
\end{table}



\paragraph{Standardgrafik}
\Index{Standardgrafik!Beamer}
\CommandIndex{tuDefaultTitlegraphic}

Zu Testzwecken oder falls kein eigenes Bild zur Hand ist,
kann alternativ ein Standardbild eingefügt werden, das die Front
des TU-Altgebäudes zeigt und gut mit den Standard-Folienfarben harmoniert.

Das Einfügen des Standardbildes ist mit dem Befehl
\linebreak\lstinline{\tuDefaultTitlegraphic} möglich,
der \lstinline!\titlegraphic{}! direkt als Argument übergeben werden kann.

\begin{example}
\begin{lstlisting}
\titlegraphic{\tuDefaultTitlegraphic}
\end{lstlisting}
\end{example}


\paragraph{Manuelle Einpassung}\hfill

\begin{Declaration}
  \Macro{titlegraphicswidth}\\
  \Macro{titlegraphicsheight}
\end{Declaration}

Zur manuellen Einpassung oder für das Erstellen von Grafiken mittels
\LaTeX-Befehlen ist die Dimension des Titelgrafik-Bereichs (Bildbereich) in den
beiden Längen \lstinline{titlegraphicswidth} und \lstinline{titlegraphicsheight}
hinterlegt.

% Das Titelbild wird generell an der unteren Kante des dafür vorgesehenen Bereichs ausgerichtet, sodass in falschem Seitenverhältnis vorliegene Bilder nach Oben 'rausgeschoben' werden.

\subsection{Logo}
\Index{Logo!Beamer}
\Index{Sekundärlogo!Beamer}

Neben dem immer vorhandenen Siegelband-Logo der TU kann noch ein sekundäres Logo
dargestellt werden, welches das jeweilige Institut bzw. die jeweilige Abteilung
repräsentiert.

\begin{Declaration}
  \Macro{logo}\Parameter{Bild}
\end{Declaration}

Das mittels \lstinline{\logo} eingebundene Logo wird in der rechten oberen
Ecke des Absenderbereichs angezeigt und standardmäßig auch auf allen weiteren
Folien im Fußbereich, sofern nicht die Option \Option{nologoinfoot}
verwendet wird.

Das eingebundene Logo wird bei Verwendung von \Macro{includegraphics}
automatisch korrekt vertikal skaliert,
sofern keine manuelle Skalierung als Option angegeben wird.

% Die Verwendung von \lstinline{\logoheight} als Höhenbeschränkung für Grafiken
% stellt deren automatische korrekte vertikale Skalierung sicher.
% Das Seitenverhältnis der verwendeten Grafik-Datei ist dabei relativ
% frei, sollte jedoch nach Möglichkeit und Gründen der Lesbarkeit breiter sein als
% hoch.

\begin{example}
\begin{lstlisting}
\logo{\includegraphics{institut.jpg}}
\end{lstlisting}
\end{example}

\begin{Declaration}
  \Macro{logoheight}
\end{Declaration}

Für allgemeine Inhalte kann auf die Höhe des Logo-Bereichs über die Länge
\Macro{logoheight} zugegriffen werden.

\section{Inhaltsfolien}
\Index{Folien!Inhalt}
\Index{Inhaltsfolien}

Das Layout der Inhaltsfolien ist relativ schlicht und bietet viel Platz.
Es zeichnet sich durch einen Kopfbereich, der Folientitel und evtl. eine
Inhaltsübersicht enthält, und einen Fußbereich, der neben Siegelband- und
Institutslogo allgemeine Informationen zur Präsentation liefert.
Auf den genaueren Aufbau des Kopfbereichs geht Kapitel~\ref{subsec:head},
auf den des Fußbereichs Kapitel~\ref{subsec:foot} ein.
% Die Überschrift ist im Kopfbereich grau hinterlegt und im Fußbereich der Folien
% findet sich das TU-Logo, sowie eine rote Trennlinie.
% Rechts unterhalb der Trennlinie wird, wenn definiert, das Logo angezeigt.
% Ist dies nicht gewünscht, kann dies mit der Klassenoption
% \Option{nologoinfoot} unterdrückt werden.

\begin{minipage}{0.5\textwidth}
\begin{verbatim}
\begin{frame}{Inhaltsseite}
  \begin{itemize}
    \item Hier steht der Inhalt
    \item Hier nicht
    \item Weitere Informationen
  \end{itemize}
\end{frame}
\end{verbatim}
\end{minipage}
\begin{minipage}{0.5\textwidth}
\fboxsep0mm
\fbox{\includegraphics[width=0.9\textwidth,page=1]{examples/inhaltsseite.pdf}}
\end{minipage}

\subsubsection{Hervorgehobene Folien}

Einzelne wichtige Folien können extra hervorgehoben werden.
Dazu dient die Option \Option{highlight},
welche einen breiteren und rot hinterlegten Titelbereich erzeugt.

\begin{minipage}{0.5\textwidth}
\begin{verbatim}
\begin{frame}[highlight]
  {Inhaltsseite -- Hervorgehoben}
  \begin{itemize}
    \item Hier steht der Inhalt
    \item Hier nicht
    \item Weitere Informationen
  \end{itemize}
\end{frame}
\end{verbatim}
\end{minipage}
\begin{minipage}{0.5\textwidth}
\fboxsep0mm
\fbox{\includegraphics[width=0.9\textwidth,page=2]{%
  examples/inhaltsseite.pdf}}
\end{minipage}

\subsubsection{Titellose Folien}

Wird bei einer Folie kein Titel angegeben, so wird der graue Titelbereich
nicht dargestellt.
Ein eventuell vorhandenes Inhaltsverzeichnis wird jedoch weiterhin samt
Hintergrund angezeigt!

% TODO: Option notitle?
\begin{minipage}{0.5\textwidth}
\begin{verbatim}
\begin{frame}
  \begin{itemize}
    \item Hier steht der Inhalt
    \item Hier auch
    \item Weitere Informationen
  \end{itemize}
\end{frame}
\end{verbatim}
\end{minipage}
\begin{minipage}{0.5\textwidth}
\fboxsep0mm
\fbox{\includegraphics[width=0.9\textwidth,page=3]{%
  examples/inhaltsseite.pdf}}
\end{minipage}

\subsection{Kopfbereich}\label{subsec:head}
\Index{Kopfbereich!Beamer}

Der Kopfbereich der Folien besteht standardmäßig aus dem grau hinterlegten
Folientitel. Optional kann darüber auch noch eine Inhaltsübersicht
(Inhaltsverzeichnis) dargestellt werden, 
die auch zur Navigation im Dokument genutzt werden kann.
Dieser Kopfbereich ist dann unabhängig vom Titelbereich und wird
auch bei titellosen Folien dargestellt.

\begin{Declaration}
  \Option{colorhead}
\end{Declaration}

Der Kopfbereich ist normalerweise in einem hellen Grauton hinterlegt.
Durch Verwendung der Option \Option{colorhead} wird der Hintergrund
in der 40\%-Version der aktiven Variante des Sekundärfarbklangs gesetzt.
Nähere Informationen und Beispiele zur Farbanpassung liefert das
Kapitel~\ref{beamer:sec:color}.

\subsubsection{Inhaltsübersicht}

Die Inhaltsübersicht (Inhaltsverzeichnis) wird bei Verwendung rechts
ausgerichtet als oberstes Element jeder Folie dargestellt.
Sie wird auch auf titellosen Folien angezeigt.


\begin{Declaration}
  \Option{tocinheader}\\
  \Option{tinytocinheader}
\end{Declaration}

Das Inhaltsverzeichnis im Folienkopf
wird mir der Option \Option{tocinheader} aktiviert.
Die \Option{tinytocinheader} bewirkt dies ebenfalls,
verwendet aber eine kleinere Schriftgröße.

Angezeigt werden standardmäßig untereinander die Gliederungsebenen
\Macro{section} und \Macro{subsection}. Dies kann mit der Option
\Option{nosubsectionsintoc} deaktiviert werden.


\begin{Declaration}
  \Option{widetoc}\\
  \Option{narrowtoc}
\end{Declaration}

Während die Option \Option{widetoc} das Inhaltsverzeichnis in der Breite
streckt, wird es durch die Option \Option{narrowtoc} gestaucht.
Somit kann eine flexiblere Platzausnutzung erreicht werden.

\begin{Declaration}
  \Option{nosubsectionsintoc}
\end{Declaration}

Soll im Inhaltsverzeichnis des Kopfbereichs die Gliederungsebene
\Macro{subsection} nicht mit angezeigt werden, so kann dies durch die Option
\Option{nosubsectionsintoc} erreicht werden.

Dies Option ist natürlich nur wirksam, wenn die Option
\lstinline{tocinheader} oder \lstinline{tinytocinheader} verwendet wird.



\subsection{Fußbereich}\label{subsec:foot}
\Index{Fußbereich!Beamer}

Der Fußbereich zeichnet sich durch das Siegelband-Logo und eine rote Trennlinie
aus. Darunter stehen standardmäßig Informationen wie Datum, Autor(en), Präsentationstitel
und Seitenzahl, sowie auf der rechten Seite ein Instituts-/bzw. Abteilungslogo,
sofern definiert.
Der Inhalt der einzelnen Felder ergibt sich aus der Benutzung der in
\chaptername~\ref{sec:titelfolie} beschriebenen Befehle.
Die konkrete Darstellung in der Fußzeile kann jedoch durch eine Reihe von
Beamer-Optionen zusätzlich angepasst werden.

Autor und Titel werden, falls vorhanden, in ihrer Kurzversion dargestellt,
welche jeweils bei \Macro{author} und \Macro{title} im optionalen Argument
definiert werden kann.

Die Darstellung der einzelnen Textelemente wird abhängig von deren
jeweilige Länge durch ein dynamisches Positionierungsschema optimal
an den verfügbaren Platz angepasst.
%TODO: Details zu automat. Umbrüchen etc.?

% Sprache
Ändert man im Babel-Paket die Sprache, ändert sich auch automatisch
die Sprache der Fußzeile (z.B. "`page"' statt "`Seite"', anderes Datumsformat, etc.).
Vollständig unterstützt sind Deutsch und Englisch.

\begin{Declaration}
  \Option{nodate}\\
  \Option{noauthor}\\
  \Option{nopagenum}
\end{Declaration}

Diese Beamer-Optionen unterdrücken jeweils die Darstellung eines Elementes in der Fußzeile.
Die Option \Option{nodate} sorgt dafür, dass das Datum nicht dargestellt wird,
\Option{noauthor} dafür, dass der Autor nicht angezeigt und
\Option{napagenum} dafür, dass die Seitenzahl ausgeblendet wird.

\begin{Declaration}
  \Option{totalpages}
\end{Declaration}

Standardmäßig wird die Seitenzahl wie folgt dargestellt: "`Seite x"'.
Die Option \Option{totalpages} bewirkt, dass zusätzlich die Gesamtseitenzahl
mit eingeblendet wird: "`Seite x von y"'.

\begin{Declaration}
  \Option{nologoinfoot}
\end{Declaration}

Diese Beamer-Option deaktiviert die Darstellung des Institutslogos im Fußbereich.

\subsection{Inhalt}

\paragraph{Blöcke}

Blöcke werden in den Vorlagen bei Standardeinstellungen einfach
mit transparentem Hintergrund mit verschiedenfarbigen Überschriften dargestellt
wie in \figurename~\ref{subfig:stdblocks} zu sehen.
Die Farben sind dabei dem CD-Farbschema entnommen.

\begin{Declaration}
  \Option{colorblocks}
\end{Declaration}

Für eine farblich abgesetztere Darstellung steht die Option \Option{colorblocks}
zur Verfügung. Diese bewirkt, dass die Blöcke in farbigen Boxen dargestellt werden
(siehe \figurename~\ref{subfig:colorblocks}).


\begin{figure}[!ht]
\begin{minipage}{0.5\textwidth}
  \centering
  \fboxsep0mm\fbox{%
  \includegraphics[width=0.95\textwidth,page=1]{examples/beamer_blocks.pdf}}
  \subcaption{Standard-Blöcke}\label{subfig:stdblocks}
\end{minipage}
\begin{minipage}{0.5\textwidth}
  \centering
  \fboxsep0mm\fbox{%
  \includegraphics[width=0.95\textwidth,page=2]{examples/beamer_blocks.pdf}}
  \subcaption{\Option{colorblocks}-Blöcke}\label{subfig:colorblocks}
\end{minipage}
\caption{Vergleich zwischen normaler und \Option{colorblocks}-Darstellung}
\end{figure}


\paragraph{Beschriftung von Abbildungen}
\Index{Abbildungen!Beamer}
\Index{caption!Beamer}

Anders als bei den Standard-Templates werden Abbildungen
automatisch durchnummeriert.

Sollte dies nicht erwünscht sein, kann dies mittels
\begin{lstlisting}
\setbeamertemplate{caption}[default]
\end{lstlisting}
deaktiviert werden.

\section{Farbanpassung}\label{beamer:sec:color}
\Index{Farben!Beamer}

Um ein individuelleres Foliendesign zu ermöglichen, können einige Elemente
(unter Verwendung des CD-Farbkontingents)
farblich angepasst werden.
Dazu zählen die Elemente der Titelseite (mit Ausnahme des Logos),
sowie die Hintergrundfarbe der Folientitel.

Zur einfachen Farbanpassung stehen einige spezielle Optionen zur Verfügung.
Eine komplett freie Gestaltung kann bei Bedarf über Änderung der
jeweiligen Templates erfolgen. Beide Möglichkeiten werden im folgenden
erläutert.

Details über die Farben der CD-Farbklänge können dem
Kapitel~\ref{chap:tubscolors} entnommen werden.

\subsection{Anpassung über Optionen}

Über die Angabe von Farbargumenten können alle zur farblichen Anpassung
vorgesehenen Elemente nach einem vordefinierten \emph{Farbschema}
verändert werden.
Dazu kann ein Farbklang sowie eine Variante gewählt werden.

Die Optionen gliedern sich in 3 Kategorien:
\begin{itemize}
  \item \emph{Farbe} -- Auswahl des Farbklangs\\
    Optionen: \texttt{yellow}, \texttt{green}, \texttt{blue}, \texttt{violet}
  \item \emph{Helligkeit} -- Auswahl der Helligkeitsabstufung
    innerhalb des gewählten Farbklangs\\
    Optionen: \texttt{dark}, \texttt{medium}, \texttt{light}
  \item \emph{Elemente} -- Färbung von Fußbereich/Kopfbereich aktivieren\\
    Diese wurden bereits in \chaptername~\ref{sec:titelfolie} und \ref{subsec:head} beschrieben.\\
    Optionen: \texttt{colorhead}, \texttt{colorfoot}
\end{itemize}

% Weitere Details zu Verwendung der Optionen sind im Kapitel \ref{beamer:optionen}
% zu finden.

\paragraph{Farbklang}
Es stehen vier verschiedene Farbklänge mit aufeinander abgestimmten Farbtonwerten
und Varianten zur Verfügung.

\begin{Declaration}
  \Option{orange}~/~\Option{yellow}\\
  \Option{green}\\
  \Option{blue}\\
  \Option{violet}
\end{Declaration}

Mit den Optionen \Option{orange} (alternativ \Option{yellow}), \Option{green}, \Option{blue}
oder \Option{violet} erfolgt die Auswahl des
gelben, grünen, blauen oder violetten Farbklangs als Farbschema.

\paragraph{Variante}
Für jeden Farbklang gibt es 3 verschiedene Varianten, die sich in ihrem Ton
und ihrer Helligkeit unterscheiden.

\begin{Declaration}
  \Option{dark}\\
  \Option{medium}\\
  \Option{light}
\end{Declaration}

Auswahl der hellen, mittleren bzw. dunklen Farbreihe aus gewähltem Farbklang.

\begin{example}
\lstinline!\usetheme[green,light]{tubs}! Wählt den grünen Farbklang und die
helle Farbreihe. Damit wird das Farbschema auf hellgrün gesetzt.
\end{example}

\paragraph{Farbmodell}
Standardmäßig werden Folien in einem optimierten RGB-Farbmodell ausgegeben.
Bei Bedarf kann ein anderes Farbmodell zur Darstellung gewäht werden.


\begin{Declaration}
  \Option{rgb}\\
  \Option{rgbprint}\\
  \Option{cmyk}
\end{Declaration}

Mit der Option \Option{rgb} erfolgt die Ausgabe im Beamer-optimierten RGB-Farbmodell.
Dies entspricht der Standardeinstellung.
Mit der Option \Option{rgbprint} erfolgt die Ausgabe in RGB-Druckfarben.
Mit der Option \Option{cmyk} erfolgt die Ausgabe im CMYK-Farbmodell.

\clearpage
\subsubsection{Beispiele}

% \begin{minipage}{0.55\textwidth}
% \lstinline!\usetheme[green,dark]{tubs}!
% \end{minipage}\hfill
\begin{center}
\begin{minipage}{0.49\textwidth}
\fboxsep0mm
\fbox{\includegraphics[width=0.99\textwidth,page=1]{%
  examples/colorscheme1.pdf}}
\end{minipage}\hfill
\begin{minipage}{0.49\textwidth}
\fboxsep0mm
\fbox{\includegraphics[width=0.99\textwidth,page=2]{%
  examples/colorscheme1.pdf}}
\end{minipage}\medskip\\
{\ttfamily \textbackslash usetheme[green,dark]\{tubs\}}
\end{center}

\begin{center}
\begin{minipage}{0.49\textwidth}
\fboxsep0mm
\fbox{\includegraphics[width=0.99\textwidth,page=1]{%
  examples/colorscheme2.pdf}}
\end{minipage}
\begin{minipage}{0.49\textwidth}
\fboxsep0mm
\fbox{\includegraphics[width=0.99\textwidth,page=2]{%
  examples/colorscheme2.pdf}}
\end{minipage}\medskip\\
{\ttfamily \textbackslash usetheme[green,dark,colorfoot,colorhead]\{tubs\}}
\end{center}

\begin{center}
\begin{minipage}{0.49\textwidth}
\fboxsep0mm
\fbox{\includegraphics[width=0.99\textwidth,page=1]{%
  examples/colorscheme3.pdf}}
\end{minipage}
\begin{minipage}{0.49\textwidth}
\fboxsep0mm
\fbox{\includegraphics[width=0.99\textwidth,page=2]{%
  examples/colorscheme3.pdf}}
\end{minipage}\medskip\\
{\ttfamily%
  \textbackslash usetheme[blue,colorhead]\{tubs\}}
\end{center}

\clearpage
\subsection{Erweiterte Anpassung über Farb-Templates}

Eine individuellere Farbanpassung ist über die Änderung der entsprechenden
Templates möglich.

\subsubsection{Paletten}

Das Grundfarbschema eines Beamer-Templates wird über sog. Paletten definiert.
Es existieren eine primäre, sekundäre, tertiäre und quaternäre Palette,
die jeweils eine bestimmte Farbgebung beeinflussen.

In den \tubslatex-Vorlagen erfüllen die Paletten die folgenden Funktionen:

\begin{description}
  \item[\ttfamily palette primary]
    Die primäre Palette definiert die Farbgebung
    von Kopfleiste und Titelleiste auf Inhaltsfoline,
    sowie von Part-Seiten.
  \item[\ttfamily palette secondary]
    Die sekundäre Palette definiert die Farbgebung der einzelnen
    Segmente der Titelseite.
  \item[\ttfamily palette tertiary]
    Die tertiäre Palette definiert die Farbe der Fußzeilen-Schriftelemente.
  \item[\ttfamily palette quaternary]
    Die quaternäre Palette wird in den Vorlagen nicht verwendet.
\end{description}

Die Paletten können jeweils mittels \Macro{setbeamercolor}
unter Angabe der Palettenbezeichnung modifiziert werden.

\begin{example}
\begin{lstlisting}
  \setbeamercolor{palette primary}{fg=tuOrange, bg=tuGreen}
\end{lstlisting}
\end{example}

\subsubsection{Einzelelemente}

Neben den Paletten können natürlich auch einzeln Elemente manuell angepasst
werden. Die wichtigsten sind in der nachfolgenden Tabelle aufgeführt:

\begin{tabularx}{\textwidth}{lX}
  Bezeichner & Wirkung\\
  \midrule
  \ttfamily titlebarfirst &
    Der Hintergrund des Grafik-Segments auf der Titelseite\\
  \ttfamily titlebarsecond &
    Das Segment unterhalb des Grafik-Segments auf der Titelseite\\
  \ttfamily titlebarlow &
    Die schmale Fußleiste auf der Titelseite\\
  \ttfamily titlehead &
    Die Kopfzeile der Inhaltsseite\\
  \ttfamily titlelike &
    Die Titelzeile der Inhaltsseite\\
  \ttfamily block body &
    Hintergrundfarbe von Blöcken\\
  \ttfamily block title &
    Titelfarbe von Blöcken\\
  \ttfamily block body alerted &
    Hintergrundfarbe von alerted-Blöcken\\
  \ttfamily block title alerted &
    Titelfarbe von alerted-Blöcken\\
  \ttfamily block body example &
    Hintergrundfarbe von example-Blöcken\\
  \ttfamily block title example &
    Titelfarbe von example-Blöcken\
\end{tabularx}

Die Anpassung weiterer Elemente ist natürlich möglich,
wird aber nicht empfohlen.
Sie können bei Bedarf den entsprechenden Paket-Dokumentationen entnommen werden.


% \section{Template-Optionen}\label{beamer:optionen}
% 
% Auflistung aller template-spezifischer Optionen zur Übergabe an die
% Dokumentenklasse bzw. das Beamer-theme.
% 
% Die template-spezifischen Argumente sollten bevorzugt
% über \lstinline!\usetheme! übergeben werden.
%TODO: Hinweis zu Beamer-Option!?

\section{Schrift}

Die voreingestellte Standardschrift der Vorlagen ist Arial in Schriftgröße 11pt.
Sie kann bei Bedarf auf Nexus gewechselt werden.

\begin{Declaration}
  \Option{arial}\\
  \Option{nexus}
\end{Declaration}

Mit der Option \Option{arial} wird die Schriftart Arial verwendet.
Die Option \Option{nexus} wählt Nexus als Schriftart aus.

\subsection{Zusätzliche Schriftgrößen}
\Index{Schriftgrößen!Beamer}
\CommandIndex{microsize}
\CommandIndex{nanosize}

Das font-Template legt zwei zusätzliche Schriftgrößen \lstinline{\microsize}
und \lstinline{\nanosize} an, die jeweils noch kleiner sind als alle restlich
verfügbaren.

Sie werden intern verwendet, können aber auch bei begründeter
Notwendigkeit von Autoren verwendet werden,
z.B. für kleinste Quellenangaben etc.

\subsection{Elemente anpassen}

Mit Hilfe der Beamer-Fonts können bei Notwendigkeit auch einzelne Elemente
separat in ihrer Schrift angepasst werden.
Dieses Mittel sollte jedoch mit Bedacht eingesetzt werden.
Informationen zur Anpassung können der Dokumentation zum Beamer-Paket entnommen
werden. Die Beschreibung modifizierbarer Elemente findet sich in der
Paket-Dokumentation zum Font-Template.

\section{Hinweise}

\subsection{Inner-, Outer-, Font-, Color-Template}

Die einzelnen Teil-Templates des beamer-Templates können,
entsprechend der beamer-Konventionen, selbstverständlich
auch unabhängig voneinander verwendet werden.
So kann zum Beispiel das color-Template in Kombination mit einem der
Standard-Template verwendet werden, etc.

Die Kombination mit anderen Templates ist im allgemeinen jedoch nicht
CorporateDesign-konform, weswegen davon \emph{ausdrücklich} abgeraten wird.

\subsection{Skalierbarkeit (Format)}
\Index{16:9}
\Index{Format!Beamer}

Die Beamer-Option \lstinline{aspectratio}, mit der die Präsentation auch auf
andere Seitenformate gebrachte werden kann, wird vom Layout generell
unterstützt, jedoch ist zu bedenken, dass dafür ein Titelbild in einem anderen
Seitenverhältnis benötigt wird, wenn es verlustfrei dargestellt werden soll
(siehe Tabelle \ref{tab:picratio}).

\begin{hint}
Die Option \lstinline{aspectratio} steht erst ab beamer-Version 3.10
zur Verfügung.
\end{hint}


\subsection{columns-Umgebung}
\Index{columns-Umgebung}

Die Verwendung der columns-Umgebung ist in vielen Fällen sinnvoll.
Jedoch sollte die Umgebung nur mit der Option \lstinline{onlytextwidth}
verwendet werden, da sonst ein zusätzlicher Abstand zwischen den
Spalten hinzugefügt werden und die Gesamtbreite zu groß wird:

\begin{lstlisting}[morekeywords={onlytextwidth},keywordstyle=\color{tuOrange}]
\begin{columns}[onlytextwidth]
  \column{0.5\textwidth}
  \column{0.5\textwidth}
\end{columns}
\end{lstlisting}


\subsection{Listings in Folien}
\Index{Listings!Beamer}

Auch wenn dies in der beamer-Dokumentation ausdrücklich erläutert ist,
sollte hier noch einmal erwähnt werden, dass bei Gebraucht von Listings etc.
in Folien die Option \lstinline{fragile} übergeben werden muss.

\begin{lstlisting}[morekeywords={fragile},keywordstyle=\color{tuOrange}]
\begin{frame}[fragile]{Titel}
  % Listings, etc.
\end{frame}
\end{lstlisting}



\subsection{PDF-Titel}

Die standardmäßige Erstellung des PDF-Titels von beamer ist dahingehend
abgeändert, dass, sofern vorhanden, die \emph{Kurzversion} des Präsentationstitels
als PDF-Titel verwendet wird.

Die Verwendung der Kurzversion ermöglicht es zum Beispiel,
kleine Grafiken in den Titel einzubringen und
trotzdem einen korrekten PDF-Titel zu erhalten.


\section{Minimalbeispiel}% TODO: Anhang?!

Im folgenden ist der Code eines Minimalbeispiels aufgeführt, zusammen mit den
daraus erzeugten Folien.

\begin{verbatim}
\documentclass{beamer}

\usetheme{tubs}
\usepackage[T1]{fontenc}
\usepackage[utf8x]{inputenc}

\title{Corporate Design}
\subtitle{Jetzt mit \LaTeX}
\author{Max Mustermann}
\titlegraphic{\includegraphics[width=\titlegraphicswidth]{titlepicture}}
\logo{\includegraphics[height=\logoheight]{institut.jpg}}

\begin{document}

\begin{frame}[plain]
  \titlepage
\end{frame}

\begin{frame}{Inhaltsseite}
  \begin{itemize}
    \item Hier steht der Inhalt
    \item Hier nicht
    \item Weitere Informationen
  \end{itemize}
\end{frame}

\end{document}
\end{verbatim}

\begin{center}
  \fbox{\includegraphics[width=0.9\textwidth]{examples/titelseite.pdf}}

  \fbox{\includegraphics[width=0.9\textwidth]{examples/inhaltsseite.pdf}}
\end{center}
  % Kapitel: 'Präsentationen'


\chapter{Hintergrundlayout}

\begin{Declaration}
  \Macro{showtubslogo}\OParameter{Position}
\end{Declaration}

Bewirkt Darstellung des TU-Logos im aktiven Layout. Die Option \PName{Position}
erlaubt die Angabe der Darstellungsseite (links/rechts). Standardmäßig
wird das Logo links bzw. innen dargestellt.

\begin{Declaration}
  \Macro{showlogo}\PParameter{Logo}
\end{Declaration}

Bewirkt Darstellung eines Individuellen Logos im aktuellen Layout.
\PName{Logo} kann dabei entweder einfacher Text oder auch ein 
mit \Makro{includegraphics} eingebundenes Bild sein.

\begin{Declaration}
  \Macro{showtopline}
\end{Declaration}

Bewirkt Darstellung einer Trennlinie zwischen Absender und Kommunikationsbereich
im aktuellen Layout.

\begin{Declaration}
  \Macro{bgelement}\OParameter{Darstellung}\PParameter{Höhe}
\end{Declaration}

Erstellt ein Hintergrundelement im Gaußraster mit angegebener \PName{Höhe}.
Der Parameter \PName{Darstellung} kann die folgenden Einstellungen verarbeiten:

\begin{Declaration}
  \KOption{bgcolor}\PName{Farbe}\\
  \KOption{bgimage}\PName{Bild-Datei}\\
  \KOption{imagefit}\PName{Darstellungsoption}
\end{Declaration}

Mit \OptionValue{bgcolor}{Farbe} wird das Hintergrundelement mit der angegebenen
Farbe gefüllt.

Die Option \OptionValue{bgimage}{Bild-Datei} erlaubt dagegen die Darstellung
eines Hintergrundbildes im Element.
Da der Darstellungsbereich fest vorgegeben ist, muss das eingebundene Bild
in diesen Bereich eingepasst werden. Dies geschieht automatisch, die Art
der Einpassung lässt sich aber mit der Option \Option{imagefit} kontrollieren.
Sie erlaubt folgende Einstellungen:


\begin{desctable}
\entry{\PValue{cropped}}{%
  Automatisches Abschneiden. Dies ist die Standardeinstellung.
}
\entry{\PValue{cropx}}{%
  Abschnitt horizontal.
}
\entry{\PValue{cropy}}{%
  Abschnitt vertikal.
}
\entry{\PValue{scaled}}{%
  Horizontale \emph{und} vertikale Skalierung.
}
\end{desctable}


\chapter{Farben}\label{chap:tubscolors}

\newcommand{\classoptionitem}[1][ ]{
  \item[\mdseries{\ttfamily%
    \textbackslash usepackage%
    {[{\color{tuRed}#1}]}%
    \{tubslogo\}}]\hfill\\
}

Die Farbdefinitionen in \tubslatex werden vom Paket \newpackage{tubscolors}
zur Verfügung gestellt. Die folgende Beschreibung bezieht sich
auf den Funktionsumfang des Paketes. Das Paket ist in allen verfügbaren
Klassen bereits korrekt eingebunden.

\newcommand{\rainbow}[2][\relax]{{\noindent\sffamily\footnotesize%
\ifx#1\relax\colorlet{fglbg}{black}\else\colorlet{fglbg}{#1}\fi
\colorbox{#2100}{\hbox to 0.188\textwidth{%
  \color{fglbg}\vphantom{Fg}#2{}100\hfill}}% 
\colorbox{#280}{\hbox to 0.188\textwidth{%
  \color{fglbg}\vphantom{Fg}#2{}80\hfill}}% 
\colorbox{#260}{\hbox to 0.188\textwidth{\vphantom{Fg}#2{}60\hfill}}% 
\colorbox{#240}{\hbox to 0.188\textwidth{\vphantom{Fg}#2{}40\hfill}}% 
\colorbox{#220}{\hbox to 0.188\textwidth{\vphantom{Fg}#2{}20\hfill}}\\% 
}}

\section{Verfügbare Farben}

Der Farbklang der TU-Braunschweig ist in eine Primär- und einen
Sekundärfarbbereich aufgeteilt.

Die Primärfarben bilden dabei Rot, Schwarz und Weiß, sowie in
20-Prozent-Schritten abgestufte Grautöne. Die Primärfarben dienen
vor allem zur Auszeichnung von Hintergrund, Textfarbe und dem TU-Logo.
Zur individuellen Gestaltung von Dokumenten ist der Sekundärfarbbereich
vorgesehen.

Die Sekundärfarben setzen sich aus 12 weiteren aufeinander abgestimmten
Farben zusammen, die 
in 4 Farbklänge (Gelb-Orange, Grün, Blau und Violett) mit je 3 Basisfarben
aufgeteilt sind.
Alle Sekundärfarben können in 20-Prozent-Schritten aufgehellt werden.

Die Namen über die die einzelnen Farben angesprochen werden können, sind in den
Beispielfeldern angegeben.

\subsection{Primärfarben}

{\sffamily\footnotesize%
\colorbox{tuRed}{\hbox to 0.188\textwidth{%
  \vphantom{Fg}tuRed\hfill}}%
\colorbox{tuBlack}{\hbox to 0.188\textwidth{%
  \color{white}\vphantom{Fg}tuBlack\hfill}}%
\fcolorbox{tuBlack}{tuWhite}{\hbox to 0.188\textwidth{%
  \vphantom{Fg}tuWhite\hfill}}\\%
}

\rainbow[tuWhite]{tuGray}

Zur Vereinfachung sind noch die Farben \lstinline{tuGray} und
\lstinline{tuLightGrey} definiert, die den Farben \lstinline{tuGray60} und
\lstinline{tuGray20} entsprechen.

Alle Graytöne sind darüber hinaus auch in britischer Schreibweise nutzbar
(\lstinline{tuGrey}).

\paragraph{Hinweis:}
\lstinline{tuRed} ist nicht zu verwechseln mit \lstinline{tuRed100} aus dem
Sekundärfarbbereich. Es handelt sich dabei um eine komplett andere Farbe.

\pagebreak
\subsection{Sekundärfarben}

  tuOrange\ldots\\
  \colorshow{Orange}{Light}
  \colorshow{Orange}{Medium}
  \colorshow{Orange}{Dark}\\[-1ex]
  tuBlue\ldots\\
  \colorshow{Blue}{Light}
  \colorshow{Blue}{Medium}
  \colorshow{Blue}{Dark}\\[-1ex]
  tuGreen\ldots\\
  \colorshow{Green}{Light}
  \colorshow{Green}{Medium}
  \colorshow{Green}{Dark}\\[-1ex]
  tuViolet\ldots\\
  \colorshow{Violet}{Light}
  \colorshow{Violet}{Medium}
  \colorshow{Violet}{Dark}
%   \caption{Im CD definierte Farben und deren Benennung (Auszug)}


Zusätzlich kann als Paketoption ein Farbklang ausgewählt werden, dessen Farben
dann über die Werte  \\\lstinline{tuSecondaryLight},
\lstinline{tuSecondaryMedium}, \lstinline{tuSecondaryDark}, sowie die
entsprechenden Prozentualwert \\(\lstinline{tuSecondaryLight20},
\lstinline{tuSecondaryLight40}, \ldots) angesprochen werden können.
Dies erlaubt eine flexible Verwendung der 4 Sekundärfarbklänge.

In folgendem Beispiel wurde \lstinline{blau} als Farbklang ausgewählt:

\colorshow{Secondary}{Light}
\colorshow{Secondary}{Medium}
\colorshow{Secondary}{Dark}\\[-1ex]

\paragraph{Hinweise:}
Die Farben des gelb-orange-Farbklangs können entsprechend der anderen
Farbmodelle auch noch einheitlich über die Alternativnamen
\lstinline{tuOrangeLight}, \lstinline{tuOrange},
\lstinline{tuOrangeDark}, sowie de entsprechenden Prozentwerte aufgerufen
werden.

Außerdem können jeweils die mittleren Farbwerte der Farbklänge auch über den
Zusatz \lstinline{Medium} angesprochen werden (statt \lstinline{tuGreen100} auch
\lstinline{tuGreenMedium100}).

Bei allen 100-Prozent-Farben (außer \lstinline{tuRed}) kann die
Zahl weggelassen werden (statt \lstinline{tuGreenLight100} auch 
\lstinline{tuGreenLight}).

\subsection{Farbmodelle}

In den Paketoptionen kann zwischen 3 Farbmodellen gewählt werden.
Das Standardmodell stellt die CD-konformen RGB-Farbwerte zur Verfügung.

Für die Ausgabe in CMYK-Farben steht die Option \lstinline!cmyk! zur Verfügung.

Darüber hinaus gibt es noch ein RGB-Farbschema, das für die Ausgabe auf
Beamern optimiert ist. Es kann über die Option \lstinline!rgbbeamer! geladen
werden.
  % Kapitel: 'Farben'

%Befehle für Glossar
\newglossaryentry{glos:siegelbandlogo}{%
  name=Siegelbandlogo,
  description={\tubslogo}
}
\newglossaryentry{glos:gaussraster}{%
  name={Gau\ss raster},
  description={Auf der \glslink{glos:summenformel}{gauß'schen Summenformel} basierende Unterteilung der Seite
    in Segmente. Benachbarte Segmente können beliebig zusammen gefasst werden}
}
\newglossaryentry{glos:modulsystem}{%
  name={Modulsystem},
  description={Flexibles Platzierungssystem für wissenschaftliche Plakate.
  Dabei wird der Darstellungsbereich komplett in einzelne Module
  verschiedener Größe aufgeteilt.}
}
\newglossaryentry{glos:cmyk}{%
  name={CMYK-Farbmodell},
  description={Das CMYK-Farbmodell ist ein subtraktives Farbmodell,
    das die technische Grundlage für den modernen Vierfarbdruck bildet.
    Die Abkürzung CMYK steht für die drei Farbbestandteile
    \emph{Cyan}, \emph{Magenta}, \emph{Yellow}
    und den Schwarzanteil \emph{Key} als Farbtiefe}
}
\newglossaryentry{glos:absenderbereich}{%
  name={Absenderbereich},
  description={
    Freier Bereich am oberen oder unteren Blattrand
    zur Darstellung eines Absenders (Institut/zentrale Eintrichtung).
    Ihm schließt sich direkt der \gls{glos:kommunikationsbereich} mit dem Inhalt an.\\
    Die Position des Absenderbereichs kontrolliert beim Gauß-Layout auch die
    Reihenfolge der Segmentaufteilung. Das größte Segment befindet sich immer
    auf der Seite des Absenderbereichs.
    }
}
\newglossaryentry{glos:kommunikationsbereich}{%
  name={Kommunikationsbereich},
  description={Bereich zur Darstellung von Inhalten}
}
\newglossaryentry{glos:spaltenraster}{%
  name={Spaltenraster},
  description={Abhängig vom Format kann der Textbereich einer Seite in
    6, 4 oder 2 Grundspalten geteilt werden, welche alle die selbe Breite und
    den selben Abstand zueinander haben.
    Benachbarte Grundspalten können variabel zu einer Darstellungsspalte
    zusammengefasst werden}
}
\newglossaryentry{glos:bindekorrektur}{%
  name={Bindekorrektur},
  description={}
}
\newglossaryentry{glos:sekundaerfarbklang}{%
  name={Sekund\"arfarbklang},
  description={}
}
\newglossaryentry{glos:mediaevalziffern}{%
  name={Medi\"avalziffern},
  description={Ziffern, die im Gegensatz zu Versalziffern Ober- und Unterlänge
  haben und sich dadurch im Mengentext besser in das Schriftbild einfügen als \gls{glos:versalziffern}}
}
\newglossaryentry{glos:versalziffern}{%
  name={Versalziffern},
  description={Auf der Grundlinie ausgerichtete Ziffern ohne Ober- und Unterlängen.
  Sie fügen sich daher meist schlechter in Schriftbild ein als \gls{glos:mediaevalziffern}}
}
\newglossaryentry{glos:dinlang}{%
  name={DIN lang},
  description={Bezeichnet hier ein Format von $1/3$ der Blattgröße \mbox{DIN\,A4},
  wie es zum Beispiel für Flyer (doppelt gefaltet) häufig verwendet wird.}
}
\newglossaryentry{glos:summenformel}{%
  name={Gauß'sche Summenformel},
  description={Berechnungsgrundlage für die formatübergreifende
  vertikale Aufteilung des Kommunikationsbereichs im CD (siehe \gls{glos:gaussraster}).
  \[1 + 2 + 3 + 4 + \ldots + n = \sum_{k=1}^n k = \frac{n(n+1)}{2}\]}
}

%Akronyme
\newacronym{CD}{CD}{Corporate Design}
\newglossaryentry{CMYK}{%
  type=\acronymtype,
  name={CMYK},
  description={Cyan, Magenta, Yellow, Key (siehe \gls{glos:cmyk})}
}
\newacronym{RGB}{RBG}{Red, Green, Blue}


\end{document}
