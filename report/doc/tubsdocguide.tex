\documentclass[a4paper,11pt]{tubsreprt}
\usepackage[utf8x]{inputenc}
% \usepackage[T1]{fontenc}
% \usepackage{ae}
% \usepackage[a4paper]{tubsdoc}

\usepackage{scrpage2}
\usepackage{amsmath}
\usepackage{tabularx}
\usepackage{booktabs}
\usepackage{listings}
\lstset{basicstyle=\ttfamily}
\usepackage[colorlinks=true]{hyperref}
\usepackage[ngerman]{babel}

%opening
\title{Das CorporateDesign in \LaTeX}
\subtitle{Anleitung und Dokumentation}
\author{Enrico Jörns}
\publishers{Institut fuer Lorem Ipsum} % to logo?


\makeatletter
\providecommand\marg[1]{%
  {\ttfamily\char`\{}\meta{#1}{\ttfamily\char`\}}}
\providecommand\oarg[1]{%
  {\ttfamily[}\meta{#1}{\ttfamily]}}
\def\cmd#1{\cs{\expandafter\cmd@to@cs\string#1}}
\def\cmd@to@cs#1#2{\char\number`#2\relax}
\DeclareRobustCommand\cs[1]{\texttt{\char`\\#1}}

\newenvironment{Declaration}{%
%    \end{macrocode}
% \begin{macro}{\new@element}
%   Help macro to define new Declaration elements.
%    \begin{macrocode}
  \newcommand*{\new@element}[1]{%
    \expandafter\newcommand\expandafter*\csname X##1\endcsname{}%
    \expandafter\let\csname X##1\expandafter\endcsname
    \csname ##1\endcsname
    \expandafter\newcommand\expandafter*\csname new##1\endcsname[1]{%
%      \begingroup
%        \let\ensuremath\@firstofone
%        \let\textit\@firstofone
%        \lowercase{\def\@tempa{##1}}%
%        \pdfstringdef\@tempb{\label@base.\@tempa.####1}%
%        \xdef\@currentHref{\@tempb}%
%        \Hy@raisedlink{\hyper@anchorstart{\@currentHref}\hyper@anchorend}%
%        \label{desc:\label@base.\@tempa.####1}%
%      \endgroup
      \csname X##1\endcsname{####1}\ignorespaces
    }%
    \expandafter\let\csname ##1\expandafter\endcsname\csname new##1\endcsname
  }%
  \newcommand*{\new@xelement}[2]{%
    \expandafter\newcommand\expandafter*\csname X##1\endcsname{}%
    \expandafter\let\csname X##1\expandafter\endcsname
    \csname ##1\endcsname
    \expandafter\newcommand\expandafter*\csname new##1\endcsname[2]{%
%      \begingroup
%        \let\ensuremath\@firstofone
%        \let\textit\@firstofone
%        \lowercase{\def\@tempa{##1}}%
%        \pdfstringdef\@tempb{\label@base.\@tempa.####1.####2}%
%        \xdef\@currentHref{\@tempb}%
%        \Hy@raisedlink{\hyper@anchorstart{\@currentHref}\hyper@anchorend}%
%        \label{desc:\label@base.\@tempa.####1.####2}%
%      \endgroup
      \csname X##1\endcsname{####1}{##2{####2}}\ignorespaces
    }%
    \expandafter\let\csname ##1\expandafter\endcsname\csname new##1\endcsname
  }%
%    \end{macrocode}
%    \begin{macrocode}
  \new@element{Option}%
  \new@element{Macro}%
  \new@element{Environment}%
  \new@element{Counter}%
  \new@element{FloatStyle}%
  \new@element{PLength}%
  \new@element{Variable}%
  \new@xelement{OptionValue}{\PValue}%
%    \end{macrocode}
% \end{macro}
%    \begin{macrocode}
  \ifvmode\else\par\fi\addvspace{2\baselineskip}%
  \vspace{-\baselineskip}%
  \vspace{\z@ plus \baselineskip}%
  \noindent
  \start@Declaration
  \tabular{|l|}\hline\ignorespaces
}{%
  \\\hline\endtabular\nobreak\after@Declaration\nobreak\par\nobreak
  \vspace{1.5\baselineskip}\nobreak\vspace{-\baselineskip}\nobreak%
  \vspace{0pt minus .5\baselineskip}\nobreak%
  \aftergroup\@afterindentfalse\aftergroup\@afterheading
}
\newcommand*{\start@Declaration}{\hspace{-1em}}
\newcommand*{\after@Declaration}{}
% \begin{macro}{\Macro}
% \begin{macro}{\Option}
% \begin{macro}{\KOption}
% \begin{macro}{\OptionValue}
% \begin{macro}{\Environment}
% \begin{macro}{\Counter}
% \begin{macro}{\Length}
% \begin{macro}{\PLength}
% \begin{macro}{\FloatStyle}
% \begin{macro}{\Pagestyle}
% \begin{macro}{\Variable}
% \begin{macro}{\FontElement}
% \begin{macro}{\PName}
% \begin{macro}{\PValue}
% \begin{macro}{\Parameter}
% \begin{macro}{\OParameter}
% \begin{macro}{\AParameter}
% \begin{macro}{\PParameter}
% \begin{macro}{\POParameter}
%   \begin{description}
%   \item[\cs{Macro}] \LaTeX{} or \TeX{} macro
%   \item[\cs{Option}] class or package option
%   \item[\cs{KOption}] |\KOMAoptions| option
%   \item[\cs{Environment}] \LaTeX{} environment
%   \item[\cs{Counter}] \LaTeX{} counter
%   \item[\cs{Length}] \LaTeX{} length
%   \item[\cs{PLength}] \KOMAScript{} pseudo length
%   \item[\cs{Variable}] \KOMAScript{} variable
%   \item[\cs{FontElement}] \KOMAScript{} element that has its own font
%     selection
%   \item[\cs{PName}] name of a parameter of a macro or environment
%   \item[\cs{PValue}] value of a parameter of a macro or environment
%   \item[\cs{Parameter}] the mandatory parameter of a macro or environment
%   \item[\cs{OParameter}] the optional parameter of a macro or environment
%   \item[\cs{AParameter}] the alternativ parameter of a macro or environment
%   \item[\cs{PParameter}] the part-of-command parameter of a macro or
%     environment
%   \end{description}
%    \begin{macrocode}
\DeclareRobustCommand*{\Macro}[1]{\mbox{\texttt{\char`\\#1}}}
\DeclareRobustCommand*{\Option}[1]{\mbox{\texttt{#1}}}
\DeclareRobustCommand*{\KOption}[1]{\mbox{\Option{#1}\texttt=}}
\DeclareRobustCommand*{\OptionValue}[2]{\mbox{\texttt{#1=#2}}}
\DeclareRobustCommand*{\FloatStyle}[1]{\mbox{\texttt{#1}}}
\DeclareRobustCommand*{\Pagestyle}[1]{\mbox{\texttt{#1}}}
\DeclareRobustCommand*{\Environment}[1]{\mbox{\texttt{#1}}}
\DeclareRobustCommand*{\Counter}[1]{\mbox{\texttt{#1}}}
\DeclareRobustCommand*{\Length}[1]{\mbox{\texttt{\char`\\#1}}}
\DeclareRobustCommand*{\PLength}[1]{\mbox{\PValue{#1}}}
\DeclareRobustCommand*{\Variable}[1]{\mbox{\PValue{#1}}}
\DeclareRobustCommand*{\FontElement}[1]{\PValue{#1}}
\DeclareRobustCommand*{\PName}[1]{\texttt{\textit{#1}}}
\DeclareRobustCommand*{\PValue}[1]{\texttt{#1}}
\DeclareRobustCommand*{\Parameter}[1]{\texttt{\{}\PName{#1}\texttt{\}}}
\DeclareRobustCommand*{\OParameter}[1]{%
  \texttt{[%]
  }\PName{#1}\texttt{%[
    ]}}
\DeclareRobustCommand*{\AParameter}[1]{%
  \texttt{(%)
  }\PName{#1}\texttt{%(
    )}}
\DeclareRobustCommand*{\PParameter}[1]{\texttt{\{#1\}}}
\DeclareRobustCommand*{\POParameter}[1]{\texttt{[#1]}}
%    \end{macrocode}
% \end{macro}
% \end{macro}
% \end{macro}
% \end{macro}
% \end{macro}
% \end{macro}
% \end{macro}
% \end{macro}
% \end{macro}
% \end{macro}
% \end{macro}
% \end{macro}
% \end{macro}
% \end{macro}
% \end{macro}
% \end{macro}
% \end{macro}
% \end{macro}
% \end{macro}
% NOTE: taken from scrguide.cls
\makeatother

\newcommand{\tubslatex}{\emph{tubslatex}}

\begin{document}

\maketitle[plain]

\tableofcontents
% \begin{abstract}
%   Das Paket \lstinline{tubsdoc} stellt umfangreiche Möglichkeiten zum Erstellen
%   von Dokumenten im CorporateDesign der TU Braunschweig zur Verfügung.
% \end{abstract}

\newcommand{\newdocumentclass}[1]{\textcolor{tuRed}{\lstinline{#1}}}

\chapter{Schnellstart}

\paragraph{Dokumente}
Die schnellste Methode ein Dokument im Corporate Design zu erstellen ist
das Laden einer der zur Verfügung stehenden Dokumentenklassen.
Für die Erstellung von Textdokumenten sind dies \newdocumentclass{tubsartcl},
\newdocumentclass{tubsreprt} und \newdocumentclass{tubsbook}.

\paragraph{Poster}
Die Klasse \newdocumentclass{tubsposter} kann zum Erstellen von Postern verwendet werden.
% tubsflyer?

\paragraph{Papierformat}
Die zu verwendende Papierformat sollte dabei als optionales Argument mit
übergeben werden. Für ein Dokument in DIN A4 ist dies \texttt{a4paper}.
Zur Verfügung stehen alle Papierformate A0 bis A6.

Standardmäßig wird die Titelseite in Dokumenten einfach mit dem Logo und einer
roten Trennlinie zwischen Kommunikations- und Absenderbereich versehen.

\paragraph{Präsentationen}
Um Präsentation zu erstellen existiert ein Style für \LaTeX-Beamer.
Dieser wird einfach mit \lstinline!\usetheme{tubs}! geladen.


\chapter{Dokumente}

Einfache Textdokumente können mit den Klassen \newdocumentclass{tubsartcl},
\newdocumentclass{tubsreprt} und \newdocumentclass{tubsbook} erstellt werden.

Da Dokumente normalerweise gedruckt werden und ggf. auch gebunden, werden
sie standardmäßig mit einer kleinen Bindekorrektur gesetzt, sodass das
Logo beim Drucken oder Abheften nicht abgeschnitten wird.
Diese Bindekorrektur kann mittels der Paketoption \texttt{bcor} angepasst werden.
Mit \texttt{bcor=0mm} wird sie beispielsweise deaktiviert.

\section{Titelseite}

Titelseiten können bei \LaTeX\ generell auf zwei verschiedene Arten erstellt
werden; entweder mit Hilfe des Befehls \lstinline{\maketitle} oder mit
der Umgebung \lstinline{titlepage}. Beide Varianten werden von tubslatex
unterstützt und leicht modifiziert.

\begin{Declaration}
  \Macro{maketitle}\OParameter{style}
\end{Declaration}

Die einfache Verwendung von \lstinline{\maketitle} erzeugt eine Titelseite
mit dem TU-Logo und einer roten Trennline zwischen Absender- und
Kommunikationsbereich.

Mit Hilfe des optionalen Arguements \PName{style} kann die Darstellung
der Titelseite geändert werden, indem aus einer Reihe vordefinierter Styles
ausgewählt wird.\bigskip

Neben den standardmäßig definierten Elementen für Haupttitelseiten werden
in \tubslatex noch ein paar zusätzliche Definiert.

\begin{Declaration}
  \Macro{logo}\Parameter{logo}\\
  \Macro{titlepicture}\Parameter{file}\\
  \Macro{titleabstract}\Parameter{text}
\end{Declaration}

\Macro{logo} dient zur Darstellung eines zusäzlichen Absenders als Schrift
oder Bild. Es wird in allen Stilen im Absenderbereich auf der dem
TU-Logo gegenüberliegenden Seite dargestellt.
Mit \Macro{titlepicture} kann eine Bilddatei angegeben werden, die bei
Verwendung eines entsprechenden Styles auf der Titelseite dargestellt werden.
\Macro{titeabstract} erlaubt die Darstellung eines kurzen zusammenfassenden
Textes auf der Titelseite, sofern der gewählte Stil dies unterstützt.


% \noindent\hspace*{-\parindent}\fbox{\cmd{\maketitle}\oarg{asdf}\marg{asdf}}
\chapter{Poster}

Poster werden mit Hilfe der Klasse \newdocumentclass{tubsposter} erstellt.
Dabei kann Mittels der Option \texttt{style} zwischen der Darstellung als
normales Poster oder als wissenschaftliches Poster gewählt werden.


\section{Format}

\section{Farben}

\section{Befehle}

\section{Optionen}


\chapter{Präsentationen}

\end{document}
