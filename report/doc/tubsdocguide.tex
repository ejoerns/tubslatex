\documentclass[a4paper,11pt]{scrartcl}
\usepackage[utf8x]{inputenc}
\usepackage[T1]{fontenc}
% \usepackage{ae}
\usepackage[a4paper]{tubsdoc}

\usepackage{scrpage2}
\usepackage{amsmath}
\usepackage{tabularx}
\usepackage{booktabs}
\usepackage{listings}
\lstset{basicstyle=\ttfamily}
\usepackage[colorlinks=true]{hyperref}
\usepackage[ngerman]{babel}

%opening
\title{tubsdoc -- Dokumentation}
\author{Enrico Jörns}
\publishers{Institut fuer Lorem Ipsum} % to logo?
\subtitle{Another approach}

\begin{document}

\maketitle


\begin{abstract}
  Das Paket \lstinline{tubsdoc} stellt umfangreiche Möglichkeiten zum Erstellen
  von Dokumenten im CorporateDesign der TU Braunschweig zur Verfügung.
\end{abstract}

\newcommand{\newdocumentclass}[1]{\textcolor{tuRed}{\lstinline{#1}}}

\section{Schnellstart}

Die schnellste Methode ein Dokument im CorporateDesign zu erstellen ist
das Laden einer der zur Verfügung stehenden Dokumentenklassen.
Für die Erstellung von Textdokumenten sind dies \newdocumentclass{tubsartcl},
\newdocumentclass{tubsreprt} und \newdocumentclass{tubsbook}.
Die Klasse \newdocumentclass{tubsposter} kann zum Erstellen von Postern verwendet werden.
% tubsflyer?


Die Poster-Klasse verwendet ein anderes Backend, dies kann aber durch Angabe
dier Options xy auch in jedem anderen Dokument verwendet werden.

\section{Titelseite}

\section{Format}

\section{Farben}

\section{Befehle}

\section{Optionen}


\end{document}
