\chapter{Dokumente}

Einfache Textdokumente können mit den Klassen \newdocumentclass{tubsartcl},
\newdocumentclass{tubsreprt} und \newdocumentclass{tubsbook} erstellt werden.

% \section{Bindekorrektur und Marginalen}

Da Dokumente normalerweise gedruckt werden und ggf. auch gebunden, werden
sie standardmäßig mit einer kleinen Bindekorrektur gesetzt, sodass das
Logo beim Drucken oder Abheften nicht abgeschnitten wird.
Diese Bindekorrektur kann mittels der Paketoption \texttt{bcor} angepasst werden.
Mit \texttt{bcor=0mm} wird sie beispielsweise deaktiviert.



\section{Titelseite}

Titelseiten können bei \LaTeX\ generell auf zwei verschiedene Arten erstellt
werden; entweder mit Hilfe des Befehls \lstinline{\maketitle} oder mit
der Umgebung \lstinline{titlepage}. Beide Varianten werden von tubslatex
unterstützt und leicht modifiziert.

\begin{Declaration}
  \Macro{maketitle}\OParameter{style}
\end{Declaration}

Die einfache Verwendung von \lstinline{\maketitle} erzeugt eine Titelseite
mit dem TU-Logo und einer roten Trennlinie zwischen Absender- und
Kommunikationsbereich.

Mit Hilfe des optionalen Arguments \PName{style} kann die Darstellung
der Titelseite geändert werden, indem aus einer Reihe vordefinierter Styles
ausgewählt wird.\bigskip

Neben den standardmäßig definierten Elementen für Haupttitelseiten werden
in \tubslatex noch ein paar zusätzliche definiert.

\begin{Declaration}
  \Macro{logo}\Parameter{logo}\\
  \Macro{titlepicture}\Parameter{file}\\
  \Macro{titleabstract}\Parameter{text}
\end{Declaration}

\Macro{logo} dient zur Darstellung eines zusätzlichen Absenders als Schrift
oder Bild. Es wird in allen Stilen im Absenderbereich auf der dem
TU-Logo gegenüberliegenden Seite dargestellt.
Mit \Macro{titlepicture} kann eine Bilddatei angegeben werden, die bei
Verwendung eines entsprechenden Styles auf der Titelseite dargestellt werden.
\Macro{titeabstract} erlaubt die Darstellung eines kurzen zusammenfassenden
Textes auf der Titelseite, sofern der gewählte Stil dies unterstützt.

\begin{Declaration}
  \XMacro{begin}\PParameter{\Environment{titlepage}}\\
  \quad\dots\\
  \XMacro{end}\PParameter{titlepage}
\end{Declaration}

Mit der \Environment{titlepage}-Umgebung können, wie von den Standardklassen
gewohnt, Titelseiten definiert werden.
% TODO: \showtubslogo  etc.

\begin{Declaration}
  \XMacro{begin}\PParameter{\Environment{titlerow}}%
    \OParameter{options}%
    \Parameter{gaussheight}\\
  \quad\dots\\
  \XMacro{end}\PParameter{titlerow}
\end{Declaration}

Die Umgebung \Environment{titlerow} erlaubt es dabei, die Titelelemente im 
Gaußraster anzulegen. Der Parameter \PName{gaussheight} gibt dabei
die Höhe des jeweiligen Elements in Segmenten an. Die Position der Elemente
ergibt sich aus der Reihenfolge der Definition.
Mit dem optionalen Parameter \PName{options} können Einstellung wie die Hintergrundfarbe oder ein Hintergrundbild übergeben werden.

\subsection{Vordefiniert Titel-Styles}

Es sind 3 einfache Styles vordefiniert.

\begin{center}
  \fboxsep0mm
  \begin{minipage}[t]{0.33\textwidth}
    \centering\sffamily
    \fbox{%
      \includegraphics[width=0.95\textwidth]{examples/article1.pdf}}
    [default]
  \end{minipage}%
  \begin{minipage}[t]{0.33\textwidth}
    \centering\sffamily
    \fbox{%
      \includegraphics[width=0.95\textwidth]{examples/titlestyle_image.pdf}}
    [image]
  \end{minipage}%
  \begin{minipage}[t]{0.33\textwidth}
    \centering\sffamily
    \fbox{%
      \includegraphics[width=0.95\textwidth]{examples/titlestyle_imagetext.pdf}}
    [imagetext]
  \end{minipage}

\end{center}

\section{Kopf-/ Fußzeile}

Standardmäßig wird die Fußzeile komplett leer gelassen und die Kopfzeile
wird mit Seitennummer rechts und Kapitelname links gesetzt. Bei zweiseitigem
Layout gilt dies für die ungeraden Seiten, gerade Seiten werden entsprechend
mit Seitennummer links und Kapitelname rechts gesetzt.

\begin{Declaration}
  \Macro{ihead}\Parameter{innen}\\
  \Macro{ohead}\Parameter{außen}
\end{Declaration}

Bei Bedarf können die Kopfzeilen individuell angepasst werden.
Dies geschieht mit den Befehlen \Macro{ihead} und \Macro{ohead}, welche
auf den gleichnamigen Kommandos aus dem Koma-Skript aufbauen und für eine
korrekte vertikale Positionierung sorgen.

\begin{Declaration}
  \Macro{headsepline}
\end{Declaration}


Die standardmäßig gesetzten kurzen Linien am jeweils oberen äußeren Ende des
Kopfbereiches sind im Makro \Macro{headsepline} definiert.
Dies kann bei Bedarf überschrieben werden.


\section{Seitenlayout}

Eine Seite im Corporate Design unterteilt sich grundlegend in 2 Bereiche.
Der Absenderbereich und der Kommunikationsbereich. Der Absenderbereich wird zur
Darstellung des TU-Siegelbandlogos und des Logos eines speziellen Instituts
oder einer Abteilung verwendet. Im Textteil dient er teilweise als Bereich für
Inhalte einer Kopf-/Fußzeile.

Der Absenderbereich kann entweder am Anfang oder am Ende der Seite platziert
werden. Dies beeinflusst auch das Gaußraster, es beginnt jeweils mit dem
größten Segment am Absenderbereich.

\begin{Declaration}
  \Option{a0paper}\\
  \Option{a1paper}\\
  \Option{a2paper}\\
  \Option{a3paper}\\
  \Option{a4paper}\\
  \Option{a5paper}\\
  \Option{a6paper}
\end{Declaration}

Für die Auswahl des verwendeten Papierformats stehen alle DIN A-Größen von
0 bis 6 zur Verfügung. Standardeinstellung ist \Option{a4paper}.

\begin{Declaration}
  \Option{landscape}
\end{Declaration}

Schaltet das Dokument in Querformat-Darstellung.

\begin{minipage}{0.45\textwidth}\sffamily\centering
  {\fboxsep0mm\fbox{\includegraphics[width=\textwidth]{examples/article_landscape.pdf}}}\\
  Dokument im Querformat
\end{minipage}


\begin{important}
  Im Querformat hat das Gaußraster eine abweichende
  Segmentanzahl. Für alle gängigen Formate sind im Querformat \emph{6 Segmente}
  definiert. Daher lösen Layouts, die für hochformatige Dokumente
  geschrieben wurden, im Querformat einen Fehler aus.
\end{important}


\begin{Declaration}
  \KOption{sender}\PName{Position}
\end{Declaration}

Mit der Option \Option{sender} kann die Position des Absenderbereiches
festgelegt werden. Mit der Einstellung \OptionValue{sender}{top} wird
der Absenderbereich am oberen Seitenende dargestellt. Dies ist auch die
Standardeinstellung. Wählt man dagegen \OptionValue{sender}{bottom}, so
wird der Absenderbereich am unteren Ende der Seite platziert

\begin{important}
  Die Positionierung des Absenderbereichs beeinflusst die Orientierung des
  Gaußrasters. Das größte Segment wird dabei immer so gesetzt, dass es direkt
  an den Absenderbereich anschließt und alle folgende von absteigender Höhe sind.
\end{important}



\begin{Declaration}
  \KOption{bcor}\PName{Wert}
\end{Declaration}

Die Bindekorrektur beschreibt einen zusätzlichen Abstand des eigentlichen
Darstellungsbereichs vom inneren Formatrand. Die Bindekorrektur
ist bei Textdokumenten standardmäßig auf Rahmenbreite voreingestellt.

Sinnvoll ist eine Bindekorrektur selbstverständlich zum einen für Bindungen,
wo sie der Breite des durch die Bindung verdeckten Bereiches entsprechen sollte.
Zum anderen ist sie aber auch für den Druckvorgang sinnvoll, da normale Drucker
keinen randlosen Druck ermöglichen. Die Bindekorrektur verhindert so \zB ein 'Abschneiden' des Siegellogos.
Soll das Dokument ohne Bindekorrektur dargestellt werden, so ist dies
mit \OptionValue{bcor}{0mm} möglich.

\begin{Declaration}
  \Option{twoside}
\end{Declaration}

Für das Setzen von zweiseitige Dokumente ist die Option \Option{twoside}
vorgesehen. Mit ihr wird von einseitigem Layout auf zweiseitiges Layout
umgeschaltet, was bedeutet, dass die Innenseite eines Dokuments abwechselnd
auf der linken und rechten Seite definiert ist. Die hat unter anderem Einfluss
auf definierte Ränder (Bindungskorrektur, Marginale).


\begin{Declaration}
  \Option{marginleft}\\
  \Option{marginright}
\end{Declaration}

Das setzen einer Marginale wird durch die Befehle \PName{marginleft} und
\PName{marginright} vereinfacht. Diese setzen jeweils auf der linken (inneren)
bzw. rechten (äußeren) Seite des Dokumentes eine Marginale, deren Breite
einem Element im Spaltenraster entspricht.


\section{Schrift}

Dokumente im Corporate Design werden allgemein in der Schrift \emph{Nexus}
gesetzt.
Diese wird standardmäßig bei allen Dokumenten geladen. Auf eine explizite
Angabe der Schriftgröße in den Dokumentenoptionen sollte nach Möglichkeit
verzichtet werden, da für die jeweiligen Papierformate spezielle
Schriftgrößen vordefiniert sind und geladen werden.
Das Laden anderer Schriftgrößen führt im Vergleich dazu teilweise zu stark abweichenden Ergebnissen.

Für wichtige Standard-Elemente gibt es des weiteren vordefinierte Koma-Fonts.


\begin{desctable}
\entry{\PValue{headline}}{%
  Einfache Überschrift (groß)
}
\entry{\PValue{subheadline}}{%
  Unterüberschrift
}
\entry{\PValue{institute}}{%
  Institutsname im Logo-Bereich des Absenders.
}
% \entry{\PValue{infotext}}{%
% }
% \entry{\PValue{infotextitem}}{%
% }
\entry{\PValue{copytext}}{%
  Mengentext
}
\end{desctable}


\subsection{Farben}

Für eine Übersicht über die verfügbaren Farben siehe \ref{sec:colors}.%TODO...

\begin{Declaration}
  \Option{mono}
\end{Declaration}

Schwarz-Weiß-Darstellung der Standardelemente

\begin{Declaration}
  \Option{cmyk}
\end{Declaration}

CMYK-Darstellung der Standardelemente
