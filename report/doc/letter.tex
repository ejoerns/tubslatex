\chapter{Briefe}

Briefe können mit der Dokumentenklasse \newdocumentclass{tubslttr2} erstellt
werden.

Der Aufbau eines Briefes unterscheidet sich zum Teil grundlegend von dem eines
normalen Dokumentes, weswegen hierfür eine spezielle Klasse vorliegt%
\footnote{Das etwas spezielle Interface liegt in der zugrunde liegenden 
KOMA-Skript-Klasse scrlttr2 begründet.}, deren Benutzung sich ein wenig von
der der anderen unterscheidet.

Die Absender-Informationen eines Briefes können mit Hilfe einer Reihe von
Variablen festgelegt werden. Näher geht darauf das Kapitel \ref{sec:lettervars}
ein.

Absender-Informationen variieren naturgemäß von Institut zu Institut
und vor allem von Person zu Person. Innerhalb eines Instituts und insbesondere
bei gleichem individuellen Absender sind diese Informationene jedoch sehr
konstant. Daher gibts es die Möglichkeit, die Absender-Informationen in
separaten Dateien als Vorlage vorzudefinieren. Mehr Informationen dazu gibt
das Kapitel \ref{sec:lco}

\section{Paketoptionen}

Generell können die meisten Optionen, die auch die zugrundeliegenden Klasse
\texttt{scrlttr2} akzeptiert, übergeben werden. Es wird jedoch keine korrekte
Funktionalität gewährleistet und darauf hingewiesen, dass diese Optionen
ggf. nich Corporate Design-Konform sind.

Als Schritgröße ist standardmäßig \PValue{10pt} eingestellt.

\begin{Declaration}
  \Option{arial}\\
  \Option{nexus}
\end{Declaration}

Standardmäßig wird ein Brief in der Schriftart \emph{Nexus} gesetzt.
Ist stattdessen eine Darstellung in der Schriftart \emph{Arial} erwünscht,
so kann dies mittels der Option \PName{arial} festgelegt werden.
Die Option \PName{nexus} sorgt für eine Darstellung in \emph{Nexus} und
kann optional angegeben werden.

\begin{Declaration}
  \Option{mono}
\end{Declaration}

Falls ein Brief nur in schwarz-weiß erstellt werden soll, können mit Hilfe
der Option \PName{mono} alle wesentlichen Elemente der Briefvorlage monochrom
dargestellt werden. Dazu zählen unter anderem Das Siegelband-Logo und die
angrenzende Trennlinie.
Auch sämtliche Elemente, die die Farbe \texttt{tuRed} verwenden, werden
in Schwarz statt in Rot dargestellt.

\section{Variablen}\label{sec:lettervars}

Eine Übersicht über alle zu verwendenden Variablen gibt
Tabelle~\ref{table:lettervars}. Es ist dabei meist nicht nötig zu wissen
wo die entsprechenden Werte verwendet werden, da das Design im allgemeinen
festgelegt ist. Zur besseren Veranschaulichung kann jedoch
Abbildung~\ref{fig:lettervarpos} zu Rate gezogen werden.

\begin{desctable}
  \caption{Von der Klasse \newdocumentclass{tubslttr2} unterstützte
    Variablen -- Inhaltsunabhängig}\label{table:lettervars}\\
  \Endfirsthead
  \caption[]{Von der Klasse \newdocumentclass{tubslttr2} unterstützte Variablen
    (\emph{Fortsetzung})}\\
  \Endhead%
  \standardfoot%
  \ventry{fromlogo}{Anweisungen zum Setzen des Institutslogos (Text oder Grafik)}%
  \ventry{frominstitute}{Name des Instituts}%
  \ventry{fromstreest}{Straßennahme und Hausnummer der Absenders}%
  \ventry{fromzipcode}{Postleitzahl des Absenders}%
  \ventry{fromtown}{Absender-Ort}%
  \ventry{fromfaxdirect}{Fax-Durchwahl}%
  \ventry{fromurl}{Internetaddresse}%
  \ventry{frombank}{Bankverbindung des Absenders}%
  \ventry{fromIBAN}{IBAN des Absender}%
  \ventry{fromBIC}{BIC-Code des Absenders}%
  \ventry{fromUStID}{Umsatzsteuer-ID des Absenders}%
  \ventry{fromSteuernummer}{Steuernummer des Absenders}%
  %
  \ventry{fromdepartment}{ggf. Abteilung oder andere Untereinheit des Instituts}%
  \ventry{fromtitle}{Titel des Absenders (Dr., Prof. Dr., \ldots)}
  \ventry{fromname}{Vollständiger Name des Absenders}
  \ventry{signature}{Signatur unter Unterschrift und Grußformel}
  \ventry{fromphonedirect}{Durchwahl}
  \ventry{fromemail}{E-Mail-Adresse}
\end{desctable}

\begin{desctable}
  \caption{Von der Klasse \newdocumentclass{tubslttr2} unterstützte
    Variablen -- Inhaltsabhängig}\label{table:lettervars}\\
  \Endfirsthead
  \ventry{specialmail}{Spezielle Versandart, wird zwischen Rücksendeadresse
    und Empfängeradresse dargestellt.}
  \ventry{yourref}{Feld \glqq Ihr Zeichen:\grqq}
  \ventry{yourmail}{Feld \glqq Ihre Nachricht vom:\grqq}
  \ventry{myref}{Feld \glqq Unser Zeichen:\grqq}
  \ventry{mymail}{Feld \glqq Unsere Nachricht vom:\grqq}
\end{desctable}


\begin{figure}[!ht]
\centering\fboxsep0mm\fbox{%
\includegraphics[width=0.9\textwidth]{../../letter/doc/examples/example_vars.pdf}
}
\caption{Darstellung der Variablen}
\label{fig:lettervarpos}
\end{figure}

\section{lco-Dateien}\label{sec:lco}

Vorschlag: Aufteilung nach Institut, Mitarbeiter.

Aufteilung ist nicht immer eindeutig.

Eine Beispielhafte Aufteilung findet sich im Anhang...

