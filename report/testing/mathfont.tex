\documentclass[a4paper,10pt,extramargin]{tubsartcl}
\usepackage[utf8x]{inputenc}
\usepackage{lipsum}

%opening
\title{}
\author{Enrico Jörns}

% \usepackage[LY1]{fontenc}
% \usepackage{mathpazo}
% \usepackage{nexus}
\usepackage{amsmath}    % need for subequations
%\usepackage{amsfonts}  note how statements can be commented out
%\usepackage{amssymb}
\usepackage{graphicx}   % for figures

\newcommand{\ajp}{AJP}  % example of a definition of a macro
\begin{document}

\title{Preparation of manuscripts for the American Journal of Physics\\
using \LaTeX}
%Lines break automatically or can be forced with \\
\author{Harvey Gould}
%  \altaffiliation[Also at ]{home.}  %  optional
%  \affiliation{Clark University, Department of Physics, Worcester,
% MA 01610}
%  \email{hgould@clarku.edu}   %optional
\author{Jan Tobochnik}
% \affiliation{Kalamazoo College, Department of Physics, Kalamazoo,
% MI 49007}
\date{\today}

\begin{abstract}
We gave some examples of the use of \LaTeX\ that we hope will be
helpful in preparing manuscripts for \ajp.
\end{abstract}

\maketitle

\section{Introduction}
\TeX\ looks more difficult than it is. It is
almost as easy as $\pi$. See how easy it is to make special
symbols such as $\alpha$,
$\beta$, $\gamma$,
$\delta$, $\sin x$, $\hbar$, $\lambda$, $\ldots$ We also can make
subscripts
$A_{x}$, $A_{xy}$ and superscripts, $e^x$, $e^{x^2}$, and
$e^{a^b}$. We will use \LaTeX, which is based on \TeX\ and has
many higher-level commands (macros) for formatting, making
tables, etc. More information can be found in
% Ref.~\onlinecite{latex}, a book written by Kopka and
Daly.\cite{latex}

We just made a new paragraph. Extra lines and spaces make no
difference. Note that all formulae are enclosed by
\$ and occur in \textit{math mode}.

The default font is Computer Modern. It includes \textit{italics}
or {\it italics}, \textbf{boldface} or {\bf boldface},
\textsl{slanted} or {\sl slanted}, and \texttt{monospaced} or {\tt
monospaced} (typewriter) fonts.

\section{Equations}
Let us see how easy it is to write equations.
\begin{equation}
\Delta =\sum_{i=1}^N w_i (x_i - \bar{x})^2 .
\end{equation}
For AJP all equations must be numbered, however for other uses, we can have an
equation without a number by writing
\begin{equation*}
P(x) = \frac{x - a}{b - a} , 
\end{equation*}
or
\begin{equation}
g = \frac{1}{2} \sqrt{2\pi} . \nonumber
\end{equation}

We can give an equation a label so that we can refer to it later.
\begin{equation}
\label{eq:ising}
E = -J \sum_{i=1}^N s_i s_{i+1} ,
\end{equation}
Equation~(\ref{eq:ising}) expresses the energy of a configuration
of spins.\cite{footnotes} If we use the amsmath package, we could write Eq.~\eqref{eq:ising}.

We can define our own macros to save typing. For example, suppose
that we introduce the macros:
\begin{verbatim}
 \newcommand{\lb}{{\langle}}
 \newcommand{\rb}{{\rangle}}
\end{verbatim}
\newcommand{\lb}{{\langle}}
\newcommand{\rb}{{\rangle}}
% verbatim useful for program listings
Then we can write the average value of $x$ as
\begin{verbatim}
\begin{equation}
\lb x \rb = 3
\end{equation}
\end{verbatim}
The result is
\begin{equation}
\lb x \rb = 3 .
\end{equation}

Examples of more complicated equations:
\begin{equation}
I = \! \int_{-\infty}^\infty f(x)\,dx \label{eq:fine}.
\end{equation}
We can do some fine tuning by adding small amounts of horizontal
spacing:
\begin{verbatim}
 \, small space       \! negative space
\end{verbatim}
as is done in Eq.~(\ref{eq:fine}).

We also can align several equations:
\begin{eqnarray}
a & =& b \\
c &=& d ,
\end{eqnarray}
or number them as subequations:
\begin{subequations}     % need amsmath package
\begin{eqnarray}
a & =& b \\
c &=& d .
\end{eqnarray}
\end{subequations}
If you have multiple lines for an equation but only want to number the whole 
equation, you suppress the equation numbers in any line by adding \nonumber before \\. 
A slightly better way to align equations is to use the amsmath package:
\begin{subequations} 
\begin{align}
a & = b \\
c &= d .
\end{align}
\end{subequations}
Can you notice the difference?
Some other examples:
\begin{equation}
\label{eq:mdiv}
m(T) =
\begin{cases}
0 & \text{$T > T_c$} \\
\bigl(1 - [\sinh 2 \beta J]^{-4} \bigr)^{\! 1/8} & \text{$T < T_c$}
\end{cases}
\end{equation}
\begin{align}
\textbf{T} &=
\begin{pmatrix}
T_{++} \hfill & T_{+-} \\
T_{-+} & T_{--} \hfill 
\end{pmatrix} \nonumber \\
& =
\begin{pmatrix}
e^{\beta (J + B)} \hfill & e^{-\beta J} \hfill \\
e^{-\beta J} \hfill & e^{\beta (J - B)} \hfill
\end{pmatrix}
\end{align}
\newcommand{\rv}{\textbf{r}}
\begin{equation}
\sum_i \vec A \cdot \vec B = -P \! \int \! \rv \cdot
\hat{\mathbf{n}}\, dA = P \! \int \! {\vec \nabla} \cdot \rv\, dV.
\end{equation}

\section{Lists}

Some example of formatted lists include the
following:

\begin{enumerate}

\item bread

\item cheese

\end{enumerate}

\begin{itemize}

\item Tom

\item Dick

\end{itemize}

An example of a table is given in Table~\ref{tab:tc} at the end
of the manuscript and examples of how to include figures are
shown in Figs.~\ref{fig:sine} and \ref{fig:lj}.

\section{Special Symbols}

\subsection{Common Greek letters}

These commands may be used only in math mode. Only the most common
letters are included.

$\alpha, 
\beta, \gamma, \Gamma,
\delta,\Delta,
\epsilon, \zeta, \eta, \theta, \Theta, \kappa,
\lambda, \Lambda, \mu, \nu,
\xi, \Xi,
\pi, \Pi,
\rho,
\sigma, 
\tau,
\phi, \Phi,
\chi,
\psi, \Psi,
\omega, \Omega$

\subsection{Special symbols}

The derivative is defined as
\begin{equation}
{dy \over dx} = \lim_{\Delta x \to 0}{\Delta y
\over
\Delta x}
\end{equation}
\begin{equation}
f(x) \to y \quad {\rm as} \quad x \to
x_{0}
\end{equation}
\begin{equation}
f(x) \mathop {\longrightarrow}
\limits_{x \to x_0} y
\end{equation}

\noindent Order of magnitude:
\begin{equation}
\log_{10}f \simeq n
\end{equation}
\begin{equation}
f(x)\sim 10^{n}
\end{equation}
Approximate equality:
\begin{equation}
f(x)\simeq g(x)
\end{equation}
\TeX\ is simple if we keep everything in proportion:
\begin{equation}
f(x) \propto x^3 .
\end{equation}

We can skip some space by using a command such as
\begin{verbatim}
\bigskip    \medskip    \smallskip    \vspace{1pc}
\end{verbatim}
The space can be negative.

And it sometimes is convenient to write $\tilde x$,
$\widetilde{xy}$, $\overline{A}$, \%, accents: Schr\"odinger
and \'e (or any other letter \'z).

\section{Literal text}
It is desirable to print program code exactly as it is typed in a
monospaced font. Use \verb=\begin{verbatim}= and
\verb=\end{verbatim}= as in the following example:
% = sign used as a delimiter
\begin{verbatim}
  public void computeArea()
  {
    this.area = this.length*this.length;
    System.out.println("Area = " + this.area);
   }
\end{verbatim}
The command \verb=\verbatiminput{programs/Square.java}= allows
the file \texttt{Square.java} in the direction \texttt{programs}
to be listed without changes.

\appendix
\section{Some dos and don'ts}

\begin{enumerate}

\item Note the \ajp\ style for books\cite{latex} and
articles.\cite{1d}

\item Also note the American convention of the positions
of citations.

\item Do not skip a line before \verb=\begin{equation}= or after 
\verb=\end{equation}=.
\end{enumerate}

% \begin{acknowledgments}
% Be sure to thank your colleagues and any granting agencies.
% \end{acknowledgments}

\begin{thebibliography}{5}

\bibitem{latex}Helmut Kopka and Patrick W. Daly, \textsl{A Guide to
\LaTeX: Document Preparation for Beginners and Advanced Users} (Addison-Wesley, Reading, MA, 1999), 3rd. ed. 

\bibitem{footnotes}It is necessary to process a file twice to
get the counters correct. \ajp\ does not use footnotes.

\bibitem{1d}B. C. Freasier, C. E. Woodward, and R. J. Bearman,
``Heat capacity extrema on isotherms in one-dimension: Two
particles interacting with the truncated Lennard-Jones potential
in the canonical ensemble,'' J. Chem. Phys. {\bf 105}, 3686--3690
(1996).


\end{thebibliography}

\newpage
\section*{Tables}

Tables are a little difficult until you get the
knack.
\LaTeX\ automatically calculates the width of the columns. Tables
should be placed at the end of the manuscript.

\begin{table}[h]
\begin{center}
\begin{tabular}{|l|l|r|l|}
\hline
lattice & $d$ & $q$ & $T_{\rm mf}/T_c$ \\
\hline
square & 2 & 4 & 1.763 \\
\hline
triangular & 2 & 6 & 1.648 \\
\hline
diamond & 3 & 4 & 1.479 \\
\hline
simple cubic & 3 & 6 & 1.330 \\
\hline
bcc & 3 & 8 & 1.260 \\
\hline
fcc & 3 & 12 & 1.225 \\
\hline
\end{tabular}
\caption{\label{tab:tc}Comparison of the mean-field predictions
for the critical temperature of the Ising model with exact results
and the best known estimates for different spatial dimensions $d$
and lattice symmetries.}
\end{center}
\end{table}

\newpage
\section*{Figures}
It is easy to include encapsulated postscript files (see Figure~\ref{fig:sine}). We can make figures bigger or smaller by scaling them. Figure~\ref{fig:lj} has been scaled by 80\%.
Figures should be placed at the end of the manuscript and sent as separate files. It also is possible to include pdf files.

\begin{figure}[h]
\begin{center}
% \includegraphics{figures/sine.eps}
\caption{\label{fig:sine}Show me a sine.}
\end{center}
\end{figure}

\begin{figure}[h]
\begin{center}
% \scalebox{0.80}{\includegraphics{figures/lj.eps}}
\caption{\label{fig:lj}Plot of the
Lennard-Jones potential
$u(r)$. The potential is characterized by a length
$\sigma$ and an energy
$\epsilon$. This potential is applied in Ref.~.}
\end{center}
\end{figure}

\end{document}

