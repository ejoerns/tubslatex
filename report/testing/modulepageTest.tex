\documentclass[a3paper,style=scifi,table,scifiposter]{tubsposter}
\usepackage[utf8]{inputenc}

\usepackage{tabularx,booktabs}
\usepackage{lipsum}

\begin{document}
  \begin{modulepage}[sender=bottom,bgcolor=tuVioletLight]{5cm,X,5.8cm,1.5cm}
  \showtubslogo[right]
  \showlogo{\fboxsep0mm\fbox{\includegraphics{dummy_institut}}}
  \begin{modrow}[bgcolor=tuGreenLight80]
    \bfseries\sffamily{\LARGE Das Corporate Design mit \LaTeX}\\[1em]
    von Max Mustermann
  \end{modrow}
  \begin{modrow}[X,X]
    \begin{modcol}
      {\usekomafont{subheadlinesmall}Lorem Ipsum}\\[2ex]
      \lipsum[1-2]
    \end{modcol}
    \begin{modcol}[X,X]
      \begin{modsubrow}[bgimage=infozentrum,imagefit=autoclip]
      \end{modsubrow}
      \begin{modsubrow}[c,bgcolor=tuOrange,fgcolor=tuGreenDark]
        \centering\itshape
        Dieser Text sollte tuGreenDark-farbig und tuOrange-hinterlegt sein.
        Außerdem sollte er horizontal und vertikal zentriert sein\par
        \raggedleft\upshape
        Loremus Ipsamus
      \end{modsubrow}
    \end{modcol}
  \end{modrow}
  \begin{modrow*}
    \large
    Dieses Modul sollte randlos dargestellt werden!
    Sowohl oben als auch unten als auch links als auch rechts.
  \end{modrow*}
  \begin{modrow}[bgcolor=tuGreen]
    \bfseries\raggedleft\textcolor{tuWhite}{\copyright 2011 TU Braunschweig}
  \end{modrow}
\end{modulepage}

\begin{modulepage}[sender=top,bgcolor=tuBlue]{5cm,X,5.8cm,1.5cm}
  \showtubslogo[right]
  \showlogo{\fboxsep0mm\fbox{\includegraphics{dummy_institut}}}
  \begin{posterrow}[bgcolor=tuOrangeDark80,fgcolor=tuWhite]
    \bfseries\sffamily{\LARGE Das Corporate Design mit \LaTeX}\\[1em]
    von Max Mustermann
  \end{posterrow}
  \begin{posterrow}[X,X]
    \begin{postercol}
      {\usekomafont{subheadlinesmall}Lorem Ipsum}\\[2ex]
      \lipsum[1-2]
    \end{postercol}
    \begin{postercol}[bgcolor=tuViolet,fgcolor=tuWhite,X,X]
      \begin{postersubrow}
        Diese Box sollte einen tuViolet-Hintergrund und tuWhite-Schrift haben
      \end{postersubrow}
      \begin{postersubrow}[frame=fbox]
        Diese Box sollte ebenfalls einen tuViolet-Hintergrund und tuWhite-Schrift haben.\\
        Darüber hinaus sollte sie von einem schwarzen Rahmen umgeben sein.
      \end{postersubrow}
    \end{postercol}
  \end{posterrow}
  \begin{modrow*}
    \large
    \rowcolors{3}{tuGreen40}{tuOrange40} \arrayrulecolor{red!75!gray}
    \begin{tabularx}{\textwidth}{lXXXXXX}
      \rowcolor{tuGray20}
      No. & 1970 & 1980 & 1990 & 2000 & 2010 & Future \\
      0   & 23   & 42   & 23   & 42   & 23   & 4711 \\
      1   & 42   & 23   & 42   & 23   & 23   & 4711 \\
      2   & 23   & 42   & 23   & 42   & 23   & 4711 \\
      3   & 42   & 23   & 42   & 23   & 23   & 4711 \\
      4   & 23   & 42   & 23   & 42   & 23   & 4711 \\
      5   & 42   & 23   & 42   & 23   & 23   & 4711 \\
    \end{tabularx}
  \end{modrow*}
  \begin{modrow}[bgcolor=tuGreen]
    \bfseries\raggedleft\textcolor{tuWhite}{\copyright 2011 TU Braunschweig}
  \end{modrow}
\end{modulepage}

  \begin{modulepage}[sender=bottom,bgcolor=tuVioletLight]{5cm,X,5.8cm,1.5cm}
  \end{modulepage}
  
\end{document}
