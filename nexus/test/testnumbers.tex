\documentclass[a4paper,11pt,parskip=full]{tubsartcl}
\usepackage[utf8]{inputenc}

%opening
\title{}
\author{}

\begin{document}

% Testen von \textlnum und \oldstylenums:
\large

\section*{Argumentkommandos}
\subsection*{rmfamily}
\rmfamily
Mediävalziffern: 1,2,3,4,5,6,7,8,0\\
Mediävalziffern: \oldstylenums{1,2,3,4,5,6,7,8,0}\\
Versalziffern: \textlnum{1,2,3,4,5,6,7,8,9,0}

\subsection*{sffamily}
\sffamily
Mediävalziffern: 1,2,3,4,5,6,7,8,0\\
Mediävalziffern: \oldstylenums{1,2,3,4,5,6,7,8,0}\\
Versalziffern: \textlnum{1,2,3,4,5,6,7,8,9,0}

% Testen von \lnum und \onum:

\section*{Schalter-Kommandos}
\subsection*{sffamily}

\newkomafont{myfont}{\sffamily}

{\usekomafont{myfont} Hier sind es Mediävalziffern: 0,1,2,3,4,5,6,7,8,9\\}
\addtokomafont{myfont}{\mdseries\lnum}
{\usekomafont{myfont} Hier sind es Versalziffern: 0,1,2,3,4,5,6,7,8,9\\}
Hier sind es wieder Mediävalziffern: 0,1,2,3,4,5,6,7,8,9

{\usekomafont{myfont} Hier sind es Versalziffern: 0,1,2,3,4,5,6,7,8,9\\}
{\usekomafont{myfont} Hier sind es Mediävalziffern: \onum 0,1,2,3,4,5,6,7,8,9}

Mix: \textlnum{1,2,3,\oldstylenums{4,5,6},7,8,9}

\subsection*{rmfamily}

\addtokomafont{myfont}{\rmfamily}

{\usekomafont{myfont} Hier sind es Mediävalziffern: 0,1,2,3,4,5,6,7,8,9\\}
\addtokomafont{myfont}{\mdseries\lnum}
{\usekomafont{myfont} Hier sind es Versalziffern: 0,1,2,3,4,5,6,7,8,9\\}
Hier sind es wieder Mediävalziffern: 0,1,2,3,4,5,6,7,8,9

{\usekomafont{myfont} Hier sind es Versalziffern: 0,1,2,3,4,5,6,7,8,9\\}
{\usekomafont{myfont} Hier sind es Mediävalziffern: \onum 0,1,2,3,4,5,6,7,8,9}

Mix: \textlnum{1,2,3,\oldstylenums{4,5,6},7,8,9}

\end{document}
